% !TEX root = main.tex

\chapter{Constructing Galois representations}





\section{Notation and necessary results}

In this chapter we loosely summarize, and adapt as needed, the results of 
\cite{khare-larsen-ramakrishna-2005,pande-2011}. Throughout, if $F$ is a field, 
$M$ a $G_F$-module, we write $\h^i(F,M)$ in place of $\h^1(G_F,M)$. All Galois 
representations will be to $\GL_2(\bZ/l^n)$ or $\GL_2(\bZ_l)$ for $l$ a (fixed) 
rational prime, and all deformations will have fixed determinant, so we only 
consider the cohomology of $\Ad^0\bar\rho$, the induced representation on 
trace-zero matrices by conjugation. 

If $S$ is a set of rational primes, $\bQ_S$ denotes the largest extension of 
$\bQ$ unramified outside $S$. So $\h^i(\bQ_S,-)$ is what is usually written as 
$\h^1(G_{\bQ,S},-)$. If $M$ is a $G_\bQ$-module and $S$ a finite set of primes, 
write 
\[
	\sha^i_S(M) = \ker\left( \h^i(\bQ_S,M) \to \prod_{p\in S} \h^i(\bQ_p,M)\right) .
\]
If $l$ is a rational prime and $S$ a finite set of primes containing $l$, then 
for any $\bF_l[G_{\bQ_S}]$-module $M$, write $M^\vee=\hom_{\bF_l}(M,\bF_l)$ 
with the obvious $G_{\bQ_S}$-action, and write $M^\ast = M^\vee(1)$ for the 
Cartier dual. By \cite[Th.~8.6.7]{neukirch-schmidt-winberg-2008}, there is an 
isomorphism $\sha^1_S(M^\ast) = \sha_S^2(M)^\vee$. 

Recall that if $\rho_n\colon G_{\bQ_S} \to \GL_2(\bZ/l^n)$ is a deformation of 
$\bar\rho$, then $\rho_n$ lifts to a $\GL_2(\bZ/l^{n+1})$-valued deformation if 
and only if the ``obstruction class'' of $\rho_n$ in 
$\h^2(\bQ_S,\Ad^0\bar\rho)$ vanishes. If $\sha_S^2(\Ad^0\bar\rho) = 0$, then 
$\rho_n$ lifts if and only if each $\left.\rho_n\right|_{G_{\bQ_p}}$ lifts. 
Moreover, if $\rho_n$ lifts to $\bZ/l^{n+1}$, the set of its lifts form a 
$\h^1(\bQ_S,\Ad^0\bar\rho)$-torsor. Write this action multiplicatively, i.e.~if 
$\rho_{n+1}$ is one lift and $f\in \h^1(\bQ_S,\Ad^0\bar\rho)$, then 
$f\cdot \rho_{n+1}$ is the lift that \cite{khare-larsen-ramakrishna-2005} write 
as $(1+p f)\rho_2$. If $\rho_n$ lifts and $\sha_S^1(\Ad^0\bar\rho) = 0$, then 
the choice of $\left. \rho_{n+1}\right|_{G_{\bQ_p}}$ for all primes $p$ can be 
satisfied by at most one specific lift $\rho_{n+1}$. 

\begin{theorem}\label{thm:always-can-lift}
Let $l\geqslant 7$ be a rational prime and 
$\bar\rho\colon G_\bQ \to \GL_2(\bF_l)$ an odd, absolutely irreducible, 
weight-$2$ representation. Then there exists a weight-$2$ lift of $\bar\rho$ to 
$\bZ_l$. 
\end{theorem}
\begin{proof}
This is \cite[Th.~1]{ramakrishna-2002}, taking into account that the paper in 
question allows for arbitrary fixed determinants. 
\end{proof}





\section{Galois representations with specified Satake parameters}

Fix a rational prime $l\geqslant 5$ and a continuous, absolutely irreducible 
representation $\bar\rho\colon G_\bQ \to \GL_2(\bF_l)$ that is weight-$2$. We 
consider weight-$2$ deformations of $\bar\rho$. The final deformation, 
$\rho\colon G_\bQ \to \GL_2(\bZ_l)$, will be constructed as the inverse limit 
of a compatible collection of lifts $\rho_n\colon G_\bQ \to \GL_2(\bZ/l^n)$. At 
any given stage, we will be concerned with making sure that a) there exists a 
lift to the next stage, and b) there is a lift with the necessary properties. 
Fix a sequence $\bx=(x_1,x_2,\dots)$ in $[-1,1]$. The set of unramified primes 
of $\rho$ is not determined at the beginning, but at each stage there will be 
a large finite set $R_n$ of primes which we know will remain unramified. 
Re-indexing $\bx$ by these unramified primes, we will construct $\rho$ so that 
for all unramified primes $p$, $\tr\rho(\frob_p)\in \bZ$, satisfies the Hasse 
bound, and has $\tr\rho(\frob_p) \approx x_p$. Moreover, we can ensure that the 
set of ramified primes has density zero in a very strong sense (controlled by a 
parameter function $h$) and that our trace of Frobenii are very close to 
specified values (the ``closeness'' again controlled by a parameter function 
$b$). 

Given any deformation $\rho_n$, write $\pi_{\nr(\rho_n)}(x)$ for the function 
which counts $\rho_n$-unramified primes $\leqslant x$. 

\begin{theorem}\label{thm:master-Galois}
Let $l$, $\bar\rho$, $\bx$ be as above. Fix functions 
$h,b\colon \bR^+\to \bR^+$ which decrease to zero (resp.~increase to infinity). 
Then there exists a weight-$2$ deformation $\rho$ of $\bar\rho$, such that 
\begin{enumerate}
\item
$\pi_{\nr(\rho)}(x) \ll h(x) \pi(x)$. 

\item
For each unramified prime $p$, $a_p=\tr\rho(\frob_p)\in \bZ$ and satisfies the 
Hasse bound. 

\item
For each unramified prime $p$, 
$\left| \frac{a_p}{2\sqrt p} - x_p\right| \leqslant \frac{l b_p}{\sqrt p}$. 
\end{enumerate}
\end{theorem}
\begin{proof}
?
\end{proof}


Basic ideas is as follows. Start with $\rho_1\colon G_\bQ \to \GL_2(\bF_l)$. At 
each stage, we have $\rho_n\colon G_\bQ \to \GL_2(\bZ/l^n)$. At that stage, 
we're allowed to choose the (integral) characteristic polynomial for Frobenius 
at an arbitrarily large set of primes $R_n$. Then some (but density zero) extra 
primes ramify, and then we get $\rho_{n+1}$ that agrees with our choices for 
$R_1\cup \cdots \cup R_n$. Then we have to choose characteristic polynomials for 
$R_{n+1}$, but these are already determined modulo $l^{n+1}$. 

Basic idea is, we can choose the $R_n$ to be so huge that the primes involved are 
way way bigger than $l^n$. For example, we could have 
$R_n=\{p\leqslant l^{l^n}\}$. Thus the set of possible $a_p$'s is very big (big 
enough) so that we can get discrepancy to behave as we like (both decaying 
slowly and decaying quickly) to any measure $\mu$ such that $\ST/\mu$ is 
bounded away from zero. (By this, we mean: if $\ST=f\cdot\lambda$ and 
$\mu = g\cdot\lambda$, where $\lambda$ is Lebesgue, then $f/g$ is bounded away 
from zero.)

We loosely summarize \cite{pande-2011}, changing the notation to make things 
clearer. 

Start with a prime $l$ and a representation 
$\bar\rho\colon G_\bQ \to \GL_2(\bZ/l)$. For any set $S$ of primes outside 
which $\bar\rho$ is unramified, let $\cX_{\bar\rho,S}$ be the space of 
deformations of $\bar\rho$ that are also unramified outside $S$. If 
$\chi\colon G_{\bQ,S} \to \bZ_l^\times$ is a character (perhaps a power of the 
cyclotomic character) let $\cX_{\bar\rho,S}^\chi$ be the subspace of 
$\cX_{\bar\rho,S}$ with determinant $\chi$. 

It is known that the obstruction to lifting an element of 
$\cX_{\bar\rho,S}^\chi(\bZ/l^n)$ to $\cX_{\bar\rho,S}^\chi(\bZ/l^{n+1})$ lies 
in $\h^2(G_{\bQ,S},\Ad^0\bar\rho)$. If the obstruction vanishes, lifts are a
torsor under a natural action of $\h^1(G_{\bQ,S},\Ad^0\bar\rho)$. 

Given $\bar\rho$, there is a set of primes $p$ called \emph{nice primes}, which 
are those $p$ for which:
\begin{enumerate}
\item
$\bar\rho$ is unramified at $p$.

\item
The ratio of eigenvalues of $\bar\rho(\frob_p)$ is $p$. 
\end{enumerate}
If $p$ is a nice prime, there is a subspace $\cC_p$ of deformations of 
$\left.\bar\rho\right|_{G_{\bQ_p}}$ consisting of \emph{nice deformations}. 
(Not too sure about that.)

At the very least, $p$ is nice iff 
$\Ad^0 \bar\rho\simeq \bF_l\oplus \bF_l(1)\oplus \bF_l(-1)$. (Maybe this is a 
nicer way to think about it.)

Then put $N_p = \h^1(G_{\bQ_p},\bF_l(1))\subset \h^1(G_{\bQ_p},\Ad^0\bar\rho)$. 
Let $C_p$ be deformations of $\bar\rho$ which are fixed by $N_p$. (What does 
this mean?)





If $\rho$ is a Galois representation, let $\pi_{\nr(\rho)}(x)$ be the number of 
$p\leqslant x$ such that $\rho$ is unramified at $p$. 

\begin{theorem}
Let $\bx=(x_1,x_2,\dots)$ be a sequence in $[-1,1]$. Let $l\geqslant 5$ be a 
rational prime, and let $\bar\rho\colon G_\bQ \to \GL_2(\bF_l)$ be a continuous, 
absolutely irreducible representation that is odd of weight $2$. Fix a function 
$h\colon \bR^+ \to \bR^+$ which decreases to zero, and a function 
$b\colon \bN \to \bN$ which increases to infinity. Then there exists a 
continuous lift of $\bar\rho$, namely  $\rho\colon G_\bQ\to \GL_2(\bZ_l)$, also 
of weight $2$, such that 
\begin{enumerate}
\item
$\pi_{\nr(\rho)}(x) \ll h(x) \pi(x)$.

\item
For each unramified prime $p$, $a_p = \tr \rho(\frob_p)\in \bZ$, satisfies 
$|a_p| \leqslant 2\sqrt p$.

\item
If $\{x_p\}$ is a renumeration of $\bx$ by unramified primes, then 
$\left|\frac{a_p}{2\sqrt p} - x_p\right| \leqslant l b(p) p^{-1/2}$. 
\end{enumerate}
\end{theorem}
\begin{proof}
By \cite[Lem.~6]{khare-larsen-ramakrishna-2005}, there exist a finite set 
$S$, containing the primes at which $\bar\rho$ ramifies, such that 
$\sha_S^1(\Ad^0\bar\rho) = \sha_S^2(\Ad^0\bar\rho) = 0$. We will 
consider deformations of $\bar\rho$ that are also of weight $2$ and ramified at 
$S$ (or a larger, but finite set). Thus, a deformation to $\bZ/l^n$ will 
admit a lift to $\bZ/l^{n+1}$ if and only if all its restrictions to 
$G_{\bQ_p}$ admit such lifts. As we progress, re-index $\bx$ so that its 
segment is indexed by unramified primes of $\rho_n$. 

Let $S_2$ be the union of $S$ and those primes $p$ for which there does 
\emph{not} exist an $a\in \bZ$ with $a\equiv \tr\bar\rho(\frob_p)\pmod{l}$ and 
$|a|\leqslant 2\sqrt p$. For the remainder of the proof, we modify the methods 
of\cite{khare-larsen-ramakrishna-2005,pande-2011}. For all primes $p\in S_2$, 
choose a ramified lift $\rho_p$ of $\left.\bar\rho\right|_{G_{\bQ_p}}$ to 
$\bZ_l$. Let $r_2$ be large enough so that $r_2 < p$ implies $l < b(p)$. Let 
$R_2 = \{p\notin S_2 : p \leqslant r_2\}$, and choose $a_p$ for $p\in R_2$ so 
that $|a_p| \leqslant 2\sqrt p$ and 
$\left|\frac{a_p}{2\sqrt p} - x_p\right| < l p^{-1/2}$. For $p\in R_2$, let 
$\rho_p$ be the unique unramified representation $G_{\bQ_p} \to \GL_2(\bZ_l)$ 
that has weight $2$ and $\tr\rho_p(\frob_p) = a_p$. 

By Theorem \ref{thm:always-can-lift}, $\bar\rho$ is modular, so in particular 
it admits a lift $\rho_2^\ast$ to $\bZ/l^2$. By 
\cite[Lem.~8]{khare-larsen-ramakrishna-2005}, there exists a finite set of 
nice primes $Q_2^\ast$, such that the map 
\[
	\h^1(\bQ_{S_2\cup Q_2^\ast},\Ad^0\bar\rho) \to \prod_{p\in S_2} \h^1(\bQ_p,\Ad^0\bar\rho) \times \prod_{p\in R_2} \h_\nr^1(\bQ_p,\Ad^0\bar\rho) 
\]
is an isomorphism. Moreover, an examination of the proof of 
\cite[Fact 5]{khare-larsen-ramakrishna-2005} shows that we may ensure that 
$\pi_{\nr(\rho_2^\ast)}(p) \leqslant h(p) \pi(p)$ for all $p\in Q_2^\ast$. 
Since the above map is an isomorphism, there exists 
$f_2\in \h^1(\bQ_{S_2\cup Q_2^\ast},\Ad^0\bar\rho)$ such that 
$f_2\cdot\bar\rho\equiv \rho_p\pmod{l^2}$ for all $p\in S_2\cup R_2$. Clearly 
$\rho_2^\ast$ can lift to $\bZ/l^3$ when restricted to all primes in 
$S_2\cup R_2$, but it might be obstructed at some primes in $Q_2^\ast$. By 
\cite[Prop.~3.10]{pande-2011}, there exists a set $A_2$ of nice primes, with 
$\# A_2\leqslant 2$ and with $\pi_\nr(p) \leqslant h(p) \pi(p)$ for 
$p\in A_2$, together with 
$g_2\in \h^1(\bQ_{S_2\cup Q_2^\ast \cup A_2},\Ad^0\bar\rho)$, such that 
$(f_2+g_2)\cdot\rho_2^\ast$ still agrees with $\rho_p$ modulo $l^2$ for all 
$p\in S_2\cup R_2$, and $(f_2+g_2)\cdot\rho_2^\ast$ is unobstructed at all 
primes in $Q_2 = Q_2^\ast\cup A_2$. 

Set $\rho_2 = (f_2+g_2)\cdot\rho_2^\ast$ and put $S_3 = S_2\cup Q_2$. Then 
$\rho_2$ admits a lift to $\bZ/l^3$; call it $\rho_3^\ast$. For all 
$p\in S_3\smallsetminus S_2$, choose a ramified, weight-$2$ lift $\rho_p$ of 
$\left.\rho_2\right|_{G_{\bQ_p}}$ to $\bZ_l$. Let $r_3$ be large enough so that 
$r_3 < p$ implies $l^2 < b(p)$, and let 
$R_3 = \{p\notin S_3 : p\leqslant r_3\}$. For all $p\in R_3\smallsetminus R_2$, 
choose $a_p$ such that $\left|\frac{a_p}{2\sqrt p} - x_p\right| < l^2 p^{-1/2}$ 
and $|a_p| \leqslant 2\sqrt p$ and $a_p \equiv \tr\rho_2(\frob_p)\pmod{l^2}$. 

Once again, there exists a finite set of primes $Q_3$ and 
$f_3\in \h^1(\bQ_{S_3\cup Q_3},\Ad^0\bar\rho)$ such that 
$f_3\cdot\rho_3^\ast\equiv \rho_p\pmod{l^3}$ for all $p\in S_3\cup R_3$, and 
$\rho_3 = f_3\cdot\rho_3^\ast$ is unobstructed for primes in $Q_3$. Moreover, 
we can ensure that $\pi_\nr(p) \leqslant h(p) \pi(p)$ for all $p\in Q_3$. Set 
$S_4 = S_3\cup Q_3$ and continue in the same manner. 

For the general inductive step, let $\rho_{n+1}^\ast$ be a lift of $\rho_n$ 
from $\bZ/l^n$ to $\bZ/l^{n+1}$. Let $r_{n+1}$ be large enough so that 
$r_{n+1} < p$ implies $l^n < b(p)$, and let 
$R_{n+1} = \{p\notin S_{n+1} : p\leqslant r_{n+1}\}$. For 
$p\in S_{n+1}\smallsetminus S_n$, choose a ramified, weight-$2$ lift $\rho_p$ 
of $\left.\rho_n\right|_{G_{\bQ_p}}$ to $\bZ_l$, and for 
$p\in R_{n+1}\smallsetminus R_n$, choose $a_p\in \bZ$ satisfying the Hasse 
bound such that $\left|\frac{a_p}{2\sqrt p} - x_p\right| < l^n p^{-1/2}$. As 
before, there exists a finite set $Q_{n+1}$ of nice primes, enforcing 
$\pi_\nr \leqslant h \pi$, and 
$f_{n+1}\in H^1(\bQ_{S_{n+1}\cup Q_{n+1}},\Ad^0\bar\rho)$ such that 
$\rho_{n+1} = f_{n+1}\cdot\rho_{n+1}^\ast$ agrees with $\rho_p$ modulo 
$l^{n+1}$ for all $p\in S_{n+1}\cup R_{n+1}$, and moreover $\rho_{n+1}$ is 
unobstructed at $Q_{n+1}$. Set $S_{n+2} = S_{n+1} \cup Q_{n+1}$, and the 
inductive step is complete. 
\end{proof}
