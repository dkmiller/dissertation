% !TEX root = main.tex

\chapter{Constructing Galois representations}





\section{Main idea}

Basic ideas is as follows. Start with $\rho_1\colon G_\bQ \to \GL_2(\bF_l)$. At 
each stage, we have $\rho_n\colon G_\bQ \to \GL_2(\bZ/l^n)$. At that stage, 
we're allowed to choose the (integral) characteristic polynomial for Frobenius 
at an arbitrarily large set of primes $R_n$. Then some (but density zero) extra 
primes ramify, and then we get $\rho_{n+1}$ that agrees with our choices for 
$R_1\cup \cdots \cup R_n$. Then we have to choose characteristic polynomials for 
$R_{n+1}$, but these are already determined modulo $l^{n+1}$. 

Basic idea is, we can choose the $R_n$ to be so huge that the primes involved are 
way way bigger than $l^n$. For example, we could have 
$R_n=\{p\leqslant l^{l^n}\}$. Thus the set of possible $a_p$'s is very big (big 
enough) so that we can get discrepancy to behave as we like (both decaying 
slowly and decaying quickly) to any measure $\mu$ such that $\ST/\mu$ is 
bounded away from zero. (By this, we mean: if $\ST=f\cdot\lambda$ and 
$\mu = g\cdot\lambda$, where $\lambda$ is Lebesgue, then $f/g$ is bounded away 
from zero.)

We loosely summarize \cite{pande-2011}, changing the notation to make things 
clearer. 

Start with a prime $l$ and a representation 
$\bar\rho\colon G_\bQ \to \GL_2(\bZ/l)$. For any set $S$ of primes outside 
which $\bar\rho$ is unramified, let $\cX_{\bar\rho,S}$ be the space of 
deformations of $\bar\rho$ that are also unramified outside $S$. If 
$\chi\colon G_{\bQ,S} \to \bZ_l^\times$ is a character (perhaps a power of the 
cyclotomic character) let $\cX_{\bar\rho,S}^\chi$ be the subspace of 
$\cX_{\bar\rho,S}$ with determinant $\chi$. 

It is known that the obstruction to lifting an element of 
$\cX_{\bar\rho,S}^\chi(\bZ/l^n)$ to $\cX_{\bar\rho,S}^\chi(\bZ/l^{n+1})$ lies 
in $\h^2(G_{\bQ,S},\Ad^0\bar\rho)$. If the obstruction vanishes, lifts are a
torsor under a natural action of $\h^1(G_{\bQ,S},\Ad^0\bar\rho)$. 

Given $\bar\rho$, there is a set of primes $p$ called \emph{nice primes}, which 
are those $p$ for which:
\begin{enumerate}
\item
$\bar\rho$ is unramified at $p$.

\item
The ratio of eigenvalues of $\bar\rho(\frob_p)$ is $p$. 
\end{enumerate}
If $p$ is a nice prime, there is a subspace $\cC_p$ of deformations of 
$\left.\bar\rho\right|_{G_{\bQ_p}}$ consisting of \emph{nice deformations}. 
(Not too sure about that.)

At the very least, $p$ is nice iff 
$\Ad^0 \bar\rho\simeq \bF_l\oplus \bF_l(1)\oplus \bF_l(-1)$. (Maybe this is a 
nicer way to think about it.)

Then put $N_p = \h^1(G_{\bQ_p},\bF_l(1))\subset \h^1(G_{\bQ_p},\Ad^0\bar\rho)$. 
Let $C_p$ be deformations of $\bar\rho$ which are fixed by $N_p$. (What does 
this mean?)
