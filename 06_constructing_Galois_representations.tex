% !TEX root = main.tex

\chapter{Constructing Galois representations}





\section{Main idea}

Basic ideas is as follows. Start with $\rho_1\colon G_\bQ \to \GL_2(\bF_l)$. At 
each stage, we have $\rho_n\colon G_\bQ \to \GL_2(\bZ/l^n)$. At that stage, 
we're allowed to choose the (integral) characteristic polynomial for Frobenius 
at an arbitrarily large set of primes $R_n$. Then some (but density zero) extra 
primes ramify, and then we get $\rho_{n+1}$ that agrees with our choices for 
$R_1\cup \cdots \cup R_n$. Then we have to choose characteristic polynomials for 
$R_{n+1}$, but these are already determined modulo $l^{n+1}$. 

Basic idea is, we can choose the $R_n$ to be so huge that the primes involved are 
way way bigger than $l^n$. For example, we could have 
$R_n=\{p\leqslant l^{l^n}\}$. Thus the set of possible $a_p$'s is very big (big 
enough) so that we can get discrepancy to behave as we like (both decaying 
slowly and decaying quickly) to any measure $\mu$ such that $\ST/\mu$ is 
bounded away from zero. (By this, we mean: if $\ST=f\cdot\lambda$ and 
$\mu = g\cdot\lambda$, where $\lambda$ is Lebesgue, then $f/g$ is bounded away 
from zero.)

We loosely summarize \cite{pande-2011}, changing the notation to make things 
clearer. 

Start with a prime $l$ and a representation 
$\bar\rho\colon G_\bQ \to \GL_2(\bZ/l)$. For any set $S$ of primes outside 
which $\bar\rho$ is unramified, let $\cX_{\bar\rho,S}$ be the space of 
deformations of $\bar\rho$ that are also unramified outside $S$. If 
$\chi\colon G_{\bQ,S} \to \bZ_l^\times$ is a character (perhaps a power of the 
cyclotomic character) let $\cX_{\bar\rho,S}^\chi$ be the subspace of 
$\cX_{\bar\rho,S}$ with determinant $\chi$. 

It is known that the obstruction to lifting an element of 
$\cX_{\bar\rho,S}^\chi(\bZ/l^n)$ to $\cX_{\bar\rho,S}^\chi(\bZ/l^{n+1})$ lies 
in $\h^2(G_{\bQ,S},\Ad^0\bar\rho)$. If the obstruction vanishes, lifts are a
torsor under a natural action of $\h^1(G_{\bQ,S},\Ad^0\bar\rho)$. 

Given $\bar\rho$, there is a set of primes $p$ called \emph{nice primes}, which 
are those $p$ for which:
\begin{enumerate}
\item
$\bar\rho$ is unramified at $p$.

\item
The ratio of eigenvalues of $\bar\rho(\frob_p)$ is $p$. 
\end{enumerate}
If $p$ is a nice prime, there is a subspace $\cC_p$ of deformations of 
$\left.\bar\rho\right|_{G_{\bQ_p}}$ consisting of \emph{nice deformations}. 
(Not too sure about that.)

At the very least, $p$ is nice iff 
$\Ad^0 \bar\rho\simeq \bF_l\oplus \bF_l(1)\oplus \bF_l(-1)$. (Maybe this is a 
nicer way to think about it.)

Then put $N_p = \h^1(G_{\bQ_p},\bF_l(1))\subset \h^1(G_{\bQ_p},\Ad^0\bar\rho)$. 
Let $C_p$ be deformations of $\bar\rho$ which are fixed by $N_p$. (What does 
this mean?)





\section{Summary}

Here we summarize the results of 
\cite{khare-larsen-ramakrishna-2005,pande-2011}, adapting them as needed. Let 
$\bar\rho\colon G_\bQ \to \GL_2(\bZ/l)$ be a continuous, absolutely irreducible 
representation. By Serre's conjecture, $\bar\rho$ is modular. 

By \cite[Lem.~6]{khare-larsen-ramakrishna-2005}, there exists a finite set $S$ 
of primes, containing all primes at which $\bar\rho$ ramifies, such that 
$\sha_S^1(\Ad^0\bar\rho) = \sha_S^2(\Ad^0\bar\rho) = 0$. 

Since $\bar\rho$ is modular, it admits a lift to $\bZ/l^2$. Call it $\rho_2$. 
Let $S_2 = S$. 

For primes $p\in S_2$, choose an arbitrary weight-$2$ lift 
$\rho_p\colon G_{\bQ_p} \to \GL_2(\bZ_l)$. 

Let $R_2 = \{p\notin S_2 : p\leqslant r_2\}$, for arbitrary choice of $r_2$ 
(ideally very large compared to $l$). 

For primes $p\in R_2$, choose an arbitrary unramified weight-$2$ lift 
$\rho_p\colon G_{\bQ_p} \to \GL_2(\bZ_l)$. This is determined by its 
characteristic polynomial, which we assume to be in $\bZ[t]$. 

By \cite[Lem.~8]{khare-larsen-ramakrishna-2005}, there is a finite set $Q_2$ of 
primes such that the following maps is an isomorphism:
\[
	\h^1(G_{S_2\cup Q_2},\Ad^0\bar\rho) 
		\to \prod_{p\in S_2} \h^1(G_{\bQ_p},\Ad^0\bar\rho)\times \prod_{p\in R_2} \h^1_\nr(G_{\bQ_p},\Ad^0\bar\rho) .
\]

Since $\rho_1$ has no obstructions to lifting, we know that 
$\h^1(G_{S_2\cup Q_2},\Ad^0\bar\rho)$ acts simply transitively on the set of 
lifts of $\rho_1$ to $\bZ/l^2$. By the above isomorphism, there exists 
$f_2\in \h^1(G_{S_2\cup Q_2},\Ad^0\bar\rho)$ such that 
$\left.f_2\cdot \rho_2\right|_{G_{\bQ_p}} \equiv \rho_p\pmod{l^2}$ for all 
$p\in S_2\cup R_2$. 

Now clearly, $f_2\cdot \rho_2$ can be lifted to $\bZ/l^3$ when restricted to 
all primes in $S_2\cup R_2$. However, it may not necessarily lift for primes in 
$Q_2$. By \cite[Lem.~3.10]{pande-2011}, there is a set $A_2$, of at most 
two primes, and $g_2\in \h^1(G_{S_2\cup Q_2\cup A_2},\Ad^0\bar\rho)$ such that 
$g_2\cdot f_2\cdot \rho_2$ still agrees with our chosen lifts on 
$S_2\cup R_2$, and moreover has no local obstructions to lifting in 
$Q_2\cup A_2$. 

Put $S_3 = S_2\cup Q_2\cup A_2$, and let 
$R_3 = R_2\cup \{p\notin S_3 : p\leqslant r_3\}$, where $r_3\gg r_2$. Choose 
arbitrary lifts of $g_2 f_2 \rho_2$ restricted to primes in 
$S_3\smallsetminus S_2$ and $R_3\smallsetminus R_2$. 

Generally, suppose we have a representation $\rho_n$ to $\bZ/l^n$, which 
agrees with our choices of lifts for all primes in $S_n$ and $R_n$, and 
we choose lifts of $\rho_n$ to $\bZ_l$ for primes in 
$R_{n+1}\smallsetminus R_n$ and $S_{n+1}\smallsetminus S_n$. There will be a 
set $Q_{n+1}^\ast$ such that if we allow ramification at $Q_{n+1}^\ast$, we can 
ensure that there is a lift of $\rho_n$ to $\bZ/l^{n+1}$ which agrees with 
our choices of lifts for primes in $S_{n+1}$ and $R_{n+1}$. 





\section{More summary}

We begin by bounding the size of some sets. In the discussion below, sets 
$Q_{n+1}^\ast$ arise. By the proof of 
\cite[Lem.~8]{khare-larsen-ramakrishna-2005}, we see that 
$\# Q_{n+1} = h^1(G_{\bQ,S_{n+1}\cup R_{n+1}},\Ad^0\bar\rho^\ast)$. Adapting 
equation (9) of the same paper with $M=\Ad^0\bar\rho$ and 
$L_p = 0$, we see that 
\[
	h^1(G_{\bQ,S_{n+1}\cup R_{n+1}},\Ad^0\bar\rho^\ast) = h^0(\bQ,\Ad\bar\rho^\ast) + \sum_{p\in S_{n+1}\cup R_{n+1}} h^0(\bQ_p,\Ad^0\bar\rho) .
\]
The $h^0(\bQ,\Ad^0\bar\rho) = 0$, at least if $\bar\rho$ surjects onto 
$\GL_2(\bZ/l)$ (prove this!). Moreover, if $p$ is a nice prime, then 
$h^0(\bQ_p,\Ad^0\bar\rho) = 1$. It follows that 
\[
	\# Q_{n+1}^\ast = \# S_{n+1} + \sum_{p\in R_{n+1}} h^0(\bQ_p,\Ad^0\bar\rho) .
\]
Now, $S_{n+1} = S_2\bigcup_{m\leqslant n} Q_m^\ast \bigcup_{m\leqslant n} A_m$, 
and $\# A_m \leqslant 2$. So 
\[
	\# Q_{n+1}^\ast \leqslant \# S_2 + \sum_{m\leqslant n} \# Q_m^\ast + 2 n + 3\# R_{n+1} .
\]
(Not sure if this matters\ldots)

Let $\bar\rho\colon G_\bQ \to \GL_2(\bZ/l)$ be a continuous, odd, absolutely 
irreducible representation. By results of Khare--Wintenberger, $\bar\rho$ is 
modular. By \cite[Lem.~6]{khare-larsen-ramakrishna-2005}, there exists a finite 
set $S$ of primes, containing the primes at which $\bar\rho$ ramifies, such 
that $\sha_S^1(\Ad^0\bar\rho) = \sha_S^2(\Ad^0\bar\rho) = 0$, and thus any 
obstructions to lifting can be detected locally. 

Let $S_2=S$, and for each $p\in S_2$, choose a lift 
$\rho_p\colon G_{\bQ_p} \to \GL_2(\bZ_l)$ of 
$\left.\bar\rho\right|_{G_{\bQ_p}}$. Choose $r_2$, and let 
$R_2 = \{p\notin S_2 : p\leqslant r_2\}$. Choose $a_p\in \bZ$ (hence a lift 
$\rho_p\colon G_{\bQ_p} \to \GL_2(\bZ_l)$ of 
$\left.\bar\rho\right|_{G_{\bQ_p}}$) for all $p\in R_2$. 

Let $\rho_2^\ast$ be any lift of $\bar\rho$ to 
$G_{\bQ,S_2} \to \GL_2(\bZ/l^2)$. Such lifts exist because $\bar\rho$ is 
modular. By \cite[Lem.~8]{khare-larsen-ramakrishna-2005}, there is a finite set $Q_2^\ast$ of primes such that the following maps is an isomorphism:
\[
	\h^1(G_{\bQ,S_2\cup Q_2^\ast},\Ad^0\bar\rho) 
		\to \prod_{p\in S_2} \h^1(G_{\bQ_p},\Ad^0\bar\rho)\times \prod_{p\in R_2} \h^1_\nr(G_{\bQ_p},\Ad^0\bar\rho) .
\]
Looking at the proof of \cite[Fact 5]{khare-larsen-ramakrishna-2005}, we can 
ensure that all primes in $Q_2^\ast$ are larger than an arbitrary fixed bound 
$q_2$. In fact, we can ensure that the primes in $Q_2^\ast$ (and $Q_2$) grow 
faster than any function we give. 
