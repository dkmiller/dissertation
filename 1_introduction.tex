% !TEX root = Daniel-Miller-thesis.tex

\chapter{Introduction}





\section{Motivation from classical analytic number theory}

Start with an old problem central to number theory: counting 
prime numbers. As usual, let $\pi(x)$ be the prime counting function and 
$\Li(x) = \int_2^x \frac{\dd t}{\log t}$ be the Eulerian logarithmic integral. 
The prime number theorem tells us that as $x\to \infty$, 
$\frac{\pi(x)}{\Li(x)} \to 1$. The standard approach 
to proving the prime number theorem is by showing that the Riemann 
$\zeta$-function has non-vanishing meromorphic continuation to $\Re = 1$.

\begin{theorem}
The function $\zeta(s)$ admits a non-vanishing meromorphic continuation to 
$\Re = 1$ with at most a simple pole at $s=1$, if and only if 
$\lim_{x\to \infty} \frac{\pi(x)}{\Li(x)} = 1$. 
\end{theorem}

Since $\zeta(s)$ does have the desired properties, the prime number 
theorem is true. It is natural to try to bound the difference 
$\pi(x) - \Li(x)$. Numerical experiements dating back to Gauss 
suggest that $|\pi(x) - \Li(x)| \ll x^{-\frac 1 2+\epsilon}$, and in fact we 
have the following result. 

\begin{theorem}[{\cite{von-koch-1901}}]
The Riemann hypothesis is true if and only if 
$|\pi(x) - \Li(x)| \ll x^{-\frac 1 2+\epsilon}$. 
\end{theorem}

Neither side of this equivalence is known for certain to be true! 

The above discussion generalizes naturally to Artin $L$-functions. 
Let $K/\bQ$ be a finite Galois extension with group $G=\Gal(K/\bQ)$. For any 
rational prime $p$ at which $K$ is unramified, let $\frob_p$ be the conjugacy 
class of the Frobenius at $p$ in $G$. For any irreducible representation 
$\rho$ of $G$, there is a corresponding $L$-function defined as 
\[
	L(\rho,s) = \prod_p \det\left(1-\rho(\frob_p) p^{-s}\right)^{-1} ,
\]
where here (and for the remainder of this thesis) we tacitly omit from the 
product those primes at which $\rho$ is ramified. Given a cutoff $x$, there is 
a natural empirical measure 
$P_x = \frac{1}{\pi(x)} \sum_{p\leqslant x} \delta_{\frob_p}$ on 
$G^\natural$, the set of conjugacy classes in $G$. Let $\mu$ be the 
(normalized) Haar measure on $G^\natural$, and let 
$\D(P_x) = \max_{S\subset G^\natural} \left| P_x(S) - \mu(S)\right|$. 
Then $P_x$ converges weakly to the Haar measure on $G^\natural$ if and 
only if $\D(P_x) \to 0$. Recall that weak convergence of $P_x$ to $\mu$ means 
$\int f\, \dd P_x \to \int f$ for all continuous functions $f$ on $G^\natural$. 
Since $G^\natural$ is a finite set, all functions on $G^\natural$ are 
continuous, but later on we will consider weak convergence on more general 
spaces.  

\begin{theorem}[{\cite[Th.~2 Cor., A.1]{serre-1989}}]
The measures $P_x$ converge weakly to the Haar measure on $G^\natural$ if and 
only if the function $L(\rho,s)$ admits a non-vanishing analytic continuation 
to $\Re = 1$ for all nontrivial $\rho$. 
\end{theorem}

Both sides of this equivalence are true, and known as the Chebotarev density 
theorem. Moreover, there is a version of the strong prime number theorem in 
this context. In this thesis, the Riemann hypothesis for a Dirichlet series 
$L(s)$ is the statement that $L(s)$ admits a non-vanishing analytic 
continuation to $\Re > \frac 1 2$. 

\begin{theorem}
The bound $\D(P_x) \ll x^{-\frac 1 2+\epsilon}$ holds if and only if each
$L(\rho,s)$, $\rho$ nontrivial, satisfies the Riemann hypothesis. 
\end{theorem}

The forward implication follows from Theorem \ref{thm:AT->RH:gp}, while the reverse implication is a result of Serre \cite[Th.~4]{serre-1981}. 
This whole discussion generalizes to a more complicated set of Galois 
representations---those arising from elliptic curves and more general motives.  





\section{Discrepancy and the Riemann hypothesis for elliptic curves}

For background on the Galois representations and $L$-functions associated to 
elliptic curves, see \cite[III\S7, C\S17]{silverman-2009}. Throughout this 
thesis, what we call the $L$-function of an elliptic curve (motive, etc.) is 
the normalized (i.e.~analytic instead of algebraic) $L$-function. 
Let $E_{/\bQ}$ be a non-CM elliptic curve. For any prime $l$, the $l$-adic Tate 
module of $E$ induces a continuous representation 
$\rho_l \colon G_\bQ \to \GL_2(\bZ_l)$. It is known that the quantities 
$a_p = \tr \rho_l(\frob_p)$ lie in $\bZ$ and satisfy the Hasse bound 
$|a_p| \leqslant 2\sqrt p$. For each unramified prime $p$, the 
corresponding Satake parameter for $E$ is 
$\theta_p = \cos^{-1}\left(\frac{a_p}{2\sqrt p}\right) \in [0,\pi]$. 
These parameters are packaged into an $L$-function as follows:
\[
	L(E,s) = \prod_p \frac{1}{(1 - e^{i \theta_p} p^{-s})(1- e^{-i \theta_p} p^{-s})} = \prod_p \det\left(1 - \smat{e^{i\theta_p}}{}{}{e^{-i \theta_p}}p^{-s}\right)^{-1}.
\]
More generally we have, for each irreducible representation $\sym^k$ of 
$\SU(2)$, the $k$-th symmetric power $L$-function: 
\[
	L(\sym^k E, s) = \prod_p \prod_{j=0}^k \frac{1}{1 - e^{i (k - 2j) \theta_p} p^{-s}} = \prod_p \det\left(1-\sym^k \smat{e^{i\theta_p}}{}{}{e^{-i \theta_p}}p^{-s}\right)^{-1}.
\]

Numerical experiments suggest that the Satake parameters are equidistributed 
with respect to the Sato--Tate distribution 
$\ST = \frac{2}{\pi} \sin^2\theta\, \dd\theta$. Indeed, for any cutoff $x$, let 
$P_x$ be the empirical measure 
$\frac{1}{\pi(x)} \sum_{p\leqslant x} \delta_{\theta_p}$. 
The convergence of $P_x$ to the Sato--Tate measure is closely related to 
the analytic properties of the $L(\sym^k E,s)$. First, here is the famous 
Sato--Tate conjecture (now a theorem) in our notation. 

\begin{theorem}[Taylor et.~al.]
The measures $P_x$ converge weakly to $\ST$. 
\end{theorem}

\begin{theorem}[Serre]
The Sato--Tate conjecture holds for $E$ if and only if each of 
the functions $L(\sym^k E,s)$ have analytic continuation to $\Re = 1$. 
\end{theorem}

The stunning recent proof of the Sato--Tate conjecture 
\cite{harris-shepherd-barron-taylor-2010} 
showed that the functions $L(\sym^k E,s)$ have the desired analytic 
continuation. In fact, they show that for all $k$, $L(\sym^k E,s)$ has 
meromorphic continuation to the whole complex plane. Even better, when $k$ is 
odd, the $L$-function is potentially automorphic. See Theorem 
\ref{thm:bad-Galois} for a result in this thesis where more can be said about 
odd symmetric power $L$-functions than even ones. 

The Riemann hypothesis, and its analogue for Artin $L$-functions, has a natural 
generalization to elliptic curves. In this context, the discrepancy of the set 
$\{\theta_p\}_{p\leqslant x}$ is 
\[
	\D_x\left(E,\ST\right) = \sup_{t\in [0,\pi]} \left| P_x[0,t] - \ST[0,t]\right| .
\]
The following conjecture is made in \cite{akiyama-tanigawa-1999}.

\begin{conjecture}[Akiyama--Tanigawa]
$\D_x\left(E,\ST\right)\ll x^{-\frac 1 2+\epsilon}$.
\end{conjecture}

Akiyama and Tanigawa go on to prove a special case of the following theorem, 
proved in full generality by Mazur. 

\begin{theorem}[{\cite[\S3.4]{mazur-2008}}]
If $\D_x\left(E,\ST\right)\ll x^{-\frac 1 2+\epsilon}$, 
then all the functions $L(\sym^k E, s)$ satisfy the Riemann hypothesis. 
\end{theorem}

This discussion also makes sense when $E$ has complex multiplication (for 
simplicity, we consider $E_{/F}$ where $F$ is the field of definition of the 
complex multiplication). The Sato--Tate measure for such $E$ is the Haar 
measure on $\SO(2)$, i.e.~the uniform measure on $[0,\pi]$. Instead of 
symmetric power $L$-functions, there is an $L$-function for each character of 
$\SO(2)$. Once again, there is a theorem ``Akiyama--Tanigawa conjecture implies 
Riemann hypothesis.'' For a precise statement and proof, see Theorem 
\ref{AT->RH:AB}. 

It is natural to assume that the converse to the implication 
``Akiyama--Tanigawa conjecture implies Riemann hypothesis'' holds. Zywina 
first suggested to the author that it might not. In this thesis, we construct a 
range of counterexamples to the implication ``Akiyama--Tanigawa conjecture implies 
generalized Riemann hypothesis'' for the case of CM abelian varieties. 
Moreover, we generalize the results of \cite{pande-2011} to show that there can 
be no purely Galois-theoretic proof of the Sato--Tate conjecture, for there are 
Galois representations with arbitrary Sato--Tate distributions! We also show 
that some of the results of \cite{sarnak-2007} about sums of the form 
$\sum_{p\leqslant x} \frac{a_p}{\sqrt p}$ cannot be generalized to 
general---in particular, infinitely ramified---Galois representations. 





\section{Notation conventions}

If $S$ is a set, $1_S$ is the characteristic function of $S$. 

Whenever $l$ is mentioned it is a rational prime $\geqslant 7$. 

Write $f\ll g$ if $f = O(g)$, i.e.~there is a constant $C>0$ such that 
$f \leqslant C g$. 

Write $f=\Omega(g)$ (in the convention of Hardy--Littlewood) if 
$\limsup \frac f g > 0$. 

The symbol $f = \Theta(g)$ means there exist constants $0<C_1<C_2$ such that 
$C_1 g \leqslant f \leqslant C_2 g$. Equivalently, $g \ll f$ and $f \ll g$. 

If $\mu$ is a measure on $\bR$, then write $\mu[a,b]$ for $\mu([a,b])$, and 
similarly for $[a,b)$, $(a,b]$, etc. Whenever it simplifies the 
notation, we will write $\mu S$ for $\mu(S)$. 

If $\mu$ is a measure on $\bR$, then \emph{cumulative distribution function 
(cdf) of $\mu$} is given by $\cdf_\mu(x) = \mu[-\infty,x]$. 

If $z\in \bC$, write $\Re z$ for the real part of $z$. 

If $\alpha\in \bR$, we write $\Re > \alpha$ for the half-plane of complex 
numbers with real part $> \alpha$. So a function has analytic continuation to 
the half-plane $\{z\in \bC : \Re z > \alpha\}$ if and only if the function 
extends to an analytic function on $\Re > \alpha$. 

We write $\bx = (x_1,x_2,\dots)$ for infinite sequences and 
$\vx = (x_1,\dots,x_d)$ for vectors. Sometimes we will have a sequence of 
vectors, written as $\bvx = (\vx_1,\vx_2,\dots)$. 

If $\bx=(x_1,x_2,\dots)$ is a sequence, write 
$P_{\bx,N} = \frac{1}{N} \sum_{n\leqslant N} \delta_{x_n}$ for the 
corresponding empirical measure. If $\bx=(x_\alpha)$ is instead indexed by 
some other discrete subset of $\bR^+$ (for example a subset of the primes), 
write 
$P_{\bx,N} = \frac{1}{\# \{\text{indices }\leqslant N\}}\sum_{\alpha \leqslant N} \delta_{x_\alpha}$. 

Omitted entries in matrices are zero, i.e.~$\smat{a}{}{}{b}$ means 
$\smat{a}{0}{0}{b}$. 
