% !TEX root = Daniel-Miller-thesis.tex

\chapter{Dirichlet series with Euler product}\label{ch:Dirichlet-series}





\section{Definitions and motivation}

We start by considering a very general class of Dirichlet series: those that 
admit a product formula with degree 1 factors. The motivating example was 
suggested to the author by Ramakrishna. Let $E_{/\bQ}$ be an elliptic 
curve and let 
\[
	L_{\sgn}(E,s) = \prod_p \frac{1}{1-\sgn(a_p) p^{-s}} .
\]
How much can we say about the behavior of $L_{\sgn}(E,s)$? For example, does it 
admit analytic continuation to $\Re = 1$? Yes, by the results of 
\cite[A.2]{serre-1989}. In fact, we will see later that the Akiyama--Tanigawa 
conjecture implies the existence of a non-vanishing analytic continuation of 
$L_{\sgn}(E,s)$ to $\Re > \frac 1 2$. Can the rank of $E$ be 
found from $L_{\sgn}(E,s)$? Theoretically yes, by the following result, which 
the author learned from Harris. 

\begin{theorem}
If $E_1$ and $E_2$ are non-CM elliptic curves over $\bQ$ with 
$\sgn a_p(E_1) = \sgn a_p(E_2)$ for all $p$, then $E_1$ and $E_2$ are 
isogenous.
\end{theorem}
\begin{proof}
Assume by way of contradiction that $E_1$ and $E_2$ are non-isogenous, non-CM 
elliptic curves over $\bQ$ with $\sgn a_p(E_1) = \sgn a_p(E_2)$ for all $p$. By 
\cite[5.4]{harris-2009}, the pairs $(\theta_p(E_1),\theta_p(E_2))$ are 
equidistributed with respect to 
$\ST\times \ST = \frac{4}{\pi^2} \sin^2\theta_1\sin^2\theta_2\, \dd\theta_1\, \dd\theta_2$ 
on $[0,\pi]\times [0,\pi]$. 

Recall that if $f(\theta) = 1_{[0,\pi/2)}(\theta) - 1_{(\pi/2,\pi]}(\theta)$, 
then $\sgn(a_p) = f(\theta_p)$. Moreover, 
$g(\theta_1,\theta_2) = |f(\theta_1) - f(\theta_2)|$ is non-negative and 
continuous almost everywhere, and $g(\theta_p(E_1),\theta_p(E_2)) = 0$ if and 
only if $\sgn a_p(E_1) = \sgn a_p(E_2)$. It is clear that 
$\int g\, \dd \ST\times \ST > 0$. Harris' equidistribution result tells us that 
\[
	\int g\, \dd \ST\times\ST = \lim_{N\to \infty} \frac{1}{\pi(N)} \sum_{p\leqslant N} g(\theta_p(E_1),\theta_p(E_2)) = 0 ,
\]
which is a contradiction. 
\end{proof}

It follows that if $L_{\sgn}(E_1,s) = L_{\sgn}(E_2,s)$ for all $s$ in some 
right half-plane, then $E_1$ and $E_2$ 
are isogenous, so in particular they have the same rank. Can we recover the 
rank of $E$ from the behavior of $L_{\sgn}(E,s)$ at $s = \frac 1 2$? For 
$k\geqslant 1$, let $r_k$ be the order of vanishing of $L(\sym^k E,s)$ at 
$s = \frac 1 2$. The heuristics in \cite{sarnak-2007} suggest that if the 
Akiyama--Tanigawa (and other natural conjectures) hold, then $L_f(E,s)$ has a 
zero of order 
$\sum_{k\geqslant 1} \widehat f(\sym^k) \left(-2 r_k + (-1)^{k+1}\right)$ at 
$s = \frac 1 2$. Here, 
$\widehat f(\sym^k) = \int_{\SU(2)^\natural} f(x) \tr\sym^k(x)\, \dd x$ is the 
$\sym^k$-Fourier coefficient of $f$. If $f = \tr\sym^1$ (this is called $U_1$ 
in Sarnak's letter), then $\widehat f(\sym^k)$ vanishes for $k\geqslant 2$, so 
the order of vanishing of $L_{U_1}(E,s)$ at $s = \frac 1 2$ is $1 - 2 r_1$, 
which, if we assume the Birch and Swinnerton-Dyer conjecture, allows us to 
``read'' the rank of $E$ from the behavior of $L_{U_1}(E,s)$ at 
$s = \frac 1 2$. However, for $f = 1_{[0,\pi/2)} - 1_{(\pi/2,\pi]}$, we have 
\[
	\widehat f(\sym^k) = 
	\begin{cases}
		0 & \text{$k$ is even} \\
		\frac 2 \pi (-1)^{\frac{k-1}{2}} \left(\frac 1 k + \frac{1}{k+2}\right) & \text{$k$ is odd} 
	\end{cases} ,
\]
so it's not clear if the rank of $E$ can be directly read from the behavior 
of $L_{\sgn}(E,s)$ at $s = \frac 1 2$. 

\begin{definition}
Let $\bx=(x_2,x_3,x_5,\dots)$ be a sequence of complex numbers indexed by the 
primes. The associated Dirichlet series is 
$L(\bx,s) = \prod_p \left(1- x_p p^{-s}\right)^{-1}$. 
\end{definition}

If $x_p$ is defined only for a subset of the primes, we tacitly set $x_p = 0$ 
(so the Euler factor is $1$) at all primes for which $x_p$ is not defined. 

\begin{lemma}
Let $\bx$ be a sequence with $|\bx|_\infty \leqslant 1$. Then $L(\bx,s)$ 
defines a holomorphic function on the region $\Re > 1$. On that region, 
$\log L(\bx,s) = \sum_{p^r} \frac{x_p^r}{r p^{r s}}$. 
\end{lemma}
\begin{proof}
Expanding the product for $L(\bx,s)$ formally, we have 
$L(\bx,s) = \sum_{n\geqslant 1} \frac{\prod_p x_p^{v_p(n)}}{n^s}$. 
An easy comparison with the Riemann zeta function tells us that this sum 
is holomorphic on $\Re > 1$. By \cite[Th.~11.7]{apostol-1976}, the 
product formula holds in the same region. The formula for $\log L(\bx,s)$ 
comes from \cite[11.9 Ex.~2]{apostol-1976}. 
\end{proof}

Abel summation is a commonly-used result that will allow us to turn questions 
on the analytic continuation and non-vanishing of $L(\bx,s)$ into questions 
about the asymptotics of $\sum_{p\leqslant N} x_p$. 

\begin{lemma}[Abel summation]\label{lem:abel-sum}
Let $\bx=(x_2,x_3,x_5,\dots)$ be a sequence of complex numbers, $f$ a smooth 
$\bC$-valued function on $\bR$. Then 
\[
	\sum_{p\leqslant N} f(p) x_p = f(N) \sum_{p\leqslant N} x_p - \int_2^N f'(t) \sum_{p\leqslant t} x_p\, \dd t .
\]
\end{lemma}
\begin{proof}
If $p_1,\dots,p_n$ is an enumeration of the primes $\leqslant N$, we have 
\begin{align*}
	\int_2^N f'(t) \sum_{p\leqslant t} x_p\, \dd t 
		&= \sum_{p\leqslant N} x_p \int_{p_n}^N f'(t)\, \dd t + \sum_{i=1}^{n-1} \sum_{p\leqslant p_i} x_p \int_{p_i}^{p_{i+1}} f'(t)\, \dd t \\
		&= \left(f(N) - f(p_n)\right) \sum_{p\leqslant N} x_p + \sum_{i=1}^{n-1} \left(f(p_{i+1}) - f(p_i)\right) \sum_{p\leqslant p_i} x_p \\
		&= f(N) \sum_{p\leqslant N} x_p - \sum_{p\leqslant N} f(p) x_p ,
\end{align*}
as desired. 
\end{proof}

\begin{theorem}\label{thm:AT->RH}
Let $|\bx|_\infty \leqslant 1$, and assume 
$|\sum_{p\leqslant N} x_p| \ll N^{\alpha+\epsilon}$ for some 
$\alpha\in [\frac 1 2,1]$. Then the series for $\log L(\bx,s)$ converges 
conditionally to a holomorphic function on $\Re > \alpha$. 
\end{theorem}
\begin{proof}
Formally split the sum for $\log L(\bx,s)$ into two pieces: 
\[
	\log L(\bx,s) = \sum_p \frac{x_p}{p^s} + \sum_p \sum_{r\geqslant 2} \frac{x_p^r}{r p^{r s}} .
\]
For each $p$, we have 
\[
	\left| \sum_{r\geqslant 2} \frac{x_p^r}{r p^{r s}}\right| \leqslant \sum_{r\geqslant 2} p^{- r \Re s} = p^{-2 \Re s} \frac{1}{1-p^{-\Re s}} .
\]
Elementary analysis gives 
$1 \leqslant \frac{1}{1-p^{-\Re s}} \leqslant 2 + 2\sqrt 2$, so the second 
piece of $\log L(\bx,s)$ converges absolutely on $\Re >\frac 1 2$. We could 
simply cite \cite[II.1 Th.~10]{tenenbaum-1995} to finish the proof; instead we 
prove directly that $\sum \frac{x_p}{p^s}$ converges absolutely to a 
holomorphic function on the region $\Re > \alpha$. 

By Lemma \ref{lem:abel-sum} (Abel summation) with $f(t) = t^{-s}$, we have 
\begin{align}
	\sum_{p\leqslant N} \frac{x_p}{p^s}
		&= N^{-s} \sum_{p\leqslant N} x_p + s \int_2^N \sum_{p\leqslant t} x_p\, \frac{\dd t}{t^{s+1}} \label{eq:log-L-sum} \\ 
		&\ll N^{-\Re s + \alpha + \epsilon} + |s| \int_2^N t^{\alpha+\epsilon} \frac{\dd t}{t^{\Re s+1}} . \nonumber
\end{align}
Since $\alpha-\Re s < 0$, the first term is converges to zero. Since 
$\Re s+1-\alpha > 1$ and 
$\epsilon$ is arbitrary, the integral converges absolutely, and the proof is 
complete. 
\end{proof}

The proof of Theorem \ref{thm:AT->RH} actually gives an absolutely 
convergent expression for $\log L(\bx,s)$ on the region $\Re >\alpha$. Since 
the term $N^{-s} \sum_{p\leqslant N} x_p$ in \eqref{eq:log-L-sum} 
converges to zero, we get 
\[
	\log L(\bx,s) = s \int_2^\infty t^{-s-1}\left(\sum_{p\leqslant t} x_p\right) \dd t + \sum_p \sum_{r\geqslant 2} \frac{x_p^r}{r p^{r s}} . 
\]

Let $X$ be a topological space, $f\colon X\to \bC$ a function with 
$|f|_\infty\leqslant 1$, and $\bx=(x_2,x_3,\dots)$ a sequence in $X$. Write 
\[
	L_f(\bx,s) = \prod_p \frac{1}{1-f(x_p) p^{-s}} ,
\]
for the associated Dirichlet series. In the remainder, we will 
exclusively focus on Dirichlet series of this type. 





\section{Relation to automorphic and motivic \texorpdfstring{$L$}{L}-functions}

Suppose $G$ is a compact group, $G^\natural$ the space of conjugacy classes in 
$G$. If $\bx = (x_2,x_3,x_5,\dots)$ is a sequence in $G^\natural$ and $\rho$ is 
a finite-dimensional unitary representation of $G$, put 
\[
	L(\rho(\bx),s) = \prod_p \frac{1}{\det(1-\rho(x_p) p^{-s})} .
\]
Clearly $L((\rho_1\oplus \rho_2)(\bx),s) = L(\rho_1(\bx),s) L(\rho_2(\bx),s)$. 
Now, suppose $G$ is a compact connected Lie group, let $T\subset G$ be a 
maximal torus, and recall that 
$T\twoheadrightarrow G^\natural$ \cite[IX.5 Prop.~5]{bourbaki-2005}. 
The representation 
$\left.\rho\right|_T$ decomposes as $\bigoplus \chi^{\oplus m_\chi}$, where 
$\chi$ ranges over characters of $T$ and the entire expression is 
$W$-invariant. We may regard the $x_p$ as lying in $T/W$, so we have 
\[
	L(\rho(\bx),s) = \prod_\chi L(\chi(\bx),s)^{m_\chi} .
\]
If the trivial representation appears in $\left.\rho\right|_T$, this product 
formula will include a copy (possibly several) of $\zeta(s)$. Since 
$\chi(x_p) \in S^1$, the above formula decomposes $L(\rho(\bx),s)$ into a 
product of Dirichlet series of the type considered above. For $G = \SU(2)$, the 
trivial representation occurs in $\left.\sym^k \right|_T$ if and only if $k$ is 
even, and the resulting $\zeta(s)$ in the product decomposition of 
$L(\sym^k \bx,s)$ may explain why some of the results in this thesis only apply 
to \emph{odd} symmetric powers. 

\begin{theorem}\label{thm:AT->RH:gp}
Let $G^\natural$ be the space of conjugacy classes in a compact group, 
$\bx = (x_2,x_3,x_5,\dots)$ a sequence in $G^\natural$. If $\rho$ is a 
nontrivial unitary representation of $G$ and 
$\left| \sum_{p\leqslant N} \tr \rho(x_p)\right| \ll N^{\alpha+\epsilon}$ for 
$\alpha\in \left[\frac 1 2,1\right]$, then $L(\rho(\bx),s)$ admits a 
non-vanishing analytic continuation to $\Re > \alpha$. 
\end{theorem}
\begin{proof}
On $\Re > 1$, we have 
$\log L(\rho(\bx),s) = \sum_{r\geqslant 1} \sum_p \frac{\tr \rho(x_p)^r}{r p^{r s}}$. 
Just as in the proof of Theorem \ref{thm:AT->RH}, we can split the sum into 
two terms, $\sum_p \frac{\tr \rho(x_p)}{p^s}$ and 
$\sum_{r\geqslant 2} \sum_p \frac{\tr \rho(x_p)^r}{r p^{r s}}$. Analytic 
continuation and nonvanishing remain the same when we omit finitely many Euler 
factors, so we may ignore all primes for which 
$\frac{\dim(\rho)}{p^{-1/2}} \geqslant 1$. Then the second sum converges on 
$\Re > \frac 1 2$, and assuming 
$\left| \sum_{p\leqslant N} \tr \rho(x_p)\right| \ll N^{\alpha+\epsilon}$, an 
argument identical to the one used in the proof of Theorem \ref{thm:AT->RH}, 
using Abel summation, shows that 
$\sum_p \frac{\tr \rho(x_p)}{p^s}$ converges to a holomorphic function on 
$\Re > \alpha$. This yields the desired non-vanishing analytic continuation. 
\end{proof}





\section{Discrepancy and the Riemann hypothesis}

\begin{definition}
We say the \emph{Riemann hypothesis} for $L(\bx,s)$ holds if the function 
$\log L(\bx,s)$ admits analytic continuation to $\Re > \frac 1 2$. 
\end{definition}

Under reasonable analytic hypotheses, namely conditional convergence of 
the Dirichlet series for 
$\log L(\bx,s)$ on $\Re > \frac 1 2$, the result 
\cite[II.1 Th.~10]{tenenbaum-1995} gives 
an estimate $|\sum_{p\leqslant N} x_p| \ll N^{\frac 1 2 + \epsilon}$. 

\begin{theorem}
Let $(X,\mu)$ be a probability space in which discrepancy and Koksma--Hlawka 
make sense (i.e., Theorem \ref{thm:koksma-hlawka} applies), and let 
$\bx=(x_2,x_3,x_5,\dots)$ be a sequence in $X$ with 
$\D_N(\bx,\mu) \ll N^{-\frac 1 2+\epsilon}$. For any function $f$ on $X$ of 
bounded variation with $\int f\, \dd\mu = 0$, $L_f(\bx,s)$ satisfies 
the Riemann hypothesis. 
\end{theorem}
\begin{proof}
By the Koksma--Hlawka inequality (Theorem \ref{thm:koksma-hlawka}), the bound 
on discrepancy yields the estimate 
$\left| \sum_{p\leqslant N} f(x_p)\right| \ll N^{\frac 1 2+\epsilon}$. 
By Theorem \ref{thm:AT->RH}, the Riemann hypothesis holds for $L_f(\bx,s)$. 
\end{proof}

The same proof shows that if $\D_N(\bx,\mu)\ll N^{-\alpha+\epsilon}$, then 
$\log L_f(\bx,s)$ conditionally converges to a holomorphic function on 
$\Re > 1 - \alpha$. This theorem applied to the function $L_{\sgn}(E,s)$ shows 
that the Akiyama--Tanigawa conjecture implies the Riemann hypothesis for 
$L_{\sgn}(E,s)$. The author is unaware of any results, conditional or 
otherwise, that suggest $L_{\sgn}(E,s)$ has analytic continuation past 
$\Re = \frac 1 2$ or has any kind of functional equation. Also, if 
$\int f\, \dd\mu \ne 0$, the function $L_f(\bx,s)$ will have a singularity at 
$s = 1$, but the author is not aware of a way to continue the function 
$L_f(\bx,s)$ past $\Re = 1$, even if $\D_N(\bx,\mu)$ decays rapidly. 

Let $F = \bF_q(t)$ be a function field, $E_{/F}$ a generic elliptic curve. 
There is, for every prime $\fp$ of $F$, a Satake parameter 
$\theta_\fp \in [0,\pi]$, defined in the usual way. It is known 
\cite[Ch.~3]{katz-1988} that
\begin{equation}\label{eq:katz-char-bound}
	\left|\sum_{\N(\fp) \leqslant x} \tr\sym^k\smat{e^{i \theta_\fp}}{}{}{e^{- i \theta_\fp}} \right| \ll k \sqrt{x} .
\end{equation}
Briefly, let $G$ be a compact Lie group, $\rho$ an irreducible unitary 
representation of $G$. For $f\in L^1(G)$, the Fourier coefficient of $f$ at 
$\rho$ is $\widehat f(\rho) = \int f(x) \overline{\tr\rho(x)}\, \dd x$. If 
$f\in L^1(G^\natural)$, then $f = \sum_\rho \widehat f(\rho) \tr \rho$. When 
$G = \SU(2)$, the nontrivial irreducible unitary representations are 
$\sym^k$ for $k\geqslant 1$. 

Equation \eqref{eq:katz-char-bound} tells us that for any 
$f\in C(\SU(2)^\natural)$ with 
$\sum_{k\geqslant 1} |\widehat f(\sym^k)|<\infty$ and $\widehat f(\sym^0) = 0$, 
the strange Dirichlet series $L_f(\btheta,s)$ satisfies the Riemann hypothesis. 

The best estimate on discrepancy is found in 
\cite{niederreiter-1991}, where it is shown that $D_x \ll N^{-\frac 1 4}$ 
by applying a generalization of the Koksma--Hlawka inequality to 
$\SU(2)^\natural$. Namely, for any odd $r$, we have 
\[
	\D_x(\btheta,\ST) \ll \frac 1 r + \sum_{k=1}^{2 r -1} \frac 1 k \left| \frac{1}{\pi_F(x)} \sum_{\N(\fp)\leqslant x} \tr\sym^k\smat{e^{i \theta_\fp}}{}{}{e^{- i \theta_\fp}}\right| .
\]
Using the estimate \eqref{eq:katz-char-bound} on character sums, Niederreiter 
is able to derive $\D_x \ll x^{-\frac 1 4}$. This fits well with the results of 
\cite{bucar-kedlaya-2015,rouse-thorner-2016} in the number field case, both of 
which derive estimates of the form $\D_N \ll N^{-\frac 1 4+\epsilon}$, assuming 
both the generalized Riemann hypothesis and the functional equation for all 
symmetric-power $L$-functions associated to the (non-CM) elliptic curve in 
question. 
