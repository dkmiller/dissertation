% !TEX root = Daniel-Miller-thesis.tex

\chapter{Pathological Galois representations}\label{ch:construct-Galois}





\section{Notation and supporting results}

In this section we loosely summarize and adapt the results of 
\cite{khare-larsen-ramakrishna-2005,pande-2011}. Throughout, if $F$ is a field 
and $M$ a $G_F$-module, we write $\h^\bullet(F,M)$ in place of 
$\h^\bullet(G_F,M)$. All Galois representations will take values in 
$\GL_2(\bZ/l^n)$ or $\GL_2(\bZ_l)$ for $l$ a (fixed) rational prime, and 
all deformations will have fixed determinant. So we consider the cohomology of 
$\Ad^0\bar\rho$, the induced representation on trace-zero matrices by 
conjugation. 

If $S$ is a set of rational primes, $\bQ_S$ denotes the largest extension of 
$\bQ$ unramified outside $S$. So $\h^i(\bQ_S,-)$ is what is usually written as 
$\h^1(G_{\bQ,S},-)$. If $M$ is a $G_\bQ$-module and $S$ a finite set of primes, 
denote the corresponding Tate--Shafarevich group by 
\[
	\Sha^i_S(M) = \ker\left( \h^i(\bQ_S,M) \to \prod_{p\in S} \h^i(\bQ_p,M)\right) .
\]
If $l$ is a rational prime and $S$ a finite set of primes containing $l$, then 
for any $\bF_l[G_{\bQ_S}]$-module $M$, write $M^\vee=\hom_{\bF_l}(M,\bF_l)$ 
with the obvious $G_{\bQ_S}$-action, and write $M^\ast = M^\vee(1)$ for the 
Cartier dual of $M^\vee$. By \cite[Th.~8.6.7]{neukirch-schmidt-winberg-2008}, 
there is an isomorphism $\Sha^1_S(M^\ast) \simeq \Sha_S^2(M)^\vee$. As a 
result, if $\Sha_S^1(M)$ and $\Sha_S^2(M)$ are trivial, and $S\subset T$, then 
$\Sha_T^1(M)$ and $\Sha_T^2(M)$ are also both trivial. 

% The l>=7 comes from \cite{ramakrishna-2002}. 
\begin{definition}
A \emph{good residual representation} is an odd, absolutely irreducible, 
weight-$2$ representation $\bar\rho\colon G_{\bQ_S} \to \GL_2(\bF_l)$, where 
$l\geqslant 7$ is a rational prime. 
\end{definition}

Recall that $\bar\rho$ is weight-$2$ if $\det\bar\rho$ is the mod-$l$ 
cyclotomic character. Similarly, $\rho\colon G_\bQ \to \GL_2(\bZ_l)$ is 
weight-$2$ if $\det\rho$ is the $l$-adic cyclotomic character. 
Roughly, ``good residual representations'' have enough properties that we can 
prove meaningful theorems about their lifts without assuming the modularity 
results of Khare--Wintenberger. 

\begin{theorem}\label{thm:always-can-lift}
Let $\bar\rho\colon G_\bQ \to \GL_2(\bF_l)$ be a good residual 
representation. Then there exists a weight-$2$ lift of $\bar\rho$ to $\bZ_l$, 
ramified at the same set of primes as $\bar\rho$. 
\end{theorem}
\begin{proof}
To-do: find a proof of this!
\end{proof}

\begin{definition}
Let $\bar\rho\colon G_{\bQ_S} \to \GL_2(\bF_l)$ be a good residual 
representation. A prime $p\not\equiv \pm 1\pmod l$ is \emph{nice} if 
$\Ad^0\bar\rho\simeq \bF_l \oplus \bF_l(1)\oplus \bF_l(-1)$, i.e.~if the 
eigenvalues of $\bar\rho(\frob_p)$ have ratio $p$. 
\end{definition}

Taylor allows $p\equiv -1\pmod l$, but the results of \cite{pande-2011} 
require $p\not\equiv -1\pmod l$. The following theorem gives a complete 
description of the versal deformation ring for 
$\left.\bar\rho\right|_{G_{\bQ_p}}$ when $p$ is nice.

\begin{theorem}[\cite{ramakrishna-1999}]
Let $\bar\rho$ be a good residual representation and $p$ a nice prime. Then 
any deformation of $\left.\bar\rho\right|_{G_{\bQ_p}}$ is induced by 
$G_{\bQ_p} \to \GL_2(\bZ_l\pow{a,b} / \langle a b\rangle)$, sending 
\[
	\frob_p \mapsto \smat{p(1+a)}{}{}{(1+a)^{-1}} \qquad \tau_p \mapsto \smat{1}{b}{}{1} ,
\]
where $\tau_p\in G_{\bQ_p}$ is a generator for tame inertia. 
\end{theorem}

We close this section by introducing some new terminology and notation to 
condense the lifting process used in \cite{khare-larsen-ramakrishna-2005}. 

Fix a good residual representation $\bar\rho$. We will consider weight-$2$ 
deformations of $\bar\rho$ to $\bZ/l^n$ and $\bZ_l$. Call such a deformation a 
``lift of $\bar\rho$ to $\bZ/l^n$ (resp.~$\bZ_l$).'' We will often restrict the 
local behavior of such lifts, i.e.~the restrictions of a lift to $G_{\bQ_p}$ 
for $p$ in some set of primes. The necessary constraints are captured in the 
following definition. 

\begin{definition}
Let $\bar\rho$ be a good residual representation, 
$h\colon \bR^+ \to \bR_{\geqslant 1}$ an 
increasing function. An \emph{$h$-bounded lifting datum} is a tuple 
$(\rho_n,R_n,U_n,\{\rho_p\}_{p\in R_n\cup U_n})$, where 
\begin{enumerate}
\item
$\rho_n\colon G_{\bQ_{R_n}} \to \GL_2(\bZ/l^n)$ is a lift of $\bar\rho$.

\item
$R_n$ and $U_n$ are finite sets of primes, $R_n$ containing $l$ and all primes at 
which $\rho_n$ ramifies. 

\item
$\pi_{R_n}(x)\leqslant h(x)$ for all $x$. 

\item
Both $\Sha_{R_n}^1(\Ad^0\bar\rho)$ and $\Sha_{R_n}^2(\Ad^0\bar\rho)$ are 
trivial. 

\item
For all $p\in R_n\cup U_n$, 
$\rho_p\colon G_{\bQ_p} \to \GL_2(\bZ_l)$ satisfies 
$\rho_p\equiv \left. \rho_n\right|_{G_{\bQ_p}}\pmod{l^n}$. 

\item
For all $p\in R_n$, $\rho_p$ is ramified. 

\item
$\rho_n$ admits a lift to $\bZ/l^{n+1}$. 
\end{enumerate}
\end{definition}

If $(\rho_n,R_n,U_n,\{\rho_p\})$ is an $h$-bounded lifting datum, we call 
another $h$-bounded lifting datum $(\rho_{n+1},R_{n+1},U_{n+1},\{\rho_p\})$ a 
\emph{lift} of $(\rho_n,R_n,U_n,\{\rho_p\})$ if $U_n\subset U_{n+1}$, 
$R_n\subset R_{n+1}$, and for all 
$p\in R_n\cup U_n$, the two possible $\rho_p$ agree. 

\begin{theorem}\label{thm:lifting-datum}
Let $\bar\rho$ be a good residual representation, 
$h\colon \bR^+ \to \bR_{\geqslant 1}$ 
increasing to infinity. If $(\rho_n,R_n,U_n,\{\rho_p\})$ is an $h$-bounded lifting 
datum, $U_{n+1}\supset U_n$ is a finite set of primes disjoint from $R_n$, and 
$\{\rho_p\}_{p\in U_{n+1}}$ extends $\{\rho_p\}_{p\in U_n}$, then there exists an 
$h$-bounded lift $(\rho_{n+1},R_{n+1},U_{n+1},\{\rho_p\})$ of 
$(\rho_n,R_n,U_n,\{\rho_p\})$. 
\end{theorem}
\begin{proof}
By \cite[Lem.~8]{khare-larsen-ramakrishna-2005}, there exists a finite set 
$N$ of nice primes such that the map 
\begin{equation}\label{eq:h1-isom}
	\h^1(\bQ_{R_n\cup N},\Ad^0\bar\rho) \to \prod_{p\in R_n} \h^1(\bQ_p,\Ad^0\bar\rho) \times \prod_{p\in U_{n+1}} \h_\nr^1(\bQ_p,\Ad^0\bar\rho) 
\end{equation}
is an isomorphism. In fact, 
$\# N = \dim\h^1(\bQ_{R_n\cup U_n},\Ad^0\bar\rho^\ast)$, and the primes in $N$ are 
chosen, one at a time, from Chebotarev sets. Since $\pi_{R_n}(x)$ is eventually 
constant and $h(x)$ increases to infinity, $h(x) \geqslant \pi_{R_n}(x)+1$ for all 
$x\geqslant C_1$ for some $C_1$. Choose the first prime $p$ in $N$ to be 
$\geqslant C_1$; then $\pi_{R_n\cup\{p\}}(x) \leqslant h(x)$ for all $x$. Repeat 
this process for for all the other primes in $N$. We can ensure 
that the bound $\pi_{R_n\cup N}(x) \leqslant h(x)$ continues to hold. We also 
choose the primes in $N$ to be larger than any prime in $U_{n+1}$. 

By our hypothesis, $\rho_n$ admits a lift to $\bZ/l^{n+1}$; call one such lift 
$\rho^\ast$. For each $p\in R_n\cup U_{n+1}$, $\h^1(\bQ_p,\Ad^0\bar\rho)$ acts 
transitively on lifts of $\left.\rho_n\right|_{G_{\bQ_p}}$ to $\bZ/l^{n+1}$. In 
particular, there are cohomology classes $f_p\in \h^1(\bQ_p,\Ad^0\bar\rho)$ 
such that $f_p\cdot \rho^\ast \equiv \rho_p\pmod{l^{n+1}}$ for all 
$p\in R_n\cup U_{n+1}$. Moreover, for all $p\in U_{n+1}$, the class $f_p$ is 
unramified. Since the map \eqref{eq:h1-isom} is an isomorphism, there exists 
$f\in \h^1(\bQ_{R_n\cup N},\Ad^0\bar\rho)$ such that 
$\left.f\cdot \rho^\ast\right|_{G_{\bQ_p}}\equiv \rho_p\pmod{l^{n+1}}$ for all 
$p\in R_n\cup U_{n+1}$. 

Clearly $\left. f\cdot \rho^\ast\right|_{G_{\bQ_p}}$ admits a lift to $\bZ_l$ 
for all $p\in R_n\cup U_{n+1}$, but it does not necessarily admit such a lift for 
$p\in N$. By repeated applications of \cite[Prop.~3.10]{pande-2011}, there 
exists a set $N'\supset N$, with $\# N'\leqslant 2\# N$, of nice primes and 
$g\in \h^1(\bQ_{R_n\cup N'},\Ad^0\bar\rho)$ such that 
$(g+f)\cdot \rho^\ast$ still agrees with $\rho_p$ for $p\in R_n\cup U'$, and 
$(g+f)\cdot \rho^\ast$ is nice for all $p\in N'$. As above, the primes in $N'$ 
are chosen one at a time from Chebotarev sets, so we can continue to ensure the 
bound $\pi_{R_n\cup N'}(x)\leqslant h(x)$ and also that all primes in $N'$ 
are larger than those in $U_{n+1}$. Let $\rho_{n+1} = (g+f) \cdot \rho^\ast$. Let 
$R_{n+1} = R_n\cup \{p\in N' : \rho_{n+1}\text{ is ramified at }p\}$. For each 
$p\in R_{n+1}\smallsetminus R_n$, choose a lift $\rho_p$ of 
$\left. \rho_{n+1}\right|_{G_{\bQ_p}}$ to $\bZ_l$. 

Since $\left.\rho_{n+1}\right|_{G_{\bQ_p}}$ admits a lift to $\bZ/l^{n+2}$ (in 
fact, it admits a lift to $\bZ_l$) for each $p$, and 
$\Sha_{R_{n+1}}^1(\Ad^0\bar\rho)$, $\Sha_{R_{n+1}}^2(\Ad^0\bar\rho)$ are 
trivial, the deformation $\rho_{n+1}$ admits a lift to $\bZ/l^{n+2}$. The tuple 
$(\rho_{n+1},R_{n+1},U_{n+1},\{\rho_p\})$ is the desired lift of 
$(\rho_n,R_n,U_n,\{\rho_p\})$ to $\bZ/l^{n+1}$. 
\end{proof}





\section{Galois representations with specified Satake parameters}

Fix a good residual representation $\bar\rho$, and consider weight-$2$ 
deformations of $\bar\rho$. The final deformation, 
$\rho\colon G_\bQ \to \GL_2(\bZ_l)$, will be constructed as the inverse limit 
of a compatible collection of lifts $\rho_n\colon G_\bQ \to \GL_2(\bZ/l^n)$. At 
any given stage, we will be concerned with making sure that there exists a 
lift to the next stage, and that there is a lift with the necessary properties. 
Fix a sequence $\bx=(x_1,x_2,\dots)$ in $[-1,1]$. The set of unramified primes 
of $\rho$ is not determined at the beginning, but at each stage there will be 
a large finite set $U$ of primes which we know will remain unramified. 
Reindexing $\bx$ by these unramified primes, we will construct $\rho$ so that 
for all unramified primes $p$, $\tr\rho(\frob_p)\in \bZ$, satisfies the Hasse 
bound, and has $\frac{\tr\rho(\frob_p)}{2\sqrt p} \approx x_p$. 
Moreover, we can ensure that the 
set of ramified primes has density zero in a very strong sense (controlled by a 
parameter function $h$) and that our trace of Frobenii are very close to 
specified values. 

Given any deformation $\rho$, write $\pi_{\ram(\rho)}(x)$ for the function 
which counts $\rho_n$-ramified primes $\leqslant x$. Since we will have 
$\pi_{\ram(\rho)}(x)\ll h(x)$ and bounds of this form are only helpful 
if $h(x) = o(\pi(x))$, we will usually assume $h(x) \ll x^\epsilon$, 
e.g.~$h(x) = \log x$ or something which grows even slower (for example, the 
inverse of the Ackermann function). In \cite{khare-rajan-2001}, it is proved 
that for \emph{any} continuous semisimple $\rho\colon G_\bQ \to \GL_2(\bZ_l)$, 
we will have $\pi_{\ram(\rho)}(x) = o(\pi(x))$. That is, any continuous 
Galois representation we consider will be ramified at a density zero set of 
primes. However, by \cite[Th.~19]{khare-larsen-ramakrishna-2005}, it is 
possible for $\pi_{\ram(\rho)}(x)$ to be 
$\Omega(\frac{x}{\log(x)^{1+\epsilon}})$. This means the ability to bound 
$\pi_{\ram(\rho)}(x)$ by slow-growing functions like $\log(x)$ in the 
following result is non-trivial. 

\begin{theorem}\label{thm:master-Galois}
Let $l$, $\bar\rho$, $\bx$ be as above. Fix a function 
$h\colon \bR^+ \to \bR_{\geqslant 1}$ which increases to infinity. Then there 
exists a weight-$2$ deformation $\rho$ of $\bar\rho$, such that: 
\begin{enumerate}
\item
$\pi_{\ram(\rho)}(x) \ll h(x)$. 

\item
For each unramified prime $p$, $a_p=\tr\rho(\frob_p)\in \bZ$ and satisfies the 
Hasse bound. 

\item
For each unramified prime $p$, 
$\left| \frac{a_p}{2\sqrt p} - x_p\right| \leqslant \frac{l h(p)}{2\sqrt p}$. 
\end{enumerate}
\end{theorem}
\begin{proof}
Begin with $\rho_1= \bar\rho$. By \cite[Lem.~6]{khare-larsen-ramakrishna-2005}, 
there exists a finite set $R$, containing the set of primes at which $\bar\rho$ 
ramifies, such that $\Sha_R^1(\Ad^0\bar\rho)$ and $\Sha_R^2(\Ad^0\bar\rho)$ are 
trivial. 
Let $R_1$ be the union of $R$ and all primes $p$ with 
$\frac{l}{2\sqrt p} > 2$. Since $\frac{l}{2\sqrt p} \to 0$ as $p\to \infty$, 
the set $R_1$ is finite. For all $p\notin R_1$ and any $a\in \bF_l$, there 
exists $a_p\in \bZ$ satisfying the Hasse bound with $a_p\equiv a\pmod l$. In 
fact, given any $x_p\in [-1,1]$, there exists $a_p\in \bZ$ satisfying the Hasse 
bound, congruent to any fixed $a\in\bF_l$ modulo $l$, such that 
$\left| \frac{a_p}{2\sqrt p} - x_p\right| \leqslant \frac{l}{2\sqrt p}$.
Choose, for all primes $p\in R_1$, a ramified 
lift $\rho_p$ of $\left. \rho_1\right|_{G_{\bQ_p}}$. Let $U_1$ be the set of 
primes $p$ not in $R_1$ such that 
$\frac{l^2}{2\sqrt p} > \min\left(2, \frac{l h(p)}{2\sqrt p}\right)$; this is 
finite because $\frac{l^2}{2\sqrt p} \to 0$ and also eventually 
$h(p) \geqslant l$. If $U_1$ is empty, then the next few sentences of the 
proof are superfluous, but the theorem still holds. 
For each $p\in U_1$, there exists $a_p\in \bZ$, satisfying the 
Hasse bound, such that 
\[
	\left| \frac{a_p}{2\sqrt p} - x_p\right| \leqslant \frac{l}{2\sqrt p} \leqslant \frac{l h(p)}{2\sqrt p} ,
\]
and moreover $a_p\equiv \tr\bar\rho(\frob_p)\pmod l$. For each $p\in U_1$, let 
$\rho_p$ be an unramified lift of $\left.\bar\rho\right|_{G_{\bQ_p}}$ with 
$\tr\rho_p$ being the desired $a_p$. It may not be that 
$\pi_{R_1}(x) \leqslant h(x)$ for all $x$. Let 
$C = \max\left\{\pi_{R_1}(x)\right\}$; this is finite because 
$R_1$ is and $\pi_{R_1}(x)$ is constant past the largest prime in $R_1$. Then 
for $h^\ast = C h$, we have $\pi_{R_1}(x) \leqslant h^\ast(x)$ for all $x$. 

We have constructed our first $h^\ast$-bounded lifting datum 
$(\rho_1,R_1,U_1,\{\rho_p\})$. We proceed to construct 
$\rho = \varprojlim \rho_n$ inductively, by constructing a new $h^\ast$-bounded 
lifting datum for each $n$. We ensure that $U_n$ contains all primes for which 
$\frac{l^{n+1}}{2\sqrt p} > \min\left(2, \frac{l h(p)}{2\sqrt p}\right)$, so 
there are always integral $a_p$ satisfying the Hasse bound which satisfy any 
mod-$l^{n+1}$ constraint, and that can always choose these $a_p$ so as to 
preserve statement 2 in the theorem. 

The base case is complete, so suppose we have 
$(\rho_{n-1},R_{n-1},U_{n-1},\{\rho_p\})$. We may assume that $U_{n-1}$ 
contains all primes for 
which $\frac{l^n}{2\sqrt p} > \min\left(2, \frac{l h(p)}{2\sqrt p}\right)$. Let 
$U_n$ be the set of all primes not in $R_{n-1}$ such that 
$\frac{l^{n+1}}{2\sqrt p} > \min\left(2, \frac{l h(p)}{2\sqrt p}\right)$. For 
each $p\in U_n\smallsetminus U_{n-1}$, there is an integer $a_p$, satisfying 
the Hasse bound, such that $a_p\equiv \rho_n(\frob_p)\pmod{l^n}$, and moreover 
$\left|\frac{a_p}{2\sqrt p} - x_p\right| \leqslant \frac{l h(p)}{2\sqrt p}$. 
For such $p$, let $\rho_p$ be an unramified lift of 
$\left. \rho_n\right|_{G_{\bQ_p}}$ such that $\tr\rho_n(\frob_p)$ is the 
desired $a_p$. By Theorem \ref{thm:lifting-datum}, there exists an 
$h^\ast$-bounded lifting datum $(\rho_n,R_n,U_n,\{\rho_p\})$ extending and 
lifting $(\rho_{n-1},R_{n-1},U_{n-1},\{\rho_p\})$. This completes the inductive 
step.  
\end{proof}

The implied constant in the bound $\pi_{\ram(\rho)}(x)\ll h(x)$ depends on 
$\bar\rho$ (and hence $l$) but not on $h$. 
We will apply this theorem to construct Galois representations with specified 
Sato--Tate distributions in the next section, but for now here is a small 
consequence, which addresses the results in \cite{sarnak-2007}. Sarnak, 
the generalized Riemann hypothesis along with linear independence of the 
zeros of $L(\sym^k E,s)$, proves 
that for $E_{/\bQ}$ a non-CM elliptic curve with rank $r$, the partial sums 
$\frac{\log x}{\sqrt x} \sum_{p\leqslant x} \frac{a_p}{\sqrt p}$ approach a 
limiting distribution with mean $1 - 2 r$. 

\begin{corollary}
Let $L \in [-\infty,\infty]$ and $\epsilon>0$ be given. Then there exists a 
weight 2 Galois representation $\rho\colon G\to \GL_2(\bZ_l)$, such that 
each $a_p = \tr\rho(\frob_p)\in \bZ$ satisfies the Hasse bound, 
\[
	L = \lim_{N\to \infty} \frac{\log N}{\sqrt N}\sum_p \frac{a_p}{\sqrt p} 
\]
and $\pi_{\ram(\rho)}(x) \ll \log(x)$. 
\end{corollary}
\begin{proof}
Begin with a sequence $(x_p)$ in $\left[-\frac 1 2,\frac 1 2\right]$ such that 
$\lim_{N\to \infty} \frac{\log N}{\sqrt N}\sum_{p\leqslant N} x_p = L$. If 
$L=\pm \infty$, we can choose $x_p = \pm \frac 1 2$. By Theorem 
\ref{thm:master-Galois}, there exists $\rho\colon G_\bQ \to \GL_2(\bZ_l)$ with 
$\pi_{\ram(\rho)}(x) \ll \log(x)$, and such that for each unramified $p$, 
$a_p = \tr\rho(\frob_p)\in \bZ$, satisfies the Hasse bound. Moreover, after 
re-indexing $(x_p)$ by the unramified primes of $\rho$, we can have 
$\left| \frac{a_p}{2\sqrt p} - x_p\right| < \frac{l \log p}{\sqrt p}$. In fact, 
by inspecting the proof of Theorem \ref{thm:master-Galois}, we can even ensure 
that
$\sum \left(\frac{a_p}{2\sqrt p} - x_p\right)$ converges conditionally, by 
forcing the sign of $\frac{a_p}{2\sqrt p} - x_p$ to alternate. This will relax 
the bound 
$\left|\frac{a_p}{2\sqrt p} - x_p\right| \leqslant \frac{l \log p}{2\sqrt p}$ 
to 
$\left|\frac{a_p}{2\sqrt p} - x_p\right| \leqslant \frac{l \log p}{\sqrt p}$. 
Write $A_N = \frac{\log N}{\sqrt N}\sum_{p\leqslant N} \frac{a_p}{2\sqrt p}$ 
and $B_N = \frac{\log N}{\sqrt N} \sum_{p\leqslant N} x_p$, both sums tacitly 
taken over $\rho$-unramified primes. Then 
\[
	\left| A_N - B_N\right| \leqslant \frac{\log N}{\sqrt N} \left| \sum_{p\leqslant N} \left(\frac{a_p}{2\sqrt p} - x_p\right)\right| ,
\]
which converges to zero because 
$\sum_{p\leqslant N} \left(\frac{a_p}{2\sqrt p} - x_p\right)$ converges to a 
finite quantity by the alternating series test, and 
$\frac{\log N}{\sqrt N} \to 0$. The proof isn't quite 
complete, because we only know that 
$\lim_{N\to \infty} \frac{\log N}{\sqrt N}\sum_{p\leqslant N} x_p = L$ when 
$x_p$ is indexed by \emph{all} the rational primes, not just by the 
$\rho$-unramified ones. We need to prove that $B_N$ converges to $L$. 
Let $C_N = \frac{\log N}{\sqrt N} \sum_{p\leqslant N} x_p$, where here the sum 
is taken with $x_p$ indexed by all primes. Write 
$M_N = \pi^{-1}(\pi(N) - \pi_{\ram(\rho)}(N))$; then the number of 
$\rho$-unramified primes $\leqslant N$ is the same as number of all primes 
$\leqslant M_N$. It follows that 
$B_N = \frac{\log N}{\sqrt N} \frac{\sqrt{M_N}}{\log{M_N}}C_{M_N}$, so to prove 
$B_N \to L$, it suffices to prove that 
$\frac{\log N}{\sqrt N} \frac{\sqrt{M_N}}{\log{M_N}} \to 1$. Convergence 
$\frac{M_N}{N} \to 1$ follows from the prime number theorem, so we show that 
this implies $\frac{\log N}{\log M_N} \to 1$. Since 
$\frac{\log M_N}{\log N} = \log_N(M_N)$, we want to prove 
that $\log_N(M_N) \to 1$. Write $M_N = N^{\alpha_N}$; then 
$\frac{M_N}{N} = N^{\alpha_N - 1}$. Since $N\to \infty$, the only way for 
$N^{\alpha_N - 1} \to 1$ is for $\alpha_N \to 1$, 
i.e.~$\frac{\log N}{\log{M_N}} \to 1$. 

When $L\ne \pm\infty$, this shows that the limit in question exists and is $L$. 
When $L=\pm \infty$, this shows that the the sums in question diverge to $L$. 
\end{proof}





\section{Galois representations with specified Sato--Tate distributions}

For $k\geqslant 1$, let 
\[
	U_k(\theta) = \tr\sym^k\smat{e^{i\theta}}{}{}{e^{- i \theta}} = \frac{\sin((k+1)\theta)}{\sin\theta} .
\]
Then $U_k(\cos^{-1} t)$ is the $k$-th Chebyshev polynomial of the 2nd kind. 
Moreover, $\{1\}\cup\{U_k\}$ forms an orthonormal basis for 
$L^2([0,\pi],\ST) = L^2(\SU(2)^\natural)$. 

This section has two parts. First, for any reasonable measure $\mu$ on 
$[0,\pi]$ invariant under the same ``flip'' automorphism as the Sato--Tate 
measure, there is a sequence $(a_p)$ of integers satisfying the Hasse 
bound $|a_p|\leqslant 2\sqrt p$, such that for 
$\theta_p = \cos^{-1}\left(\frac{a_p}{2\sqrt p}\right)$, the discrepancy 
$\D_N(\btheta,\mu)$ behaves like $\pi(N)^{-\alpha}$ for predetermined 
$\alpha\in \left(0,\frac 1 2\right)$, while for any odd $k$, the strange 
Dirichlet series $L_{U_k}(\btheta,s)$, which we will write as 
$L(\sym^k \btheta,s)$, satisfies the Riemann hypothesis. In the second part of 
this section, we associate Galois representations to these fake Satake 
parameters. 

\begin{definition}
Let $\mu = f(\theta)\, \dd \theta$ be an absolutely continuous probability 
measure on $[0,\pi]$. If $f(\theta) \ll \sin(\theta)$ on $[0,\pi]$, then $\mu$ 
is a \emph{Sato--Tate compatible measure}. 
\end{definition}

Recall that $\cos_\ast\mu = \frac{f(\cos^{-1} t)}{\sqrt{1-t^2}}\, \dd t$. So 
the Radon--Nikodym derivative of $\cos_\ast\mu$ is bounded if and only if 
$\frac{f(\cos^{-1} t)}{\sqrt{1-t^2}}$ is bounded. Plugging in 
$t = \cos\theta$, we see that $\cos_\ast\mu$ has bounded Radon--Nikodym 
derivative if and only if $\frac{f(\theta)}{\sin\theta}$ is bounded, 
i.e.~$f(\theta) \ll \sin\theta$. So we could rephrase the definition of a 
Sato--Tate compatible measure to be ``an absolutely continuous measure $\mu$ 
such that $\cos_\ast\mu$ has bounded Radon--Nikodym derivative.'' 
Since $\ST = \frac 2 \pi \sin^2\theta\, \dd\theta$ clearly satisfies this 
definition, the Sato--Tate measure is itself Sato--Tate compatible. 

If $\mu$ is Sato--Tate compatible, then $\cos_\ast\mu$ 
satisfies the hypotheses of Theorem \ref{thm:discrepancy-arbitrary}, so 
there are ``$N^{-\alpha}$-decaying van der Corput sequences'' for 
$\cos_\ast\mu$, and also that since $\cos\colon [0,\pi] \to [-1,1]$ is 
strictly decreasing, we know that for any sequence $\bx$ on $[-1,1]$, 
$\D_N(\bx,\cos_\ast\mu) \approx \D_N(\cos^{-1}\bx,\mu)$, with the 
difference being $O(N^{-1})$.  Finally, the 
Radon--Nikodym derivative of $\mu$ (and also $\cos_\ast\mu$) is bounded , so 
Lemma \ref{lem:disc-of-two-seq} applies to both $\mu$ and $\cos_\ast\mu$. 
Recall that for deceasing functions 
$\varphi_1,\varphi_2$, we write $\varphi_1(N) = \Theta(\varphi_2(N))$ if 
there exists constants $0 < C_1 < C_2$ such that 
$C_1 \varphi_2(N) \leqslant \varphi_1(N) \leqslant C_2 \varphi_2(N)$. 


\begin{theorem}\label{thm:integral-a_p-alpha}
Let $\mu$ be a Sato--Tate compatible measure, and fix 
$\alpha\in \left(0,\frac 1 3\right)$. 
Then there exists a sequence of integers $a_p$ satisfying the Hasse bound, 
such that if we set $\theta_p = \cos^{-1}\left(\frac{a_p}{2\sqrt p}\right)$, 
then $\D_N(\btheta,\mu) = \Theta(\pi(N)^{-\alpha})$. 
\end{theorem}
\begin{proof}
Apply Theorem \ref{thm:discrepancy-arbitrary} to find a sequence $\bx$ such 
that $\D_N(\bx,\cos_\ast \mu) = \Theta(\pi(N)^{-\alpha})$. For each prime 
$p$, there exists an integer $a_p$ such that $|a_p|\leqslant 2\sqrt p$ and 
$\left| \frac{a_p}{2\sqrt p} - x_p\right| \leqslant \frac{1}{2\sqrt p}$. Let 
$y_p = \frac{a_p}{2\sqrt p}$, and apply Lemma \ref{lem:disc-of-two-seq} with 
$\epsilon = N^{-1/2}$. We obtain 
\[
	\left| \D_N(\bx,\cos_\ast \mu) - \D_N(\by, \cos_\ast \mu)\right| \ll  N^{-1/2} + \frac{\pi(N^{1/2})}{\pi(N)} ,
\]
which tells us that $\D_N(\by,\cos_\ast\mu) = \Theta(\pi(N)^{-\alpha})$. 
Now let $\btheta = \cos^{-1}(\by)$. Apply Lemma \ref{lem:push-discrepancy} to 
$\btheta = \cos^{-1}(\by)$, and we see that 
$\D_N(\btheta,\mu) = \Theta(\pi(N)^{-\alpha})$. 
\end{proof}

We can improve this example by controlling the behavior of the sums 
$\sum_{p\leqslant N} U_k(\theta_p)$ for odd $k$. Let $\sigma$ be 
the involution of $[0,\pi]$ given by $\sigma(\theta) = \pi-\theta$. Note that 
$\sigma_\ast \ST = \ST$. Moreover, note that for any odd $k$, 
$U_k\circ\sigma = - U_k$, so $\int U_k\, \dd\ST = 0$. Of course, 
$\int U_k \,\dd\ST = 0$ for the reason that $U_k$ is the trace of a 
non-trivial unitary representation, but we will directly use the ``oddness'' 
of $U_k$ in what follows.

\begin{theorem}\label{thm:int-flip-seq}
Let $\mu$ be a $\sigma$-invariant Sato--Tate compatible measure. Fix 
$\alpha\in \left(0,\frac 1 3\right)$. Then there is a sequence of integers 
$a_p$, satisfying the Hasse bound, such that for 
$\theta_p =\cos^{-1}\left( \frac{a_p}{2\sqrt p}\right)$, we have
\begin{enumerate}
\item
$\D_N(\btheta,\mu) = \Theta(\pi(N)^{-\alpha})$. 

\item
For all odd $k$, 
$\left| \sum_{k\leqslant N} U_k(\theta_p)\right| \ll \pi(N)^{1/2}$. 
\end{enumerate}
\end{theorem}
\begin{proof}
The basic ideas is as follows. Enumerate the primes 
\[
	p_1 = 2, q_1 = 3, p_2 = 5, q_2 = 7, p_3 = 11, q_3 = 13, \dots .
\]
Consider the measure $\left.\mu\right|_{[0,\pi/2)}$. This is supported on 
$[0,\pi/2)$, but we extend it by zero to $[0,\pi]$. An argument 
nearly identical to the proof of Theorem \ref{thm:integral-a_p-alpha} shows 
that we can choose $a_{p_i}$ satisfying the Hasse bound so that 
\[
	\D_N\left( \left\{\theta_{p_i}\right\},\left.\mu\right|_{[0,\pi/2)}\right) = \Theta(N^{-\alpha}) .
\]
Since $\frac{a_{p_i}}{2\sqrt{p_i}}\in \frac{1}{2\sqrt{p_i}} \bZ$ and 
$\frac{a_{q_i}}{2\sqrt{q_i}}\in \frac{1}{2\sqrt{q_i}}\bZ$, we cannot obtain 
$\frac{a_{p_i}}{2\sqrt{p_i}} = - \frac{a_{q_i}}{2\sqrt{q_i}}$, but we can get 
quite close to equality. That is, we can also choose the $a_{q_i}$ such that 
$\frac{a_{q_i}}{2\sqrt{q_i}}\in [-1,0)$ and 
$\left| \frac{a_{p_i}}{2\sqrt{p_i}} + \frac{a_{q_i}}{2\sqrt{q_i}}\right| \ll \frac{1}{\sqrt{p_i}}$. 

Let $\bx$ be the sequence of the $\frac{a_{p_i}}{2\sqrt{p_i}}$ and $\by$  
the corresponding sequence with the $q_i$-s. Then Lemma 
\ref{lem:flip-discrepancy} with $\sigma(t) = -t$ tells us that the discrepancy 
of $\by$ decays at the same rate as $-\by$, and then Corollary 
\ref{cor:close-seq-disc} with $\alpha = \frac 1 2$ tells us that the 
discrepancy of $-\by$ decays at the same rate (within $O(N^{-1/3})$) as the 
discrepancy of $\bx$. Thus the discrepancies of both $\bx$ and $\by$ decay as 
$\Theta(N^{-\alpha})$. Finally, Theorem \ref{thm:wreath-seq} tell us that 
$\D_N(\bx\wr\by,\mu) = \Theta(N^{-\alpha})$. 

The function $U_k(\cos^{-1} t)$ is an odd polynomial in $t$, so for 
$t_1,t_2\in [-1,1]$, 
\[
	|U_k(\cos^{-1} t_1) + U_k(\cos^{-1} t_2)| = |U_k(\cos^{-1} t_1) - U_k(\cos^{-1}(-t_2))| \ll |t_1 - (-t_2)|.
\]
It follows that since 
$\left|\frac{a_{p_i}}{2\sqrt{p_i}} - \left(- \frac{a_{q_i}}{2\sqrt{q_i}}\right)\right| \ll p_i^{-1/2}$, 
then $|U_k(\theta_{p_i}) + U_k(\theta_{q_i})|\ll p_i^{-1/2}$. 
We can then bound 
\[
	\left| \sum_{i\leqslant N} \left(U_k(\theta_{p_i}) + U_k(\theta_{q_i})\right)\right| \ll \sum_{p\leqslant N} p^{-1/2} \ll \pi(N)^{1/2} .
\]
\end{proof}

Note that this proof actually shows that for any $f\in C([0,\pi])$ such 
that $f\circ \cos^{-1}$ is Lipschitz, and $f(\pi-\theta) = -f(\theta)$, the 
estimate $\left| \sum_{p\leqslant N} f(\theta_p)\right| \ll \pi(N)^{1/2}$ 
holds. 

\begin{theorem}\label{thm:bad-Galois}
Let $\mu$ be a Sato--Tate compatible $\sigma$-invariant measure on $[0,\pi]$. 
Fix $\alpha\in \left(0,\frac 1 3\right)$ and a good residual representation 
$\bar\rho\colon G_\bQ \to \GL_2(\bF_l)$. Then there exists a weight-$2$ lift 
$\rho\colon G_\bQ \to \GL_2(\bZ_l)$ of $\bar\rho$ such that 
\begin{enumerate}
\item
$\pi_{\ram(\rho)}(x) \ll \log(x)$. 

\item
For each unramified prime $p$, $a_p = \tr\rho(\frob_p)\in \bZ$ and satisfies 
the Hasse bound. 

\item
If, for unramified $p$ we set 
$\theta_p = \cos^{-1}\left(\frac{a_p}{2\sqrt p}\right)$, then 
$\D_N(\btheta,\mu) = \Theta(\pi(N)^{-\alpha})$. 

\item
For each odd $k$, the function $L(\sym^k \rho,s)$ satisfies the Riemann 
hypothesis. 
\end{enumerate}
\end{theorem}
\begin{proof}
Let $\bx$ be an $N^{-\alpha}$-decay van der Corput sequence for 
$\cos_\ast \left.\mu\right|_{[0,\pi/2)}$, so that $\bx$ is contained in 
$(0,1]$. Let $\by = -\bx$ (contained in $[-1,0)$), and put 
$\bz = \bx\wr\by$, reindexed by the prime numbers. We have
$\D_N(\bz,\cos_\ast\mu) = \Theta(\pi(N)^{-\alpha})$ just as in the proof 
of Theorem \ref{thm:int-flip-seq}. Set $h(x) = \log(x)$. By Theorem 
\ref{thm:master-Galois}, there is a $\rho\colon G_\bQ \to \GL_2(\bZ_l)$ lifting 
$\bar\rho$ such that $\pi_{\ram(\rho)}(x) \ll \log x$, the $\tr \rho(\frob_p)$ 
are integral, satisfy the Hasse bound, and 
$\left| \frac{a_p}{2\sqrt p} - z_p\right| \leqslant \frac{l \log p}{2\sqrt p}$. 
This implies, just as in the proof of Theorem \ref{thm:master-Galois}, that 
the discrepancy of the sequence $\left\{\frac{a_p}{2\sqrt p}\right\}$ decays 
as $\Theta(\pi(N)^{-\alpha})$ and by Lemma \ref{lem:push-discrepancy} with 
$f(t) = \cos^{-1}(t)$, the 
discrepancies of $\left\{\frac{a_p}{2\sqrt p}\right\}$ and $\{\theta_p\}$ 
decay at the same rate. 

We've proved statements 1--3 in the theorem, which follow essentially for free 
from Theorem \ref{thm:master-Galois} and its proof. All that remains is to prove 
the Riemann hypothesis for odd symmetric powers. The proof of Theorem 
\ref{thm:int-flip-seq} gives us an estimate 
$\left| \sum_{p\leqslant N} U_k(\theta_p)\right| \ll N^{\frac 1 2+\epsilon}$, 
and this combined with Theorem \ref{thm:AT->RH:gp} yields the result. 
\end{proof}

This entire discussion works with absolutely continuous measure $\mu$. For 
example, let $I$ be an arbitrarily small subinterval of $[0,\pi]$ 
(e.g.~$I = \left[\frac \pi 2 - \epsilon,\frac \pi 2 + \epsilon\right]$), let 
$B_I(t)$ be a bump function for $I$, normalized to have total mass one. Then 
Theorem \ref{thm:bad-Galois} gives Galois representations with empirical 
Sato--Tate distribution converging at any specified rate to 
$\mu_I = B_I(t)\, \dd t$. This is a strictly stronger result than 
\cite[Th.~5.2]{pande-2011}. Moreover, the proof of Theorem 
\ref{thm:int-flip-seq} shows that in fact for any $f\in C([0,\pi])$ with 
$f\circ \cos^{-1}\in C^1([-1,1])$ and $f(\pi-\theta) = -f(\theta)$, the 
Dirichlet series 
$L_f(\rho,s) = \prod \left( 1 - f(\theta_p) p^{-s}\right)^{-1}$ satisfies the 
Riemann hypothesis. 
