% !TEX root = main.tex

\chapter{Strange Dirichlet series}

To-do: show that \cite[A.1]{serre-1989} works for $L_f(\bx,s)$, $f$ 
almost-everywhere continuous. 





\section{Definitions}

We start by considering a very general class of Dirichlet series. In fact, they 
are all Dirichlet series that admit a product formula with degree-1 factors, 
but in this thesis they will be called strange Dirichlet series. The motivating 
example was suggested by Ramakrishna. Let $E_{/\bQ}$ be an elliptic curve and 
let 
\[
	L_{\sgn}(E,s) = \prod_p \frac{1}{1-\sgn(a_p) p^{-s}} .
\]
How much can we say about the behavior of $L_{\sgn}(E,s)$? For example, does it 
``know'' the rank of $E$?

\begin{definition}
Let $\bz=(z_2,z_3,z_5,\dots)$ be a sequence of complex numbers indexed by the 
primes. The associated \emph{strange Dirichlet series} is 
\[
	L(\bz,s) = \prod_p \frac{1}{1- z_p p^{-s}} .
\]
\end{definition}

If $z_p$ is only defined for all but finitely many primes, then we tacitly set 
$\bz_p = 0$ for all primes for which $z_p$ is not defined. 

\begin{lemma}
Let $\bz$ be a sequence with $\|\bz\|_\infty \leqslant 1$. Then $L(\bz,s)$ 
defines a holomorphic function on the region $\{\Re s>1\}$. Moreover, on that 
region, 
\[
	\log L(\bz,s) = \sum_{p^r} \frac{z_p^n}{n p^{n s}} .
\]
\end{lemma}
\begin{proof}
Expanding the product for $L(\bz,s)$ formally, we have 
\[
	L(\bz,s) = \sum_{n\geqslant 1} \frac{\prod_p z_p^{v_p(n)}}{n^s} .
\]
An easy comparison with the Riemann zeta function tells us that this sum 
is holomorphic on $\{\Re s>1\}$. By \cite[Th.~11.7]{apostol-1976}, the 
product formula holds in the same region. The formula for $\log L(\bz,s)$ 
comes from \cite[11.9 Ex.2]{apostol-1976}. 
\end{proof}

\begin{lemma}[Abel summation]\label{lem:abel-sum}
Let $\bz=(z_2,z_3,z_5,\dots)$ be a sequence of complex numbers, $f$ a smooth 
complex-valued function on $\bR$. Then 
\[
	\sum_{p\leqslant N} f(p) z_p = f(N) \sum_{p\leqslant N} z_p - \int_2^N f'(x) \sum_{p\leqslant x} z_p\, \dd x .
\]
\end{lemma}
\begin{proof}
Simply note that if $p_1,\dots,p_n$ is an enumeration of the primes 
$\leqslant N$, we have 
\begin{align*}
	\int_2^N f'(x) \sum_{p\leqslant x} z_p\, \dd x 
		&= \sum_{p\leqslant N} z_p \int_{p_n}^N f' + \sum_{i=1}^{n-1} \sum_{p\leqslant p_{i+1}} z_p \int_{p_i}^{p_{i+1}} f' \\
		&= (f(N) - f(p_n)) \sum_{p\leqslant N} z_p + \sum_{i=1}^{n-1} (f(p_{i+1}) - f(p_i)) \sum_{p\leqslant p_{i+1}} z_p \\
		&= f(N) \sum_{p\leqslant N} z_p - \sum_{p\leqslant N} f(p) z_p ,
\end{align*}
as desired. 
\end{proof}

\begin{theorem}
Assume $|\sum_{p\leqslant x} z_p| \ll x^{\alpha+\epsilon}$ for some 
$\alpha\in [\frac 1 2,1]$. Then the series for $\log L(\bz,s)$ converges to a 
holomorphic function on the region $\{\Re s>\alpha\}$. 
\end{theorem}
\begin{proof}
Formally split the sum for $\log L(\bz,s)$ into two pieces: 
\[
	\log L(\bz,s) = \sum_p \frac{z_p}{p^s} + \sum_p \sum_{r\geqslant 2} \frac{z_p^r}{r p^{r s}} .
\]
For each $p$, we have 
\[
	\left| \sum_{r\geqslant 2} \frac{z_p^r}{r p^{r s}}\right| \leqslant \sum_{r\geqslant 2} p^{- r \Re s} = p^{-2 \Re s} \frac{1}{1-p^{-\Re s}} .
\]
Elementary analysis gives 
\[
	1 \leqslant \frac{1}{1-p^{-\Re s}} \leqslant 2 + 2\sqrt 2 ,
\]
so the second piece of $\log L(\bz,s)$ converges absolutely when 
$\Re s>\frac 1 2$. We could simply cite \cite[II.1 Th.~10]{tenenbaum-1995}; 
instead we prove directly that $\sum_p \frac{z_p}{p^s}$ converges absolutely 
to a holomorphic function on the region $\{\Re s>\alpha\}$. 

By Lemma \ref{lem:abel-sum} with $f(x) = x^{-s}$, we have 
\begin{align*}
	\sum_{p\leqslant N} \frac{z_p}{p^s}
		&= N^{-s} \sum_{p\leqslant N} z_p + s \int_2^N \sum_{p\leqslant x} z_p\, \frac{\dd x}{x^{s+1}} \\
		&\ll N^{-\Re s + \alpha + \epsilon} + s \int_2^N x^{\alpha+\epsilon} \frac{\dd x}{x^{s+1}} .
\end{align*}
Since $\alpha-\Re s < 0$, the first term is bounded. Since $s+1-\alpha > 1$ and 
$\epsilon$ is arbitrary, the integral converges absolutely, and the proof is 
complete. 
\end{proof}

\begin{theorem}
Let $\bz=(z_2,z_3,\dots)$ be a sequence with $\|\bz\|_\infty\leqslant 1$, and 
assume $\log L(\bz,s)$ has analytic continuation to $\{\Re s>\alpha\}$ for some 
$\alpha\in \frac 1 2,1]$, and that for $\sigma>\alpha$, we have 
$|\log L(\bz,\sigma+i t)| \ll |t|^{1-\epsilon}$ (implied constant independent 
of $\sigma$.) Then $|\sum_{p\leqslant N} z_p| \ll N^{\alpha+\epsilon}$. 
\end{theorem}
\begin{proof}
Recall that we can write 
\[
	\log L(\bz,s) = \sum_p \frac{z_p}{p^s} + \sum_p \sum_{r\geqslant 2} \frac{z_p^r}{r p^{r s}} = \sum_p \frac{z_p}{p^s} + O(\zeta(2 \Re s)) .
\]
Thus, for any $\epsilon>0$, analytic continuation and the bound on 
$|\log L(\bz,\sigma+i t)|$ implies the same analytic continuation and bound for 
$\sum \frac{z_p}{p^s}$ on $\{\Re s>\alpha+\epsilon\}$. 

For any $T>0$, let 
$\gamma_T = \gamma_{1,T} + \gamma_{2,T} + \gamma_{3,T} + \gamma_{4,T}$ be the 
following contour: 
\begin{align*}
	\gamma_{1,T}(t) &= (\alpha+\epsilon)+i t\qquad t\in [-T,T] \\
	\gamma_{2,T}(t) &= t+i T \qquad t\in [\alpha+\epsilon,1+\epsilon] \\
	\gamma_{3,T}(t) &= (1+\epsilon) + i t \qquad t\in [T,-T] \\
	\gamma_{4,T}(t) &= t - i T \qquad t\in [1+\epsilon,\alpha+\epsilon] .
\end{align*}
Graphically, the contour looks like this: 
\begin{center}
\begin{tikzpicture}[
	decoration={%
		markings,
		mark=at position 2cm with {\arrow[line width=1pt]{>}},
		}]
	\draw [help lines,->] (-1,0) -- (4,0) coordinate (xaxis);
	\draw [help lines,->] (0,-4) -- (0,4) coordinate (yaxis);
	\node [below] at (xaxis) {$\bR$};
	\node [left] at (yaxis) {$i\bR$};
	\node at (0.6,1) {$\gamma_{1,T}$};
	\node at (2, 2.5) {$\gamma_{2,T}$};
	\node at (3.6, 1) {$\gamma_{3,T}$};
	\node at (-2.5, 2) {$\gamma_{4,T}$};
	\path[draw,postaction=decorate] 
		(1,-3) node[below] {$\alpha+\epsilon - i T$} -- 
		(1,3) node[above] {$\alpha+\epsilon + i T$} -- 
		(3,3) node[above] {$1+\epsilon + i T$} -- 
		(3,-3) node[below] {$1+\epsilon - i T$} --
		(1,-3);
\end{tikzpicture}
\end{center}
By Perron's formula \cite[Th.~11.18]{apostol-1976}, 
\[
	\lim_{T\to \infty} \frac{1}{2\pi i} \int_{-\gamma_{3,T}} \sum_p \frac{z_p}{p^s} N^z\, \frac{\dd z}{z} = \frac 1 2 \sum_{p\leqslant N} z_p .
\]
for $N\in \bZ$, and the same without the $\frac 1 2$ on the right-hand side 
when $N\notin \bZ$. 

Let $h(s)$ be the analytic continuation of $\sum z_p p^{-s}$ to 
$\{\Re s>\alpha\}$. Since $\int_{\gamma_T} h(s)\, \frac{\dd s}{s}=0$, we obtain 
\[
	\left|\sum_{p\leqslant N} z_p\right| 
		\ll \lim_{T\to \infty} \left(\left| \int_{\gamma_{1,T}} h(s) N^s\frac{\dd s}{s}\right| + \left|\int_{\gamma_{2,T}} h(s) N^s \frac{\dd s}{s}\right| + \left|\int_{\gamma_{4,T}} h(s) N^s \frac{\dd s}{s}\right| \right).
\]
We know that $|h(\sigma+i t)| \ll |t|^{1-\epsilon}$, so we can bound 
\[
	\left| \int_{\gamma_{2,T}} h(s)N^s \frac{\dd s}{s}\right| = \left| \int_{\alpha+\epsilon}^{1+\epsilon} \frac{h(t+i T) N^{t+i T}}{t+i T}\, \dd t\right| \ll \frac{N^{1+\alpha}}{T^\epsilon} ,
\]
and similarly for $\gamma_{4,T}$. Finally, note that 
\[
	\left| \int_{\gamma_{1,T}} h(s) N^s\, \frac{\dd s}{s}\right| \ll \int_{-T}^T |t|^{1-\epsilon} \frac{N^{\alpha+\epsilon}}{(\alpha+\epsilon)^2 + t^2} \, \dd t \ll N^{\alpha+\epsilon} .
\]
Letting $T\to \infty$ we obtain the desired result. 
\end{proof}

In this thesis, we are interested in the following sort of strange Dirichlet 
series. Let $X$ be a space, $f\colon X\to \bC$ a function with 
$\|f\|_\infty\leqslant 1$, and $\bx=(x_2,x_3,\dots)$ a sequence in $X$. Write 
\[
	L_f(\bx,s) = \prod_p \frac{1}{1-f(x_p) p^{-s}} ,
\]
for the associated strange Dirichlet series. 





\section{Relation to automorphic and motivic \texorpdfstring{$L$}{L}-functions}





\section{The Riemann Hypothesis}





\section{Discrepancy of sequences and the Riemann Hypothesis}





\section{Strange Dirichlet series over function fields}
