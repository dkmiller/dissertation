\documentclass[phd,cornellheadings]{cornell}

\usepackage{
	amsmath,
	amssymb,
	amsthm,
	mathrsfs,     % \mathscr
	stmaryrd,     % llbracket
	thesis-style, % custom math commands
	tikz-cd       % commutative diagrams
}
\usepackage[hidelinks]{hyperref}
\usetikzlibrary{decorations.markings}

\usepackage[
	backend  = bibtex,    % use bibtex instead of biber
	sorting  = nyt,       % sort by (name, year, title)
	style    = alphabetic % citations look like [Har77]
]{biblatex}

\DeclareFieldFormat{postnote}{#1}
\DeclareFieldFormat{multipostnote}{#1}
\addbibresource{thesis-sources.bib}

\title{Counterexamples related to the strong Sato--Tate conjecture}
\author{Daniel Miller}
\conferraldate{May}{2017}


\begin{document}
\maketitle
\makecopyright

\begin{abstract}
Let $E_{/\bQ}$ be an elliptic curve. The Sato--Tate conjecture (now a theorem) 
tells us that the angles $\theta_p =\cos^{-1}\left(\frac{a_p}{2\sqrt p}\right)$ 
are equidistributed in $[0,\pi]$ with respect to the measure 
$\frac{2}{\pi}\sin^2\theta$ if $E$ is not CM, and uniformly distributed if 
$E$ has CM over $\bQ$. Akiyama and Tanigawa conjecture that in fact, the 
discrepancy 
\[
	D_N = \sup_{x\in [0,\pi]} \left| \frac{1}{\pi(N)} \sum_{p\leqslant N} 1_{[0,x]}(\theta_p) - \int_0^x \frac{2}{\pi} \sin^2\theta\, \dd\theta\right| 
\]
decays as $D_N \ll N^{-\frac 1 2+\epsilon}$, as is suggested by computational 
evidence and certain natural heuristics. This conjecture implies the Riemann 
Hypothesis for all $L(\sym^k E,s)$. It is natural to assume that the converse 
holds, as is suggested by analogy with the Riemann Hypothesis for Artin 
$L$-functions. We show that at least when $E$ has CM over $\bQ$, there is no 
reason to believe that the converse holds, as there are ``fake Satake 
parameters'' yielding $L$-functions which satisfy the Generalized Riemann 
Hypothesis, but for which the discrepancy decays like $N^{-\epsilon}$. 

Returning to the non-CM case, we show that there are Galois representations 
$\rho\colon \Gal(\overline \bQ /\bQ) \to \GL_2(\bZ_l)$, ramified at an 
arbitrarily ``thin'' density-zero set of primes, whose ``Satake parameters'' 
converge at any specified rate to any reasonable specified measure on 
$[0,\pi]$. 
\end{abstract}

\begin{biosketch}
Daniel Miller was born in St.~Paul, Minnesota. He completed his Bachelor of 
Science at the University of Nebraska--Omaha, during which he attended 
Cornell's Summer Mathematics Institute in 2011. He started his Ph.D.~at 
Cornell planning on a career in academia, but halfway through had a change of 
heart, and will be joining Microsoft's Analysis and Experimentation team as a 
Data Scientist after graduation. 
\end{biosketch}

\begin{dedication}
This thesis is dedicated to my undergraduate thesis advisor, Griff Elder. 
He is the reason I considered a career in math, his infectious enthusiasm for 
number theory has inspired me more than I can say. 
\end{dedication}

\begin{acknowledgements}
For starters, I'd like to thank my parents Jay and Cindy for noticing and 
fostering my mathematical interests early on, and for being loving and 
supportive the whole way through. I'd also like to thank my undergraduate 
thesis advisor, Griffith Elder, without whose encouragement and inspiration 
I'd probably never have considered a career in math. 

I'd like to thank Tara Holm for organizing Cornell's Summer Mathematics 
Institute in 2011, Jason Boynton for teaching a fantastic algebra class, and 
Anthony Weston for introducing me to the world of nonlinear functional 
analysis. 

Thanks to my graduate student friends Sasha Patotski and Bal\'azs Elek for 
sharing my early love of algebraic geometry, for laughing with me at the 
absurdities of academic life, and listening to my ramblings about number 
theory long after they'd stopped being interesting. 

I owe a big debt of gratitude to the mathematics department at Cornell---so 
many professors were generous with their time and ideas. I especially 
appreciate Birgit Speh, Yuri Berest, David Zywina, Farbod Shokrieh, and John 
Hubbard for letting me bounce ideas off them, helping me add rigor to 
half-baked ideas, and pointing me in new and exciting directions. 

I am especially thankful to my advisor Ravi. He kindled my first love for 
number theory, and stayed supportive as my research bounced all over the place, 
and helped focus and ground my thesis when I needed concrete results. 

Lastly, and most importantly, I thank my loving wife Ivy for being there for me 
through the highs and the lows---both when I (prematurely) thought my thesis was 
complete, and when I thought my results were completely in shambles. I couldn't 
have done it without her. 
\end{acknowledgements}

\contentspage

\normalspacing
\setcounter{page}{1}
\pagenumbering{arabic}
\pagestyle{cornell}





% !TEX root = main.tex

\chapter{Introduction}





\section{Motivation from classical analytic number theory}

We start with a problem near and dear to every number theorist's heart: 
counting prime numbers. As usual, let $\pi(x)$ be the number of rational 
primes $\leqslant x$ and $\Li(x) = \int_2^x \frac{\dd t}{\log t}$ be the 
logarithmic integral. For any $x$, we have the normalized empirical measure 
\[
	P_x = \frac{1}{\pi(x)} \sum_{p\leqslant x} \delta_{p/x} ,
\]
which is supported on the unit interval $[0,1]$. The prime number theorem 
tells us that as $x\to \infty$, these empirical measures approach the 
``true'' measure $L_x = \frac{\Li(t x)}{\Li(x)}\, \dd t$. This is proven via 
demonstrating analytic properties of the Riemann zeta function. 

\begin{theorem}
We have $P_x \to L_x$ (in the weak sense) if and only if $\zeta(s)$ admits 
a meromorphic continuation past $\{\Re =1\}$, with at most a simple pole at 
$s=1$. 
\end{theorem}

Since $\zeta(s)$ has the desired properties, the prime number theorem is true. 
It is natural to try to quantify the rate of converge of $P_x$ to $L_x$. One 
easy way to do this is via the discrepancy 
\[
	\disc(P_x,L_x) 
		= \sup_{t\in [0,1]} \left| P_x[0,t] - L_x[0,t]\right|
		= \sup_{t\in [0,1]} \left| \frac{\pi(t x)}{\pi(x)} - \frac{\int_2^{tx} \frac{\dd s}{\log s}}{\int_2^x \frac{\dd s}{\log s}}\right| .
\]
Numerical experiments suggest that 
$\disc(P_x,L_x) \ll x^{-\frac 1 2+\epsilon}$, and in fact we have the following 
result. 

\begin{theorem}
We have $\disc(P_x,L_x) \ll x^{-\frac 1 2+\epsilon}$ if and only if the 
Riemann Hypothesis holds. 
\end{theorem}

Of course here, neither side is known for certain to be true! This discussion 
finds a natural generalization in Artin $L$-functions. 

Let $K/\bQ$ be a finite Galois extension with group $G=\Gal(K/\bQ)$. For any 
irreducible representation $\rho\colon G\to \GL_d(\bC)$, there is a 
corresponding $L$-function defined as 
\[
	L(\rho,s) = \prod_p \frac{1}{\det(1-\rho(\frob_p) p^{-s})} ,
\]
where we tacitly omit those primes $p$ at which $\rho$ is ramified. Given a 
cutoff $x$, there is a natural empirical measure 
\[
	P_x = \frac{1}{\pi(x)} \sum_{p\leqslant x} \delta_{\frob_p} ,
\]
where $\frob_p$ is a conjugacy class in $G$. Let 
\[
	\disc(P_x) = \sup_{c\in G^\natural} \left| P_x(c) - \frac{1}{\# G}\right|,
\]
where $G^\natural$ is the set of conjugacy classes in $G$. 

\begin{theorem}
We have convergence $P_x \to L$ if and only if $L(\rho,s)$ admits analytic 
continuation past $\{\Re =1\}$ for all nontrivial $\rho$. 
\end{theorem}

Both sides of this equivalence are true, and known as the Chebotarev density 
theorem. Moreover, there is a version of the strong Prime Number Theorem in 
this context. 

\begin{theorem}
We have $\disc(P_x) \ll x^{-\frac 1 2+\epsilon}$ if and only if $L(\rho,s)$ 
satisfies the Riemann Hypothesis for all nontrivial $\rho$. 
\end{theorem}





\section{Discrepancy and Riemann Hypothesis for elliptic curves}

Let's start with something basic, an elliptic curve $E_{/\bQ}$. For any 
prime $l$, we have the Tate module of $E$, written $\tate_l E$. This is a 
rank-$2$ $\bZ_l$-module with continuous $G_\bQ$-action, so it induces a 
continuous representation 
\[
	\rho_{E,l} \colon G_\bQ \to \GL_2(\bZ_l) .
\]
It is known (citation?) that the quantities $a_p(E) = \tr \rho_l(\frob_p)$ lie 
in $\bZ$ and satisfy the Hasse bound 
\[
	|a_p(E)| \leqslant 2\sqrt p .
\]
Thus we can define, for each prime $p$, the corresponding Satake parameter for 
$E$. 
\[
	\theta_p(E) = \cos^{-1}\left(\frac{a_p(E)}{2\sqrt p}\right) \in [0,\pi) .
\]
The Satake parameters are packaged into an $L$-function as follows:
\[
	L^\an(E,s) = \prod_p \frac{1}{(1 - e^{i \theta_p(E)} p^{-s})(1- e^{-i \theta_p(E)} p^{-s})} .
\]
More generally we have, for each $k\geqslant 1$, the $k$-th symmetric power 
$L$-function 
\[
	L^\an(\sym^k E, s) = \prod_p \prod_{j=0}^k \frac{1}{1 - e^{i (k - 2j) \theta_p(E)} p^{-s}} .
\]

Numerical experiments suggest that the Satake parameters are distributed with 
respect to the Sato--Tate distribution 
$\ST = \frac{2}{\pi} \sin^2\theta\, \dd\theta$. The ``goodness of fit'' of the 
Satake parameters to the Sato--Tate distribution is quantified by the 
\emph{discrepancy}:
\[
	\disc^\star(\{\theta_p(E)\}_{p\leqslant X}, \ST) = \sup_{x\in [0,\pi]} \left| \frac{\#\{p\leqslant X : \theta_p(E)\in [0,x)\}}{\pi(X)} - \int_0^x \, \dd \ST\right| .
\]
The decay of the discrepancy is closely related to the analytic properties of 
the $L(\sym^k E,s)$. First, here is the famous Sato--Tate conjecture (now a 
theorem) in the language we have defined. 

\begin{theorem}[Sato--Tate conjecture]
$\disc^\star(\{\theta_p(E)\}_{p\leqslant X}, \ST) \to 0$.  
\end{theorem}

\begin{theorem}
The Sato--Tate conjecture for $E$ holds if and only if each of the functions 
$L(\sym^k E,s)$ have analytic continuation past $\Re s=1$. 
\end{theorem}

The stunning recent proof of the Sato--Tate conjecture (citation) in fact 
showed that the functions $L(\sym^k E,s)$ were potentially automorphic, which 
gives analytic continuation. 

There is an analogy between the above equivalence and classical analytic number 
theory. Let $K/\bQ$ be a finite Galois extension, and 
$\rho\colon \Gal(K/\bQ) \to \GL_n(\bC)$ an irreducible representation. Recall 
the Artin $L$-function is 
\[
	L(\rho,s) = \prod_p \frac{1}{1-\tr \rho(\frob_p) p^{-s}} .
\]
Let $\Gal(K/\bQ)^\natural$ be the set of conjugacy classes in $\Gal(K/\bQ)$. 
The analogue of discrepancy here is: 
\[
	\disc(\{\frob_p\}_{p\leqslant X}) = \sup_{c\in \Gal(K/\bQ)^\natural} \left| \frac{\# \{p\leqslant X : \rho(\frob_p) \in c\}}{\pi(X)} - \frac{1}{\# \Gal(K/\bQ)^\natural}\right| .
\]

\begin{theorem}
The ``discrepancy'' $\disc(\{\frob_p\}_{p\leqslant X})\to 0$ if and only 
if $L(\rho,s)$ has analytic continuation past $\Re s=1$ for all non-trivial 
irreducible representations $\rho$ of $\Gal(K/\bQ)$. 
\end{theorem}

In the case of Artin $L$-functions, we know moreover that 

\begin{theorem}
The ``discrepancy'' satisfies the bound 
$\disc(\{\frob_p\}_{p\leqslant X}) \ll X^{-1/2+\epsilon}$ if and only if 
$L(\rho,s)$ satisfies the Riemann Hypothesis for all non-trivial irreducible 
representation $\rho$ of $\Gal(K/\bQ)$. 
\end{theorem}

In this context, the ``Riemann Hypothesis'' for $L(\rho,s)$ means exactly that 
$\log L(\rho,s)$ has analytic continuation to $\Re s=1/2$. 

The connection between the Riemann Hypothesis and ``strong Sato--Tate'' 
generalizes to elliptic curves and more general motives. For the moment, we 
stick to elliptic curves. In this case, ``strong Sato--Tate'' was conjectured 
by Akiyama--Tanigawa. More precisely, 

Conjecture:

Let $E_{/\bQ}$ be a non-CM elliptic curve. Then 
$\disc^\star(\{\theta_p(E)\}_{p\leqslant X}, \ST) \ll X^{-1/2+\epsilon}$. 


Moreover, one side of the equivalence ``Riemann Hypothesis $\Leftrightarrow$ 
strong Sato--Tate'' is known. 

\begin{theorem}
Let $E_{/\bQ}$ be an elliptic curve. If the Akiyama--Tanigawa conjecture for 
$E$ holds, then all $L(\sym^k E, s)$ satisfy the Riemann Hypothesis. 
\end{theorem}

It is natural to assume that the converse to this theorem holds. However (and 
that is the main point of this thesis) it does not! In this thesis, I construct 
a range of counterexamples to the implication ``strong Sato--Tate implies 
Riemann,'' and explore why the two are equivalent for Artin $L$-functions. 

I also provide computational evidence for the Akiyama--Tanigawa conjecture 
(for elliptic curves and also generic abelian $2$-folds). 

Similar work: \cite{pande-2011}. 

To-do: conjectural framework? Can I find the rank from $\{sign(a_p)\}$?

See \cite{mazur-2008} for a nice discussion. 

% !TEX root = main.tex

\chapter{Discrepancy}





\section{Equidistribution}

The discrepancy (also known as the Kolmogorov--Smirnov statistic) is a way of 
measuring how closely sample data fits a predicted distribution. It has many 
applications in computer science and statistics, but here we will focus on only 
the basic known properties, as well as how discrepancy changes when sequences 
are tweaked and/or combined. 

First, recall that the discrepancy is a way of sharpening the ``soft'' 
convergence results of, say \cite[A.1]{serre-1989}. Let $X$ be a compact 
topological space, $\{x_p\}$ a sequence of points in $X$ indexed by the prime 
numbers. 

\begin{definition}
Let $\mu$ be a continuous probability measure on $X$. The sequence $\{x_p\}$ is 
\emph{equidistributed} with respect to $\mu$ if for all $f\in C(X)$, we have 
\[
	\lim_{x\to \infty} \frac{1}{\pi(x)} \sum_{p\leqslant x} f(x_p) \to \int f\, \dd \mu .
\]
\end{definition}

In other words, $\{x_p\}$ is $\mu$-equidistributed if the empirical measures 
$P_x = \frac{1}{\pi(x)} \sum_{p\leqslant x} \delta_{x_p}$ converge to $\mu$ in 
the weak topology. It is easy to see that $\{x_p\}$ is $\mu$-equidistributed if 
and only if $\left| \sum_{p\leqslant x} f(x_p)\right| = o(x)$ for all 
continuous $f$ having $\int f\, \dd\mu = 0$. In fact, one can restrict to a 
set of $f$ which generate a dense subpace of $C(X)^{\mu =0}$. 

In the discussion in \cite[A.1]{serre-1989}, $X$ is the space of conjugacy 
classes in a compact Lie group, and $f$ is allowed to range over the characters 
of irreducible, nontrivial representations of the group. In this section, we 
will show that the entire discussion can be generalized to a much broader class 
of \emph{strange Dirichlet series}, which are of the form 
\[
	L_f(\{x_p\},s) = \prod_p \frac{1}{1-f(x_p)p^{-s}} .
\]
A useful, but not too well known, result, is that we in fact can consider 
functions $f$ which are only continuous almost everywhere. 

\begin{theorem}
Let $X$ be a compact separable metric space with no isolated points. Let $\mu$ 
be a Borel measure on $X$ and let $f\colon X\to \bC$ be bounded and measurable. 
Then $f$ is continuous almost everywhere if and only if 
\[
	\lim_{x\to \infty} \frac{1}{\pi(x)} \sum_{p\leqslant x} f(x_p) = \int f\, \dd\mu
\]
for all $\mu$-equidistributed sequences $\{x_p\}$. 
\end{theorem}
\begin{proof}
This follows immediately from the proof of \cite[Th.~1]{mazzone-1995}
\end{proof}





\section{Definitions and first results}

We will define discrepancy for measures on the $d$-dimensional half-open box 
$[0,\infty)^d$. For vectors $x,y\in [0,\infty)^d$, we say $x<y$ if 
$x_1<y_1$,\dots,$x_d<y_d$, and in that case write $[x,y)$ for the half-open 
box $[x_1,y_1)\times \cdots \times [x_d,y_d)$. 

\begin{definition}
Let $\mu, \nu$ be probability measures on $[0,\infty)^d$. The 
\emph{discrepancy} of $\mu$ with respect to $\nu$ is 
\[
	\disc(\mu,\nu) = \sup_{x < y} \left| \mu[x,y) - \nu[x,y)\right| ,
\]
where $x<y$ range over $[0,\infty)^d$.

The \emph{star discrepancy} of $\mu$ with respect to $\nu$ is 
\[
	\disc^\star(\mu,\nu) = \sup_{0<y} \left| \mu[0,y) - \nu[0,y)\right| ,
\]
where $y$ ranges over $[0,\infty)^d$. 
\end{definition}

\begin{lemma}
Let $\mu,\nu$ be Borel measures on $\bR^d$. Then 
\[
	\disc^\star(\mu,\nu) \leqslant \disc(\mu,\nu) \leqslant 2^d \disc^\star(\mu,\nu) .
\]
\end{lemma}
\begin{proof}
The first inequality holds because the supremum defining the discrepancy is 
taken over a larger set than that defining star discrepancy. To prove the 
second inequality, let $x<y$ be in $[0,\infty)^d$. For 
$S\subset \{1,\dots,d\}$, let 
\[
	I_S = \{ t \in [0,y) : t_i < x_i \text{ for all }i\in S\} .
\]
The inclusion-exclusion principle for measures tells us that: 
\[
	\mu[x,y) = \sum_{S\subset \{1,\dots,d\}} (-1)^{\# S} \mu(I_S) ,
\]
and similarly for $\nu$. Since each of the $I_S$ are ``half-open boxes'' 
we know that $|\mu(I_S) - \nu(I_S)| \leqslant \disc^\star(\mu,\nu)$. It 
follows that 
\[
	|\mu[x,y) - \nu[x,y)| \leqslant \sum_{S\subset \{1,\dots,d\}} |\mu(I_S) - \nu(I_S)| \leqslant 2^d \disc^\star(\mu,\nu) .
\]
For a discussion and related context, see 
\cite[Ch.~2 Ex.~1.2]{kuipers-niederreiter-1974}. 
\end{proof}

We are usually interested in comparing empirical measures and their conjectured 
distribution. Namely, let $\bx = \{x_p\}$ be a sequence in $[0,\infty)^d$ 
indexed by the prime numbers, and $\mu$ a Borel measure on $[0\infty)^d$. For 
any real number $N\geqslant 2$, we write $\bx^N$ for the empirical measure 
given by 
\[
	\bx^N(S) = \frac{1}{\pi(N)} \sum_{p\leqslant N} \delta_{x_p}(S) = \frac{\# \{p\leqslant N : x_p\in S\}}{\pi(N)} .
\]
Also, we write $\bx_{\geqslant N}$ for the truncated sequence 
$(x_p)_{p\geqslant N}$, and similarly for $\bx_{\leqslant N}$, etc. In this 
context, 
\[
	\disc^\star(\bx^N,\nu) = \sup_{y\in [0,\infty)^d} \left| \frac{\# \{p\leqslant N : x_p \in [0,y)\}}{\pi(N)} - \int_{[0,y)} \, \dd\nu\right| .
\]

If the measure $\nu$ is only defined on a subset of $[0,\infty)^d$, we will 
tacitly extend it by zero. Moreover, if the sequence $\bx$ actually lies in a 
torus $(\bR/a \bZ)^d$, we identify that torus with the 
$[0,a)^d\subset [0,\infty)^d$. If $\nu$ is the Lebesgue measure (on 
$[0,\infty)^d$) or the normalized Haar measure on the torus, we write 
$\disc^\star(\bx^N)$ in place of $\disc^\star(\bx^N, \nu)$. 

Sometimes the sequence $\bx$ will not be indexed by the prime numbers, but 
rather by some other discrete subset of $\bR^+$. In that case we will still 
use the notations $\bx^N$, $\bx_{\geqslant N}$, etc., keeping in mind that 
$\pi(N)$ is replaced by $\#\{\textnormal{indices }\leqslant N\}$. 

Todo: give some basic examples of equidistributed sequences, talk about 
equidistribution, almost-everywhere continuous functions. Prove basic facts 
about van der Corput sequence for arbitrary measures. 





\section{The Koksma--Hlawka inequality}

Here we summarize the results of the paper \cite{okten-1999}, generalizing them 
as needed for our context. Recall that a function $f$ on $[0,\infty)^d$ is 
said to be of \emph{bounded variation} if there is a finite Radon measure $\nu$ 
such that $f(x) - f(0) = \nu[0,x]$. In such a case we write 
$\Var(f) = |\nu|$. If the appropriate differentiability conditions are 
satisfied, then 
\[
	\Var(f) = \int_{[0,\infty)^d} \left|\frac{\dd^d f}{\dd x_1 \dots \dd x_d} \right|.
\]

\begin{theorem}[Koksma--Hlawka]
Let $\mu$ be a probability measure on $[0,\infty)^d$, $f$ a function of 
bounded variation. Then for any sequence $\bx$ in $[0,\infty)^d$, we have 
\[
	\left| \frac{1}{\pi(x)} \sum_{p\leqslant x} f(x_p) - \int f\, \dd\mu \right| \leqslant \Var(f) \disc(\bx^N,\mu) .
\]
\end{theorem}
\begin{proof}
By our assumptions there is a Radon measure $\nu$ such that 
$f(y) - f(0) = \nu[0,y]$. What follows is essentially trivial, noting that 
$1_{[0,x]}(y) = 1_{[y,\infty)^d}(x)$. 
\begin{align*}
	\frac{1}{\pi(x)} \sum_{p\leqslant x} f(x_p) - \int f\, \dd\mu 
		&= \frac{1}{\pi(x)} \sum_{p\leqslant x} \left(f(x_p) - f(0)\right) - \int \left(f - f(0)\right)\, \dd\mu \\
		&= \frac{1}{\pi(x)} \sum_{p\leqslant x} \int 1_{[y,\infty)^d}(x_p)\, \dd \nu(y) - \int \int 1_{[0,y]}\, \dd\nu \, \dd\mu(y) \\
		&= \int \frac{1}{\pi(x)} \sum_{p\leqslant x} 1_{[y,\infty)^d}(x_p) - \int 1_{[y,\infty)^d}\, \dd\mu \, \dd\nu(y)
\end{align*}
It follows that 
\[
	\left| \frac{1}{\pi(x)} \sum_{p\leqslant x} f(x_p) - \int f\, \dd\mu \right|
		\leqslant \sup_{y\in [0,\infty)} \left| \frac{1}{\pi(x)} \sum_{p\leqslant x} 1_{[y,\infty)}(x_p) - \int 1_{[y,\infty)}\, \dd\mu\right| \cdot |\nu| .
\]
The supremum in question is clearly bounded above by $\disc(\bx^N,\mu)$, so the 
proof is complete. 
\end{proof}





\section{Comparing sequences}

\begin{lemma}
Let $\bx$ and $\by$ be sequences in $[0,\infty)$. Suppose 
$\nu = f\cdot \lambda$ for $f$ a bounded continuous function and $\lambda$ the 
Lebesgue measure. Then 
\[
	\left|\disc^\star(\bx^N, \nu) - \disc^\star(\by^N,\nu)\right| \leqslant \|f\|_\infty \epsilon + \disc^\star(\bx^N,\nu) + \frac{\#\{ p\leqslant N : \|x_p - y_p\|_\infty \geqslant \epsilon\}}{\pi(N)} .
\]
\end{lemma}
\begin{proof}
Let $\epsilon>0$ and $t\in [0,\infty)$ be arbitrary. For all $p\leqslant N$ 
such that $y_p<t$, either $x_p < t+\epsilon$ or 
$\|x_p - y_p\|_\infty \geqslant \epsilon$. It follows that 
\[
	\by^N[0,t) \leqslant \bx^N[0,t+\epsilon) + \frac{\#\{ p\leqslant N : \|x_p - y_p\|_\infty \geqslant \epsilon\}}{\pi(N)} .
\]
Moreover, we trivially have 
\[
	\left| \bx^N[0,t+\epsilon) - \nu[0,t+\epsilon)\right| \leqslant \disc^\star(\bx^N,\nu) .
\]
Putting these together, we get: 
\begin{align*}
	\by^N[0,t) - \nu[0,t) 
		&\leqslant \bx^N[0,t+\epsilon) - \nu[0,t) + \frac{\#\{ p\leqslant N : \|x_p - y_p\|_\infty \geqslant \epsilon\}}{\pi(N)} \\
		&\leqslant \nu[t,t+\epsilon) + \disc^\star(\bx^N,\nu) + \frac{\#\{ p\leqslant N : \|x_p - y_p\|_\infty \geqslant \epsilon\}}{\pi(N)} \\
		&\leqslant \|f\|_\infty \epsilon + \disc^\star(\bx^N,\nu) + \frac{\#\{ p\leqslant N : \|x_p - y_p\|_\infty \geqslant \epsilon\}}{\pi(N)} 
\end{align*}
as desired. 
\end{proof}

\begin{lemma}
Let $\sigma$ be an isometry of $\bR$, and $\bx$ a sequence in $[0,\infty)$ 
such that $\sigma(\bx)$ is also in $[0,\infty)$. Let $\nu$ be an absolutely 
continuous measure on $[0,\infty)$ such that $\sigma_\ast \nu$ is also 
supported on $[0,\infty)$. Then 
\[
	\left|\disc(\bx^N, \nu) - \disc(\sigma_\ast \bx^N, \sigma_\ast \nu)\right| \leqslant \frac{2}{\pi(N)} .
\]
\end{lemma}
\begin{proof}
Every isometry of $\bR$ is a combination of translations and reflections. 
The statement is clear with translations (the two discrepancies are equal). So, 
suppose $\sigma(t) = a - t$ for some $a>0$. Since $\nu$ is absolutely 
continuous, $\nu\{t\}=0$ for all $t\geqslant 0$. In particular, 
$\nu[s,t) = \nu(s,t]$. In contrast, $\bx^N\{t\}\leqslant \pi(N)^{-1}$. For any 
interval $[s,t)$ in $[0,\infty)$, we know that 
\[
	\left| \bx^N[s,t) - \bx^N(s,t]\right| \leqslant \frac{2}{\pi(N)}  ,
\]
hence 
\[
	\left| \bx^N[s,t) - \nu[s,t) - (\sigma_\ast \bx^N)[a-t,a-s) - (\sigma_\ast \nu)[a-t,a-s)\right| \leqslant \frac{2}{\pi(N)} .
\]
This proves the result. 
\end{proof}





\section{Combining sequences}

\begin{definition}
Let $\bx$ and $\by$ be sequences in $[0,\infty)^d$. We write $\bx\wr\by$ for 
the interleaved sequence 
\[
	(x_2,y_2,x_3,y_3,x_5,y_5,\dots,x_p,y_p,\dots) .
\]
\end{definition}

For the interleaved sequence $\bx\wr\by$, we write $(\bx\wr\by)^N$ for the 
empirical measure 
\[
	(\bx\wr\by)^N = \frac{1}{2\pi(N)} \sum_{p\leqslant N} \delta_{x_p} + \delta_{y_p} .
\]

\begin{theorem}
Let $I$ and $J$ be disjoint open boxes in $[0,\infty)^d$, and let $\mu$, 
$\nu$ be absolutely continuous probability measures on $I$ and $J$, 
respectively. Let $\bx$ be a sequence in $I$ and $\by$ be a sequence in $J$. 
Then 
\[
	\max\{\disc(\bx^N,\mu),\disc(\by^N,\nu)\} \leqslant \disc((\bx\wr\by)^N, \mu+\nu) \leqslant \disc(\bx^N,\mu) + \disc(\by^N,\nu)
\]
\end{theorem}
\begin{proof}
Any half-open box in $[0,\infty)^d$ can be split by a coordinate 
hyperplane into two disjoint half-open boxes $[a,b)\sqcup [s,t)$, each of which 
intersects at most one of $I$ and $J$. We may assume that 
$[a,b)\cap J=\varnothing$ and $[s,t)\cap I = \varnothing$. Then 
\begin{align*}
	\left| (\bx\wr\by)^N([a,b)\sqcup [s,t)) - (\mu+\nu)([a,b)\sqcup[s,t))\right| 
		&\leqslant |\bx^N[a,b) - \mu[a,b)| + |\by^N[s,t) - \nu[s,t)| \\
		&\leqslant \disc(\bx^N,\mu) + \disc(\by^N,\nu) .
\end{align*}
This yields the second inequality in the statement of the theorem. To see the 
first, assume that the maximum discrepancy is $\disc(\bx^N,\mu)$, and let 
$[s,t)$ be a half-open box such that $|\bx^N[s,t) - \mu[s,t)|$ is within an 
arbitrary $\epsilon$ of $\disc(\bx^N,\mu)$. We can assume that $[s,t)$ does not 
intersect $J$, and thus 
\[
	\left|(\bx\wr\by)^N[s,t) - (\mu+\nu)[s,t)\right| = |\bx^N[s,t) - \mu[s,t)| ,
\]
which yields the result. 
\end{proof}

% !TEX root = main.tex

\chapter{Strange Dirichlet series}

To-do: show that \cite[A.1]{serre-1989} works for $L_f(\bx,s)$, $f$ 
almost-everywhere continuous. 





\section{Definitions}

We start by considering a very general class of Dirichlet series. In fact, they 
are all Dirichlet series that admit a product formula with degree-1 factors, 
but in this thesis they will be called strange Dirichlet series. The motivating 
example was suggested by Ramakrishna. Let $E_{/\bQ}$ be an elliptic curve and 
let 
\[
	L_{\sgn}(E,s) = \prod_p \frac{1}{1-\sgn(a_p) p^{-s}} .
\]
How much can we say about the behavior of $L_{\sgn}(E,s)$? For example, does it 
``know'' the rank of $E$?

\begin{definition}
Let $\bz=(z_2,z_3,z_5,\dots)$ be a sequence of complex numbers indexed by the 
primes. The associated \emph{strange Dirichlet series} is 
\[
	L(\bz,s) = \prod_p \frac{1}{1- z_p p^{-s}} .
\]
\end{definition}

If $z_p$ is only defined for all but finitely many primes, then we tacitly set 
$\bz_p = 0$ for all primes for which $z_p$ is not defined. 

\begin{lemma}
Let $\bz$ be a sequence with $\|\bz\|_\infty \leqslant 1$. Then $L(\bz,s)$ 
defines a holomorphic function on the region $\{\Re s>1\}$. Moreover, on that 
region, 
\[
	\log L(\bz,s) = \sum_{p^r} \frac{z_p^n}{n p^{n s}} .
\]
\end{lemma}
\begin{proof}
Expanding the product for $L(\bz,s)$ formally, we have 
\[
	L(\bz,s) = \sum_{n\geqslant 1} \frac{\prod_p z_p^{v_p(n)}}{n^s} .
\]
An easy comparison with the Riemann zeta function tells us that this sum 
is holomorphic on $\{\Re s>1\}$. By \cite[Th.~11.7]{apostol-1976}, the 
product formula holds in the same region. The formula for $\log L(\bz,s)$ 
comes from \cite[11.9 Ex.2]{apostol-1976}. 
\end{proof}

\begin{lemma}[Abel summation]\label{lem:abel-sum}
Let $\bz=(z_2,z_3,z_5,\dots)$ be a sequence of complex numbers, $f$ a smooth 
complex-valued function on $\bR$. Then 
\[
	\sum_{p\leqslant N} f(p) z_p = f(N) \sum_{p\leqslant N} z_p - \int_2^N f'(x) \sum_{p\leqslant x} z_p\, \dd x .
\]
\end{lemma}
\begin{proof}
Simply note that if $p_1,\dots,p_n$ is an enumeration of the primes 
$\leqslant N$, we have 
\begin{align*}
	\int_2^N f'(x) \sum_{p\leqslant x} z_p\, \dd x 
		&= \sum_{p\leqslant N} z_p \int_{p_n}^N f' + \sum_{i=1}^{n-1} \sum_{p\leqslant p_{i+1}} z_p \int_{p_i}^{p_{i+1}} f' \\
		&= (f(N) - f(p_n)) \sum_{p\leqslant N} z_p + \sum_{i=1}^{n-1} (f(p_{i+1}) - f(p_i)) \sum_{p\leqslant p_{i+1}} z_p \\
		&= f(N) \sum_{p\leqslant N} z_p - \sum_{p\leqslant N} f(p) z_p ,
\end{align*}
as desired. 
\end{proof}

\begin{theorem}
Assume $|\sum_{p\leqslant x} z_p| \ll x^{\alpha+\epsilon}$ for some 
$\alpha\in [\frac 1 2,1]$. Then the series for $\log L(\bz,s)$ converges to a 
holomorphic function on the region $\{\Re s>\alpha\}$. 
\end{theorem}
\begin{proof}
Formally split the sum for $\log L(\bz,s)$ into two pieces: 
\[
	\log L(\bz,s) = \sum_p \frac{z_p}{p^s} + \sum_p \sum_{r\geqslant 2} \frac{z_p^r}{r p^{r s}} .
\]
For each $p$, we have 
\[
	\left| \sum_{r\geqslant 2} \frac{z_p^r}{r p^{r s}}\right| \leqslant \sum_{r\geqslant 2} p^{- r \Re s} = p^{-2 \Re s} \frac{1}{1-p^{-\Re s}} .
\]
Elementary analysis gives 
\[
	1 \leqslant \frac{1}{1-p^{-\Re s}} \leqslant 2 + 2\sqrt 2 ,
\]
so the second piece of $\log L(\bz,s)$ converges absolutely when 
$\Re s>\frac 1 2$. We could simply cite \cite[II.1 Th.~10]{tenenbaum-1995}; 
instead we prove directly that $\sum_p \frac{z_p}{p^s}$ converges absolutely 
to a holomorphic function on the region $\{\Re s>\alpha\}$. 

By Lemma \ref{lem:abel-sum} with $f(x) = x^{-s}$, we have 
\begin{align*}
	\sum_{p\leqslant N} \frac{z_p}{p^s}
		&= N^{-s} \sum_{p\leqslant N} z_p + s \int_2^N \sum_{p\leqslant x} z_p\, \frac{\dd x}{x^{s+1}} \\
		&\ll N^{-\Re s + \alpha + \epsilon} + s \int_2^N x^{\alpha+\epsilon} \frac{\dd x}{x^{s+1}} .
\end{align*}
Since $\alpha-\Re s < 0$, the first term is bounded. Since $s+1-\alpha > 1$ and 
$\epsilon$ is arbitrary, the integral converges absolutely, and the proof is 
complete. 
\end{proof}

\begin{theorem}
Let $\bz=(z_2,z_3,\dots)$ be a sequence with $\|\bz\|_\infty\leqslant 1$, and 
assume $\log L(\bz,s)$ has analytic continuation to $\{\Re s>\alpha\}$ for some 
$\alpha\in \frac 1 2,1]$, and that for $\sigma>\alpha$, we have 
$|\log L(\bz,\sigma+i t)| \ll |t|^{1-\epsilon}$ (implied constant independent 
of $\sigma$.) Then $|\sum_{p\leqslant N} z_p| \ll N^{\alpha+\epsilon}$. 
\end{theorem}
\begin{proof}
Recall that we can write 
\[
	\log L(\bz,s) = \sum_p \frac{z_p}{p^s} + \sum_p \sum_{r\geqslant 2} \frac{z_p^r}{r p^{r s}} = \sum_p \frac{z_p}{p^s} + O(\zeta(2 \Re s)) .
\]
Thus, for any $\epsilon>0$, analytic continuation and the bound on 
$|\log L(\bz,\sigma+i t)|$ implies the same analytic continuation and bound for 
$\sum \frac{z_p}{p^s}$ on $\{\Re s>\alpha+\epsilon\}$. 

For any $T>0$, let 
$\gamma_T = \gamma_{1,T} + \gamma_{2,T} + \gamma_{3,T} + \gamma_{4,T}$ be the 
following contour: 
\begin{align*}
	\gamma_{1,T}(t) &= (\alpha+\epsilon)+i t\qquad t\in [-T,T] \\
	\gamma_{2,T}(t) &= t+i T \qquad t\in [\alpha+\epsilon,1+\epsilon] \\
	\gamma_{3,T}(t) &= (1+\epsilon) + i t \qquad t\in [T,-T] \\
	\gamma_{4,T}(t) &= t - i T \qquad t\in [1+\epsilon,\alpha+\epsilon] .
\end{align*}
Graphically, the contour looks like this: 
\begin{center}
\begin{tikzpicture}[
	decoration={%
		markings,
		mark=at position 2cm with {\arrow[line width=1pt]{>}},
		}]
	\draw [help lines,->] (-1,0) -- (4,0) coordinate (xaxis);
	\draw [help lines,->] (0,-4) -- (0,4) coordinate (yaxis);
	\node [below] at (xaxis) {$\bR$};
	\node [left] at (yaxis) {$i\bR$};
	\node at (0.6,1) {$\gamma_{1,T}$};
	\node at (2, 2.5) {$\gamma_{2,T}$};
	\node at (3.6, 1) {$\gamma_{3,T}$};
	\node at (-2.5, 2) {$\gamma_{4,T}$};
	\path[draw,postaction=decorate] 
		(1,-3) node[below] {$\alpha+\epsilon - i T$} -- 
		(1,3) node[above] {$\alpha+\epsilon + i T$} -- 
		(3,3) node[above] {$1+\epsilon + i T$} -- 
		(3,-3) node[below] {$1+\epsilon - i T$} --
		(1,-3);
\end{tikzpicture}
\end{center}
By Perron's formula \cite[Th.~11.18]{apostol-1976}, 
\[
	\lim_{T\to \infty} \frac{1}{2\pi i} \int_{-\gamma_{3,T}} \sum_p \frac{z_p}{p^s} N^z\, \frac{\dd z}{z} = \frac 1 2 \sum_{p\leqslant N} z_p .
\]
for $N\in \bZ$, and the same without the $\frac 1 2$ on the right-hand side 
when $N\notin \bZ$. 

Let $h(s)$ be the analytic continuation of $\sum z_p p^{-s}$ to 
$\{\Re s>\alpha\}$. Since $\int_{\gamma_T} h(s)\, \frac{\dd s}{s}=0$, we obtain 
\[
	\left|\sum_{p\leqslant N} z_p\right| 
		\ll \lim_{T\to \infty} \left(\left| \int_{\gamma_{1,T}} h(s) N^s\frac{\dd s}{s}\right| + \left|\int_{\gamma_{2,T}} h(s) N^s \frac{\dd s}{s}\right| + \left|\int_{\gamma_{4,T}} h(s) N^s \frac{\dd s}{s}\right| \right).
\]
We know that $|h(\sigma+i t)| \ll |t|^{1-\epsilon}$, so we can bound 
\[
	\left| \int_{\gamma_{2,T}} h(s)N^s \frac{\dd s}{s}\right| = \left| \int_{\alpha+\epsilon}^{1+\epsilon} \frac{h(t+i T) N^{t+i T}}{t+i T}\, \dd t\right| \ll \frac{N^{1+\alpha}}{T^\epsilon} ,
\]
and similarly for $\gamma_{4,T}$. Finally, note that 
\[
	\left| \int_{\gamma_{1,T}} h(s) N^s\, \frac{\dd s}{s}\right| \ll \int_{-T}^T |t|^{1-\epsilon} \frac{N^{\alpha+\epsilon}}{(\alpha+\epsilon)^2 + t^2} \, \dd t \ll N^{\alpha+\epsilon} .
\]
Letting $T\to \infty$ we obtain the desired result. 
\end{proof}

In this thesis, we are interested in the following sort of strange Dirichlet 
series. Let $X$ be a space, $f\colon X\to \bC$ a function with 
$\|f\|_\infty\leqslant 1$, and $\bx=(x_2,x_3,\dots)$ a sequence in $X$. Write 
\[
	L_f(\bx,s) = \prod_p \frac{1}{1-f(x_p) p^{-s}} ,
\]
for the associated strange Dirichlet series. 





\section{Relation to automorphic and motivic \texorpdfstring{$L$}{L}-functions}





\section{The Riemann Hypothesis}





\section{Discrepancy of sequences and the Riemann Hypothesis}





\section{Strange Dirichlet series over function fields}

% !TEX root = main.tex

\chapter{Irrationality exponents}

\input{05_deformation_theory}
% !TEX root = main.tex

\chapter{Constructing Galois representations}





\section{Main idea}

Basic ideas is as follows. Start with $\rho_1\colon G_\bQ \to \GL_2(\bF_l)$. At 
each stage, we have $\rho_n\colon G_\bQ \to \GL_2(\bZ/l^n)$. At that stage, 
we're allowed to choose the (integral) characteristic polynomial for Frobenius 
at an arbitrarily large set of primes $R_n$. Then some (but density zero) extra 
primes ramify, and then we get $\rho_{n+1}$ that agrees with our choices for 
$R_1\cup \cdots \cup R_n$. Then we have to choose characteristic polynomials for 
$R_{n+1}$, but these are already determined modulo $l^{n+1}$. 

Basic idea is, we can choose the $R_n$ to be so huge that the primes involved are 
way way bigger than $l^n$. For example, we could have 
$R_n=\{p\leqslant l^{l^n}\}$. Thus the set of possible $a_p$'s is very big (big 
enough) so that we can get discrepancy to behave as we like (both decaying 
slowly and decaying quickly) to any measure $\mu$ such that $\ST/\mu$ is 
bounded away from zero. (By this, we mean: if $\ST=f\cdot\lambda$ and 
$\mu = g\cdot\lambda$, where $\lambda$ is Lebesgue, then $f/g$ is bounded away 
from zero.)

We loosely summarize \cite{pande-2011}, changing the notation to make things 
clearer. 

Start with a prime $l$ and a representation 
$\bar\rho\colon G_\bQ \to \GL_2(\bZ/l)$. For any set $S$ of primes outside 
which $\bar\rho$ is unramified, let $\cX_{\bar\rho,S}$ be the space of 
deformations of $\bar\rho$ that are also unramified outside $S$. If 
$\chi\colon G_{\bQ,S} \to \bZ_l^\times$ is a character (perhaps a power of the 
cyclotomic character) let $\cX_{\bar\rho,S}^\chi$ be the subspace of 
$\cX_{\bar\rho,S}$ with determinant $\chi$. 

It is known that the obstruction to lifting an element of 
$\cX_{\bar\rho,S}^\chi(\bZ/l^n)$ to $\cX_{\bar\rho,S}^\chi(\bZ/l^{n+1})$ lies 
in $\h^2(G_{\bQ,S},\Ad^0\bar\rho)$. If the obstruction vanishes, lifts are a
torsor under a natural action of $\h^1(G_{\bQ,S},\Ad^0\bar\rho)$. 

Given $\bar\rho$, there is a set of primes $p$ called \emph{nice primes}, which 
are those $p$ for which:
\begin{enumerate}
\item
$\bar\rho$ is unramified at $p$.

\item
The ratio of eigenvalues of $\bar\rho(\frob_p)$ is $p$. 
\end{enumerate}
If $p$ is a nice prime, there is a subspace $\cC_p$ of deformations of 
$\left.\bar\rho\right|_{G_{\bQ_p}}$ consisting of \emph{nice deformations}. 
(Not too sure about that.)

At the very least, $p$ is nice iff 
$\Ad^0 \bar\rho\simeq \bF_l\oplus \bF_l(1)\oplus \bF_l(-1)$. (Maybe this is a 
nicer way to think about it.)

Then put $N_p = \h^1(G_{\bQ_p},\bF_l(1))\subset \h^1(G_{\bQ_p},\Ad^0\bar\rho)$. 
Let $C_p$ be deformations of $\bar\rho$ which are fixed by $N_p$. (What does 
this mean?)

% !TEX root = main.tex

\chapter{Counterexample via Diophantine Approximation}

\section{Supporting results}

Give $(\bR/\bZ)^d$ the natural Haar measure normalized to have total mass one. 
Recall that for any $f\in L^1((\bR/\bZ)^d)$, the Fourier coefficients of $f$ 
are, for $m\in \bZ^d$ 
\[
	\widehat f(m) = \int_{(\bR/\bZ)^d} e^{2\pi i \langle m,x\rangle} \, \dd x ,
\]
where $\langle m,x\rangle = m_1 x_1 + \cdots + m_d x_d$ is the usual inner 
product. 

\begin{theorem}
Fix $x\in (\bR/\bZ)^d$ with $\omega_{d-1}(x)$ finite. Then 
\[
	\left| \sum_{n\leqslant N} e^{2\pi i \langle m, n x\rangle}\right| \ll |m|^{\omega_{d-1}(x) + \epsilon} 
\]
as $m$ ranges over $\bZ^r\smallsetminus 0$. 
\end{theorem}
\begin{proof}
From Lemma \ref{lem:bound-exp-sum} we know that 
\[
	\left| \sum_{n\leqslant N} e^{2\pi i \langle m, n x\rangle}\right| \ll \frac{1}{d(\langle m, x\rangle,\bZ)} ,
\]
and from Lemma \ref{lem:bound-distance}, we know that 
$d(\langle m,x\rangle, \bZ)^{-1} \ll |m|^{\omega_{d-1}(x)+\epsilon}$. The 
result follows. 
\end{proof}

\begin{theorem}
Let $x\in \bR^d$ with $\omega_{d-1}(x)$ finite. Then let 
$f\in L^1((\bR/\bZ)^d)$ with $\widehat f(0)=0$ and suppose the Fourier 
coefficients of $f$ satisfy the bound 
$|\widehat f(m)| \ll |m|^{-\frac{1}{d-1} - \omega_{d-1}(x)-\epsilon}$. Then 
\[
	\left| \sum_{n\leqslant N} f(n x)\right| \ll 1. 
\]
\end{theorem}
\begin{proof}
Write $f$ as a Fourier series:
\[
	f(x) = \sum_{m\in \bZ^r} \widehat f(m) e^{2\pi pi \langle m,x\rangle} .
\]
Since $\widehat f(0)=0$, we can compute:
\begin{align*}
	\left| \sum_{n\leqslant N} f(n x)\right| 
		&= \left| \sum_{n\leqslant N} \sum_{m\in \bZ^d\smallsetminus 0} \widehat f(m) e^{2\pi i \langle m,x\rangle}\right| \\
		&\leqslant \sum_{m\in \bZ^d\smallsetminus 0} |\widehat f(m)| \left| \sum_{n\leqslant N} e^{2\pi i n \langle m,x\rangle}\right| \\
		&\ll \sum_{m\in \bZ^d\smallsetminus 0} |m|^{-\frac{1}{d-1} - \omega_{d-1}(x) - \epsilon} |m|^{\omega_{d-1}(x) + \epsilon/2} \\
		&\ll \sum_{m\in \bZ^d\smallsetminus 0} |m|^{-\frac{1}{d-1} - \epsilon/2} .
\end{align*}
The sum converges since the exponent is less than $-\frac{1}{d-1}$, and it 
doesn't depend on $N$, hence the result. 
\end{proof}





\section{Pathological Satake parameters}

d





\section{Some remarks on isotropic discrepancy}

d

% !TEX root = main.tex

\chapter{Direct counterexample}

%% !TEX root = main.tex

\chapter{Computational evidence for the Akiyama--Tanigawa conjecture}

% !TEX root = main.tex

\chapter{Concluding remarks and future directions}






\printbibliography[heading=bibintoc]
\end{document}
