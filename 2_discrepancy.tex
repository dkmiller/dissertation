% !TEX root = Daniel-Miller-thesis.tex

\chapter{Discrepancy}





\section{Equidistribution}

Discrepancy (also known as the Kolmogorov--Smirnov statistic) is a way of 
measuring how closely sample data fits a predicted distribution. It has many 
applications in computer science and statistics, but here we will focus on only 
the basic properties, such as how discrepancy changes when sequences 
are ``tweaked'' and combined. 

First, recall that discrepancy provides a way of sharpening the ``soft'' 
convergence results of, say \cite[A.1]{serre-1989}. Let $X$ be a compact 
topological space, $\bx = (x_2,x_3,x_5,\dots)$ a sequence of points in $X$ 
indexed by the rational primes. 

\begin{definition}
Let $\mu$ be a continuous probability measure on $X$. The sequence $\bx$ is 
\emph{equidistributed} with respect to $\mu$ if for all $f\in C(X)$, we have 
\[
	\lim_{N\to \infty} \frac{1}{\pi(N)} \sum_{p\leqslant N} f(x_p) \to \int f\, \dd \mu .
\]
\end{definition}

In other words, $\bx$ is $\mu$-equidistributed if the empirical measures 
$P_N = \frac{1}{\pi(N)} \sum_{p\leqslant N} \delta_{x_p}$ converge to $\mu$ in 
the weak topology. It is easy to see that $\bx$ is $\mu$-equidistributed if 
and only if $\left| \sum_{p\leqslant N} f(x_p)\right| = o(N)$ for all 
continuous $f$ having $\int f\, \dd\mu = 0$. In fact, one can restrict to a 
set of $f$ which generate a dense subpace of $C(X)^{\mu = 0}$. 

In the discussion in \cite[A.1]{serre-1989}, $X$ is the space of conjugacy 
classes in a compact Lie group, and $f$ is allowed to range over the characters 
of irreducible, nontrivial representations of the group. We will see that the 
entire discussion can be generalized to a much broader class of 
Dirichlet series, which are of the form 
\[
	L_f(\bx,s) = \prod_p \frac{1}{1-f(x_p)p^{-s}} .
\]
In fact, we can consider functions $f$ which are only only continuous almost 
everywhere. 

\begin{theorem}
Let $X$ be a compact separable metric space with no isolated points. Let $\mu$ 
be a Borel measure on $X$ and let $f\colon X\to \bC$ be bounded and measurable. 
Then $f$ is continuous almost everywhere if and only if 
\[
	\lim_{N\to \infty} \frac{1}{\pi(N)} \sum_{p\leqslant N} f(x_p) = \int f\, \dd\mu
\]
for all $\mu$-equidistributed sequences $\bx$. 
\end{theorem}
\begin{proof}
This follows immediately from the proof of \cite[Th.~1]{mazzone-1995}
\end{proof}





\section{Definitions and first results}

We will define discrepancy for measures on the $d$-dimensional half-open box 
$[0,\vinfty) = [0,\infty)^d\subset \bR^d$. For vectors 
$\vx,\vy\in [0,\infty)^d$, we say $\vx<\vy$ if $x_1<y_1$,\dots,$x_d<y_d$, and 
in that case write $[\vx,\vy)$ for the half-open box 
$[x_1,y_1)\times \cdots \times [x_d,y_d)$. 

\begin{definition}
Let $\mu, \nu$ be probability measures on $[0,\vinfty)$. The 
\emph{discrepancy} of $\mu$ with respect to $\nu$ is 
\[
	\D(\mu,\nu) = \sup_{\vx < \vy} \left| \mu[\vx,\vy) - \nu[\vx,\vy)\right| ,
\]
where $\vx<\vy$ range over $[0,\vinfty)$.

The \emph{star discrepancy} of $\mu$ with respect to $\nu$ is 
\[
	\D^\star(\mu,\nu) = \sup_{0<\vy} \left| \mu[0,\vy) - \nu[0,\vy)\right| ,
\]
where $\vy$ ranges over $[0,\vinfty)$. 
\end{definition}

\begin{lemma}
Let $\mu,\nu$ be Borel measures on $\bR^d$. Then 
\[
	\D^\star(\mu,\nu) \leqslant \D(\mu,\nu) \leqslant 2^d \D^\star(\mu,\nu) .
\]
\end{lemma}
\begin{proof}
The first inequality holds because the supremum defining the discrepancy is 
taken over a larger set than that defining star discrepancy. To prove the 
second inequality, let $\vx<\vy$ be in $[0,\vinfty)$. For 
$S\subset \{1,\dots,d\}$, let 
$I_S = \{ \vt \in [0,\vy) : t_i < x_i \,\forall\,i\in S\}$.
\begin{figure}[h]
\caption{The sets $I_{\{1\}}$ and $I_{\{2\}}$ when $d = 2$. $I_\varnothing$ is 
the large square and $I_{\{1,2\}}$ is the small square.}
\centering
\begin{tikzpicture}
\draw (1,1) rectangle (4,4);
\draw (0,0) rectangle (4,4);
\draw (0,0) rectangle (1,4);
\draw (0,0) rectangle (4,1);
\node [above right] at (1, 1) {$\vx$};
\node at (1,1) {\textbullet};
\node [below left] at (4,4) {$\vy$};
\node at (.5,1) {$I_{\{1\}}$};
\node at (1,.5) {$I_{\{2\}}$};
\end{tikzpicture}
\end{figure}
Inclusion-exclusion tells us that 
$\mu[\vx,\vy) = \sum_{S\subset \{1,\dots,d\}} (-1)^{\# S} \mu(I_S)$, 
and similarly for $\nu$. Since each of the $I_S$ are ``half-open boxes'' 
we know that $|\mu(I_S) - \nu(I_S)| \leqslant \D^\star(\mu,\nu)$. It 
follows that 
\[
	|\mu[\vx,\vy) - \nu[\vx,\vy)| \leqslant \sum_{S\subset \{1,\dots,d\}} |\mu(I_S) - \nu(I_S)| \leqslant 2^d \D^\star(\mu,\nu) .
\]
For a discussion and related context, see 
\cite[Ch.~2 Ex.~1.2]{kuipers-niederreiter-1974}. 
\end{proof}

Since we are only interested in the asymptotics of discrepancy, we will 
gloss over the distinction between discrepancy and star discrepancy, using 
whichever makes a proof easier to follow. 

We are usually interested in comparing empirical measures and their conjectured 
distribution. Namely, let $\bx = (\vx_1,\vx_2,\dots)$ be a sequence in 
$[0,\vinfty)$, and $\mu$ a probability measure on $[0,\vinfty)^d$. For any real 
number $N\geqslant 2$, the empirical measure associated to the set 
$\{\vx_n\}_{n\leqslant N}$ is 
$\frac{1}{N} \sum_{n\leqslant N} \delta_{\vx_n}$. Write 
\[
	\D_N(\bx,\mu) = \D_N\left( \frac{1}{N} \sum_{n\leqslant N} \delta_{\vx_n},\mu\right) ,
\]
and likewise for the star discrepancy. Also, we write 
$\bx_{\geqslant N}$ for the truncated sequence 
$(\vx_n)_{n\geqslant N}$, and similarly for $\bx_{\leqslant N}$, etc. In this 
context, 
\[
	\D^\star_N(\bx,\mu) = \sup_{\vy\in [0,\vinfty)} \left| \frac{\# \{n\leqslant N : \vx_n \in [0,\vy)\}}{N} - \int_{[0,\vy)} \, \dd\mu\right| .
\]

If the measure $\mu$ is only defined on a Borel subset of $[0,\vinfty)^d$, we w
tacitly extend it to $\bR^d$ by zero. Moreover, if the sequence $\bx$ actually 
lies in a torus $(\bR/a \bZ)^d$, we identify that torus with  
$[0,a)^d\subset [0,\infty)^d$. If $\mu$ is normalized Haar measure on the 
torus, we write $\D_N^\star(\bx)$ in place of $\D_N^\star(\bx, \mu)$. 

If the sequence $\bx$ lies in the space $G^\natural$ of conjugacy classes in a 
compact Lie group $G$, choose a maximal torus $T\subset G$, and recall that 
$G^\natural = T/W$, where $W$ is the Weyl group of $T$. There is a ``half-open 
box'' in $\ft = \lie(T)$ which maps bijectively to $T$ under the exponential 
map. Choose a ``half-open'' polyhedral set $Q$ that maps bijectively to $T/W$. 
Then $Q\subset \ft$ and, if we choose a basis for $\ft$ mapping to zero in 
$T$, then it makes sense to talk about the discrepancy of a sequence in 
$G^\natural$ with respect to the Haar measure. The paper 
\cite{rosengarten-2013} has a different definition of discrepancy which only 
works for semisimple simply-connected groups, but also proves an 
Erd\"os--Tur\'an inequality in that context. It is likely that a reasonable 
application of isotropic discrepancy would render these definitions equivalent, 
at least for asymptotic purposes, but as the two definitions coincide for 
$\SU(2)$, we do not explore this further. 

Sometimes the sequence $\bx$ will not be indexed by the natural numbers, but 
by the rational primes, or some other discrete subset of $\bR^+$. In that case 
we will still use the notations $\D_N(\bx,\mu)$, $\bx_{\geqslant N}$, etc., 
keeping in mind that the set $\{\vx_\alpha : \alpha \leqslant N\}$ is involved, 
and that in formulas $\frac{1}{N}$ is replaced by 
$\#\{\textnormal{indices }\leqslant N\}^{-1}$. 

\emph{Why half-open boxes?} The choice of sets of the form $[\vx,\vy)$ in the 
definition of discrepancy seems rather arbitrary. It is. Discrepancy could just 
as well be defined as a supremum over all open (or closed) balls---this 
definition generalizes to arbitrary metric spaces. There are more subtle 
notions involving suprema over open or closed convex sets. See 
\cite{kuipers-niederreiter-1974} for a discussion and comparison of these 
differing definitions. In this thesis, we restrict to half-open boxes because 
they're computationally tractable, fit well with diophantine approximation on 
tori, and the theory is best developed in that context. 





\section{Statistical heuristics}

Replace the Satake parameters $\theta_p$ of an elliptic curve with 
a sequence $\{\theta_p\}$ of iid random variables with common distribution 
$\ST = \frac{2}{\pi} \sin^2 \theta\, \dd\theta$ supported on $[0,\pi]$. Then 
the discrepancy (known as the Kolmogorov--Smirnov statistic in this context) is 
the random variable 
\[
	D_N = \sup_{x\in [0,\pi]} \left|\frac{1}{\pi(N)} \sum_{p\leqslant N} 1_{[0,x]}\circ \theta_p - \int 1_{[0,x]}\, \dd\ST\right| .
\]
Kolmogorov and Smirnov proved that the inside of the absolute value converges 
to zero. The Glivenko--Cantelli Theorem says that $D_N \to 0$ almost surely, 
and even better that the normalized discrepancy $\sqrt{\pi(N)} D_N$ approaches 
a limiting distribution $K$ (supremum of the Brownian Bridge) which does not 
depend on $\mu$. The rate of convergence of $\sqrt{\pi(N)} D_N$ to that 
distribution is quantified by the Dvoretzky--Kiefer--Wolfowitz inequality, 
which tells us that 
\[
	\mathrm{P}\left(\sqrt{\pi(N)} D_N > z\right) \leqslant 2 e^{-2 z^2} .
\]
These theorems suggest that for $E_{/\bQ}$ a non-CM elliptic 
curve, the ``true'' discrepancy $\D_N(\btheta,\ST)$ should decay 
like $\pi(N)^{-\frac 1 2}$, or at least $N^{-\frac 1 2+\epsilon}$. 
Ideally, the normalized discrepancy $\sqrt{\pi(N)}\D_N^\star(\btheta,\ST)$ 
would also be equidistributed, but sadly numerical experiments suggest that 
this is not the case. 





\section{The Koksma--Hlawka inequality}

Here we summarize the results of the paper \cite{okten-1999}, generalizing them 
as needed for our context. Recall that a function $f$ on 
$[0,\vinfty)\subset \bR^d$ is said to be of \emph{bounded variation} (in the 
measure-theoretic sense) if there is a finite Radon measure $\nu$ such that 
$f(\vx) - f(0) = \nu[0,\vx]$. In such a case we write $\Var(f) = |\nu|$. If 
the appropriate differentiability conditions are satisfied, then 
\[
	\Var(f) = \int_{[0,\vinfty)} \left|\frac{\dd^d f}{\dd t_1 \dots \dd t_d} \right|.
\]

\begin{theorem}[Koksma--Hlawka]
Let $\mu$ be a probability measure on $[0,\vinfty)$, $f$ a function of 
bounded variation. Then for any sequence $\bx = (\vx_1,\vx_2,\dots)$ in 
$[0,\vinfty)$, we have 
\[
	\left| \frac{1}{N} \sum_{n\leqslant N} f(\vx_n) - \int f\, \dd\mu \right| \leqslant \Var(f) \D_N(\bx,\mu) .
\]
\end{theorem}
\begin{proof}
By our assumptions there is a finite Radon measure $\nu$ such that 
$f(\vy) - f(0) = \nu[0,\vy]$. What follows is essentially trivial, noting that 
$1_{[0,\vx]}(\vy) = 1_{[\vy,\vinfty)}(\vx)$. 
\begin{align*}
	\frac{1}{N} \sum_{n\leqslant N} f(\vx_n) - \int f\, \dd\mu 
		&= \frac{1}{N} \sum_{n\leqslant N} \left(f(\vx_n) - f(0)\right) - \int \left(f(\vy) - f(0)\right)\, \dd\mu(\vy) \\
		&= \frac{1}{N} \sum_{n\leqslant N} \int 1_{[\vy,\vinfty)}(\vx_n)\, \dd \nu(\vy) - \int \int 1_{[0,\vy]}\, \dd\nu \, \dd\mu(\vy) \\
		&= \int \left(\frac{1}{N} \sum_{n\leqslant N} 1_{[\vy,\vinfty)}(\vx_n) - \int 1_{[\vy,\vinfty)}\, \dd\mu \right) \dd\nu(\vy) .
\end{align*}
It follows that 
\[
	\left| \frac{1}{N} \sum_{n\leqslant N} f(\vx_n) - \int f\, \dd\mu \right|
		\leqslant \sup_{\vy\in [0,\infty)} \left| \frac{1}{N} \sum_{n\leqslant N} 1_{[\vy,\vinfty)}(\vx_n) - \int 1_{[\vy,\vinfty)}\, \dd\mu\right| \cdot |\nu| .
\]
The supremum in question is clearly bounded above by $\D_N(\bx,\mu)$, so the 
proof is complete. 
\end{proof}

This theorem is proved in a somewhat restrictive setting, and can be 
generalized. For $f$ a function on $\bR^+$ that is bounded variation in 
the traditional sense (for example, piecewise continuous) and $\mu$ a 
continuous probability measure, the inequality 
\[
	\left| \frac{1}{N} \sum_{n\leqslant N} f(x_n) - \int f\, \dd\mu\right| \leqslant \Var(f) \D_N^\star(\bx,\mu) 
\]
holds \cite[Ch.~2, Th.~5.1]{kuipers-niederreiter-1974}. In particular, when 
$\mu$ is the Sato--Tate measure and $f$ is piecewise continuous, we can apply 
this inequality. 





\section{Comparing and combining sequences}

Throughout this section, $\lambda$ is the Lebesgue measure. 

\begin{lemma}\label{lem:disc-of-two-seq}
Let $\bx$ and $\by$ be sequences in $[0,\infty)$. Suppose $\mu$ is an 
absolutely continuous probability measure on $[0,\infty)$ with bounded 
Radon--Nikodym derivative $\frac{\dd \mu}{\dd\lambda}$. Let $\epsilon>0$ be 
arbitrary. Then 
\[
	\left|\D_N^\star(\bx, \nu) - \D_N^\star(\by,\nu)\right| \leqslant \left\|\frac{\dd\mu}{\dd\lambda}\right\|_\infty \epsilon + \frac{\#\{ n\leqslant N : |x_n - y_n| \geqslant \epsilon\}}{N} .
\]
\end{lemma}
\begin{proof}
Let $\epsilon>0$ and $t\in [0,\infty)$ be arbitrary. In this proof, write 
$P_{\bx,N} = \frac 1 N \sum_{n\leqslant N} x_n$ for the empirical measure 
associated to $\bx$, and similarly for $\by$. For all $n\leqslant N$ 
such that $y_n<t$, either $x_n < t+\epsilon$ or 
$|x_n - y_n| \geqslant \epsilon$. It follows that 
\[
	P_{\by,N}[0,t) \leqslant P_{\bx,N}[0,t+\epsilon) + \frac{\#\{ n\leqslant N : |x_n - y_n| \geqslant \epsilon\}}{N} .
\]
Moreover, we have 
$\left| P_{\bx,N}[0,t+\epsilon) - \nu[0,t+\epsilon)\right| \leqslant \D_N^\star(\bx,\nu)$. Putting these together, we get: 
\begin{align*}
	P_{\by,N}[0,t) - \nu[0,t) 
		&\leqslant P_{\bx,N}[0,t+\epsilon) - \nu[0,t) + \frac{\#\{ n\leqslant N : |x_n - y_n| \geqslant \epsilon\}}{N} \\
		&\leqslant \nu[t,t+\epsilon) + \D_N^\star(\bx,\nu) + \frac{\#\{ n\leqslant N : |x_n - y_n| \geqslant \epsilon\}}{N} \\
		&\leqslant \left\|\frac{\dd\mu}{\dd\lambda}\right\|_\infty \epsilon + \D_N^\star(\bx,\nu) + \frac{\#\{ n\leqslant N : |x_n - y_n| \geqslant \epsilon\}}{N} 
\end{align*}
This tells us that 
$\D_N^\star(\by,\nu) \leqslant \left\|\frac{\dd\mu}{\dd\lambda}\right\|_\infty \epsilon + \D_N^\star(\bx,\nu) + \frac{\#\{ n\leqslant N : |x_n - y_n| \geqslant \epsilon\}}{N}$. 
Reversing the roles of $\bx$ and $\by$, we obtain the desired result. 
\end{proof}

\begin{lemma}\label{lem:flip-discrepancy}
Let $\sigma$ be an isometry of $\bR$, and $\bx$ a sequence in $[0,\infty)$ 
such that $\sigma(\bx)$ is also in $[0,\infty)$. Let $\nu$ be an absolutely 
continuous measure on $[0,\infty)$ such that $\sigma_\ast \nu$ is also 
supported on $[0,\infty)$. Then 
\[
	\left|\D_N(\bx, \nu) - \D_N(\sigma_\ast \bx, \sigma_\ast \nu)\right| \leqslant \frac{2}{N} .
\]
\end{lemma}
\begin{proof}
Every isometry of $\bR$ is a combination of translations and reflections. 
The statement is clear with translations (the two discrepancies are equal). So, 
suppose $\sigma(t) = a - t$ for some $a>0$. Since $\nu$ is absolutely 
continuous, $\nu\{t\}=0$ for all $t\geqslant 0$. In particular, 
$\nu[s,t) = \nu(s,t]$. As before, write $P_{\bx,N}$ for empirical measures. By 
definition, $P_{\bx,N}\{t\}\leqslant N^{-1}$. For any 
interval $[s,t)$ in $[0,\infty)$, we know that 
$\left| P_{\bx,N}[s,t) - P_{\bx,N}(s,t]\right| \leqslant \frac{2}{N}$, hence 
\[
	\left| P_{\bx,N}[s,t) - \nu[s,t) - P_{\sigma_\ast \bx,N}[a-t,a-s) - \sigma_\ast \nu[a-t,a-s)\right| \leqslant \frac{2}{N} .
\]
This proves the result. 
\end{proof}

A technique we will use throughout this thesis involves comparing the 
discrepancy of a sequence with the discrepancy of a pushforward sequence, 
with respect to the pushforward measure. 

\begin{lemma}\label{lem:push-discrepancy}
Let $I$, $J$ be closed connected intervals and $f\colon I\twoheadrightarrow J$ 
a continuous monotonic map. If $\bx$ is a sequence in $I$ and $\mu$ is an 
absolutely continuous probability measure on $I$, then 
$\left|\D_N(\bx,\mu) - \D_N(f_\ast \bx,f_\ast\mu)\right| \leqslant \frac 4 N$.
\end{lemma}
\begin{proof}
Because $f$ is continuous and monotonic, given $[u,v)\subset I$, there exists 
$[x,y)\subset J$ such that $f[u,v)$ differs from $[x,y)$ by at most two 
elements. Similarly, if $[x,y)\subset J$, there exists $[u,v)\subset I$ such 
that $f^{-1}[x,y)$ differs from $[u,v)$ by at most two elements. Writing 
$P_{\bx,N}$ as usual for the empirical measure, we compute (in the first 
case):
\[
	\left| P_{f_\ast\bx,N}[x,y) - f_\ast\mu[x,y) - \left(P_{\bx,N}[u,v) - \mu[u,v)\right)\right| \leqslant \frac 4 N ,
\]
equality $f_\ast\mu[u,v) = \mu[x,y)$ following from the continuity of $\mu$. 
It follows that 
$\D_N(\bx,\mu) \leqslant \D_N(f_\ast \bx,f_\ast\mu) + \frac 4 N$. Reversing 
the roles of $\bx$ and $f_\ast \bx$, the result follows. 
\end{proof}

Now we show that the discrepancy behaves as expected when two sequences are 
interleaved. 

\begin{definition}
Let $\bx$ and $\by$ be sequences in $[0,\vinfty)\subset \bR^d$. We write 
$\bx\wr\by$ for the interleaved sequence $(x_1,y_1,x_2,y_2,x_3,y_3,\dots)$. 
\end{definition}

Write 
$P_{\bx\wr\by,N} = \frac{1}{2N} \sum_{n\leqslant N} \left(\delta_{x_n} + \delta_{y_n}\right)$
for the combined empirical measure of the interleaved sequence $\bx\wr\by$. 

\begin{theorem}\label{thm:wreath-seq}
Let $I$ and $J$ be disjoint open boxes in $[0,\vinfty)$, and let $\mu$, 
$\nu$ be absolutely continuous probability measures on $I$ and $J$, 
respectively. Let $\bx$ be a sequence in $I$ and $\by$ be a sequence in $J$. 
Then 
\[
	\max\{\D_N(\bx,\mu),\D_N(\by,\nu)\} \leqslant \D_N(\bx\wr\by, \mu+\nu) \leqslant \D_N(\bx,\mu) + \D(\by,\nu)
\]
\end{theorem}
\begin{proof}
Any half-open box in $[0,\vinfty)$ can be split by a coordinate 
hyperplane into two disjoint half-open boxes $[\va,\vb)\sqcup [\vs,\vt)$, each 
of which intersects at most one of $I$ and $J$. We may assume that 
$[\va,\vb)\cap J=\varnothing$ and $[\vs,\vt)\cap I = \varnothing$. Then 
\begin{align*}
	\left| P_{\bx\wr\by,N}([\va,\vb)\sqcup [\vs,\vt)) - (\mu+\nu)([\va,\vb)\sqcup[\vs,\vt))\right| 
		&\leqslant |P_{\bx,N}[\va,\vb) - \mu[\va,\vb)| + |P_{\by,N}[\vs,\vt) - \nu[\vs,\vt)| \\
		&\leqslant \D_N(\bx,\mu) + \D_N(\by,\nu) .
\end{align*}
This yields the second inequality in the statement of the theorem. To see the 
first, assume that the maximum discrepancy is $\D_N(\bx,\mu)$, and let 
$[\vs,\vt)$ be a half-open box such that $|P_{\bx,N}[\vs,\vt) - \mu[\vs,\vt)|$ 
is within some arbitrary $\epsilon$ of $\D_N(\bx,\mu)$. We can replace 
$[\vs,\vt)$ with a smaller box to ensure it does not intersect $J$. Assuming 
$[\vs,\vt)\cap J=\varnothing$, we have
\[
	\left|P_{\bx\wr\by,N}[\vs,\vt) - (\mu+\nu)[\vs,\vt)\right| = |P_{\bx,N}[\vs,\vt) - \mu[\vs,\vt)| ,
\]
which yields the result. 
\end{proof}





\section{Examples}

Historically, one of the first interesting examples of an equidistributed 
sequence is the set of translates  of an irrational number modulo one. 

\begin{theorem}
Let $a\in \bR$ be irrational. Then the sequence 
$\bx=(a\mod 1,2a\mod 1,3a\mod 1,\dots)$ is equidistributed in $[0,1]$. 
\end{theorem}

We will prove this result (originally due to Weyl) in Chapter \ref{chapter:irrationality-exponent}. It is known that sequences of this form 
have discrepancy which decays like $N^{-\alpha\pm \epsilon}$, for 
$\alpha\in (0,1/2)$. It can be useful to have a sequence whose discrepancy 
decays faster. The best known rate of decay is achieved by the following 
sequence. 

\begin{definition}
The \emph{van der Corput sequence} is 
$\bv = \left(\frac 1 2,\frac 1 4,\frac 3 4,\dots\right)$. More precisely, 
write $n$ in base $2$ as $n = \sum a_i 2^i$. Then $v_n = \sum a_i 2^{-(i+1)}$. 
\end{definition}

The van der Corput sequence has generalizations to other bases and higher 
dimensions. It is well-known for being ``very equidistributed''---i.e., its 
discrepancy has extremely fast convergence to zero. 

\begin{lemma}
Let $\bv$ be the van der Corput sequence. Then
$\D_N(\bv) \leqslant \frac{\log(N+1)}{N\log 2}$. 
\end{lemma}
\begin{proof}
This is \cite[Ch.~2 Th.~3.5]{kuipers-niederreiter-1974}. In particular, we 
will use often that $\D_N(\bv)\ll \frac{\log N}{N}$. 
\end{proof}

The van der Corput sequence is uniformly distributed, but we will use the 
``almost-invariance of discrepancy under transform'' results of the previous 
section to construct sequences equidistributed with respect to more 
general measures. 

\begin{theorem}\label{thm:van-der-corput}
Let $\mu$ be an absolutely continuous probability measure on an interval $I$  
whose cdf is strictly increasing on $I$. Then there exists a sequence 
$\bx=(x_1,x_2,\dots)$ in $I$ such that $\D_N(\bx,\mu) \ll \frac{\log(N)}{N}$. 
\end{theorem}
\begin{proof}
Lemma \ref{lem:push-discrepancy} tells us that for $\bv$ the van der Corput 
sequence on $[0,1]$, we have 
$\left|\D_N(\cdf_\mu^{-1}(\bv), \mu) - \D_N(\bv,\mu)\right| \ll N^{-1}$, 
which gives us the desired result with $\bx = \cdf_\mu^{-1}(\bv)$. 
\end{proof}

\begin{theorem}\label{thm:discrepancy-arbitrary}
Let $\mu$ be an absolutely continuous probability measure, supported on $I$, 
whose cdf is strictly increasing on $I$. Fix $\alpha\in (0,1)$. Then there 
exists a sequence $\bx=(x_1,x_2,\dots)$ such that 
$\D_N(\bx,\mu) = \Theta(N^{-\alpha})$. 
\end{theorem}
\begin{proof}
Let $I = [a,b]$. If $\bx_{\leqslant N}$ is a sequence of length $N$, let 
$\bx_{\leqslant N}:a^M$ be the sequence $(x_1,\dots,x_N,a,\dots,a)$ ($M$ copies 
of $a$). Then 
\[
	\D(\bx_{\leqslant N}:a^M,\mu)
		\geqslant \left| \frac{\#\{ n\leqslant N+M : x_n = a\}}{N+M} - \mu\{a\}\right| 
		\geqslant \frac{M}{N+M} .
\]
On the other hand, 
\begin{align*}
	\left| P_{N,M}[s,t) - P_N[s,t)\right| 
		&\leqslant \frac{\left|\#\{n\leqslant N : s\leqslant x_n< t\} + M - \frac{M+N}{N}\#\{n\leqslant N : s\leqslant x_n < t\}\right|}{M+N} \\
%		&\leqslant \frac{\left|M - \frac{M}{N} \#\{n\leqslant N : x_n \leqslant t\} \right|}{M+N} \\
		&\leqslant \frac{2M}{M+N} ,
\end{align*}
which implies that 
$\D\left(\bx_{\leqslant N}:a^M,\mu\right) \leqslant \D\left(\bx^N,\mu\right) + \frac{2 M}{M+N}$. 
Let $\bv$ be the $\mu$-equidistributed van der Corput sequence of 
Theorem \ref{thm:van-der-corput}, possibly transformed linearly to lie in 
$[a,b]$. We know that $\D(\bv^N, \mu) \ll \frac{\log N}{N}$, which converges 
to zero faster than $N^{-\alpha}$. 

We construct the sequence $\bx$ via the following recipe. Start with 
$(x_1 = v_1,x_2 = v_2,\dots)$ until, for some $N_1$, 
$\D_{N_1}(\bx,\mu) < N_1^{-\alpha}$. Then set $x_{N_1+1} = a$, 
$x_{N_1+2} = a$, \dots, until 
$\D_{N_1+M_1}(\bx,\mu) > (N_1+M_1)^{-\alpha}$. Then set 
$x_{N_1+M_1+1} = v_{N_1+1}$, $x_{N_1+M_1+2} = v_{N_1+2}$, \dots, 
until once again 
$\D_{N_1+M_1+N_2}(\bx,\mu) < (N_1+M_1+N_2)^{-\alpha}$. Repeat 
indefinitely. We will show first, that the two steps are possible, and that 
nowhere does $\D_N(\bx,\mu)$ differ by too much from $N^{-\alpha}$. 

Note that $\frac{M+1}{N+M+1} - \frac{M}{N+M} \leqslant N^{-1}$. This tells 
us that when we are adding $a$'s at the end of $\bx^N$, the discrepancy of 
$\bx_{\leqslant N}:a_{\leqslant M}$ increases by at most $N^{-1}$ at each 
step. So if $\D_N(\bx,\mu) < N^{-\alpha}$, we can ensure that 
$\D_N(\bx_{\leqslant N}:a^M,\mu)$ is at most $N^{-1}$ greater than 
$N^{-\alpha}$. 

Moreover, we know that $\D_N(\bx:a,\mu)$ is at most 
$\frac{2}{N+1}$ away from $\D_N(\bx,\mu)$. So when adding van der 
Corput elements to the end of the sequence, its discrepancy cannot decay any 
faster than by $\frac{2}{N+1}$ per $a$ added. This yields 
\[
	\left|\D_N(\bx,\mu) - N^{-\alpha}\right| \ll N^{-1} , 
\]
which is even stronger than we need.
\end{proof}

In the remainder of this thesis, we will refer to the sequences constructed in 
this theorem as ``$N^{-\alpha}$-decay van der Corput sequences.''

