% !TEX root = Daniel-Miller-thesis.tex

\chapter{Discrepancy}\label{ch2:discrepancy}





\section{Equidistribution}

Discrepancy (also known as the Kolmogorov--Smirnov statistic) is a way of 
measuring how closely sample data fits a predicted distribution. It has many 
applications in computer science and statistics, but here we will focus on only 
its basic properties, such as how discrepancy changes when sequences 
are ``tweaked'' and combined. 

First, recall that discrepancy provides a way of sharpening the soft 
convergence results in \cite[A.1]{serre-1989}. Let $X$ be a compact 
topological space, $\bx = (x_2,x_3,x_5,\dots)$ a sequence in $X$ indexed by 
the rational primes. 

\begin{definition}
Let $\mu$ be a continuous probability measure on $X$. The sequence $\bx$ is 
\emph{equidistributed} with respect to $\mu$ if for all $f\in C(X)$, we have 
\[
	\lim_{N\to \infty} \frac{1}{\pi(N)} \sum_{p\leqslant N} f(x_p) = \int f\, \dd \mu .
\]
\end{definition}

In other words, $\bx$ is $\mu$-equidistributed if the empirical measures 
$P_{\bx,N} = \frac{1}{\pi(N)} \sum_{p\leqslant N} \delta_{x_p}$ converge to 
$\mu$ in the weak topology. It is easy to see that $\bx$ is 
$\mu$-equidistributed if and only if 
$\left| \sum_{p\leqslant N} f(x_p)\right| = o(\pi(N))$ for all $f\in C(X)$ 
having $\int f\, \dd\mu = 0$. One can restrict to any set of functions which 
generates a dense subpace of $C(X)^{\mu = 0}$. 

In the discussion in \cite[A.1]{serre-1989}, $X$ is the space of conjugacy 
classes in a compact Lie group, and $f$ is allowed to range over the characters 
of irreducible, nontrivial, unitary representations of the group. Serre's 
results can be generalized to a much broader class of Dirichlet series, 
which are of the form 
\[
	L_f(\bx,s) = \prod_p \frac{1}{1-f(x_p)p^{-s}} .
\]
In fact, in light of the following theorem, we can consider functions $f$ 
which are only continuous almost everywhere. This allows us to consider 
step functions like $1_{[0,\pi/2)} - 1_{(\pi/2,\pi]}$ on $[0,\pi]$. 

\begin{theorem}
Let $X$ be a compact topological space, $\mu$ a Radon probability measure on 
$X$, and $f\colon X\to \bC$ bounded and continuous $\mu$-almost everywhere. If 
$\bx$ is $\mu$-equidistributed, then 
\[
	\lim_{N\to \infty} \frac{1}{\pi(N)} \sum_{p\leqslant N} f(x_p) = \int f\, \dd\mu .
\]
\end{theorem}
\begin{proof}
We prove the more general result that if $\{\mu_n\}$ is a sequence of Radon 
probability measures on $X$ which converges weakly to $\mu$, then 
$\mu_n(f) \to \mu(f)$ for all $f$ which are bounded and continuous 
$\mu$-almost everywhere. 

Let $D$ be the set of points at which $f$ is not continuous. For every 
$\epsilon>0$, there exists an open $U_\epsilon\supset D$ with 
$\mu(U_\epsilon) < \epsilon$, and $f_\epsilon\in C(X)$ such that 
$|f_\epsilon|_\infty \leqslant |f|_\infty$, $f_\epsilon|_D = 0$, and 
$f_\epsilon|_{X\smallsetminus U_\epsilon} = f|_{X\smallsetminus U_\epsilon}$. 
Note that 
\begin{equation}\label{eq:measures-ineq}
	|\mu_n f - \mu f| \leqslant |\mu_n f - \mu_n f_\epsilon| + |\mu_n f_\epsilon - \mu f_\epsilon| + |\mu f_\epsilon - \mu f| .
\end{equation}
Now $|\mu_n f - \mu_n f_\epsilon| \leqslant \mu_n(U_\epsilon) |f|_\infty$. Since 
$U_\epsilon$ is open, this converges to 
$\mu(U_\epsilon) |f|_\infty < \epsilon |f|_\infty$. The second term in 
\eqref{eq:measures-ineq} converges to zero because $f_\epsilon$ is continuous, 
and the third term can be bounded as 
$|\mu f_\epsilon - \mu f| \leqslant \mu(U_\epsilon) |f|_\infty < \epsilon |f|_\infty$. 
We have shown that 
$\limsup_{n\to \infty} |\mu_n f - \mu f| \leqslant 2 \epsilon |f|_\infty$. 
Since $\epsilon$ was arbitrary, the result follows. 
\end{proof}





\section{Definitions and first results}

We will define discrepancy for measures on the $d$-dimensional half-open box 
$[\vzero,\vinfty) = [\vzero,\infty)^d\subset \bR^d$. For vectors 
$\vx,\vy\in [\vzero,\vinfty)$, we say $\vx<\vy$ if $x_i<y_i\,\forall i$, and 
in that case write $[\vx,\vy)$ for the half-open box 
$[x_1,y_1)\times \cdots \times [x_d,y_d)$. 

\begin{definition}
Let $\mu, \nu$ be probability measures on $[\vzero,\vinfty)$. The 
\emph{discrepancy between $\mu$ and $\nu$} is 
$\D(\mu,\nu) = \sup_{\vx < \vy} \left| \mu[\vx,\vy) - \nu[\vx,\vy)\right|$, 
where $\vx$ and $\vy$ range over $[\vzero,\vinfty)$.
The \emph{star discrepancy between $\mu$ and $\nu$} is 
$\D^\star(\mu,\nu) = \sup_{\vzero\leqslant \vy} \left| \mu[\vzero,\vy) - \nu[\vzero,\vy)\right|$, 
where $\vy$ ranges over $[\vzero,\vinfty)$. 
\end{definition}

\begin{lemma}\label{lem:star-reg-disc}
$\D^\star(\mu,\nu) \leqslant \D(\mu,\nu) \leqslant 2^d \D^\star(\mu,\nu)$. 
\end{lemma}
\begin{proof}
The first inequality holds because the supremum defining discrepancy is 
taken over a larger set than that defining star discrepancy. To prove the 
second inequality, let $\vx<\vy$ be in $[\vzero,\vinfty)$. For 
$S\subset \{1,\dots,d\}$, let 
$I_S = \{ \vt \in [\vzero,\vy) : t_i < x_i \,\forall\,i\in S\}$.
\begin{figure}[h]
\caption{The sets $I_{\{1\}}$ and $I_{\{2\}}$ when $d = 2$.}
\centering
\begin{tikzpicture}
\draw (1,1) rectangle (4,4);
\draw (0,0) rectangle (4,4);
\draw (0,0) rectangle (1,4);
\draw (0,0) rectangle (4,1);
\node [above right] at (1, 1) {$\vx$};
\node at (1,1) {\textbullet};
\node [below left] at (4,4) {$\vy$};
\node at (.5,1) {$I_{\{1\}}$};
\node at (1,.5) {$I_{\{2\}}$};
\end{tikzpicture}
\end{figure}
Inclusion-exclusion tells us that 
$\mu[\vx,\vy) = \sum_{S\subset \{1,\dots,d\}} (-1)^{\# S} \mu(I_S)$, 
and similarly for $\nu$. Since each of the $I_S$ are half-open boxes 
intersecting the origin, we know that 
$|\mu(I_S) - \nu(I_S)| \leqslant \D^\star(\mu,\nu)$. It follows that 
\[
	|\mu[\vx,\vy) - \nu[\vx,\vy)| \leqslant \sum_{S\subset \{1,\dots,d\}} |\mu(I_S) - \nu(I_S)| \leqslant 2^d \D^\star(\mu,\nu) .
\]
For a discussion and related context, see 
\cite[Ch.~2 Ex.~1.2]{kuipers-niederreiter-1974}. 
\end{proof}

Since we are only interested in the asymptotics of discrepancy, we will 
sometimes gloss over the distinction between discrepancy and star discrepancy, 
using whichever type of discrepancy makes a proof easier to follow. 

We are usually interested in comparing empirical measures and their conjectured 
asymptotic distribution. 
Let $\bvx = (\vx_1,\vx_2,\vx_3,\dots)$ be a sequence in 
$[\vzero,\vinfty)$, and $\mu$ a probability measure on $[\vzero,\vinfty)$. For any 
$N\geqslant 1$, the empirical measure associated to the truncated 
sequence $\bvx_{\leqslant N} = (\vx_n)_{n\leqslant N}$ is 
$P_{\bvx,N} = \frac{1}{N} \sum_{n\leqslant N} \delta_{\vx_n}$. Write 
$\D_N(\bvx,\mu) = \D\left( P_{\bvx,N},\mu\right)$, and similarly for star 
discrepancy. In this context, 
\[
	\D^\star_N(\bvx,\mu) = \sup_{\vy\in [\vzero,\vinfty)} \left| \frac{\# \{n\leqslant N : \vx_n \in [\vzero,\vy)\}}{N} - \int_{[\vzero,\vy)} \, \dd\mu\right| .
\]
When the measure $\mu$ is clear from the context, we will refer to 
$\D_N(\bvx,\mu)$ (resp.~$\D_N^\star(\bvx,\mu)$) as the discrepancy (resp.~star 
discrepancy) of the sequence $\bx$.

If the measure $\mu$ is only defined on a Borel subset of $[\vzero,\vinfty)$, we 
tacitly extend it by zero to $\bR^d$. It is also possible to define discrepancy 
for sequences lying in compact Lie groups. For example, if the sequence $\bvx$ 
lies in a real torus $T$, choose an isomorphism $\bT^d \simeq T$, and 
using that isomorphism identify the torus with  $[0,1)^d\subset [\vzero,\vinfty)$. 
This gives a definition of discrepancy for sequences in $T$. Two different 
isomorphisms $T\simeq \bT^d$ will give two different definitions of 
discrepancy, but asymptotically they will be bounded above and below by 
constant multiples of each other as long as the measure in question is the 
normalized Haar measure. In that case, we write $\D_N(\bvx)$ in place of 
$\D_N(\bvx, \mu)$. 

We can even define discrepancy for sequences in compact, connected Lie groups, 
though we will only use $G = \SU(2)$. Let $G$ be such a group, and consider a 
sequence lying in the space $G^\natural$ of conjugacy classes of $G$. Choose a 
maximal torus $T\subset G$, and recall that 
$G^\natural = T/W$, for $W$ the Weyl group of $T$. There is a half-open 
box in $\ft = \Lie(T)$ which maps bijectively to $T$ under the exponential 
map. Choose a ``half-open'' polyhedral set $Q$ which is a fundamental domain for 
the action of $W$ on $T$. Then $Q\subset \ft$ and, if we choose a basis for 
$\ft$ mapping to zero in $T$, we can identify $G^\natural$ with a subset of 
$[0,1)^d$, $d = \dim T$. This gives a definition of discrepancy for 
a sequence in $G^\natural$ with respect to the Haar measure. Of course this 
definition depends on the choice of $T$, $Q$, and the basis of $\ft$, but 
asymptotically these definitions are all the same. The paper 
\cite{rosengarten-2013} gives a different definition of discrepancy which only 
works for semisimple simply-connected groups, but unlike this thesis, proves an 
Erd\"os--Tur\'an inequality in that context. It is likely that a reasonable 
application of isotropic discrepancy would render these definitions equivalent, 
at least for asymptotic purposes, but as the two definitions coincide for 
$\SU(2)$, we do not explore this further. 

Sometimes the sequence $\bx$ will not be indexed by the natural numbers, but 
by the rational primes, or some other discrete subset of $\bR^+$. In that case 
we will still use the notations $\D_N(\bx,\mu)$, 
$\bx_{\leqslant N} = (\vx_1,\dots,\vx_N)$, 
$\bx_{\geqslant N} = (\vx_N,\vx_{N+1},\dots)$, etc., 
keeping in mind that the set $\{\vx_\alpha : \alpha \leqslant N\}$ is involved, 
and that in formulas $\frac{1}{N}$ is replaced by 
$\#\{\textnormal{indices }\leqslant N\}^{-1}$. 

\emph{Why half-open boxes?} The choice of sets of the form $[\vx,\vy)$ in the 
definition of discrepancy seems rather arbitrary, and it is. One could easily 
define another kind of discrepancy as a supremum over all open or closed 
balls---and 
those  definitions generalize to arbitrary metric spaces. There are also more 
subtle definitions involving suprema over open or closed convex sets 
(isotropic discrepancy). See \cite{kuipers-niederreiter-1974} for a discussion 
and comparison of these differing definitions. In this thesis, we restrict to 
half-open boxes because they are computationally tractable, fit well with 
Diophantine approximation on tori, and the theory is most well-developed for 
this definition. 





\section{Statistical heuristics}

Let $\Omega$ be a probability space, and let $\{\theta_p\}$ be a collection of 
prime-indexed independent, identically distributed, random variables on 
$\Omega$ with continuous joint distribution $\mu$. By this, we mean each 
$\theta_p\colon \Omega \to \bR$ is measurable, and if $P$ is the probability 
measure on $\Omega$, then 
$\left(\prod_{p\in S} \theta_p\right)_\ast P = \prod_{p\in S} \mu$ for all 
sets $S$ of primes. In 
the language of statistics, $\{\theta_p\}$ is a sequence of iid random 
variables with joint distribution $\mu$. For the sake of concreteness, the 
reader may take $\mu = \frac{2}{\pi} \sin^2 \theta\, \dd\theta$, 
supported on $[0,\pi]$. Then the discrepancy (known as the 
Kolmogorov--Smirnov statistic in this context) is the random variable 
\[
	D_N = \sup_{x\in [0,\pi]} \left|\frac{1}{\pi(N)} \sum_{p\leqslant N} 1_{[0,x]}\circ \theta_p - \int 1_{[0,x]}\, \dd\mu\right| .
\]
Kolmogorov and Smirnov proved that the inside of the absolute value, a 
function-valued random variable, converges 
to zero. The Glivenko--Cantelli theorem says that $D_N \to 0$ almost everywhere, 
and even better, the normalized discrepancy $\sqrt{\pi(N)} D_N$ approaches 
a limiting distribution (supremum of the Brownian bridge) which does not 
depend on $\mu$. The rate of convergence of $\sqrt{\pi(N)} D_N$ to that 
distribution is quantified by the Dvoretzky--Kiefer--Wolfowitz inequality, 
which tells us that 
$\mathrm{P}\left(\sqrt{\pi(N)} D_N > t\right) \leqslant 2 e^{-2 t^2}$. 

Now let $E_{/\bQ}$ be a non-CM elliptic curve, and suppose the Satake parameters 
$\theta_p$ are being randomly drawn from the distribution $\ST$. Then the 
above theorems suggest that as $N\to \infty$, we should have 
$\D_N(\btheta,\ST)\sim \pi(N)^{-\frac 1 2}$, or, if we are not so lucky, at 
least $\D_N(\btheta,\ST) \ll N^{-\frac 1 2+\epsilon}$. Ideally, the 
normalized discrepancy $\sqrt{\pi(N)}\D_N(\btheta,\ST)$ would also 
be equidistributed, but sadly, numerical experiments conducted by the author 
suggest this is not the case. 





\section{The Koksma--Hlawka inequality}

In this section we summarize the results of the paper \cite{okten-1999}, 
generalizing them as needed for our context. Recall that a function $f$ on 
$[\vzero,\vinfty)\subset \bR^d$ is said to be of \emph{bounded variation} (in 
the measure-theoretic sense) if there is a finite Radon measure (i.e., 
finite and inner regular) $\nu$ such that 
$f(\vx) - f(\vzero) = \nu[\vzero,\vx]$. In such a case we write 
$\Var(f) = |\nu|$. If $f\in C^d(\bR^d)$, then 
$\Var(f) = \int_{[\vzero,\vinfty)} \left|\frac{\dd^d f}{\dd t_1 \dots \dd t_d} \right|$. 

\begin{theorem}[Koksma--Hlawka]\label{thm:koksma-hlawka}
Let $\mu$ be a probability measure on $[\vzero,\vinfty)$, $f$ of bounded 
variation. For any sequence $\bvx = (\vx_1,\vx_2,\dots)$ in $[\vzero,\vinfty)$, 
we have 
\[
	\left| \frac{1}{N} \sum_{n\leqslant N} f(\vx_n) - \int f\, \dd\mu \right| \leqslant \Var(f) \D_N(\bvx,\mu) .
\]
\end{theorem}
\begin{proof}
By assumption, there is a finite Radon (that is, inner regular) measure $\nu$ 
such that $f(\vy) - f(\vzero) = \nu[\vzero,\vy]$. What follows relies on the 
fact that $1_{[\vzero,\vx]}(\vy) = 1_{[\vy,\vinfty)}(\vx)$. 
\begin{align*}
	\frac{1}{N} \sum_{n\leqslant N} f(\vx_n) - \int f\, \dd\mu 
		&= \frac{1}{N} \sum_{n\leqslant N} \left(f(\vx_n) - f(\vzero)\right) - \int \left(f(\vx) - f(\vzero)\right)\, \dd\mu(\vx) \\
		&= \frac{1}{N} \sum_{n\leqslant N} \int 1_{[\vzero,\vx_n]}(\vy)\, \dd \nu(\vy) - \int \int 1_{[\vzero,\vx]}(\vy)\, \dd\nu(\vy) \, \dd\mu(\vx) \\
		&= \int \left(\frac{1}{N} \sum_{n\leqslant N} 1_{[\vy,\vinfty)}(\vx_n) - \int 1_{[\vy,\vinfty)}\, \dd\mu \right) \dd\nu(\vy) .
\end{align*}
It follows that 
\[
	\left| \frac{1}{N} \sum_{n\leqslant N} f(\vx_n) - \int f\, \dd\mu \right|
		\leqslant \sup_{\vy\in [\vzero,\vinfty)} \left| \frac{1}{N} \sum_{n\leqslant N} 1_{[\vy,\vinfty)}(\vx_n) - \int 1_{[\vy,\vinfty)}\, \dd\mu\right| \cdot |\nu| .
\]
The supremum is bounded above by $\D_N(\bvx,\mu)$, so the proof is complete. 
\end{proof}

This theorem is proved in a somewhat restrictive setting, and there are more 
general versions of the theorem for less restrictive notions of bounded 
variation. For example,  $f$ a function on $\bR^+$ that is bounded variation 
in the traditional sense (for example, piecewise continuous) and $\mu$ a 
continuous probability measure, the inequality 
\[
	\left| \frac{1}{N} \sum_{n\leqslant N} f(x_n) - \int f\, \dd\mu\right| \leqslant \Var(f) \D_N^\star(\bx,\mu) 
\]
still holds \cite[Ch.~2, Th.~5.1]{kuipers-niederreiter-1974}. In particular, 
when $\mu$ is the Sato--Tate measure and $f$ is piecewise continuous, we can 
apply this inequality. When $d>1$, the non-measure-theoretic definition of 
variation is more general, but much more complicated, than the 
measure-theoretic one. See \cite[2\S5]{kuipers-niederreiter-1974} for a 
discussion of this. 





\section{Comparing and combining sequences}

Throughout this section, $\lambda$ is the Lebesgue measure on $\bR$. Recall 
that for another measure $\mu$ on $\bR$, the \emph{Radon--Nikodym derivative} 
of $\mu$ with respect to $\lambda$ (when it exists) is uniquely determined by 
$\cdf_\mu(x) = \int_{-\infty}^x \frac{\dd\mu}{\dd\lambda}(t)\, \dd t$. The 
Radon--Nikodym derivative exists whenever $\mu$ is absolutely continuous with 
respect to $\lambda$, i.e.~$\mu(S) = 0$ whenever $\lambda(S) = 0$. The 
following result, a generalization of 
\cite[Ch.~2 Th.~4.1]{kuipers-niederreiter-1974}, quantifies how much the 
discrepancy of a sequence changes when the elements of the sequence are changed 
by a small amount. 

\begin{lemma}\label{lem:disc-of-two-seq}
Let $\bx$ and $\by$ be sequences in $[0,\infty)$. Suppose $\mu$ is an 
absolutely continuous probability measure on $[0,\infty)$ with bounded 
Radon--Nikodym derivative $\frac{\dd \mu}{\dd\lambda}$. Let $\epsilon>0$ be 
arbitrary. Then 
\[
	\left|\D_N^\star(\bx, \mu) - \D_N^\star(\by,\mu)\right| \leqslant \left|\frac{\dd\mu}{\dd\lambda}\right|_\infty \epsilon + \frac{\#\{ n\leqslant N : |x_n - y_n| \geqslant \epsilon\}}{N} .
\]
\end{lemma}
\begin{proof}
Fix $\epsilon>0$, and let $t\in [0,\infty)$ be arbitrary. For all 
$n\leqslant N$ such that $y_n<t$, either $x_n < t+\epsilon$ or 
$|x_n - y_n| \geqslant \epsilon$. It follows that 
\[
	P_{\by,N}[0,t) \leqslant P_{\bx,N}[0,t+\epsilon) + \frac{\#\{ n\leqslant N : |x_n - y_n| \geqslant \epsilon\}}{N} .
\]
Moreover, we have 
$\left| P_{\bx,N}[0,t+\epsilon) - \mu[0,t+\epsilon)\right| \leqslant \D_N^\star(\bx,\mu)$. Putting these together, we get: 
\begin{align*}
	P_{\by,N}[0,t) - \mu[0,t) 
		&\leqslant P_{\bx,N}[0,t+\epsilon) - \mu[0,t) + \frac{\#\{ n\leqslant N : |x_n - y_n| \geqslant \epsilon\}}{N} \\
		&\leqslant \mu[t,t+\epsilon) + \D_N^\star(\bx,\mu) + \frac{\#\{ n\leqslant N : |x_n - y_n| \geqslant \epsilon\}}{N} \\
		&\leqslant \left|\frac{\dd\mu}{\dd\lambda}\right|_\infty \epsilon + \D_N^\star(\bx,\mu) + \frac{\#\{ n\leqslant N : |x_n - y_n| \geqslant \epsilon\}}{N} 
\end{align*}
This tells us that 
$\D_N^\star(\by,\mu) \leqslant \left|\frac{\dd\mu}{\dd\lambda}\right|_\infty \epsilon + \D_N^\star(\bx,\mu) + \frac{\#\{ n\leqslant N : |x_n - y_n| \geqslant \epsilon\}}{N}$. 
Reversing the roles of $\bx$ and $\by$, we obtain the desired result. 
\end{proof}

This next result shows that if a sequence is transformed by an isometry of 
$\bR$, the discrepancy of the transformed sequence is roughly the same as the 
discrepancy of the original sequence.

\begin{lemma}\label{lem:flip-discrepancy}
Let $\sigma$ be an isometry of $\bR$, and $\bx$ a sequence in $[0,\infty)$ 
such that $\sigma(\bx)$ is also in $[0,\infty)$. Let $\mu$ be an absolutely 
continuous measure on $[0,\infty)$ such that $\sigma_\ast \mu$ is supported on 
$[0,\infty)$. Then 
$\left|\D_N(\bx, \mu) - \D_N(\sigma_\ast \bx, \sigma_\ast \mu)\right| \leqslant \frac{2}{N}$. 
\end{lemma}
\begin{proof}
Every isometry of $\bR$ is a composition of a translation and a reflection. 
The statement is clear if $\sigma$ is a translation, as then the two 
discrepancies are equal. So, suppose $\sigma(t) = a - t$ for some $a>0$. Since 
$\mu$ is absolutely continuous, $\mu\{t\}=0$ for all $t\geqslant 0$. In 
particular, $\mu[s,t) = \mu(s,t]$. By definition, 
$P_{\bx,N}\{t\}\leqslant \frac 1 N$. For any interval $[s,t)$ in $[0,\infty)$, 
we know that 
$\left| P_{\bx,N}[s,t) - P_{\bx,N}(s,t]\right| \leqslant \frac{2}{N}$, hence 
\[
	\left| P_{\bx,N}[s,t) - \mu[s,t) - \left(P_{\sigma_\ast \bx,N}[a-t,a-s) - \sigma_\ast \mu[a-t,a-s)\right)\right| \leqslant \frac{2}{N} .
\]
This proves the result. 
\end{proof}

A technique we will use throughout this thesis involves comparing the 
discrepancy of a sequence with the discrepancy of a pushforward sequence, 
with respect to the pushforward measure. 

\begin{lemma}\label{lem:push-discrepancy}
Let $I$, $J$ be closed connected intervals and $f\colon I\twoheadrightarrow J$ 
a continuous monotonic map. If $\bx$ is a sequence in $I$ and $\mu$ is an 
absolutely continuous probability measure on $I$, then 
$\left|\D_N(\bx,\mu) - \D_N(f_\ast \bx,f_\ast\mu)\right| \leqslant \frac 4 N$.
\end{lemma}
\begin{proof}
Because $f$ is continuous and monotonic, given $[u,v)\subset I$, there exists 
$[x,y)\subset J$ such that $f[u,v)$ differs from $[x,y)$ by at most two 
elements. Compute:
\[
	\left| P_{f_\ast\bx,N}[x,y) - f_\ast\mu[x,y) - \left(P_{\bx,N}[u,v) - \mu[u,v)\right)\right| \leqslant \frac 4 N ,
\]
the equality $f_\ast\mu[u,v) = \mu[x,y)$ following from the continuity of $\mu$. 
Since $[u,v)\subset I$ was arbitrary, it follows that 
$\D_N(\bx,\mu) \leqslant \D_N(f_\ast \bx,f_\ast\mu) + \frac 4 N$. 

Similarly, for any $[x,y)\subset J$, there exists $[u,v)\subset I$ such that 
$f^{-1}[x,y)$ differs from $[u,v)$ by at most two elements. Compute:
\[
	\left| P_{\bx,N}[u,v) - \mu[u,v) - \left(P_{f_\ast \bx,N}[x,y) - f_\ast\mu[x,y)\right)\right| \leqslant \frac 4 N .
\]
Here, $[x,y)$ is arbitrary, so 
$\D_N(f_\ast \bx,f_\ast\mu) \leqslant \D_N(\bx,\mu) + \frac 4 N$. The result 
follows. 
\end{proof}

Now we show that the discrepancy behaves as expected when two sequences are 
interleaved. 

\begin{definition}
Let $\bx$ and $\by$ be sequences in $[0,\vinfty)\subset \bR^d$. We write 
$\bx\wr\by$ for the interleaved sequence $(x_1,y_1,x_2,y_2,x_3,y_3,\dots)$. 
\end{definition}

Write 
$P_{\bx\wr\by,N} = \frac{1}{2} \left( P_{\bx,N} + P_{\by,N}\right)$ for the 
combined empirical measure of the interleaved sequence $\bx\wr\by$. 

\begin{theorem}\label{thm:wreath-seq}
Let $I$ and $J$ be disjoint open boxes in $[0,\vinfty)$, and let $\mu$, 
$\nu$ be absolutely continuous probability measures on $I$ and $J$, 
respectively. Let $\bx$ be a sequence in $I$ and $\by$ be a sequence in $J$. 
Then 
\[
	\max\{\D_N(\bx,\mu),\D_N(\by,\nu)\} \leqslant \D_N(\bx\wr\by, \mu+\nu) \leqslant \D_N(\bx,\mu) + \D(\by,\nu)
\]
\end{theorem}
\begin{proof}
Any half-open box in $[0,\vinfty)$ can be split by a coordinate 
hyperplane into two disjoint half-open boxes $[\va,\vb)\sqcup [\vs,\vt)$, each 
of which intersects at most one of $I$ and $J$. We may assume that 
$[\va,\vb)\cap J=\varnothing$ and $[\vs,\vt)\cap I = \varnothing$. Then 
\begin{align*}
	\left| P_{\bx\wr\by,N}([\va,\vb)\sqcup [\vs,\vt)) - (\mu+\nu)([\va,\vb)\sqcup[\vs,\vt))\right| 
		&\leqslant |P_{\bx,N}[\va,\vb) - \mu[\va,\vb)| + |P_{\by,N}[\vs,\vt) - \nu[\vs,\vt)| \\
		&\leqslant \D_N(\bx,\mu) + \D_N(\by,\nu) .
\end{align*}
This yields the second inequality in the statement of the theorem. To see the 
first, assume that the maximum discrepancy is $\D_N(\bx,\mu)$, and let 
$[\vs,\vt)$ be a half-open box such that $|P_{\bx,N}[\vs,\vt) - \mu[\vs,\vt)|$ 
is within some arbitrary $\epsilon$ of $\D_N(\bx,\mu)$. Use a hyperplane 
(parallel to a coordinate hyperplane) 
between $I$ and $J$ to ``cut off'' the part of $[\vs,\vt)$ that 
does not intersect $I$. Replacing $[\vs,\vt)$ with this smaller box, we may 
assume it does not intersect $J$. 
\begin{figure}[h]
\caption{The sets $I$, $J$, and $[\vs,\vt)$ when $d = 2$.}
\centering
\begin{tikzpicture}
\draw (0,0) rectangle (3,1);
\draw (1,2) rectangle (4,4);
\draw (-1,1.5) -- (5,1.5);
\draw (2,0.5) rectangle (4,1.5);
\node at (6,1.5) {hyperplane};
\node at (3.5,1) {$[\vs,\vt)$};
\node at (1,0.5) {$I$};
\node at (2.5,3) {$J$};
\end{tikzpicture}
\end{figure}
Under the assumption  
$[\vs,\vt)\cap J=\varnothing$, we have
$\left|P_{\bx\wr\by,N}[\vs,\vt) - (\mu+\nu)[\vs,\vt)\right| = |P_{\bx,N}[\vs,\vt) - \mu[\vs,\vt)|$, 
which yields the result. 
\end{proof}





\section{Examples}

Historically, one of the first interesting examples of an equidistributed 
sequence is the set of translates  of an irrational number modulo one. 

\begin{theorem}[Weyl, Sierpi\'nski, Bohl]
Let $a\in \bR$ be irrational. Then the sequence 
$\bx=(a\mod 1,2a\mod 1,3a\mod 1,\dots)$ is equidistributed in $[0,1)$. 
\end{theorem}

We will prove this result in Chapter \ref{chapter:irrationality-exponent}. It 
is known, and we will prove, that 
sequences of this form have discrepancy which decays roughly like 
$N^{-\alpha}$, for some $\alpha\in \left(0,\frac 1 2\right)$ which controls the 
``goodness'' of rational approximations of $x$. It is useful to have a sequence 
whose discrepancy decays faster. The best known rate of decay is achieved by 
the following example. 

\begin{definition}
For $n\in \bN$, write $n$ in base $2$ as $n = \sum a_i 2^i$, and put 
$v_n = \sum a_i 2^{-(i+1)}$. The \emph{van der Corput sequence} is 
$\bv = (v_1,v_2,v_3,\dots)$.
\end{definition}

The van der Corput sequence has generalizations to other bases and higher 
dimensions, but we will not use them. The discrepancy of the van der Corput 
sequence has extremely fast convergence to zero. 

\begin{lemma}[{\cite[Ch.~2 Th.~3.5]{kuipers-niederreiter-1974}}]
$\D_N(\bv) \leqslant \frac{\log(N+1)}{N\log 2}$. 
\end{lemma}

The van der Corput sequence is uniformly distributed (equidistributed with 
respect to the Lebesgue measure). We can use the results of the previous 
section to construct sequences equidistributed with respect to more general 
measures. 

\begin{theorem}\label{thm:van-der-corput}
Let $\mu$ be an absolutely continuous probability measure on an interval $I$  
whose cdf is strictly increasing on $I$. Then there exists a sequence 
$\bx=(x_1,x_2,\dots)$ in $I$ such that $\D_N(\bx,\mu) \ll \frac{\log(N)}{N}$. 
\end{theorem}
\begin{proof}
Since $\left(\cdf_\mu\right)_\ast\mu$ is the Lebesgue measure, Lemma 
\ref{lem:push-discrepancy} tells us that 
$\left|\D_N(\cdf_\mu^{-1}(\bv), \mu) - \D_N(\bv)\right| \ll N^{-1}$. Since 
$\cdf_\mu$ is strictly increasing, it is an order-preserving bijection between 
$I$ and $[0,1]$, which gives us the desired result with 
$\bx = \cdf_\mu^{-1}(\bv)$. The implied constant in the result does not depend 
on $\mu$. 
\end{proof}

Now that we can construct sequences with discrepancy decaying rapidly (with 
respect to a fixed measure $\mu$), we use the sequences with rapid discrepancy 
decay to construct sequences whose discrepancy decays at any specified rate. 
The $N^{-\alpha}$ in the following theorem could actually be specified by any 
decreasing function of $N$ which converges to zero, but doesn't decay faster 
than $N^{-1}$. 

\begin{theorem}\label{thm:discrepancy-arbitrary}
Let $\mu$ be an absolutely continuous probability measure, supported on $I$, 
whose cdf is strictly increasing on $I$. Fix $\alpha\in (0,1)$. Then there 
exists a sequence $\bx=(x_1,x_2,\dots)$ such that 
$\D_N(\bx,\mu) = \Theta(N^{-\alpha})$. 
\end{theorem}

The proof is similar in concept to the proof that a conditionally (but not 
absolutely!) convergent sequence may be rearranged to sum to any desired value. 
We start with a van der Corput sequence with rapidly decaying discrepancy. Our 
sequence begins by adding van der Corput elements until the discrepancy is 
smaller than $N^{-\alpha}$, then repeatedly adds the same element to the end 
of the sequence, pushing up the discrepancy until it is bigger than 
$N^{-\alpha}$. There are two main difficulties. First, we need to show that 
repeatedly adding the same element to the end of a sequence eventually forces 
the discrepancy to increase, and that when doing this, the discrepancy does not 
increase or decrease too rapidly. 

\begin{proof}
Let $I = [a,b]$. If $\bx_{\leqslant N}$ is a sequence of length $N$, let 
$\bx_{\leqslant N}: a^M$ be the sequence $(x_1,\dots,x_N,a,\dots,a)$ ($M$ copies 
of $a$). We begin by showing that the discrepancy of $\bx_{\leqslant N}:a^M$ 
is eventually large relative to $N^{-\alpha}$. Recalling that $\mu\{a\} = 0$, 
we have:
\[
	\D(\bx_{\leqslant N}: a^M,\mu)
		\geqslant \left| \frac{\#\{ n\leqslant N+M : x_n = a\}}{N+M} - \mu\{a\}\right| 
		\geqslant \frac{M}{N+M} .
\]
So for fixed $N$, if we add enough $a$'s to the end of $\bx_{\leqslant N}$, the 
discrepancy $\D(\bx_{\leqslant N};a^M,\mu)$ will be larger than 
$(N+M)^{-\alpha}$. On the other hand for $J = [s,t)\subset I$, 
\begin{align*}
	\left| P_{\bx_{\leqslant N}: a^M}(J) - P_{\bx_{\leqslant N}}(J)\right| 
		&\leqslant \frac{\left|\#\{n\leqslant N : x_n\in J\} + M - \frac{M+N}{N}\#\{n\leqslant N : x_n\in J\}\right|}{M+N} \\
		&= \frac{\left|M - \frac{M}{N} \#\{n\leqslant N : x_n \leqslant t\} \right|}{M+N} \\
		&\leqslant \frac{M}{M+N} ,
\end{align*}
which implies that 
$\D\left(\bx_{\leqslant N}: a^M,\mu\right) \leqslant \D\left(\bx^N,\mu\right) + \frac{M}{M+N}$. This lets us control how rapidly the discrepancy can increase. 

Let $\bv$ be the $\mu$-equidistributed van der Corput sequence of 
Theorem \ref{thm:van-der-corput}, possibly transformed linearly to lie in 
$[a,b]$. We know that $\D(\bv^N, \mu) \ll \frac{\log N}{N}$, which converges 
to zero faster than $N^{-\alpha}$. 

We construct the sequence $\bx$ via the following recipe. Start with 
$(x_1 = v_1,x_2 = v_2,\dots)$ until, for some $N_1$, 
$\D_{N_1}(\bx,\mu) < N_1^{-\alpha}$. Then set $x_{N_1+1} = a$, 
$x_{N_1+2} = a$, \dots, until 
$\D_{N_1+M_1}(\bx,\mu) > (N_1+M_1)^{-\alpha}$. Then set 
$x_{N_1+M_1+1} = v_{N_1+1}$, $x_{N_1+M_1+2} = v_{N_1+2}$, \dots, 
until once again 
$\D_{N_1+M_1+N_2}(\bx,\mu) < (N_1+M_1+N_2)^{-\alpha}$. Repeat 
indefinitely. We will show first, that the two steps are possible, and that 
nowhere does $\D_N(\bx,\mu)$ differ by too much from $N^{-\alpha}$. 

Note that $\frac{M+1}{N+M+1} - \frac{M}{N+M} \leqslant N^{-1}$. This tells 
us that when we are adding $a$'s at the end of $\bx_{\leqslant N}$, the 
discrepancy of 
$\bx_{\leqslant N}:^M a$ is eventually increasing, and can increase by at most 
$N^{-1}$ at each 
step. So if $\D(\bx_{\leqslant N},\mu) < N^{-\alpha}$, we can ensure that 
$\D(\bx_{\leqslant N}: a^M,\mu)$ is at most $N^{-1}$ greater than 
$N^{-\alpha}$. 
Moreover, we know that $\D(\bx_{\leqslant N}: a,\mu)$ is at most 
$\frac{2}{N+1}$ away from $\D(\bx_{\leqslant N},\mu)$. So when adding van der 
Corput elements to the end of the sequence, its discrepancy cannot decay any 
faster than by $\frac{2}{N+1}$ per $a$ added. This yields 
\[
	\left|\D_N(\bx,\mu) - N^{-\alpha}\right| \ll N^{-1} , 
\]
which implies $\D_N(\bx,\mu) \sim N^{-\alpha}$, both of which are even 
stronger than we need.
\end{proof}
