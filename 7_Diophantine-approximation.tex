% !TEX root = Daniel-Miller-thesis.tex

\chapter{Counterexample via Diophantine Approximation}

\section{Supporting results}

Give $(\bR/\bZ)^d$ the natural Haar measure normalized to have total mass one. 
Recall that for any $f\in L^1((\bR/\bZ)^d)$, the Fourier coefficients of $f$ 
are, for $m\in \bZ^d$ 
\[
	\widehat f(m) = \int_{(\bR/\bZ)^d} e^{2\pi i \langle m,x\rangle} \, \dd x ,
\]
where $\langle m,x\rangle = m_1 x_1 + \cdots + m_d x_d$ is the usual inner 
product. 

\begin{theorem}
Fix $x\in (\bR/\bZ)^d$ with $\omega_{d-1}(x)$ finite. Then 
\[
	\left| \sum_{n\leqslant N} e^{2\pi i \langle m, n x\rangle}\right| \ll |m|^{\omega_{d-1}(x) + \epsilon} 
\]
as $m$ ranges over $\bZ^r\smallsetminus 0$. 
\end{theorem}
\begin{proof}
From Lemma \ref{lem:bound-exp-sum} we know that 
\[
	\left| \sum_{n\leqslant N} e^{2\pi i \langle m, n x\rangle}\right| \ll \frac{1}{d(\langle m, x\rangle,\bZ)} ,
\]
and from Lemma \ref{lem:bound-distance}, we know that 
$d(\langle m,x\rangle, \bZ)^{-1} \ll |m|^{\omega_{d-1}(x)+\epsilon}$. The 
result follows. 
\end{proof}

\begin{theorem}
Let $x\in \bR^d$ with $\omega_{d-1}(x)$ finite. Then let 
$f\in L^1((\bR/\bZ)^d)$ with $\widehat f(0)=0$ and suppose the Fourier 
coefficients of $f$ satisfy the bound 
$|\widehat f(m)| \ll |m|^{-\frac{1}{d-1} - \omega_{d-1}(x)-\epsilon}$. Then 
\[
	\left| \sum_{n\leqslant N} f(n x)\right| \ll 1. 
\]
\end{theorem}
\begin{proof}
Write $f$ as a Fourier series:
\[
	f(x) = \sum_{m\in \bZ^r} \widehat f(m) e^{2\pi pi \langle m,x\rangle} .
\]
Since $\widehat f(0)=0$, we can compute:
\begin{align*}
	\left| \sum_{n\leqslant N} f(n x)\right| 
		&= \left| \sum_{n\leqslant N} \sum_{m\in \bZ^d\smallsetminus 0} \widehat f(m) e^{2\pi i \langle m,x\rangle}\right| \\
		&\leqslant \sum_{m\in \bZ^d\smallsetminus 0} |\widehat f(m)| \left| \sum_{n\leqslant N} e^{2\pi i n \langle m,x\rangle}\right| \\
		&\ll \sum_{m\in \bZ^d\smallsetminus 0} |m|^{-\frac{1}{d-1} - \omega_{d-1}(x) - \epsilon} |m|^{\omega_{d-1}(x) + \epsilon/2} \\
		&\ll \sum_{m\in \bZ^d\smallsetminus 0} |m|^{-\frac{1}{d-1} - \epsilon/2} .
\end{align*}
The sum converges since the exponent is less than $-\frac{1}{d-1}$, and it 
doesn't depend on $N$, hence the result. 
\end{proof}





\section{Pathological Satake parameters}

Let $p_1 = 2, p_2 = 3, p_3 = 5, \dots$ be an enumeration of the prime numbers. 
Let $y\in \bR^d$ with $y_1,\dots,y_d$ linearly independent over $\bQ$. The 
associated sequence of ``fake Satake parameters'' is 
\[
	\bx = (y, 2y, 3 y, 4 y, \dots) ,
\]
where we put $x_{p_n} = n y\mod \bZ^d$. By Theorem \ref{thm:jarnik}, we can 
arrange for $\omega_0(y) = w$ and $\omega_{d-1}(y) = d w + d - 1$. 

\begin{theorem}
The sequence $\bx$ is equidistributed in $(\bR/\bZ)^d$, with discrepancy 
decaying as 
\[
	\disc(\bx^N) \ll N^{-\frac{1}{d w+d} + \epsilon} 
\]
and for which 
\[
	\disc(\bx^N) = \Omega\left(N^{-\frac{d}{w} - \epsilon}\right) .
\]
However, for any $f\in C^\infty((\bR/\bZ)^d)$ with $\widehat f(0)=0$, the 
strange Dirichlet series  $L_f(\bx,s)$ satisfies the Riemann Hypothesis. 
\end{theorem}

Now, as the sequences in this theorem are uniformly distributed, not 
equidistributed with respect to the Haar measure on $\SU(2)$ or any other 
semisimple group, this example may not say too much about non-CM elliptic 
curves. However, let $E_1,\dots,E_n$ be pairwise non-isogenous elliptic curves 
with complex multiplication defined over $\bQ$. Then the Sato--Tate group of 
the abelian variety $A = E_1\times \cdots \times E_n$ is 
$\ST(A) = \SO(2)^n = (S^1)^n$. The theorem above shows that the truth of the 
Generalized Riemann Hypothesis for $A$ says nothing about the rate of decay of 
the discrepancy for the Satake parameters of $A$. 
