% !TEX root = Daniel-Miller-thesis.tex

\chapter{Counterexample via Diophantine Approximation}

\section{Supporting results}

Give $(\bR/\bZ)^d$ the natural Haar measure normalized to have total mass one. 
Recall that for any $f\in L^1((\bR/\bZ)^d)$, the Fourier coefficients of $f$ 
are, for $m\in \bZ^d$ 
\[
	\widehat f(m) = \int_{(\bR/\bZ)^d} e^{2\pi i \langle m,x\rangle} \, \dd x ,
\]
where $\langle m,x\rangle = m_1 x_1 + \cdots + m_d x_d$ is the usual inner 
product. 

\begin{theorem}
Fix $x\in (\bR/\bZ)^d$ with $\omega_{d-1}(x)$ finite. Then 
\[
	\left| \sum_{n\leqslant N} e^{2\pi i \langle m, n x\rangle}\right| \ll |m|^{\omega_{d-1}(x) + \epsilon} 
\]
as $m$ ranges over $\bZ^r\smallsetminus 0$. 
\end{theorem}
\begin{proof}
From Lemma \ref{lem:bound-exp-sum} we know that 
\[
	\left| \sum_{n\leqslant N} e^{2\pi i \langle m, n x\rangle}\right| \ll \frac{1}{d(\langle m, x\rangle,\bZ)} ,
\]
and from Lemma \ref{lem:bound-distance}, we know that 
$d(\langle m,x\rangle, \bZ)^{-1} \ll |m|^{\omega_{d-1}(x)+\epsilon}$. The 
result follows. 
\end{proof}

\begin{theorem}
Let $x\in \bR^d$ with $\omega_{d-1}(x)$ finite. Then let 
$f\in L^1((\bR/\bZ)^d)$ with $\widehat f(0)=0$ and suppose the Fourier 
coefficients of $f$ satisfy the bound 
$|\widehat f(m)| \ll |m|^{-\frac{1}{d-1} - \omega_{d-1}(x)-\epsilon}$. Then 
\[
	\left| \sum_{n\leqslant N} f(n x)\right| \ll 1. 
\]
\end{theorem}
\begin{proof}
Write $f$ as a Fourier series:
\[
	f(x) = \sum_{m\in \bZ^r} \widehat f(m) e^{2\pi pi \langle m,x\rangle} .
\]
Since $\widehat f(0)=0$, we can compute:
\begin{align*}
	\left| \sum_{n\leqslant N} f(n x)\right| 
		&= \left| \sum_{n\leqslant N} \sum_{m\in \bZ^d\smallsetminus 0} \widehat f(m) e^{2\pi i \langle m,x\rangle}\right| \\
		&\leqslant \sum_{m\in \bZ^d\smallsetminus 0} |\widehat f(m)| \left| \sum_{n\leqslant N} e^{2\pi i n \langle m,x\rangle}\right| \\
		&\ll \sum_{m\in \bZ^d\smallsetminus 0} |m|^{-\frac{1}{d-1} - \omega_{d-1}(x) - \epsilon} |m|^{\omega_{d-1}(x) + \epsilon/2} \\
		&\ll \sum_{m\in \bZ^d\smallsetminus 0} |m|^{-\frac{1}{d-1} - \epsilon/2} .
\end{align*}
The sum converges since the exponent is less than $-\frac{1}{d-1}$, and it 
doesn't depend on $N$, hence the result. 
\end{proof}





\section{Pathological Satake parameters}

Let $p_1 = 2, p_2 = 3, p_3 = 5, \dots$ be an enumeration of the prime numbers. 
Let $y\in \bR^d$ with $y_1,\dots,y_d$ linearly independent over $\bQ$. The 
associated sequence of ``fake Satake parameters'' is 
\[
	\bx = (y, 2y, 3 y, 4 y, \dots) ,
\]
where we put $x_{p_n} = n y\mod \bZ^d$. By Theorem \ref{thm:jarnik}, we can 
arrange for $\omega_0(y) = w$ and $\omega_{d-1}(y) = d w + d - 1$. 

\begin{theorem}
The sequence $\bx$ is equidistributed in $(\bR/\bZ)^d$, with discrepancy 
decaying as 
\[
	\disc(\bx^N) \ll N^{-\frac{1}{d w+d} + \epsilon} 
\]
and for which 
\[
	\disc(\bx^N) = \Omega\left(N^{-\frac{d}{w} - \epsilon}\right) .
\]
However, for any $f\in C^\infty((\bR/\bZ)^d)$ with $\widehat f(0)=0$, the 
strange Dirichlet series  $L_f(\bx,s)$ satisfies the Riemann Hypothesis. 
\end{theorem}

Let's apply this theorem to abelian varieties with complex multiplication. 
Let $A_{/\bQ}$ be an $g$-dimensional abelian variety, $F/\bQ$ a number field 
of degree $g$ with $F\simeq \End^\circ(A) = \End_\bQ(A)\otimes \bQ$. Let 
$\cA_{/\bZ}$ be the N\'eron model of $A$. Then there is a natural map 
$\End^\circ(A) \hookrightarrow \End^\circ(\cA_{\bF_p})$. If $A$ has good 
reduction at $p$, the Frobenius of $\cA_{\bF_p}$ is an element of $F^\times$, 
which is $p$-Weil of weight $1$. Write $\pi_p\in F^\times$ for this element. It 
is easy to check that 
$\det(1-\rho_{A,l}(\frob_p) t) = \prod_{\sigma\colon F\hookrightarrow \bC} (1 - \sigma(\pi_p) t)$. 
Moreover, Shimura and Taniyama have proved that there exists a unique Hecke 
character $\chi_\sigma\colon \bA^\times/\bQ^\times \to \bC^\times$, algebraic 
of weight $1$, such that $\chi_\sigma((1,\dots,p,\dots,1)) = \sigma(\pi_p)$ 
for each unramified prime $p$. It follows that 
$L^\alg(A,s) = \prod_{\sigma\colon F\hookrightarrow \bC} L(s,\chi_\sigma)$. 
Let $\Phi$ be the CM type of $F$. Then for $\sigma\in \Phi$, let 
$\omega_\sigma = \chi_\sigma \|\cdot\|^{-1/2}$; this is a Hecke character of 
weight $1/2$, and for $L(A,s) = L^\alg(A,s+1/2)$, 
$L(A,s) = \prod_{\sigma\in\Phi} L(s,\omega_\sigma) L(s,\omega_\sigma^{-1})$. 
Identify $\widehat{\ST(A)} = \bZ^\Phi$, and for $m\in \bZ^\Phi$, write 
$\omega_m = \prod_{\sigma\in \Phi} \omega_\sigma^{m_\sigma}$. Then the 
Akiyama--Tanigawa conjecture for $A$ implies the Riemann Hypothesis for all 
$L(s,\omega_m)$, $m\in \bZ^\Phi\smallsetminus 0$. The above theorem shows that 
the converse doesn't hold. 
