% !TEX root = Daniel-Miller-thesis.tex

\chapter{Counterexample for CM abelian varieties}

\section{Supporting results}

Give $(\bR/\bZ)^d$ the natural Haar measure normalized to have total mass one. 
Recall that for any $f\in L^1((\bR/\bZ)^d)$, the Fourier coefficients of $f$ 
are, for $m\in \bZ^d$ 
\[
	\widehat f(m) = \int_{(\bR/\bZ)^d} e^{2\pi i \langle m,x\rangle} \, \dd x ,
\]
where $\langle m,x\rangle = m_1 x_1 + \cdots + m_d x_d$ is the usual inner 
product. 

\begin{theorem}
Fix $x\in (\bR/\bZ)^d$ with $\omega_{d-1}(x)$ finite. Then 
\[
	\left| \sum_{n\leqslant N} e^{2\pi i \langle m, n x\rangle}\right| \ll |m|^{\omega_{d-1}(x) + \epsilon} 
\]
as $m$ ranges over $\bZ^r\smallsetminus 0$. 
\end{theorem}
\begin{proof}
From Lemma \ref{lem:bound-exp-sum} we know that 
\[
	\left| \sum_{n\leqslant N} e^{2\pi i \langle m, n x\rangle}\right| \ll \frac{1}{d(\langle m, x\rangle,\bZ)} ,
\]
and from Lemma \ref{lem:bound-distance}, we know that 
$d(\langle m,x\rangle, \bZ)^{-1} \ll |m|^{\omega_{d-1}(x)+\epsilon}$. The 
result follows. 
\end{proof}

\begin{theorem}
Let $x\in \bR^d$ with $\omega_{d-1}(x)$ finite. Then let 
$f\in L^1((\bR/\bZ)^d)$ with $\widehat f(0)=0$ and suppose the Fourier 
coefficients of $f$ satisfy the bound 
$|\widehat f(m)| \ll |m|^{-\frac{1}{d-1} - \omega_{d-1}(x)-\epsilon}$. Then 
\[
	\left| \sum_{n\leqslant N} f(n x)\right| \ll 1. 
\]
\end{theorem}
\begin{proof}
Write $f$ as a Fourier series:
\[
	f(x) = \sum_{m\in \bZ^r} \widehat f(m) e^{2\pi pi \langle m,x\rangle} .
\]
Since $\widehat f(0)=0$, we can compute:
\begin{align*}
	\left| \sum_{n\leqslant N} f(n x)\right| 
		&= \left| \sum_{n\leqslant N} \sum_{m\in \bZ^d\smallsetminus 0} \widehat f(m) e^{2\pi i \langle m,x\rangle}\right| \\
		&\leqslant \sum_{m\in \bZ^d\smallsetminus 0} |\widehat f(m)| \left| \sum_{n\leqslant N} e^{2\pi i n \langle m,x\rangle}\right| \\
		&\ll \sum_{m\in \bZ^d\smallsetminus 0} |m|^{-\frac{1}{d-1} - \omega_{d-1}(x) - \epsilon} |m|^{\omega_{d-1}(x) + \epsilon/2} \\
		&\ll \sum_{m\in \bZ^d\smallsetminus 0} |m|^{-\frac{1}{d-1} - \epsilon/2} .
\end{align*}
The sum converges since the exponent is less than $-\frac{1}{d-1}$, and it 
doesn't depend on $N$, hence the result. 
\end{proof}





\section{Pathological Satake parameters for CM abelian varieties}

Let $p_1 = 2, p_2 = 3, p_3 = 5, \dots$ be an enumeration of the prime numbers. 
Let $y\in \bR^d$ with $y_1,\dots,y_d$ linearly independent over $\bQ$. The 
associated sequence of ``fake Satake parameters'' is 
\[
	\bx = (y, 2y, 3 y, 4 y, \dots) ,
\]
where we put $x_{p_n} = n y\mod \bZ^d$. By Theorem \ref{thm:jarnik}, we can 
arrange for $\omega_0(y) = w$ and $\omega_{d-1}(y) = d w + d - 1$. 

\begin{theorem}
The sequence $\bx$ is equidistributed in $(\bR/\bZ)^d$, with discrepancy 
decaying as 
\[
	\disc(\bx^N) \ll N^{-\frac{1}{d w+d} + \epsilon} 
\]
and for which 
\[
	\disc(\bx^N) = \Omega\left(N^{-\frac{d}{w} - \epsilon}\right) .
\]
However, for any $f\in C^\infty((\bR/\bZ)^d)$ with $\widehat f(0)=0$, the 
strange Dirichlet series  $L_f(\bx,s)$ satisfies the Riemann Hypothesis. 
\end{theorem}

Let's apply this theorem to abelian varieties with complex multiplication. Let 
$K/\bQ$ be a finite Galois extension, $A_{/K}$ an abelian variety with complex 
multiplication by $F$, defined over $K$. Let $\fa = \lie(A)$ be the Lie algebra 
of $A$; this is a $K$-vector space with natural $F$-action. The determinant of 
this action gives a homomorphism 
$\det_\fa\colon \R_{K/\bQ}\Gm \to \R_{F/\bQ}\Gm$; the image of this map is 
the motivic Galois group $G_A$ \cite{yu-2015}. Let 
$\N_{F/\bQ}\colon \R_{F/\bQ}\Gm \to \Gm$ be the norm map; then 
$G_A^1 = \im(\det_\fa)^{\N_{F/\bQ} = 1}$, and $\ST(A)$ is a maximal compact 
subgroup of $G_A^1(\bC)$. Every unitary representation of $\ST(A)$ is the 
restriction of a character of $G_A$; in fact it is the restriction of a 
character of $\R_{F/\bQ} \Gm$. 

Fix a rational prime $l$; then the Galois representation associated to $A$ is 
$\rho_l\colon G_\bQ \to G_A(\bQ_l)\subset (F\otimes\bQ_l)^\times$. For 
$\sigma\colon F\hookrightarrow \bC$, also 
write $\sigma$ for the corresponding character of $\R_{F/\bQ}\Gm$. Shimura and 
Taniyama proved \cite{serre-tate-1968} that there exists a Hecke character 
$\chi_\sigma\colon \bA_K^\times / K^\times \to \bC^\times$ such that 
$L^\alg(\sigma\circ \rho_l,s) = L(s,\chi_\sigma)$. The analytic $L$-function of 
$A$ is normalized as $L(A,s) = \prod_\sigma L^\alg(\sigma\circ \rho_l,s+1/2)$, 
so for $\omega_\sigma = \chi_\sigma \|\cdot\|^{-1/2}$, we have 
\[
	L(A,s) = \prod_{\sigma\colon F\hookrightarrow \bC} L(s,\omega_\sigma) .
\]
A character $r$ of $\R_{F/\bQ} \Gm$ is of the form $r = \sum a_\sigma \sigma$ 
with $a\in \bZ$. Write $\omega_r = \prod \omega_\sigma^{a_\sigma}$. Then 
\[
	L(r_\ast \rho_l,s) = L(s,\omega_r) .
\]
The Riemann Hypothesis holds for all $L(r_\ast \rho_l,s)$ if and only if it 
holds for each $L(\sigma \circ \rho_l,s) = L(s,\omega_\sigma)$. Since we 
already know analytic continuation of $\log L(s,\omega_\sigma)$ past 
$\Re = 1$, the Sato--Tate conjecture holds for $A$. However, the above theorem 
shows that even if each $L(r_\ast \rho_l,s)$ satisfies the Riemann Hypothesis, 
we may not immediately conclude that the Akiyama--Tanigawa conjecture holds for 
$A$. 
