% !TEX root = main.tex

\chapter{Introduction}





\section{Motivation from classical analytic number theory}

We start with a problem near and dear to every number theorist's heart: 
counting prime numbers. As usual, let $\pi(x)$ be the number of rational 
primes $\leqslant x$ and $\Li(x) = \int_2^x \frac{\dd t}{\log t}$ be the 
logarithmic integral. For any $x$, we have the normalized empirical measure 
\[
	P_x = \frac{1}{\pi(x)} \sum_{p\leqslant x} \delta_{p/x} ,
\]
which is supported on the unit interval $[0,1]$. The prime number theorem 
tells us that as $x\to \infty$, these empirical measures approach the 
``true'' measure $L_x = \frac{\Li(t x)}{\Li(x)}\, \dd t$. This is proven via 
demonstrating analytic properties of the Riemann zeta function. 

\begin{theorem}
We have $P_x \to L_x$ (in the weak sense) if and only if $\zeta(s)$ admits 
a meromorphic continuation past $\{\Re =1\}$, with at most a simple pole at 
$s=1$. 
\end{theorem}

Since $\zeta(s)$ has the desired properties, the prime number theorem is true. 
It is natural to try to quantify the rate of converge of $P_x$ to $L_x$. One 
easy way to do this is via the discrepancy 
\[
	\disc(P_x,L_x) 
		= \sup_{t\in [0,1]} \left| P_x[0,t] - L_x[0,t]\right|
		= \sup_{t\in [0,1]} \left| \frac{\pi(t x)}{\pi(x)} - \frac{\int_2^{tx} \frac{\dd s}{\log s}}{\int_2^x \frac{\dd s}{\log s}}\right| .
\]
Numerical experiments suggest that 
$\disc(P_x,L_x) \ll x^{-\frac 1 2+\epsilon}$, and in fact we have the following 
result. 

\begin{theorem}
We have $\disc(P_x,L_x) \ll x^{-\frac 1 2+\epsilon}$ if and only if the 
Riemann Hypothesis holds. 
\end{theorem}

Of course here, neither side is known for certain to be true! This discussion 
finds a natural generalization in Artin $L$-functions. 

Let $K/\bQ$ be a finite Galois extension with group $G=\Gal(K/\bQ)$. For any 
irreducible representation $\rho\colon G\to \GL_d(\bC)$, there is a 
corresponding $L$-function defined as 
\[
	L(\rho,s) = \prod_p \frac{1}{\det(1-\rho(\frob_p) p^{-s})} ,
\]
where we tacitly omit those primes $p$ at which $\rho$ is ramified. Given a 
cutoff $x$, there is a natural empirical measure 
\[
	P_x = \frac{1}{\pi(x)} \sum_{p\leqslant x} \delta_{\frob_p} ,
\]
where $\frob_p$ is a conjugacy class in $G$. Let 
\[
	\disc(P_x) = \sup_{c\in G^\natural} \left| P_x(c) - \frac{1}{\# G}\right|,
\]
where $G^\natural$ is the set of conjugacy classes in $G$. 

\begin{theorem}
We have convergence $P_x \to L$ if and only if $L(\rho,s)$ admits analytic 
continuation past $\{\Re =1\}$ for all nontrivial $\rho$. 
\end{theorem}

Both sides of this equivalence are true, and known as the Chebotarev density 
theorem. Moreover, there is a version of the strong Prime Number Theorem in 
this context. 

\begin{theorem}
We have $\disc(P_x) \ll x^{-\frac 1 2+\epsilon}$ if and only if $L(\rho,s)$ 
satisfies the Riemann Hypothesis for all nontrivial $\rho$. 
\end{theorem}





\section{Discrepancy and Riemann Hypothesis for elliptic curves}

Let's start with something basic, an elliptic curve $E_{/\bQ}$. For any 
prime $l$, we have the Tate module of $E$, written $\tate_l E$. This is a 
rank-$2$ $\bZ_l$-module with continuous $G_\bQ$-action, so it induces a 
continuous representation 
\[
	\rho_{E,l} \colon G_\bQ \to \GL_2(\bZ_l) .
\]
It is known (citation?) that the quantities $a_p(E) = \tr \rho_l(\frob_p)$ lie 
in $\bZ$ and satisfy the Hasse bound 
\[
	|a_p(E)| \leqslant 2\sqrt p .
\]
Thus we can define, for each prime $p$, the corresponding Satake parameter for 
$E$. 
\[
	\theta_p(E) = \cos^{-1}\left(\frac{a_p(E)}{2\sqrt p}\right) \in [0,\pi) .
\]
The Satake parameters are packaged into an $L$-function as follows:
\[
	L^\an(E,s) = \prod_p \frac{1}{(1 - e^{i \theta_p(E)} p^{-s})(1- e^{-i \theta_p(E)} p^{-s})} .
\]
More generally we have, for each $k\geqslant 1$, the $k$-th symmetric power 
$L$-function 
\[
	L^\an(\sym^k E, s) = \prod_p \prod_{j=0}^k \frac{1}{1 - e^{i (k - 2j) \theta_p(E)} p^{-s}} .
\]

Numerical experiments suggest that the Satake parameters are distributed with 
respect to the Sato--Tate distribution 
$\ST = \frac{2}{\pi} \sin^2\theta\, \dd\theta$. The ``goodness of fit'' of the 
Satake parameters to the Sato--Tate distribution is quantified by the 
\emph{discrepancy}:
\[
	\disc^\star(\{\theta_p(E)\}_{p\leqslant X}, \ST) = \sup_{x\in [0,\pi]} \left| \frac{\#\{p\leqslant X : \theta_p(E)\in [0,x)\}}{\pi(X)} - \int_0^x \, \dd \ST\right| .
\]
The decay of the discrepancy is closely related to the analytic properties of 
the $L(\sym^k E,s)$. First, here is the famous Sato--Tate conjecture (now a 
theorem) in the language we have defined. 

\begin{theorem}[Sato--Tate conjecture]
$\disc^\star(\{\theta_p(E)\}_{p\leqslant X}, \ST) \to 0$.  
\end{theorem}

\begin{theorem}
The Sato--Tate conjecture for $E$ holds if and only if each of the functions 
$L(\sym^k E,s)$ have analytic continuation past $\Re s=1$. 
\end{theorem}

The stunning recent proof of the Sato--Tate conjecture (citation) in fact 
showed that the functions $L(\sym^k E,s)$ were potentially automorphic, which 
gives analytic continuation. 

There is an analogy between the above equivalence and classical analytic number 
theory. Let $K/\bQ$ be a finite Galois extension, and 
$\rho\colon \Gal(K/\bQ) \to \GL_n(\bC)$ an irreducible representation. Recall 
the Artin $L$-function is 
\[
	L(\rho,s) = \prod_p \frac{1}{1-\tr \rho(\frob_p) p^{-s}} .
\]
Let $\Gal(K/\bQ)^\natural$ be the set of conjugacy classes in $\Gal(K/\bQ)$. 
The analogue of discrepancy here is: 
\[
	\disc(\{\frob_p\}_{p\leqslant X}) = \sup_{c\in \Gal(K/\bQ)^\natural} \left| \frac{\# \{p\leqslant X : \rho(\frob_p) \in c\}}{\pi(X)} - \frac{1}{\# \Gal(K/\bQ)^\natural}\right| .
\]

\begin{theorem}
The ``discrepancy'' $\disc(\{\frob_p\}_{p\leqslant X})\to 0$ if and only 
if $L(\rho,s)$ has analytic continuation past $\Re s=1$ for all non-trivial 
irreducible representations $\rho$ of $\Gal(K/\bQ)$. 
\end{theorem}

In the case of Artin $L$-functions, we know moreover that 

\begin{theorem}
The ``discrepancy'' satisfies the bound 
$\disc(\{\frob_p\}_{p\leqslant X}) \ll X^{-1/2+\epsilon}$ if and only if 
$L(\rho,s)$ satisfies the Riemann Hypothesis for all non-trivial irreducible 
representation $\rho$ of $\Gal(K/\bQ)$. 
\end{theorem}

In this context, the ``Riemann Hypothesis'' for $L(\rho,s)$ means exactly that 
$\log L(\rho,s)$ has analytic continuation to $\Re s=1/2$. 

The connection between the Riemann Hypothesis and ``strong Sato--Tate'' 
generalizes to elliptic curves and more general motives. For the moment, we 
stick to elliptic curves. In this case, ``strong Sato--Tate'' was conjectured 
by Akiyama--Tanigawa. More precisely, 

Conjecture:

Let $E_{/\bQ}$ be a non-CM elliptic curve. Then 
$\disc^\star(\{\theta_p(E)\}_{p\leqslant X}, \ST) \ll X^{-1/2+\epsilon}$. 


Moreover, one side of the equivalence ``Riemann Hypothesis $\Leftrightarrow$ 
strong Sato--Tate'' is known. 

\begin{theorem}
Let $E_{/\bQ}$ be an elliptic curve. If the Akiyama--Tanigawa conjecture for 
$E$ holds, then all $L(\sym^k E, s)$ satisfy the Riemann Hypothesis. 
\end{theorem}

It is natural to assume that the converse to this theorem holds. However (and 
that is the main point of this thesis) it does not! In this thesis, I construct 
a range of counterexamples to the implication ``strong Sato--Tate implies 
Riemann,'' and explore why the two are equivalent for Artin $L$-functions. 

I also provide computational evidence for the Akiyama--Tanigawa conjecture 
(for elliptic curves and also generic abelian $2$-folds). 

Similar work: \cite{pande-2011}. 

To-do: conjectural framework? Can I find the rank from $\{sign(a_p)\}$?

See \cite{mazur-2008} for a nice discussion. 
