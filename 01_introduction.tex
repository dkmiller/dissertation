% !TEX root = main.tex

\chapter{Introduction}





Let's start with something basic, an elliptic curve $E_{/\bQ}$. For any 
prime $l$, we have the Tate module of $E$, written $\tate_l E$. This is a 
rank-$2$ $\bZ_l$-module with continuous $G_\bQ$-action, so it induces a 
continuous representation 
\[
	\rho_{E,l} \colon G_\bQ \to \GL_2(\bZ_l) .
\]
It is known (citation?) that the quantities $a_p(E) = \tr \rho_l(\frob_p)$ lie 
in $\bZ$ and satisfy the Hasse bound 
\[
	|a_p(E)| \leqslant 2\sqrt p .
\]
Thus we can define, for each prime $p$, the corresponding Satake parameter for 
$E$. 
\[
	\theta_p(E) = \cos^{-1}\left(\frac{a_p(E)}{2\sqrt p}\right) \in [0,\pi) .
\]
The Satake parameters are packaged into an $L$-function as follows:
\[
	L^\an(E,s) = \prod_p \frac{1}{(1 - e^{i \theta_p(E)} p^{-s})(1- e^{-i \theta_p(E)} p^{-s})} .
\]
More generally we have, for each $k\geqslant 1$, the $k$-th symmetric power 
$L$-function 
\[
	L^\an(\sym^k E, s) = \prod_p \prod_{j=0}^k \frac{1}{1 - e^{i (k - 2j) \theta_p(E)} p^{-s}} .
\]

Numerical experiments suggest that the Satake parameters are distributed with 
respect to the Sato--Tate distribution 
$\ST = \frac{2}{\pi} \sin^2\theta\, \dd\theta$. The ``goodness of fit'' of the 
Satake parameters to the Sato--Tate distribution is quantified by the 
\emph{discrepancy}:
\[
	\disc^\star(\{\theta_p(E)\}_{p\leqslant X}, \ST) = \sup_{x\in [0,\pi]} \left| \frac{\#\{p\leqslant X : \theta_p(E)\in [0,x)\}}{\pi(X)} - \int_0^x \, \dd \ST\right| .
\]
The decay of the discrepancy is closely related to the analytic properties of 
the $L(\sym^k E,s)$. First, here is the famous Sato--Tate conjecture (now a 
theorem) in the language we have defined. 

\begin{theorem}[Sato--Tate conjecture]
$\disc^\star(\{\theta_p(E)\}_{p\leqslant X}, \ST) \to 0$.  
\end{theorem}

\begin{theorem}
The Sato--Tate conjecture for $E$ holds if and only if each of the functions 
$L(\sym^k E,s)$ have analytic continuation past $\Re s=1$. 
\end{theorem}

The stunning recent proof of the Sato--Tate conjecture (citation) in fact 
showed that the functions $L(\sym^k E,s)$ were potentially automorphic, which 
gives analytic continuation. 

There is an analogy between the above equivalence and classical analytic number 
theory. Let $K/\bQ$ be a finite Galois extension, and 
$\rho\colon \Gal(K/\bQ) \to \GL_n(\bC)$ an irreducible representation. Recall 
the Artin $L$-function is 
\[
	L(\rho,s) = \prod_p \frac{1}{1-\tr \rho(\frob_p) p^{-s}} .
\]
Let $\Gal(K/\bQ)^\natural$ be the set of conjugacy classes in $\Gal(K/\bQ)$. 
The analogue of discrepancy here is: 
\[
	\disc(\{\frob_p\}_{p\leqslant X}) = \sup_{c\in \Gal(K/\bQ)^\natural} \left| \frac{\# \{p\leqslant X : \rho(\frob_p) \in c\}}{\pi(X)} - \frac{1}{\# \Gal(K/\bQ)^\natural}\right| .
\]

\begin{theorem}
The ``discrepancy'' $\disc(\{\frob_p\}_{p\leqslant X})\to 0$ if and only 
if $L(\rho,s)$ has analytic continuation past $\Re s=1$ for all non-trivial 
irreducible representations $\rho$ of $\Gal(K/\bQ)$. 
\end{theorem}

In the case of Artin $L$-functions, we know moreover that 

\begin{theorem}
The ``discrepancy'' satisfies the bound 
$\disc(\{\frob_p\}_{p\leqslant X}) \ll X^{-1/2+\epsilon}$ if and only if 
$L(\rho,s)$ satisfies the Riemann Hypothesis for all non-trivial irreducible 
representation $\rho$ of $\Gal(K/\bQ)$. 
\end{theorem}

In this context, the ``Riemann Hypothesis'' for $L(\rho,s)$ means exactly that 
$\log L(\rho,s)$ has analytic continuation to $\Re s=1/2$. 

The connection between the Riemann Hypothesis and ``strong Sato--Tate'' 
generalizes to elliptic curves and more general motives. For the moment, we 
stick to elliptic curves. In this case, ``strong Sato--Tate'' was conjectured 
by Akiyama--Tanigawa. More precisely, 

Conjecture:

Let $E_{/\bQ}$ be a non-CM elliptic curve. Then 
$\disc^\star(\{\theta_p(E)\}_{p\leqslant X}, \ST) \ll X^{-1/2+\epsilon}$. 


Moreover, one side of the equivalence ``Riemann Hypothesis $\Leftrightarrow$ 
strong Sato--Tate'' is known. 

\begin{theorem}
Let $E_{/\bQ}$ be an elliptic curve. If the Akiyama--Tanigawa conjecture for 
$E$ holds, then all $L(\sym^k E, s)$ satisfy the Riemann Hypothesis. 
\end{theorem}

It is natural to assume that the converse to this theorem holds. However (and 
that is the main point of this thesis) it does not! In this thesis, I construct 
a range of counterexamples to the implication ``strong Sato--Tate implies 
Riemann,'' and explore why the two are equivalent for Artin $L$-functions. 

I also provide computational evidence for the Akiyama--Tanigawa conjecture 
(for elliptic curves and also generic abelian $2$-folds). 

Similar work: \cite{pande-2011}. 

% TODO

Claim: the Riemann Hypothesis is equivalent to the following description of the 
distribution of prime numbers. For a real number $x$, let 
\[
	P_x = \frac{1}{\pi(x)} \sum_{p\leqslant x} \delta_{p/x} .
\]
This is a discrete probability measure supported on $[0,1]$. Moreover, let 
$L_x$ be the continuous probability measure with cdf 
\[
	L_x[0,t] = \frac{\Li(tx)}{\Li(x)} .
\]
Then $\disc(P_x,L_x) \ll x^{-\frac 1 2+\epsilon}$. I'm pretty sure that this 
statement is equivalent to $|\pi(x)-\Li(x)| \ll x^{\frac 1 2+\epsilon}$, which 
is already known to be equivalent to RH. 
