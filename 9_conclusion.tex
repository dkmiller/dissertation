% !TEX root = thesis.tex

\chapter{Concluding remarks and future directions}





\section{Fake modular forms}

The Galois representations of Theorem \ref{thm:bad-Galois} have ``fake modular 
forms'' associated to them. Namely, there is a representation of $\GL_2(\bA)$ 
with the specified Satake parameters at each prime (for now, set $\theta_p = 0$ 
at ramified primes). It is natural to ask if these ``fake modular forms'' have 
any interesting properties. For example, we know that all their odd symmetric 
powers satisfy the Riemann Hypothesis. The author is unaware of any further 
results (say about analytic continuation or functional equation) concerning 
these fake modular forms. 





\section{Dense free subgroups of compact semisimple groups}

Let $G$ be a compact semisimple Lie group, for example $\SU(2)$. By 
\cite{breuillard-gelander-2003}, $G$ contains a dense free subgroup 
$\Gamma = \langle \gamma_1,\gamma_2\rangle$. We will now follow the argument of 
\cite{arnold-krylov-1963} to hint at how $\Gamma$ may yield equidistributed 
sequences with ``bad'' discrepancy and small character sums. 

Given an integer $N$, let $B_N$ be the ``closed ball of size $N$'' in $\Gamma$, 
that is the set of products $\gamma_{\sigma(1)} \dots \gamma_{\sigma(n)}$, 
where $n\leqslant N$ and $\sigma\colon \{1,\dots,n\} \to \{1,2\}$ is a 
function. We will write $\sigma\colon [n] \to [2]$ in this case. Given an 
irreducible unitary representation $\rho\in \widehat G$, we wish to control 
the behavior of $\sum_{\gamma\in B_N} \tr\rho(\gamma)$, ideally to show an 
estimate of the form 
\[
	\left| \sum_{\gamma\in B_N} \tr\rho(\gamma)\right| \ll \left(\# B_N\right)^{\frac 1 2 + \epsilon} .
\]
In fact, $\# B_N = \sum_{n=0}^N 2^n = 2^{N+1} - 1$. We can encode these sums 
in terms of convolutions of a measure as follows. Let $\mu$ be the measure 
$\delta_{\gamma_1^{-1}} + \delta_{\gamma_2^{-1}}$ on $G$. 
If $\rho$ is any unitary representation (not necessarily irreducible or even 
finite-dimensional) then $\mu$ acts on $\rho$ via 
$\rho(\mu) \int \rho\, \dd\mu$. So, if $\rho = L^2(G)$ via the left regular 
representation, then $(\mu\cdot f)(x) = f(\gamma_1 x) + f(\gamma_2 x)$, while 
if $\rho\in \widehat G$ and $v\in \rho$, then 
$\mu\cdot v = \rho(\gamma_1) v + \rho(\gamma_2) v$. Note that 
\[
	\mu^{\ast n} = \sum_{\sigma\colon [n] \to [2]} \delta_{\gamma_{\sigma(1)} \dots \gamma_{\sigma(n)}} .
\]
This tells us that 
$\sum_{\gamma\in B_N} f(\gamma) = \sum_{n\leqslant N} \mu^{\ast n}(f)$. So 
we really only need to study how $\mu$ and its powers act on the functions 
$\tr\rho$, $\rho\in \widehat G$. 

First note that $\tr\rho$ generates a subrepresentation of $L^2(G)$ which is 
isomorphic to $\rho$. On that representation, we claim that $\mu$ is 
invertible, hence 
$\sum_{n=0}^N \mu^{\ast n} = (\mu^{\ast(N+1)} - 1)(\mu - 1)^{-1}$.It 
follows that 
$\| \sum_{n=0}^N \mu^{\ast n}\| \leqslant \frac{\|\mu\|^{N+1}}{\|\mu - 1\|}$, 

Note that $\|\mu\|^{N+1} \leqslant 2^{(N+1)\alpha}$ if and only if 
$\|\mu\| \leqslant 2^\alpha$. In other words, to get the Riemann Hypothesis for 
$L$-functions coming from $\Gamma$, we need $\|\mu\| \leqslant \sqrt 2$. 
If $v\in \rho$ has norm $1$, then 
\begin{align*}
	\|\rho(\mu) v\|^2
		&= \langle \rho(\gamma_1^{-1}) v + \rho(\gamma_2^{-1}) v, \rho(\gamma_1^{-1}) v + \rho(\gamma_2^{-1}) v\rangle \\
%		&= 2 \|v\|^2 + 2 \Re\langle \rho(\gamma_1^{-1}) v,\rho(\gamma_2^{-1}) v\rangle \\
		&= 2\|v\|^2 + 2\Re \langle \rho(\gamma_2 \gamma_1^{-1}) v,v\rangle .
\end{align*}
So, we want $\Re \langle \rho(\gamma_2 \gamma_1^{-1}) v,v\rangle \leqslant 0$ 
for all irreducible $\rho$. Sadly, even for $\SU(2)$, this is not possible. 

Write $\gamma = \gamma_2 \gamma_1^{-1}$, then the identity 
$\langle \rho(\gamma)\rho(\delta)v,\rho(\delta)v\rangle = \langle \rho(\delta^{-1} \gamma\delta) v,v\rangle$ tells us that we can restrict our search to 
$\gamma$ of the form $\smat{a}{}{}{\overline a}$ with $|a|=1$. Now 
\[
	\left\langle \smat{a}{}{}{\overline a} \svec{u}{v}, \svec{u}{v}\right\rangle = \Re(a) ,
\]
which appears to be promising. But a similar computation with $\sym^2$ shows 
that one can always get $\langle \sym^2 \gamma  v,v\rangle = 1$, so the above 
approach fails. 

There may be alternative ways of bounding the sums $\sum\mu^{\ast n}(\tr\rho)$, 
but we do not investigate them here. 
