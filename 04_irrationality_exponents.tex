% !TEX root = main.tex

\chapter{Irrationality exponents}





\section{Definitions and first results}

We follow the notation of \cite{laurent-2009}. Let 
$x=(x_1,\dots,x_d)\in \bR^d$ be such that the $x_i$ are $\bQ$-linearly 
independent. 

\begin{definition}\label{def:approx-exp}
Let $\omega_0(x)$ (resp.~$\omega_{d-1}(x)$) be the supremum of the set of real 
numbers $\omega$ for which there exist infinitely many 
$m=(m_0,\dots,m_d)\in \bZ^{r+1}$ such that 
\begin{align*}
	\max\{|m_0 x_i - m_i|\} 
		&\leqslant \|m\|_\infty^{-\omega}  \qquad\text{(resp.} \\
	|m_0 + m_1 x_1 + \dots + m_r x_r| 
		&\leqslant \|m\|_\infty^{-\omega} \text{).}
\end{align*}
\end{definition}

These two quantities are related by Khintchine's Transference Principle, namely 
\[
	\frac{\omega_{d-1}(x)}{(d-1) \omega_{d-1}(x)+d} \leqslant \omega_(x) \leqslant \frac{\omega_{d-1}(x)-d+1}{d} .
\]
Moreover, these inequalities are sharp in a very strong sense. 

\begin{theorem}[Jarn\'ik]
Let $w\geqslant 1/d$. Then there exists $x\in \bR^d$ such that $\omega_0(x)=w$ 
and $\omega_{d-1}(x) = d x+d-1$. 
\end{theorem}

% TODO
\begin{theorem}
When $d=1$, relate $\omega_0(x)$ to the irrationality measure. 
\end{theorem}
\begin{proof}
Recall that the irrationality measure $\mu(x)$ is the infimum of the set of 
positive reals $\mu$ such that 
\[
	0 < \left| x - \frac p q\right| < q^{-\mu} 
\]
has only finitely many solutions $p/q$ with $p,q$ integers. 
\end{proof}

Mention Roth's theorem\ldots generalize to higher dimension?

Now given $x\in \bR^d$, we write $d(x,\bZ^d)=\min_{m\in \bZ^d} |x-m|$, where 
$|\cdot|$ is any fixed norm on $\bR^d$. Note that $d(x,\bZ^d)=0$ if and only if 
$x\in \bZ^d$. 

\begin{lemma}
Let $x\in \bR^d$ with $\|x\|_\infty\leqslant 1$ and $\omega_0(x)$ 
(resp.~$\omega_{d-1}(x)$) finite. Then 
\begin{align*}
	\frac{1}{d(n x,\bZ^d)} 
		&\ll |n|^{\omega_0(x)+\epsilon}\qquad \text{ (resp.} \\
	\frac{1}{d(\langle m,x\rangle, \bZ)} 
		&\ll |m|^{\omega_{d-1}(x)+\epsilon} \qquad\text{ for $m\in \bZ^d$).}
\end{align*}
\end{lemma}
\begin{proof}
Let $\epsilon>0$. Then there are only finitely many $n\in \bZ$ 
(resp.~$m\in \bZ^d$) such that the inequalities in Definition 
\ref{def:approx-exp} hold with $\omega_0(x)+\epsilon$ 
(resp.~$\omega_{d-1}(x)+\epsilon$). In other words, there exist constants 
$C_0, C_{d-1}>0$ such that 
\begin{align*}
	\max\{|m_0 x_i - m_i|\} 
		&\geqslant C_0 \|m\|_\infty^{-\omega_0(x)-\epsilon} ,\\
	|m_0 + m_1 x_1 + \cdots + m_d x_d| 
		&\geqslant C_{d-1} \|m\|_\infty^{-\omega_{d-1}(x)-\epsilon} 
\end{align*}
for all $m\ne 0$. 

Start with the first inequality in the statement of the result, where up to 
constant, we may assume that $|\cdot| = \|\cdot\|_\infty$ in the definition of 
$d(n x,\bZ^d)$. Let $m=(m_1,\dots,m_d)$ be the lattice point achieving the 
minimum $|n x - m|$. Then we know that 
\[
	d(n x,\bZ^d) \geqslant C_0 \|(m_1,\dots,m_d)\|_\infty^{-\omega_0(x)-\epsilon} .
\]
Moreover, since $|n x-m|<1$, there exists a constant $C_0'$ such that 
\[
	d(n x,\bZ^d) \geqslant C_0' |n|^{-\omega_0(x)-\epsilon} .
\]
It follows that 
\[
	\frac{1}{d(n x,\bZ^d)} \ll |n|^{\omega_0(x)+\epsilon} ,
\]
the implied constant depending on $x$, $\epsilon$, and the choice of norm 
$|\cdot|$. 

Now let's consider the second inequality in the statement of the result. Note 
that $d(m_1 x_1 + \dots + m_d x_d,\bZ) = |m_0 + m_1x_1 + \cdots + m_d x_d$ for 
some $m_0$ with $|m_0| \leqslant \|(m_1,\dots,m_d)\|_2 \|x\|_2 + 1$. Thus 
$\|(m_1,\dots,m_d)\|_\infty \ll \|x\|_2 \|(m_1,\dots,m_d)\|_2$, which gives us 
\[
	d(m_1 x_1 + \cdots + m_d x_d,\bZ) \geqslant C_{d-1} \|(m_1,\dots,m_d)\|_2^{-\omega_{d-1}(x)-\epsilon} .
\]
This implies 
\[
	\frac{1}{d(\langle m,x\rangle,\bZ)} \ll |m|^{\omega_{r-1}(x)+\epsilon} ,
\]
the implied constant depending on $x$, $\epsilon$, and the choice of $|\cdot|$. 
\end{proof}





\section{Irrationality exponents and discrepancy}

Let $x\in \bR^d$ with $x_1,\dots,x_d$ linearly independent over $\bQ$. We wish 
to control the discrepancy of the sequence $\{x,2x,3x,\dots\}$ in 
$(\bR/\bZ)^d$. 

\begin{theorem}
Let $\bx=(x,2x,3x,\dots)$ in $(\bR/\bZ)^d$. Then 
\[
	\disc(\bx^N) \ll \frac{\log N}{N^{\frac{1}{d ? + 1}} ?}
\]
\end{theorem}
\begin{proof}
We follow \cite[Ch.~2, Ex.~3.15, 16, 17]{kuipers-niederreiter-1974}. 
\end{proof}





\section{Irrationality exponents and bounding sums}

d
