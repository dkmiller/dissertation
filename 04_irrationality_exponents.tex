% !TEX root = main.tex

\chapter{Irrationality exponents}





\section{Definitions and first results}

We follow the notation of \cite{laurent-2009}. Let 
$x=(x_1,\dots,x_d)\in \bR^d$ be such that the $x_i$ are $\bQ$-linearly 
independent. 

\begin{definition}\label{def:approx-exp}
Let $\omega_0(x)$ (resp.~$\omega_{d-1}(x)$) be the supremum of the set of real 
numbers $\omega$ for which there exist infinitely many 
$m=(m_0,\dots,m_d)\in \bZ^{r+1}$ such that 
\begin{align*}
	\max\{|m_0 x_i - m_i|\} 
		&\leqslant \|m\|_\infty^{-\omega}  \qquad\text{(resp.} \\
	|m_0 + m_1 x_1 + \dots + m_r x_r| 
		&\leqslant \|m\|_\infty^{-\omega} \text{).}
\end{align*}
\end{definition}

These two quantities are related by Khintchine's Transference Principle, namely 
\[
	\frac{\omega_{d-1}(x)}{(d-1) \omega_{d-1}(x)+d} \leqslant \omega_(x) \leqslant \frac{\omega_{d-1}(x)-d+1}{d} .
\]
Moreover, these inequalities are sharp in a very strong sense. 

\begin{theorem}[Jarn\'ik]\label{thm:jarnik}
Let $w\geqslant 1/d$. Then there exists $x\in \bR^d$ such that $\omega_0(x)=w$ 
and $\omega_{d-1}(x) = d w+d-1$. 
\end{theorem}
\begin{proof}
% TODO
Do this. 
\end{proof}

% TODO
\begin{theorem}
When $d=1$, relate $\omega_0(x)$ to the irrationality measure. 
\end{theorem}
\begin{proof}
Recall that the irrationality measure $\mu(x)$ is the infimum of the set of 
positive reals $\mu$ such that 
\[
	0 < \left| x - \frac p q\right| < q^{-\mu} 
\]
has only finitely many solutions $p/q$ with $p,q$ integers. 
\end{proof}

Mention Roth's theorem\ldots generalize to higher dimension?

Now given $x\in \bR^d$, we write $d(x,\bZ^d)=\min_{m\in \bZ^d} |x-m|$, where 
$|\cdot|$ is any fixed norm on $\bR^d$. Note that $d(x,\bZ^d)=0$ if and only if 
$x\in \bZ^d$. 

\begin{lemma}\label{lem:bound-distance}
Let $x\in \bR^d$ with $\|x\|_\infty\leqslant 1$ and $\omega_0(x)$ 
(resp.~$\omega_{d-1}(x)$) finite. Then 
\begin{align*}
	\frac{1}{d(n x,\bZ^d)} 
		&\ll |n|^{\omega_0(x)+\epsilon}\qquad \text{ (resp.} \\
	\frac{1}{d(\langle m,x\rangle, \bZ)} 
		&\ll |m|^{\omega_{d-1}(x)+\epsilon} \qquad\text{ for $m\in \bZ^d$).}
\end{align*}
\end{lemma}
\begin{proof}
Let $\epsilon>0$. Then there are only finitely many $n\in \bZ$ 
(resp.~$m\in \bZ^d$) such that the inequalities in Definition 
\ref{def:approx-exp} hold with $\omega_0(x)+\epsilon$ 
(resp.~$\omega_{d-1}(x)+\epsilon$). In other words, there exist constants 
$C_0, C_{d-1}>0$ such that 
\begin{align*}
	\max\{|m_0 x_i - m_i|\} 
		&\geqslant C_0 \|m\|_\infty^{-\omega_0(x)-\epsilon} ,\\
	|m_0 + m_1 x_1 + \cdots + m_d x_d| 
		&\geqslant C_{d-1} \|m\|_\infty^{-\omega_{d-1}(x)-\epsilon} 
\end{align*}
for all $m\ne 0$. 

Start with the first inequality in the statement of the result, where up to 
constant, we may assume that $|\cdot| = \|\cdot\|_\infty$ in the definition of 
$d(n x,\bZ^d)$. Let $m=(m_1,\dots,m_d)$ be the lattice point achieving the 
minimum $|n x - m|$. Then we know that 
\[
	d(n x,\bZ^d) \geqslant C_0 \|(m_1,\dots,m_d)\|_\infty^{-\omega_0(x)-\epsilon} .
\]
Moreover, since $|n x-m|<1$, there exists a constant $C_0'$ such that 
\[
	d(n x,\bZ^d) \geqslant C_0' |n|^{-\omega_0(x)-\epsilon} .
\]
It follows that 
\[
	\frac{1}{d(n x,\bZ^d)} \ll |n|^{\omega_0(x)+\epsilon} ,
\]
the implied constant depending on $x$, $\epsilon$, and the choice of norm 
$|\cdot|$. 

Now let's consider the second inequality in the statement of the result. Note 
that $d(m_1 x_1 + \dots + m_d x_d,\bZ) = |m_0 + m_1x_1 + \cdots + m_d x_d$ for 
some $m_0$ with $|m_0| \leqslant \|(m_1,\dots,m_d)\|_2 \|x\|_2 + 1$. Thus 
$\|(m_1,\dots,m_d)\|_\infty \ll \|x\|_2 \|(m_1,\dots,m_d)\|_2$, which gives us 
\[
	d(m_1 x_1 + \cdots + m_d x_d,\bZ) \geqslant C_{d-1} \|(m_1,\dots,m_d)\|_2^{-\omega_{d-1}(x)-\epsilon} .
\]
This implies 
\[
	\frac{1}{d(\langle m,x\rangle,\bZ)} \ll |m|^{\omega_{r-1}(x)+\epsilon} ,
\]
the implied constant depending on $x$, $\epsilon$, and the choice of $|\cdot|$. 
\end{proof}





\section{Irrationality exponents and discrepancy}

Let $x\in \bR^d$ with $x_1,\dots,x_d$ linearly independent over $\bQ$. We wish 
to control the discrepancy of the sequence $\{x,2x,3x,\dots\}$ in 
$(\bR/\bZ)^d$. 

\begin{theorem}[Erd\"os--Tur\'an--Koksma]
Let $\bx$ be a sequence in $\bR^d$ and $h$ an arbitrary integer. Then 
\[
	\disc(\bx^N) \ll \frac 1 h + \sum_{0\leqslant \|m\|_\infty \leqslant h} \frac{1}{r(m)} \left| \frac 1 N \sum_{n\leqslant N} e^{2\pi i \langle m,x_n\rangle}\right| ,
\]
where the first sum ranges over $m\in \bZ^d$, 
$r(m) = \prod \max\{1,|m_i|\}$, and the implied constant depends only on $d$. 
\end{theorem}
\begin{proof}
This is \cite[Th.~1.21]{drmota-tichy-1997}. 
\end{proof}

\begin{lemma}\label{lem:bound-exp-sum}
Let $x\in \bR$. Then 
\[
	\left| \sum_{n\leqslant N} e^{2\pi i n x}\right| \ll \frac{1}{d(x, \bZ)} .
\]
\end{lemma}
\begin{proof}
We begin with an easy bound: 
\[
	\left| \sum_{n\leqslant N} e^{2\pi i n x}\right| = \frac{|e^{2\pi i (N+1) x} - 1|}{|e^{2\pi i x} - 1|} \leqslant \frac{2}{|e^{2\pi i x} - 1|} .
\]
Since $|e^{2\pi i m x} - 1| = \sqrt{2-2\cos(2\pi x)}$ and 
$\cos(2\theta) = 1-2\sin^2\theta$, we obtain 
\[
	\left|\sum_{n\leqslant N} e^{2\pi i n x}\right| \leqslant \frac{1}{|\sin(\pi x)|} .
\]
It is easy to check that $|\sin(\pi x)| \geqslant d(x,\bZ)$, whence the result.  
\end{proof}

\begin{corollary}\label{cor:bound-disc-distance}
Let $x\in (\bR/\bZ)^d$ with $(x_1,\dots,x_d)$ linearly independent over $\bQ$. 
Then for $\bx=(x,2x,3x,\dots)$, we have 
\[
	\disc(\bx^N) \ll \frac 1 h + \frac 1 N \sum_{0<\|m\|_\infty \leqslant h} \frac{1}{r(m) d(\langle m,x\rangle,\bZ)} 
\]
for any integer $h$, with the implied constant depending only on $d$. 
\end{corollary}
\begin{proof}
Apply the Erd\"os--Tur\'an--Koksma inequality and bound the exponential sums 
using Lemma \ref{lem:bound-exp-sum}. 
\end{proof}

\begin{theorem}
Let $\bx=(x,2x,3x,\dots)$ in $(\bR/\bZ)^d$. Then 
\[
	\disc(\bx^N) \ll N^{-\frac{1}{\omega_{d-1}(x)+1}+\epsilon} .
\]
\end{theorem}
\begin{proof}
Choose $\delta>0$ such that 
$\frac{1}{\omega_{d-1}(x)+1+\delta} = \frac{1}{\omega_{d-1}(x)+1} - \epsilon$. 

By Corollary \ref{cor:bound-disc-distance}, we know that 
\[
	\disc(\bx^N) \ll \frac 1 h + \frac 1 N \sum_{0<\|m\|_\infty \leqslant h} \frac{1}{r(m) d(\langle m,x\rangle,\bZ)} ,
\]
and by Lemma \ref{lem:bound-distance}, we know that 
$d(\langle m,x\rangle,\bZ)^{-1} \ll |m|^{\omega_{d-1}(x)+\delta}$. 
It follows that 
\[
	\disc(\bx^N) \ll \frac 1 h + \frac 1 N \sum_{0 < \|m\|_\infty \leqslant h} \frac{|m|^{\omega_{d-1}(x)+\delta}}{r(m)} .
\]
The only tricky part is bounding the sum. 
\begin{align*}
	\sum_{0< \|m\|_\infty \leqslant h} \frac{|m|_\infty^{\omega_{d-1}(x)+\delta}}{r(m)} 
		&\ll \int_1^h \int_1^{t_d} \dots \int_1^{t_2} \frac{t_d^{\omega_{d-1}(x)+\delta}}{t_1 \dots t_d}\, \dd t_1 \dots \dd t_d \\
		&\ll \int_1^h t^{\omega_{d-1}(x)+\delta-1}\, \dd t \prod_{j=1}^{d-1} \int_1^h \frac{\dd t}{t} \\
		&\ll (\log h)^{d-1} h^{\omega_{d-1}(x)+\delta} .
\end{align*}
It follows that 
\[
	\disc(\bx^N) \ll \frac 1 h + \frac 1 N (\log h)^{d-1} h^{\omega_{d-1}(x)+\delta} .
\]
Setting $h\approx N^{\frac{1}{1+\omega_{d-1}(x)+\delta}}$, we see that 
\[
	D(\bx^N) \ll N^{-\frac{1}{\omega_{d-1}(x)+1+\delta}} = N^{-\frac{1}{\omega_{d-1}(x)+1} + \epsilon} .
\]

For a slightly different proof of a similar result (given as a sequence of 
exercises), see  \cite[Ch.~2, Ex.~3.15, 16, 17]{kuipers-niederreiter-1974}. 
\end{proof}

\begin{theorem}
Let $x\in \bR$ be such that $x_1,\dots,x_d$ are linearly independent over 
$\bQ$, and let $\bx=(x,2x,3x,\dots)$ in $(\bR/\bZ)^d$. Then 
\[
	\disc(\bx^N) = \Omega\left(N^{-\frac{d}{\omega_0(x)}-\epsilon} \right).
\]
\end{theorem}
\begin{proof}
Here $f=\Omega(g)$ in the sense of Hardy, namely that $\limsup \frac f g>0$. We 
follow the proof of \cite[Ch.~2, Th.~3.3]{kuipers-niederreiter-1974}. Given 
$\epsilon>0$, there exists $\delta>0$ such that 
$\frac{d}{\omega_0(x)-\delta} = \frac{d}{\omega_0(x)} + \epsilon$. 

By the definition of $\omega_0(x)$, there exist infinitely many 
$(q,m_1,\dots,m_d)$ with $q>0$ such that 
\[
	\|q x - m\|_\infty \leqslant \|(q,m_1,\dots,m_d)\|_\infty^{-\omega_0(x)+\delta/2} .
\]
Since $\|(q,m_1,\dots,m_d)\|_\infty \geqslant q$, we derive the stronger 
statement that for infinitely many $q\to \infty$, there exists 
$m=(m_1,\dots,m_d)\in \bZ^d$ such that 
$\|q x-m\|_\infty \leqslant q^{-\omega_0(x)+\delta/2}$ or, equivalently, 
$|x-\frac m q| \leqslant q^{-1-\omega_0(x)+\delta/2}$. Pick such a $q$, and let 
$N=\lfloor q^{\omega_0(x)-\delta}\rfloor$. Then for each $n\leqslant N$, we 
have $\|n x - \frac n q m\|_\infty \leqslant q^{-1-\delta/2}$. Thus, for each 
$n\leqslant N$, each $n x$ is within $q^{-1-\delta/2}$ of the grid 
$\frac 1 q \bZ^d\subset (\bR/\bZ)^d$. Thus, they miss a box with side lengths 
$q^{-1} - 2 q^{-1-\delta/2}$. For $q$ sufficiently large, 
$q^{-1} - 2 q^{-1-\delta/2} \geqslant 1/2q$, so the discrepancy of $\bx^N$ is 
bounded below by $2^{-d} q^{-d}$. Since $q^{\omega_0(x)-\delta} \leqslant 2 N$, 
the discrepancy at $N$ is bounded below by 
\[
	2^{-d} \left( (2 N)^{-\frac{1}{\omega_0(x)+\delta}}\right)^{-d} 
		= 2^{-d-\frac{d}{\omega_0(x)+\delta}} N^{-\frac{d}{\omega_0(x)+\delta}}
		= 2^{-d\left(1+\frac{1}{\omega_0(x)}\right)-\epsilon} N^{-\frac{d}{\omega_0(x)}-\epsilon} .
\]
\end{proof}
