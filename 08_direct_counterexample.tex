% !TEX root = thesis.tex

\chapter{Direct counterexample}





\section{Main ideas}

For $k\geqslant 1$, let 
\[
	U_k(\theta) = \tr\sym^k\smat{e^{i\theta}}{}{}{e^{- i \theta}} = \frac{\sin((k+1)\theta)}{\sin\theta} .
\]
Then $U_k(\cos^{-1} t)$ is the $k$-th Chebyshev polynomial of the 2nd kind. 
Moreover, $\{U_k\}$ forms an orthonormal basis for 
$L^2([0,\pi],\ST) = L^2(\SU(2)^\natural)$. 

This chapter has two parts. First, for any reasonable measure $\mu$ on 
$[0,\pi]$ invariant under the same ``flip'' automorphism as the Sato--Tate 
measure, there is a sequence $\{a_p\}$ of integers satisfying the Hasse 
bound $|a_p|\leqslant 2\sqrt p$, such that for 
$\theta_p = \cos^{-1}\left(\frac{a_p}{2\sqrt p}\right)$, the discrepancy 
$\disc(\btheta^N,\mu)$ behaves like $\pi(N)^{-\alpha}$ for 
predetermined $\alpha\in (0,1/2]$, while for any odd $k$, the strange 
Dirichlet series $L_{U_k}(\btheta,s)$, which we will write 
$L(\sym^k \btheta,s)$, satisfies the Riemann Hypothesis. 

In the second part of this chapter, we associate (infinitely ramified) Galois 
representations to the fake Satake parameters above, using the techniques 
in Chapter \ref{ch:construct-Galois}. 





\section{Construction}


\begin{definition}
Let $\mu = f(t)\, \dd t$ be a good measure on $[0,\pi]$. If 
$f(t) \ll \sin(t)$, then $\mu$ is a \emph{Sato--Tate compatible measure}. 
\end{definition}

The key facts about Sato--Tate compatible measures are that $\cos_\ast\mu$ 
satisfies the hypotheses of Theorem \ref{thm:discrepancy-arbitrary}, so 
there are ``$N^{-\alpha}$-decaying van der Corput sequences'' for 
$\cos_\ast\mu$, and also that since $\cos\colon [0,\pi] \to [-1,1]$ is an 
order anti-isomorphism, we know that for any sequence $\bx$ on $[-1,1]$, there 
is equality $\disc(\bx^N,\cos_\ast\mu) = \disc(\cos^{-1}\bx^N,\mu)$. 


\begin{theorem}\label{thm:integral-a_p-alpha}
Let $\mu$ be a Sato--Tate compatible measure, and fix $\alpha\in (0,1/2)$. 
Then there exists a sequence of integers $a_p$ satisfying the Hasse bound, 
such that if we set $\theta_p = \cos^{-1}\left(\frac{a_p}{2\sqrt p}\right)$, 
then $\disc^\star(\btheta^N,\mu) = \Theta(\pi(N)^{-\alpha})$. 
\end{theorem}
\begin{proof}
Apply Theorem \ref{thm:discrepancy-arbitrary} to find a sequence $\bx$ such 
that $\disc(\bx^N,\cos_\ast \mu) = \Theta(\pi(N)^{-\alpha})$. For each prime 
$p$, there exists an integer $a_p$ such that $|a_p|\leqslant 2\sqrt p$ and 
$\left| \frac{a_p}{2\sqrt p} - x_p\right| \leqslant p^{-1/2}$. Let 
$y_p = \frac{a_p}{2\sqrt p}$. Now apply 
Lemma \ref{lem:disc-of-two-seq} with $\epsilon = N^{-1/2}$. We obtain 
\[
	\left| \disc(\bx^N,\cos_\ast \mu) - \disc(\by^N, \cos_\ast \mu)\right| \ll  N^{-1/2} + \frac{\pi(N^{1/2})}{\pi(N)} ,
\]
which tells us that $\disc(\by^N,\cos_\ast\mu) = \Theta(\pi(N)^{-\alpha})$. 
Now let $\btheta = \cos^{-1}(\by)$. Apply Lemma \ref{lem:pushforward-reverse} to 
$\btheta = \cos^{-1}(\by)$, and we see that 
$\disc(\btheta^N,\mu) = \Theta(\pi(N)^{-\alpha})$. 
\end{proof}

We can improve this example by controlling the behavior of sums of the form 
$\sum_{p\leqslant N} U_k(\theta_p)$ for odd $k$. Let $\sigma$ be 
the involution of $[0,\pi]$ given by $\sigma(\theta) = \pi-\theta$. Note that 
$\sigma_\ast \ST = \ST$. Moreover, note that for any odd $k$, 
$U_k\circ\sigma = - U_k$, so $\int U_k\, \dd\ST = 0$. (Of course, 
$\int U_k = 0$ for the reason that $U_k$ is the trace of a non-trivial unitary 
representation, but we will directly exploit the ``oddness'' of $U_k$ in what 
follows.)

\begin{theorem}\label{thm:int-flip-seq}
Let $\mu$ be a $\sigma$-invariant Sato--Tate compatible measure. Fix 
$\alpha\in (0,1/2)$. Then there is a sequence of integers $a_p$, satisfying 
the Hasse bound, such that for 
$\theta_p =\cos^{-1}\left( \frac{a_p}{2\sqrt p}\right)$, we have
\begin{enumerate}
\item
$\disc(\btheta^N,\mu) = \Theta(\pi(N)^{-\alpha})$. 

\item
For all odd $k$, 
$\left| \sum_{k\leqslant N} U_k(\theta_p)\right| \ll \pi(N)^{1/2}$. 
\end{enumerate}
\end{theorem}
\begin{proof}
The basic ideas is as follows. Enumerate the primes 
\[
	p_1 = 2, q_1 = 3, p_2 = 5, q_2 = 7, p_3 = 11, q_3 = 13, \dots .
\]
Consider the measure $\left.\mu\right|_{[0,\pi/2)}$. An argument 
nearly identical to the proof of Theorem \ref{thm:integral-a_p-alpha} shows 
that we can choose $a_{p_i}$ satisfying the Hasse bound so that 
\[
	\disc\left( \left\{\theta_{p_i}\right\}_{i\leqslant N},\left.\mu\right|_{[0,\pi/2)}\right) = \Theta(N^{-\alpha}) .
\]
We can also choose the $a_{q_i}\in [\pi/2,\pi]$ so that 
\[
	\left| \frac{a_{p_i}}{2\sqrt{p_i}} + \frac{a_{q_i}}{2\sqrt{q_i}}\right| \ll \frac{1}{\sqrt{p_i}} .
\]
If $\bx$ is the sequence of the $\frac{a_{p_i}}{2\sqrt{p_i}}$ and $\by$ is 
the similar sequence with the $q_i$-s, then Lemma \ref{lem:flip-discrepancy}, 
Lemma \ref{lem:disc-of-two-seq}, and Theorem \ref{thm:wreath-seq} tell us 
that $\disc((\bx\wr\by)^N,\mu) = \Theta(N^{-\alpha})$. 

Moreover, $U_k(\cos^{-1} t)$ is an odd polynomial in $t$, so if 
$|x_i - (-y_i)| \ll p_i^{-1/2}$, then 
$|U_k(\theta_{p_i}) + U_k(\theta_{q_i})| \ll p_i^{-1/2}$. We can then bound 
\[
	\left| \sum_{i\leqslant N} U_k(\theta_{p_i}) + U_k(\theta_{q_i})\right| \ll \sum_{p\leqslant N} p^{-1/2} \ll \pi(N)^{1/2} .
\]
\end{proof}





\section{Associated Galois representation}

Now we combine the results of the last section and Chapter 
\ref{ch:construct-Galois} to obtain a ``beefed-up'' version of Theorem 
\ref{thm:int-flip-seq}. 

\begin{theorem}
Let $\mu$ be a Sato--Tate compatible $\sigma$-invariant measure on $[0,\pi]$. 
Fix $\alpha\in (0,1/2)$ and a good residual representation 
$\rho\colon G_\bQ \to \GL_2(\bF_l)$. Then there exists a weight-$2$ lift 
$\rho\colon G_\bQ \to \GL_2(\bZ_l)$ of $\bar\rho$ such that 
\begin{enumerate}
\item
$\pi_{\ram(\rho)}(x) \ll e^{-x}\pi(x)$. 

\item
For each unramified prime $p$, $a_p = \tr\rho(\frob_p)\in \bZ$ and satisfies 
the Hasse bound. 

\item
If, for unramified $p$ we set 
$\theta_p = \cos^{-1}\left(\frac{a_p}{2\sqrt p}\right)$, then 
$\disc(\btheta^N,\mu) = \Theta(\pi(N)^{-\alpha})$. 

\item
For each odd $k$, the function $L(\sym^k \rho,s)$ satisfies the Riemann 
Hypothesis. 
\end{enumerate}
\end{theorem}
\begin{proof}
Let $\bx$ be an $N^{-\alpha}$-decay van der Corput sequence for 
$\cos_\ast \left.\mu\right|_{[0,\pi/2)}$. Let $\by = -\bx$. Then 
$\disc((\bx\wr\by)^N,\cos_\ast\mu) = \Theta(N^{-\alpha})$. Set $h(x) = e^{-x}$ 
and $b(x) = \log(x)$. By Theorem \ref{thm:master-Galois}, there is a 
$\rho\colon G_\bQ \to \GL_2(\bZ_l)$ lifting $\bar\rho$ such that parts 
1 and 2 of the theorem hold. The discrepancy estimate comes from Lemma 
\ref{lem:flip-discrepancy}, Lemma \ref{lem:disc-of-two-seq}, and Theorem 
\ref{thm:wreath-seq} as above, while the Riemann Hypothesis for odd symmetric 
powers follows from the proof of Theorem \ref{thm:int-flip-seq}. 
\end{proof}
