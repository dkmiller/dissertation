% !TEX root = main.tex

\chapter{Direct counterexample}





\section{Main ideas}

This chapter has two parts. First, for any reasonable measure $\mu$ on 
$[0,\pi]$ invariant under the same ``flip'' automorphism as the Sato--Tate 
measure, there is a sequence $\{a_p\}$ of integers satisfying the Hasse 
bound $|a_p|\leqslant 2\sqrt p$, such that for 
$\theta_p = \cos^{-1}\left(\frac{a_p}{2\sqrt p}\right)$, the discrepancy 
$\disc(\{\theta_p\}_{p\leqslant x},\mu)$ behaves like $x^{-\alpha}$ for 
predetermined $\alpha\in (0,1/2]$, while for any smooth $f$ satisfying 
$f(\pi-\theta) = -f(\theta)$ (and hence $\int f\, \dd\mu = 0$), the 
strange Dirichlet series $L_f(\{\theta_p\},s)$ satisfies the Riemann 
Hypothesis. 

In the second part of this chapter, we associate (infinitely ramified) Galois 
representations to the fake Satake parameters above, using techniques from 
\cite{pande-2011,khare-larsen-ramakrishna-2005}. 

Let $\mu=f(t)\, \dd t$ be a measure on $[0,\pi]$, where $f$ is a continuous 
function, nonzero except on $\{0,\pi\}$, such that $f(t) \ll \sin(t)$. Then 
$\cos_\ast\mu$ satisfies the hypotheses of 





\section{Construction}

\begin{theorem}
Let $\mu$ be a probability measure on $[0,\pi]$ such that $\cos_\ast\mu$ is 
good, and fix $\alpha\in (0,1/2)$. Then there exists a sequence of integers 
$a_p\in \bZ$ with $|a_p|\leqslant 2\sqrt p$, such that if we set 
$\theta_p = \cos^{-1}\left(\frac{a_p}{2\sqrt p}\right)$, then 
$\disc^\star(\btheta^N,\mu) = \Theta(\pi(N)^{-\alpha})$. 
\end{theorem}
\begin{proof}
Apply Theorem \ref{thm:discrepancy-arbitrary} to find a sequence $\bx$ such 
that $\disc(\bx^N,\cos_\ast \mu) = \Theta(\pi(N)^{-\alpha})$. For each prime 
$p$, there exists an integer $a_p$ such that $|a_p|\leqslant 2\sqrt p$ and 
$\left| \frac{a_p}{2\sqrt p} - x_p\right| \leqslant p^{-1/2}$. Let 
$y_p = \frac{a_p}{2\sqrt p}$. Now apply 
Lemma \ref{lem:disc-of-two-seq} with $\epsilon = N^{-1/2}$. We obtain 
\[
	\left| \disc(\bx^N,\cos_\ast \mu) - \disc(\by^N, \cos_\ast \mu)\right| \ll  N^{-1/2} + \frac{\pi(N^{1/2})}{\pi(N)} ,
\]
which tells us that $\disc(\by^N,\cos_\ast\mu) = \Theta(\pi(N)^{-\alpha})$. 
Now let $\btheta = \cos^{-1}(\by)$. Apply Lemma \ref{lem:pushforward-reverse} to 
$\btheta = \cos^{-1}(\by)$, and we see that 
$\disc(\btheta^N,\mu) = \Theta(\pi(N)^{-\alpha})$. 
\end{proof}

We can improve this example by controlling the behavior of sums of the form 
$\sum_{p\leqslant N} f(\theta_p)$, at least for ``odd'' $f$. Let $\sigma$ be 
the involution of $[0,\pi]$ given by $\sigma(\theta) = \pi-\theta$. Note that 
$\sigma_\ast \ST = \ST$. Moreover, note that for any $f$ with 
$f\circ \sigma = -f$, then $\int f\, \dd\ST = 0$. 

\begin{theorem}
Let $\mu$ be a probability measure on $[0,\pi]$ such that 
$\sigma_\ast\mu = \mu$ and $\left.\mu\right|_{[0,\pi/2)}$ is good. Fix 
$\alpha\in (0,1/2)$. Then there exists a sequence of integers $a_p$ with 
$|a_p|\leqslant 2\sqrt p$ such that for 
$\theta_p =\cos^{-1}\left( \frac{a_p}{2\sqrt p}\right)$, we have 
$\disc(\btheta^N,\mu) = \Theta(\pi(N)^{-\alpha})$, and moreover 
$\left| \sum_{p\leqslant N} f(\theta_p)\right| \ll N^{-1/2+\epsilon}$ 
whenever $f\circ \sigma = - f$, and $f$ is the restriction to $[0,\pi]$ of a 
smooth periodic function on $[-\pi,\pi]$ satisfying $f(-\theta) = f(\theta)$. 
\end{theorem}
\begin{proof}
The basic ideas is as follows. Enumerate the primes 
$p_1 =2, p_2 = 3, p_3 = 5$, and divide them into the ``odd indexed primes'' and 
the ``even indexed primes.'' For $n$ odd, choose $a_{p_n}$ so that 
$\theta_{p_n}\in [0,\pi/2)$ are equidistributed with respect to 
$\left.\mu\right|_{[0,\pi/2)}$ with desired rate of convergence. Then choose, 
for $n$ odd, $a_{p_{n+1}}$ so that $\theta_{p_{n+1}}\in [\pi/2,\pi]$ is very 
close to $\pi-\theta_{p_n}$. We can ensure that the discrepancy of the 
combined sequence decays at the correct rate. Moreover, for functions with  
$f(\pi-\theta) = -f(\theta)$, sums like $\sum_{p\leqslant N} f(\theta_p)$ 
have a bunch of terms looking like 
\[
	f(\theta_{p_n}) + f(\theta_{p_{n+1}})\approx f(\theta) + f(\pi-\theta) \approx f(\theta) - f(\theta) \approx 0 .
\]
We proceed to do this rigorously. 

Let $\bx=(x_1,x_2,\dots)$ be the sequence of Theorem 
\ref{thm:discrepancy-arbitrary} for $\cos_\ast \left. \mu\right|_{[0,\pi/2]}$ 
and $\alpha$. This is supported on $[0,1]$. Choose $a_{p_{2n-1}}$ so that 
$\left| \frac{a_{p_{2n-1}}}{2\sqrt{p_{2n-1}}} - x_n\right| \leqslant p^{-1/2}$ 
and also\ldots

Let $p<q$ be successive primes. Suppose we have already chosen $a_p<0$. Then 
we can choose $a_q>0$ to guarantee that 
\[
	\left| \frac{a_p}{2\sqrt p} + \frac{a_q}{2\sqrt q}\right| \leqslant \frac{1}{\sqrt q} .
\]

\end{proof}





\section{Associated Galois representation}



Fix, for the remainder of this section, a continuous representation 
\[
	\bar\rho_l \colon G_\bQ \to \GL_2(\bF_l) .
\]
For each $p$ at which $\bar\rho_l$ is unramified, we write 
\[
	\Theta_p(\bar\rho_l) = \left\{\cos^{-1}\left(\frac{a}{2\sqrt p}\right) : a\in \bZ\text{, }|a|\leqslant 2\sqrt p\text{, and }a\equiv \tr\bar\rho_l(\frob_p)\pmod l\right\} .
\]
For the finitely many primes $p$ for which $\Theta_p(\bar\rho_l)$ is empty, 
redefine $\Theta_p(\bar\rho_l)$ to include some elements for which 
$|a|>2\sqrt p$. We have a sequence of $\Theta_p(\bar\rho_l)$ for which at most 
finitely many do not satisfy the Hasse bound. 

\begin{theorem}
There exists a choice of $\theta_p\in \Theta_p(\bar\rho_l)$ for odd-indexed 
primes $\{2,5,11,\dots\}$ such that 
\begin{enumerate}
\item
$\theta_p\in [0,\pi/2)$ for all but finitely many $p$. 

\item
$\disc\left(\btheta_\odd^N, \left.\ST\right|_{[0,\pi/2)}\right) \to 0$, 
but is not $\ll N^{-\epsilon}$ for any $\epsilon>0$. 
\end{enumerate}
\end{theorem}
\begin{proof}
This is intuitively obvious, but a bit tricky to prove rigorously. 

Two key ideas: 

1. If we're given a ``bad'' finite distribution $\nu$, we can choose ``good'' 
$\theta_p$'s to make the combined distribution close enough (discrepancy-wise) 
to $\ST$. 

2. If we're given a ``good'' finite distribution $\nu$, we can choose ``bad'' 
$\theta_p\sim \pi/2$ to make the combined distribution far away 
(discrepancy-wise) from $\ST$. 
\end{proof}

Claim: let $\mu,\nu$ be two absolutely continuous distributions. Suppose there 
is a sequence $\{T_p\}$ of $\mu$-distributed sets, such that 
$\disc(T_p,\mu) \ll p^{-1/2}$. Suppose moreover that $\mu/\nu$ is bounded away 
from zero (at the pdf side). Then we can choose $t_p\in T_p$ so that 
$\{t_p\}$ is $\nu$-equidistributed with good discrepancy. 

Let $\mu$ be an absolutely continuous measure on $[0,\pi]$ such that the 
pushforward $\cos_\ast \mu$ is bounded (this is true for the Sato--Tate 
measure). Fix a prime $l\geqslant 5$ and a constant $\alpha\in (0,1/2]$. We 
want to construct a weight-$2$ Galois representation 
$\rho_l\colon G_\bQ \to \GL_2(\bZ_l)$, ramified at a density zero set of 
primes, such that 
\begin{enumerate}
\item
If $\rho_l$ is unramified at $p$, then $a_p = \tr \rho_l(\frob_p)\in \bZ$ and 
satisfies the Hasse bound $|a_p| \leqslant 2\sqrt p$. 

\item
If we write $\theta_p = \cos^{-1}(a_p / 2\sqrt p)$ for the Satake parameters at 
unramified primes, then $\disc(\btheta^N, \mu) \ll N^{-\alpha+\epsilon}$ and 
$\disc(\btheta^N,\mu) = \Omega(N^{-\alpha-\epsilon})$. 
\end{enumerate}

Recall the van der Corput sequence $\{x_p\}$ satisfies 
$\disc(\bx^N) \ll N^{-1+\epsilon}$. Let $\nu = \cos_\ast \mu$; this is an 
absolutely continuous measure supported on $[-1,1]$. By transforming the 
van der Corput sequence by a continuous map, we may assume that in fact 
$\disc(\bx^N,\nu) \ll N^{-1+\epsilon}$. In fact, by alternating between 
``van der Corput elements'' and ``bad elements'' we can ensure that not 
only does $\disc(\bx^N,\nu) \ll N^{-\alpha+\epsilon}$, but also 
$\disc(\bx^N,\nu) = \Omega(N^{-\alpha-\epsilon})$. 

We start by choosing a modular mod-$l$ representation 
$\rho_1\colon G_\bQ \to \GL_2(\bZ/l)$, which is ramified at a finite set of 
primes $S_1$. Let $R_1 = \{p\leqslant r_1 : p \notin S_1\}$. For $p\in R_1$, 
we can choose $a_p\in \bZ$ subject only to the condition 
$a_p \equiv \tr \rho_1(\frob_p)\pmod l$. For any $p\in R_1$, the set 
\[
	T_p(l) = \left\{\frac{a}{2\sqrt p} : |a| \leqslant 2\sqrt p\text{ and }a\equiv \tr \rho_1(\frob_p)\pmod l\right\}
\]
has an element within $l p^{-1/2}$ of any element of $[-1,1]$. Choose 
$a_p\in T_p(l)$ so that 
$\left|\frac{a_p}{2\sqrt p} - x_p\right| \leqslant l p^{-1/2}$. It follows 
that for $p\in R_1$, we have 
\[
	|\disc(\{a_p/2\sqrt p\}_{p\leqslant N}, \nu) - \disc(\bx^N,\nu)| \ll l N^{-1/2}
\]
We get a lift of $\rho_1$ to $\rho_2\colon G_\bQ \to \GL_2(\bZ/l^2)$ respecting 
our choices of the $a_p$ for $p\in R_1$, which is ramified at one (perhaps two) 
extra primes. 


What happens next is in stages. We'll already have a mod-$l^{n+1}$ 
representation $\rho_{n+1}\colon G_\bQ \to \GL_2(\bZ/l^n)$, together with 
choices of $a_p$ for $p\in R_1\cup \cdots \cup R_n$ that ensure 
$|\disc(\{a_p / 2\sqrt p\}_{p\leqslant N}, \nu) - \disc(\bx^N,\nu)| \ll ?$


The main question is: how do we choose $r_1$, and the later $r_n$? We ensure 
that a) the set $T_p(l^n)$ are non-empty, and that b) $l^n < \log(r_n)$. This 
gives us that for $N\leqslant r_n$, we have 
\[
	|\disc(\{a_p / 2\sqrt p\}_{p\leqslant N}, \nu) - \disc(\bx^N,\nu)| \ll N^{-\frac 1 2+\epsilon} .
\]

Todo: can I make $\sum a_p = ?$ anything from $-\infty$ to $\infty$?

What if I make a fake modular form with these ``bad'' Satake parameters? What 
can I say about it?





\section{Informal approach}

This discussion is inspired by \cite{pande-2011}. Throughout, all Galois 
representations have weight $2$, i.e.~determinant is the cyclotomic power. 

Fix a prime $l\geqslant 5$ and a (modular) representation 
$\rho_1\colon G_\bQ \to \GL_2(\bZ/l)$. We claim that there is a finite set 
$S$ such that $\sha_S^1(\Ad^0\rho_1) = \sha_S^2(\Ad^0 \rho_1) = 0$. Moreover, 
all the local deformation spaces are smooth? (Why?)

Set $S=S_2$. Choose lifts $\rho_p\colon G_{\bQ_p} \to \GL_2(\bZ_l)$ of 
$\left.\rho_1\right|_{G_{\bQ_p}}$ for all $p\in S_2$. We can ensure that the 
$\rho_p$ are ramified. (Can we also ensure that their characteristic 
polynomials are well behaved? By \cite{khare-rajan-2001}, these 
characteristic polynomials are well-defined for all but finitely many primes.)
Now let $R_2 = \{p\notin S_2 : p\leqslant r_2\}$, where $r_2$ is a yet 
unspecified large constant (say $l^{l^{100}}$). Choose $a_p$ for all 
$p\in R_2$. By \cite[Lem.~5.1]{pande-2011}, there is a set $Q_2$ (bound the 
size of $Q_2$!) 
