\documentclass[phd,cornellheadings,draft]{cornell}

\usepackage{
  amsmath,
  amssymb,
  amsthm,
  thesis-style          % custom math commands
}
\usepackage[
  backend  = bibtex,    % use bibtex instead of biber
  sorting  = nyt,       % sort by (name, year, title)
  style    = alphabetic % citations look like [Har77]
]{biblatex}
\addbibresource{thesis-sources.bib}

\title{Modularity lifting theorems for 2-dimensional Galois representations}
\author{Daniel Miller}
\conferraldate{June}{2017}

\begin{document}
\maketitle
\makecopyright

\begin{abstract}
Abstract here. 
\end{abstract}

\begin{biosketch}
Brief biographical sketch.
\end{biosketch}

\begin{dedication}
\emph{Soli Deo gloria}.
\end{dedication}

\begin{acknowledgements}
Thank Ravi, Yuri, Sasha, Birgit. 
\end{acknowledgements}

\contentspage

\normalspacing
\setcounter{page}{1}
\pagenumbering{arabic}
\pagestyle{cornell}


% ---------- begin main content ----------





\chapter{Introduction}

\section{Main theorem}

Let $\cO$ be a complete discrete valuation ring of characteristic zero with maximal ideal $\fm$ and finite residue field $\kk$. 
Let $\rho_0:G_\dQ\to \GL_2(\kk)$ be a continuous, odd, absolutely irreducible representation. 
By work of Khare and Wintenberger \cite{khare-wintenberger-2009-i,khare-wintenberger-2009-ii}, we know that $\rho_0$ is modular. 
Choose a deformation $\rho_n:G_\dQ\to \GL_2(\cO/\fm^{n+1})$ of $\rho_0$, which also has the same weight $k$ as $\rho_0$. 
Our goal is to prove that $\rho_n$ is modular of weight $k$. 
\[
  \sha^1(\adjoint^0 \rho_0)
\]
Another good source is \cite{neukirch-schmidt-winberg-2008}. 





\chapter{Deformation theory}

\section{General deformation theory}

Let $\cO$ be a complete discrete valuation ring with residue field $k$. Write 
$\artinian$ for the category of artinian $\cO$-algebras, and write 
$\hat\artinian$ be the category of \emph{pseudocompact} $\cO$-algebras, that is, 
topological $\cO$-algebras that are filtered inverse limits of rings in 
$\artinian$. Given $A\in \hat\artinian$, write $\formalspectrum(A)$ for the 
functor $\hom(A,-):\hat\artinian\to \sets$. By 
\cite[$\mathrm{VII_B}$ \S 0.4]{sga3-i}, a functor $X:\hat\artinian\to \sets$ is 
\emph{representable} (of the form $\formalspectrum(A)$ for some $A$) if and 
only if it is left-exact. 

Let $G$ be a representable group functor on $\hat\artinian$. 
Let $\Gamma$ be a profinite group. 
Define a functor $[\Gamma,G]:\hat\artinian\to \sets$ by 
\[
  [\Gamma,G](A) = \hom_\mathsf{TopGp}(\Gamma,G(A)) .
\]
\begin{proposition}
The functor $[\Gamma,G]$ is representable. 
\end{proposition}
\begin{proof}
First consider the case when $\Gamma$ is finite. In this case, 
$[\Gamma,G]$ is the subscheme of $\prod_\Gamma G$ cut out by 
$g_{\gamma\delta} = g_\gamma g_\delta$. In general, we note that for 
$A\in \artinian$, 
\begin{align*}
  [\Gamma,G](A) 
    &= \varinjlim_{\Delta\subset \Gamma} \hom_\mathsf{Gp}(\Gamma/\Delta,G(A)) \\
    &= \varinjlim_{\Delta\subset \Gamma} [\Gamma/\Delta,G](A) \\
    &= \left(\varinjlim_{\Delta\subset \Gamma} [\Gamma/\Delta,G]\right)(A) .
\end{align*}
By \cite[$\mathrm{VII_B}$ 1.2.B]{sga3-i}, the category $\hat\artinian$ is 
closed under filtered inductive limits, so we can set 
\[
  [\Gamma,G] = \varinjlim_{\Delta\subset \Gamma} [\Gamma/\Delta,G] .
\]
\end{proof}





\section{Basic definitions}

Let $\kk$ be a finite field, $W(\kk)$ its ring of Witt vectors. 
Let $\artinian$ be the category of artinian $W(\kk)$-algebras with residue field $\kk$. 
Let $\hat\artinian$ be the category of local \emph{pseudocompact} $W(\kk)$-algebras, that is topological $W(\kk)$-algebras that are filtered inverse limits of rings in $\artinian$. 
We will define various functors $\hat\artinian\to \sets$; such a functor $X$ is called \emph{representable} if there is $R\in \hat\artinian$ such that $X\simeq \hom(R,-)$. 
Write $\formalspectrum(R)=\hom(R,-)$. 

Fix a number field $F$. 
For $S$ a finite set of places of $F$, write $G_{F,S}$ for the Galois group of the maximal extension of $F$ unramified outside $S$. 
When $F=\dQ$, we will just write $G_S$. For simplicity, we write 
$\h^\bullet(S,M)$ instead of $\h^\bullet(G_{F,S},M)$ and 
$\h^\bullet(v,M)$ instead of $\h^\bullet(G_{F,v},M)$. 

Fix a continuous representation $\rho_0:G_F\to \GL_2(\kk)$. A \emph{deformation theory} is a tuple $\deformationtheory=(S,T,\cL,\ast)$, where 
\begin{itemize}
  \item $S$ is a finite set of places of $F$, containing all places at which $\rho_0$ is ramified, 
  \item $T$ is a finite set of places of $F$, containing $S$, 
  \item $\cL=(\cL_v)_{v\in S}$, where $\cL_v\subset \h^1(v,\adjoint^?\rho_0)$, where $?$ is $0$ if $\ast$ is defined, 
  \item $\ast$ is an optional integer $w$. 
\end{itemize}

\begin{definition}
Let $\deformationtheory=(S,T,\cL,\ast)$ be a deformation theory. 
The associated deformation functor $\deformationfunctor_\deformationtheory$ is defined on $\hat\artinian$ by: $\deformationfunctor_\deformationtheory(A)=$ all deformations of $\rho_0$ to $A$ that are unramified outside $S$, $T\smallsetminus S$-new, satisfy the local conditions $\cL$, and have weight $w$ if $\ast=w$. 
\end{definition}

For $S=\{\text{ramified places of }\rho_0\}$, we put $\deformationfunctor=\deformationfunctor_{(S,S,0)}$. 

Define carefully: $T$-new deformations, weight-$w$ deformations. The source for 
all of this is \cite{mazur-1995}. 





\printbibliography
\end{document}
