\documentclass[phd,cornellheadings,draft]{cornell}

\usepackage{
  amsmath,
  amssymb,
  amsthm,
  fixltx2e,               % textsubscript
  thesis-style          % custom math commands
}
\usepackage[
  backend  = bibtex,    % use bibtex instead of biber
  sorting  = nyt,       % sort by (name, year, title)
  style    = alphabetic % citations look like [Har77]
]{biblatex}
\DeclareFieldFormat{postnote}{#1}
\DeclareFieldFormat{multipostnote}{#1}
\addbibresource{thesis-sources.bib}

\title{Modularity lifting theorems for 2-dimensional Galois representations}
\author{Daniel Miller}
\conferraldate{June}{2017}

\begin{document}
\maketitle
\makecopyright

\begin{abstract}
Abstract here. 
\end{abstract}

\begin{biosketch}
Brief biographical sketch.
\end{biosketch}

\begin{dedication}
\emph{Soli Deo gloria}.
\end{dedication}

\begin{acknowledgements}
Thank Ravi, Yuri, Sasha, Birgit. 
\end{acknowledgements}

\contentspage

\normalspacing
\setcounter{page}{1}
\pagenumbering{arabic}
\pagestyle{cornell}


% ---------- begin main content ----------





\chapter{Introduction}

\section{Main theorem}

Let $\cO$ be a complete discrete valuation ring of characteristic zero with maximal ideal $\fm$ and finite residue field $\kk$. 
Let $\rho_0:G_\dQ\to \GL_2(\kk)$ be a continuous, odd, absolutely irreducible representation. 
By work of Khare and Wintenberger \cite{khare-wintenberger-2009-i,khare-wintenberger-2009-ii}, we know that $\rho_0$ is modular. 
Choose a deformation $\rho_n:G_\dQ\to \GL_2(\cO/\fm^{n+1})$ of $\rho_0$, which also has the same weight $k$ as $\rho_0$. 
Our goal is to prove that $\rho_n$ is modular of weight $k$. 
\[
  \sha^1(\adjoint^0 \rho_0)
\]
Another good source is \cite{neukirch-schmidt-winberg-2008}. 





\chapter{Deformation theory}

\section{General deformation theory}

Let $A$ be a pseudocompact ring. Let $\artinian_A$ be the opposite to the 
category of finite-length $A$-algebras. Our category of ``formal schemes'' will 
be $\indization(\artinian_A)$, the category of ind-objects of $\artinian_A$. 

\begin{theorem}
Let $\mathcal C$ be a category closed under finite colimits. Then the category 
$\indization(\mathcal C)$ is a cartesian closed subcategory of 
$\widehat{\mathcal C}$. 
\end{theorem}
\begin{proof}
Roughly, this is a computation. By \cite[6.1.7]{kashiwara-schapira-2006}, the 
category $\indization(\mathcal C)$ is the full subcategory of $\widehat{\mathcal C}$ 
consisting of left-exact functors. So, if $S=\varinjlim S_\alpha$ is a filtered 
colimit in $\mathcal C$, we compute:
\begin{align*}
	P^Q(S)
		&= \hom_{\widehat{\mathcal C}}(Q\times S,P) \\
		&= \hom_{\widehat{\mathcal C}}\left(Q\times \varinjlim S_\alpha,P\right) \\
		&= \hom_{\widehat{\mathcal C}}\left(\varinjlim Q\times S_\alpha,P\right) \\
		&= \varprojlim\hom_{\widehat{\mathcal C}}(Q\times S_\alpha,P) \\
		&= \varprojlim P^Q(S_\alpha) .
\end{align*}
\end{proof}

It follows that if $G,H\in \indization(\mathcal C)$, then 
\[
  [G,H](S) = \hom_{S\textnormal{-}\mathsf{gp}}(G_S,H_S) 
\]
is an object in $\indization(\mathcal C)$. 

By \cite[C 4.2.3]{johnstone-2002}, if $\mathcal C$ is cocartesian (closed under finite 
colimits) and has pullbacks, such that finite colimits are stable under pullbacks, 
then $\indization(\mathcal C)$ is locally cartesian closed, i.e.~each 
$\indization(\mathcal C)_{/P}$ is closed. In particular, 
$\indization(\artinian_A)$ is cartesian closed. Even better, Johnstone proves that 
if $P\in \indization(\mathcal C)$ (not over any presheaf) and $Q\in \widehat{\mathcal C}$,
then $P^Q\in\indization(\mathcal C)$, i.e.~$\indization(\mathcal C)$ is an exponential 
ideal in $\widehat{\mathcal C}$. 

Now compare $\indization(\artinian_A)$ with $\operatorname{Sh}_\mathrm{fl}(\artinian_A)$. 
We should have $\indization(\artinian_A)\subset \operatorname{Sh}_\mathrm{fl}(\artinian_A)$. 
Coverings look like $\formalspectrum(B)\to \formalspectrum(A)$, where $A\to B$ is flat and 
$\operatorname{mSpec}(B)\to \operatorname{mSpec}(A)$ is surjective. In particular, if 
$A$ is local, then all flat non-zero $A\to B$ are flat covers. 



\section{General deformation theory}

The material in this section is a summary of the results in 
\cite[7\textsubscript{b} \S0--2]{sga3-i}. Recall that a \emph{pseudocompact} 
ring $A$ is a (commutative, unital) topological ring admitting a basis of 
open neighborhoods of $0$ consisting of ideals with finite colength. For 
example, any complete local noetherian ring is pseudocompact. If 
$\cO$ is a pseudocompact ring, write $\mathsf{C}_\cO$ for the opposite category 
of the category of artinian (=finite length) $\cO$-algebras. We are interested 
in 
\[
  \mathrm{Sh}_\mathrm{fl}(\mathsf{C}_\cO) = \widetilde{\mathsf{C}_\cO} = \mathsf{C}_\cO
\]
By \cite[IV 4.3.5]{sga3-i} (see \cite[VII\textsubscript{B} 1.5]{sga3-i}, a 
functor $\mathsf{C}_\cO^\circ\to\sets$ is a sheaf for the flat topology if and 
only if $F$ sends disjoint unions to direct products and if 

Let $\cO$ be a complete discrete valuation ring with residue field $k$. Write 
$\artinian$ for the category of artinian $\cO$-algebras, and write 
$\hat\artinian$ be the category of \emph{pseudocompact} $\cO$-algebras, that is, 
topological $\cO$-algebras that are filtered inverse limits of rings in 
$\artinian$. Given $A\in \hat\artinian$, write $\formalspectrum(A)$ for the 
functor $\hom(A,-):\hat\artinian\to \sets$. By 
\cite[$\mathrm{VII_B}$ \S 0.4]{sga3-i}, a functor $X:\hat\artinian\to \sets$ is 
\emph{representable} (of the form $\formalspectrum(A)$ for some $A$) if and 
only if it is left-exact. 

Let $G$ be a representable group functor on $\hat\artinian$. 
Let $\Gamma$ be a profinite group. 
Define a functor $[\Gamma,G]:\hat\artinian\to \sets$ by 
\[
  [\Gamma,G](A) = \hom_\mathsf{TopGp}(\Gamma,G(A)) .
\]
\begin{proposition}
The functor $[\Gamma,G]$ is representable. 
\end{proposition}
\begin{proof}
First consider the case when $\Gamma$ is finite. In this case, 
$[\Gamma,G]$ is the subscheme of $\prod_\Gamma G$ cut out by 
$g_{\gamma\delta} = g_\gamma g_\delta$. In general, we note that for 
$A\in \artinian$, 
\begin{align*}
  [\Gamma,G](A) 
    &= \varinjlim_{\Delta\subset \Gamma} \hom_\mathsf{Gp}(\Gamma/\Delta,G(A)) \\
    &= \varinjlim_{\Delta\subset \Gamma} [\Gamma/\Delta,G](A) \\
    &= \left(\varinjlim_{\Delta\subset \Gamma} [\Gamma/\Delta,G]\right)(A) .
\end{align*}
By \cite[$\mathrm{VII_B}$ 1.2.B]{sga3-i}, the category $\hat\artinian$ is 
closed under filtered inductive limits, so we can set 
\[
  [\Gamma,G] = \varinjlim_{\Delta\subset \Gamma} [\Gamma/\Delta,G] .
\]
\end{proof}





\section{Following SGA 3}

Let $\cO$ be a pseudocompact ring. Let $\mathsf{Vaf}_\cO$ be the opposite 
category of pseudocompact $\cO$-algebras. This is naturally identified with the 
category of left-exact functors $\mathsf{Alf}_\cO\to \sets$, where 
$\mathsf{Alf}_\cO$ is the category of finite-length $\cO$-algebras. By 
definition, objects in $\mathsf{Vaf}_\cO$ are of the form 
$\formalspectrum(A)$ for $A$ a pseudocompact $\cO$-algebra. Since limits 
commute with limits, $\mathsf{Vaf}_\cO$ has all limits, and since 
filtered colimits commute with finite limits, $\mathsf{Vaf}_\cO$ has all 
filtered colimits. 

We'd like a good obstruction theory for objects in $\mathsf{Vaf}_\cO$. That 
is, let $X\in \mathsf{Vaf}_\cO$. If $0 \to \fa\to A_1\to A_2\to 0$ is a 
square-zero extension in $\mathsf{Alf}_\cO$ and $x\in X(A_2)$, we would like 
some kind of ``obstruction class'' $o(x)$ that vanishes precisely when 
$x$ lifts to $X(A_1)$. If $o(x)=0$, then lifts of $x$ to $X(A_1)$ should 
be a torsor under some kind of Ext-group. 

As above, let $\mathtt{Vaf}_{/\cO}$ be the opposite category of pseudocompact 
$\cO$-algebras. 

--------------------------------------------------

Recall from \cite[V.2.b]{sga3-i} that a monomorphism $R\to X\times X$ is an 
\emph{equivalence relation} if for each $T$, the image of 
$R(T)\to X(T)\times X(T)$ is an equivalence relation on $X(T)$. Let 
$d_0,d_1\colon R\to X$ be the components of $R\to X\times X$. The 
\emph{(categorical) quotient} of $X$ by $R$ is by definition:
\[
  X/R = \varinjlim\left(R\to_{d_0}^{d_1} X\right) .
\]
We want $X/R$ to be the ``honest quotient'' of $X$ by $R$, i.e.~for each $T$, 
$(X/R)(T) = X(T)/R(T)$. Note that 
\[
  R\xrightarrow\sim X\times_{X/R} X
\]
if and only if $R(T)=\{(x,y)\in X\times X\colon x=y\text{ in }X/R\}$. If this 
holds, then $X(T)\to (X/R)(T)$ is not necessarily surjective. But the induced 
map $X(T)/R(T)\to (X/R)(T)$ is injective. 

We're interested in the special case 
\[
  R = G\times X\to X\times X\qquad (g,x)\mapsto (g x,x) .
\]

From \cite[VII\textsubscript{B}.1.5]{sga3-i}, the quotient $X/R$ is actually 
the quotient in the category of flat sheaves. 

Let $R\to X\times X$ be an equivalence relation in an arbitrary category. Suppose 
the quotient $X/R$ exists. It admits an morphism $\pi\colon X\to X/R$ such that 
$\pi d_1 = \pi d_0$, that is universal with initial with respect to these 
properties. One can show that $\pi$ is epic. 

In the category of presheaves, colimits are computed pointwise, so 
$(X/R)(T) = X(T)/R(T)$. 

------------------------------------------------

[also see Tilouine]

Following [B\"ockle], the basic idea is as follows. Let $\Gamma$ be a profinite 
group, $\cG_{/\cO}$ a smooth formal group. We have already proved that 
$[\Gamma,\cG]$ is representable (though it may have several points). Let 
$\cZ=\zentrum(\cG)$; we assume $\cZ$ is also smooth. First question: is 
$\cG\to \cG/\cZ$ a ``pointwise surjection''?
\[
  \cG\times \cZ\to \cG\times \cG \qquad (g,z)\mapsto (g z,g) .
\]
We'd also look at 
\[
  (\cG/\cZ)\times [\Gamma,\cG]\to [\Gamma,\cG]\times [\Gamma,\cG]\qquad (g,\rho) \mapsto (g \rho g^{-1}, \rho) .
\]
For this to be an injection, we need $g\rho g^{-1} = \rho$ to imply 
$g=1$. 

------------------------------------------------





\section{Basic definitions}

Let $\kk$ be a finite field, $W(\kk)$ its ring of Witt vectors. 
Let $\artinian$ be the category of artinian $W(\kk)$-algebras with residue field $\kk$. 
Let $\hat\artinian$ be the category of local \emph{pseudocompact} $W(\kk)$-algebras, that is topological $W(\kk)$-algebras that are filtered inverse limits of rings in $\artinian$. 
We will define various functors $\hat\artinian\to \sets$; such a functor $X$ is called \emph{representable} if there is $R\in \hat\artinian$ such that $X\simeq \hom(R,-)$. 
Write $\formalspectrum(R)=\hom(R,-)$. 

Fix a number field $F$. 
For $S$ a finite set of places of $F$, write $G_{F,S}$ for the Galois group of the maximal extension of $F$ unramified outside $S$. 
When $F=\dQ$, we will just write $G_S$. For simplicity, we write 
$\h^\bullet(S,M)$ instead of $\h^\bullet(G_{F,S},M)$ and 
$\h^\bullet(v,M)$ instead of $\h^\bullet(G_{F,v},M)$. 

Fix a continuous representation $\rho_0:G_F\to \GL_2(\kk)$. A \emph{deformation theory} is a tuple $\deformationtheory=(S,T,\cL,\ast)$, where 
\begin{itemize}
  \item $S$ is a finite set of places of $F$, containing all places at which $\rho_0$ is ramified, 
  \item $T$ is a finite set of places of $F$, containing $S$, 
  \item $\cL=(\cL_v)_{v\in S}$, where $\cL_v\subset \h^1(v,\adjoint^?\rho_0)$, where $?$ is $0$ if $\ast$ is defined, 
  \item $\ast$ is an optional integer $w$. 
\end{itemize}

\begin{definition}
Let $\deformationtheory=(S,T,\cL,\ast)$ be a deformation theory. 
The associated deformation functor $\deformationfunctor_\deformationtheory$ is defined on $\hat\artinian$ by: $\deformationfunctor_\deformationtheory(A)=$ all deformations of $\rho_0$ to $A$ that are unramified outside $S$, $T\smallsetminus S$-new, satisfy the local conditions $\cL$, and have weight $w$ if $\ast=w$. 
\end{definition}

For $S=\{\text{ramified places of }\rho_0\}$, we put $\deformationfunctor=\deformationfunctor_{(S,S,0)}$. 

Define carefully: $T$-new deformations, weight-$w$ deformations. The source for 
all of this is \cite{mazur-1995}. 





\printbibliography
\end{document}
