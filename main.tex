\documentclass[phd,cornellheadings,final]{cornell}

\usepackage{
  amsmath,
  amssymb,
  thesis-style          % custom math commands
}
\usepackage[
  backend  = bibtex,    % use bibtex instead of biber
  sorting  = nyt,       % sort by (name, year, title)
  style    = alphabetic % citations look like [Har77]
]{biblatex}
\addbibresource{thesis-sources.bib}

\title{Modularity lifting theorems for 2-dimensional Galois representations}
\author{Daniel Miller}
\conferraldate{June}{2017}

\begin{document}
\maketitle
\makecopyright

\begin{abstract}
Abstract here. 
\end{abstract}

\begin{biosketch}
Brief biographical sketch.
\end{biosketch}

\begin{dedication}
\emph{Soli Deo gloria}.
\end{dedication}

\begin{acknowledgements}
Thank Ravi, Yuri, Sasha, Birgit. 
\end{acknowledgements}

\contentspage

\normalspacing
\setcounter{page}{1}
\pagenumbering{arabic}
\pagestyle{cornell}


% ---------- begin main content ----------





\chapter{Introduction}

\section{Main theorem}

Let $\cO$ be a complete discrete valuation ring of characteristic zero with maximal ideal $\fm$ and finite residue field $\kk$. 
Let $\rho_0:G_\dQ\to \GL_2(\kk)$ be a continuous, odd, absolutely irreducible representation. 
By work of Khare and Wintenberger \cite{khare-wintenberger-2009-i,khare-wintenberger-2009-ii}, we know that $\rho_0$ is modular. 
Choose a deformation $\rho_n:G_\dQ\to \GL_2(\cO/\fm^{n+1})$ of $\rho_0$, which also has the same weight $k$ as $\rho_0$. 
Our goal is to prove that $\rho_n$ is modular of weight $k$. 
\[
  \sha^1(\adjoint^0 \rho_0)
\]
Another good source is \cite{neukirch-schmidt-winberg-2008}. 

\printbibliography
\end{document}
