\documentclass[phd,cornellheadings]{cornell}

\usepackage{
	amsmath,
	amssymb,
	amsthm,
	hyperref,
	mathrsfs,
	stmaryrd, % llbracket
	thesis-style, % custom math commands
	tikz-cd
}
\usetikzlibrary{decorations.markings}

\usepackage[
	backend  = bibtex,    % use bibtex instead of biber
	sorting  = nyt,       % sort by (name, year, title)
	style    = alphabetic % citations look like [Har77]
]{biblatex}

\DeclareFieldFormat{postnote}{#1}
\DeclareFieldFormat{multipostnote}{#1}
\addbibresource{thesis-sources.bib}

\title{Constructing Galois representations with prescribed Sato--Tate 
distribution}
\author{Daniel Miller}
\conferraldate{May}{2017}


\begin{document}
\maketitle
\makecopyright

\begin{abstract}
Abstract here. 
\end{abstract}

\begin{biosketch}
Daniel Miller was born in St.~Paul, Minnesota. He completed his Bachelor of 
Science at the University of Nebraska--Omaha, during which he attended 
Cornell's Summer Mathematics Institute in 2011. He started his Ph.D.~at 
Cornell planning on a career in academia, but halfway through had a change of 
heart, and will be joining Microsoft's Analysis and Experimentation team as a 
Data Scientist after graduation. 
\end{biosketch}

\begin{dedication}
This thesis is dedicated to my undergraduate thesis advisor, Griff Elder. 
He is the reason I considered a career in math, his infectious enthusiasm for 
number theory has inspired me more than I can say. 
\end{dedication}

\begin{acknowledgements}
For starters, I'd like to thank my parents Jay and Cindy for noticing and 
fostering my mathematical interests early on, and for being loving and 
supportive the whole way through. I'd also like to thank my undergraduate 
thesis advisor, Griffith Elder, without whose encouragement and inspiration 
I'd probably never have considered a career in math. 

I'd like to thank Tara Holm for organizing Cornell's Summer Mathematics 
Institute in 2011, Jason Boynton for teaching a fantastic algebra class, and 
Anthony Weston for introducing me to the world of nonlinear functional 
analysis. 

Thanks to my graduate student friends Sasha Patotski and Bal\'azs Elek for 
sharing my early love of algebraic geometry, for laughing with me at the 
absurdities of academic life, and listening to my ramblings about number 
theory long after they'd stopped being interesting. 

I owe a big debt of gratitude to the mathematics department at Cornell---so 
many professors were generous with their time and ideas. I especially 
appreciate Birgit Speh, Yuri Berest, David Zywina, Farbod Shokrieh, and John 
Hubbard for letting me bounce ideas off them, helping me add rigor to 
half-baked ideas, and pointing me in new and exciting directions. 

I am especially thankful to my advisor Ravi. He kindled my first love for 
number theory, and stayed supportive as my research bounced all over the place, 
and helped focus and ground my thesis when I needed concrete results. 

Lastly, I thank my loving wife Ivy for being there for me through the highs and 
the lows---both when I (prematurely) thought my thesis was complete, and when I 
thought my results were completely in shambles. I couldn't have done it 
without her. 
\end{acknowledgements}

\contentspage

\normalspacing
\setcounter{page}{1}
\pagenumbering{arabic}
\pagestyle{cornell}





% !TEX root = main.tex

\chapter{Introduction}





Let's start with something basic, an elliptic curve $E_{/\bQ}$. For any 
prime $l$, we have the Tate module of $E$, written $\tate_l E$. This is a 
rank-$2$ $\bZ_l$-module with continuous $G_\bQ$-action, so it induces a 
continuous representation 
\[
	\rho_{E,l} \colon G_\bQ \to \GL_2(\bZ_l) .
\]
It is known (citation?) that the quantities $a_p(E) = \tr \rho_l(\frob_p)$ lie 
in $\bZ$ and satisfy the Hasse bound 
\[
	|a_p(E)| \leqslant 2\sqrt p .
\]
Thus we can define, for each prime $p$, the corresponding Satake parameter for 
$E$. 
\[
	\theta_p(E) = \cos^{-1}\left(\frac{a_p(E)}{2\sqrt p}\right) \in [0,\pi) .
\]
The Satake parameters are packaged into an $L$-function as follows:
\[
	L^\an(E,s) = \prod_p \frac{1}{(1 - e^{i \theta_p(E)} p^{-s})(1- e^{-i \theta_p(E)} p^{-s})} .
\]
More generally we have, for each $k\geqslant 1$, the $k$-th symmetric power 
$L$-function 
\[
	L^\an(\sym^k E, s) = \prod_p \prod_{j=0}^k \frac{1}{1 - e^{i (k - 2j) \theta_p(E)} p^{-s}} .
\]

Numerical experiments suggest that the Satake parameters are distributed with 
respect to the Sato--Tate distribution 
$\ST = \frac{2}{\pi} \sin^2\theta\, \dd\theta$. The ``goodness of fit'' of the 
Satake parameters to the Sato--Tate distribution is quantified by the 
\emph{discrepancy}:
\[
	\disc^\star(\{\theta_p(E)\}_{p\leqslant X}, \ST) = \sup_{x\in [0,\pi]} \left| \frac{\#\{p\leqslant X : \theta_p(E)\in [0,x)\}}{\pi(X)} - \int_0^x \, \dd \ST\right| .
\]
The decay of the discrepancy is closely related to the analytic properties of 
the $L(\sym^k E,s)$. First, here is the famous Sato--Tate conjecture (now a 
theorem) in the language we have defined. 

\begin{theorem}[Sato--Tate conjecture]
$\disc^\star(\{\theta_p(E)\}_{p\leqslant X}, \ST) \to 0$.  
\end{theorem}

\begin{theorem}
The Sato--Tate conjecture for $E$ holds if and only if each of the functions 
$L(\sym^k E,s)$ have analytic continuation past $\Re s=1$. 
\end{theorem}

The stunning recent proof of the Sato--Tate conjecture (citation) in fact 
showed that the functions $L(\sym^k E,s)$ were potentially automorphic, which 
gives analytic continuation. 

There is an analogy between the above equivalence and classical analytic number 
theory. Let $K/\bQ$ be a finite Galois extension, and 
$\rho\colon \Gal(K/\bQ) \to \GL_n(\bC)$ an irreducible representation. Recall 
the Artin $L$-function is 
\[
	L(\rho,s) = \prod_p \frac{1}{1-\tr \rho(\frob_p) p^{-s}} .
\]
Let $\Gal(K/\bQ)^\natural$ be the set of conjugacy classes in $\Gal(K/\bQ)$. 
The analogue of discrepancy here is: 
\[
	\disc(\{\frob_p\}_{p\leqslant X}) = \sup_{c\in \Gal(K/\bQ)^\natural} \left| \frac{\# \{p\leqslant X : \rho(\frob_p) \in c\}}{\pi(X)} - \frac{1}{\# \Gal(K/\bQ)^\natural}\right| .
\]

\begin{theorem}
The ``discrepancy'' $\disc(\{\frob_p\}_{p\leqslant X})\to 0$ if and only 
if $L(\rho,s)$ has analytic continuation past $\Re s=1$ for all non-trivial 
irreducible representations $\rho$ of $\Gal(K/\bQ)$. 
\end{theorem}

In the case of Artin $L$-functions, we know moreover that 

\begin{theorem}
The ``discrepancy'' satisfies the bound 
$\disc(\{\frob_p\}_{p\leqslant X}) \ll X^{-1/2+\epsilon}$ if and only if 
$L(\rho,s)$ satisfies the Riemann Hypothesis for all non-trivial irreducible 
representation $\rho$ of $\Gal(K/\bQ)$. 
\end{theorem}

In this context, the ``Riemann Hypothesis'' for $L(\rho,s)$ means exactly that 
$\log L(\rho,s)$ has analytic continuation to $\Re s=1/2$. 

The connection between the Riemann Hypothesis and ``strong Sato--Tate'' 
generalizes to elliptic curves and more general motives. For the moment, we 
stick to elliptic curves. In this case, ``strong Sato--Tate'' was conjectured 
by Akiyama--Tanigawa. More precisely, 

Conjecture:

Let $E_{/\bQ}$ be a non-CM elliptic curve. Then 
$\disc^\star(\{\theta_p(E)\}_{p\leqslant X}, \ST) \ll X^{-1/2+\epsilon}$. 


Moreover, one side of the equivalence ``Riemann Hypothesis $\Leftrightarrow$ 
strong Sato--Tate'' is known. 

\begin{theorem}
Let $E_{/\bQ}$ be an elliptic curve. If the Akiyama--Tanigawa conjecture for 
$E$ holds, then all $L(\sym^k E, s)$ satisfy the Riemann Hypothesis. 
\end{theorem}

It is natural to assume that the converse to this theorem holds. However (and 
that is the main point of this thesis) it does not! In this thesis, I construct 
a range of counterexamples to the implication ``strong Sato--Tate implies 
Riemann,'' and explore why the two are equivalent for Artin $L$-functions. 

I also provide computational evidence for the Akiyama--Tanigawa conjecture 
(for elliptic curves and also generic abelian $2$-folds). 

Similar work: \cite{pande-2011}. 

% TODO

Claim: the Riemann Hypothesis is equivalent to the following description of the 
distribution of prime numbers. For a real number $x$, let 
\[
	P_x = \frac{1}{\pi(x)} \sum_{p\leqslant x} \delta_{p/x} .
\]
This is a discrete probability measure supported on $[0,1]$. Moreover, let 
$L_x$ be the continuous probability measure with cdf 
\[
	L_x[0,t] = \frac{\Li(tx)}{\Li(x)} .
\]
Then $\disc(P_x,L_x) \ll x^{-\frac 1 2+\epsilon}$. I'm pretty sure that this 
statement is equivalent to $|\pi(x)-\Li(x)| \ll x^{\frac 1 2+\epsilon}$, which 
is already known to be equivalent to RH. 

To-do: conjectural framework? Can I find the rank from $\{sign(a_p)\}$?

% !TEX root = main.tex

\chapter{Discrepancy}





\section{Equidistribution}

The discrepancy (also known as the Kolmogorov--Smirnov statistic) is a way of 
measuring how closely sample data fits a predicted distribution. It has many 
applications in computer science and statistics, but here we will focus on only 
the basic known properties, as well as how discrepancy changes when sequences 
are tweaked and/or combined. 

First, recall that the discrepancy is a way of sharpening the ``soft'' 
convergence results of, say \cite[A.1]{serre-1989}. Let $X$ be a compact 
topological space, $\{x_p\}$ a sequence of points in $X$ indexed by the prime 
numbers. 

\begin{definition}
Let $\mu$ be a continuous probability measure on $X$. The sequence $\{x_p\}$ is 
\emph{equidistributed} with respect to $\mu$ if for all $f\in C(X)$, we have 
\[
	\lim_{x\to \infty} \frac{1}{\pi(x)} \sum_{p\leqslant x} f(x_p) \to \int f\, \dd \mu .
\]
\end{definition}

In other words, $\{x_p\}$ is $\mu$-equidistributed if the empirical measures 
$P_x = \frac{1}{\pi(x)} \sum_{p\leqslant x} \delta_{x_p}$ converge to $\mu$ in 
the weak topology. It is easy to see that $\{x_p\}$ is $\mu$-equidistributed if 
and only if $\left| \sum_{p\leqslant x} f(x_p)\right| = o(x)$ for all 
continuous $f$ having $\int f\, \dd\mu = 0$. In fact, one can restrict to a 
set of $f$ which generate a dense subpace of $C(X)^{\mu =0}$. 

In the discussion in \cite[A.1]{serre-1989}, $X$ is the space of conjugacy 
classes in a compact Lie group, and $f$ is allowed to range over the characters 
of irreducible, nontrivial representations of the group. In this section, we 
will show that the entire discussion can be generalized to a much broader class 
of \emph{strange Dirichlet series}, which are of the form 
\[
	L_f(\{x_p\},s) = \prod_p \frac{1}{1-f(x_p)p^{-s}} .
\]
A useful, but not too well known, result, is that we in fact can consider 
functions $f$ which are only continuous almost everywhere. 

\begin{theorem}
Let $X$ be a compact separable metric space with no isolated points. Let $\mu$ 
be a Borel measure on $X$ and let $f\colon X\to \bC$ be bounded and measurable. 
Then $f$ is continuous almost everywhere if and only if 
\[
	\lim_{x\to \infty} \frac{1}{\pi(x)} \sum_{p\leqslant x} f(x_p) = \int f\, \dd\mu
\]
for all $\mu$-equidistributed sequences $\{x_p\}$. 
\end{theorem}
\begin{proof}
This follows immediately from the proof of \cite[Th.~1]{mazzone-1995}
\end{proof}





\section{Definitions and first results}

We will define discrepancy for measures on the $d$-dimensional half-open box 
$[0,\infty)^d$. For vectors $x,y\in [0,\infty)^d$, we say $x<y$ if 
$x_1<y_1$,\dots,$x_d<y_d$, and in that case write $[x,y)$ for the half-open 
box $[x_1,y_1)\times \cdots \times [x_d,y_d)$. 

\begin{definition}
Let $\mu, \nu$ be probability measures on $[0,\infty)^d$. The 
\emph{discrepancy} of $\mu$ with respect to $\nu$ is 
\[
	\disc(\mu,\nu) = \sup_{x < y} \left| \mu[x,y) - \nu[x,y)\right| ,
\]
where $x<y$ range over $[0,\infty)^d$.

The \emph{star discrepancy} of $\mu$ with respect to $\nu$ is 
\[
	\disc^\star(\mu,\nu) = \sup_{0<y} \left| \mu[0,y) - \nu[0,y)\right| ,
\]
where $y$ ranges over $[0,\infty)^d$. 
\end{definition}

\begin{lemma}
Let $\mu,\nu$ be Borel measures on $\bR^d$. Then 
\[
	\disc^\star(\mu,\nu) \leqslant \disc(\mu,\nu) \leqslant 2^d \disc^\star(\mu,\nu) .
\]
\end{lemma}
\begin{proof}
The first inequality holds because the supremum defining the discrepancy is 
taken over a larger set than that defining star discrepancy. To prove the 
second inequality, let $x<y$ be in $[0,\infty)^d$. For 
$S\subset \{1,\dots,d\}$, let 
\[
	I_S = \{ t \in [0,y) : t_i < x_i \text{ for all }i\in S\} .
\]
The inclusion-exclusion principle for measures tells us that: 
\[
	\mu[x,y) = \sum_{S\subset \{1,\dots,d\}} (-1)^{\# S} \mu(I_S) ,
\]
and similarly for $\nu$. Since each of the $I_S$ are ``half-open boxes'' 
we know that $|\mu(I_S) - \nu(I_S)| \leqslant \disc^\star(\mu,\nu)$. It 
follows that 
\[
	|\mu[x,y) - \nu[x,y)| \leqslant \sum_{S\subset \{1,\dots,d\}} |\mu(I_S) - \nu(I_S)| \leqslant 2^d \disc^\star(\mu,\nu) .
\]
For a discussion and related context, see 
\cite[Ch.~2 Ex.~1.2]{kuipers-niederreiter-1974}. 
\end{proof}

We are usually interested in comparing empirical measures and their conjectured 
distribution. Namely, let $\bx = \{x_p\}$ be a sequence in $[0,\infty)^d$ 
indexed by the prime numbers, and $\mu$ a Borel measure on $[0\infty)^d$. For 
any real number $N\geqslant 2$, we write $\bx^N$ for the empirical measure 
given by 
\[
	\bx^N(S) = \frac{1}{\pi(N)} \sum_{p\leqslant N} \delta_{x_p}(S) = \frac{\# \{p\leqslant N : x_p\in S\}}{\pi(N)} .
\]
Also, we write $\bx_{\geqslant N}$ for the truncated sequence 
$(x_p)_{p\geqslant N}$, and similarly for $\bx_{\leqslant N}$, etc. In this 
context, 
\[
	\disc^\star(\bx^N,\nu) = \sup_{y\in [0,\infty)^d} \left| \frac{\# \{p\leqslant N : x_p \in [0,y)\}}{\pi(N)} - \int_{[0,y)} \, \dd\nu\right| .
\]

If the measure $\nu$ is only defined on a subset of $[0,\infty)^d$, we will 
tacitly extend it by zero. Moreover, if the sequence $\bx$ actually lies in a 
torus $(\bR/a \bZ)^d$, we identify that torus with the 
$[0,a)^d\subset [0,\infty)^d$. If $\nu$ is the Lebesgue measure (on 
$[0,\infty)^d$) or the normalized Haar measure on the torus, we write 
$\disc^\star(\bx^N)$ in place of $\disc^\star(\bx^N, \nu)$. 

Sometimes the sequence $\bx$ will not be indexed by the prime numbers, but 
rather by some other discrete subset of $\bR^+$. In that case we will still 
use the notations $\bx^N$, $\bx_{\geqslant N}$, etc., keeping in mind that 
$\pi(N)$ is replaced by $\#\{\textnormal{indices }\leqslant N\}$. 





\section{Examples}
Todo: give some basic examples of equidistributed sequences, 


\begin{definition}
The \emph{van der Corput sequence} is given by 
$\{\frac 1 2,\frac 1 4,\frac 3 4,\dots\}$. More precisely, write $n$ in base 
$2$ as $n = \sum a_i 2^i$. Then $x_n = \sum a_i 2^{-(i+1)}$. 
\end{definition}

The van der Corput sequence has generalizations to other bases and higher 
dimensions. It is well-known for being ``very equidistributed''---i.e.~, its' 
discrepancy has extremely fast convergence to zero. 

\begin{lemma}
We have $\disc(\{x_n\}_{n\leqslant N}) \leqslant \frac{\log(N+1)}{N\log 2}$. 
\end{lemma}
\begin{proof}
This is \cite[Ch.~2 Th.~3.5]{kuipers-niederreiter-1974}. 
\end{proof}

Now, the van der Corput sequence is uniformly distributed, but there is a 
convenient trick to construct sequences equidistributed with respect to more 
general measures. 

\begin{theorem}\label{thm:van-der-corput}
Let $\mu$ be a measure on a closed interval $[a,b]$ such that the cumulative 
distribution function $\cdf_\mu(x) = \mu[a,x]$ is continuous and strictly 
increasing. Then there exists a sequence $\bx=(x_1,x_2,\dots)$ such that 
$\disc(\bx^N,\mu) \ll \frac{\log(N)}{N}$. 
\end{theorem}
\begin{proof}
Since $\cdf_\mu$ is a continuous bijection and its domain is a compact set, 
$\cdf_\mu$ is an order isomorphism. Then Lemma \ref{lem:push-discrepancy} tells 
us that for $\bv$ the van der Corput sequence on $[0,1]$, we have 
$\disc(\cdf_\mu^{-1}(\bv)^N, \mu) = \disc(\bv^N,\mu)$, which gives us the 
desired result with $\bx = \cdf_\mu^{-1}(\bv)$. 
\end{proof}

\begin{theorem}\label{thm:discrepancy-arbitrary}
Let $\mu$ be a measure on $[a,b]$ such that $\cdf_\mu$ is continuous, 
sends $a$ to $0$, and strictly increasing. Fix $\alpha\in (0,1)$. Then there 
exists a sequence $\bx=(x_1,x_2,\dots)$ such that 
$\disc(\bx^N,\mu) = \Theta(N^{-\alpha})$. 
\end{theorem}
\begin{proof}
If $\bx_{\leqslant N}$ is a sequence of length $N$, let 
$\bx_{\leqslant N}:a_{\leqslant M}$ be the sequence $x_1,\dots,x_N,a,\dots,a$ 
($M$ copies of $a$). Then 
\[
	\disc^\star(\bx^N:a^M,\mu)
		\geqslant \left| \frac{\#\{ n\leqslant N+M : x_n = a\}}{N+M} - \mu\{a\}\right| 
		\geqslant \frac{M}{N+M} .
\]
On the other hand, 
\begin{align*}
	\left| \cdf_{N,M}(t) - \cdf_N(t)\right| 
		&\leqslant \frac{\left|\#\{n\leqslant N : x_n\leqslant t\} + M - \frac{M+N}{N}\#\{n\leqslant N : x_n\leqslant t\}\right|}{M+N} \\
%		&\leqslant \frac{\left|M - \frac{M}{N} \#\{n\leqslant N : x_n \leqslant t\} \right|}{M+N} \\
		&\leqslant \frac{2M}{M+N} ,
\end{align*}
which implies that 
$\disc^\star\left(\bx^N:a^M,\mu\right) \leqslant \disc^\star\left(\bx^N,\mu\right) + \frac{2 M}{M+N}$. 
Let $\bv$ be the $\mu$-equidistributed van der Corput sequence of 
Theorem \ref{thm:van-der-corput}, possibly transformed linearly to lie in 
$[a,b]$. We know that $\disc(\{v_n\}_{n\leqslant N}, \mu) \ll N^{-\alpha}$, 
with the constant in question depending only on $\alpha$. 

We construct the sequence $\bx$ via the following recipe. Start with 
$(x_1 = v_1,x_2 = v_2,\dots)$ until, for some $N_1$, 
$\disc(\bx^{N_1},\mu) < N^{-\alpha}$. Then set $x_{N_1+1} = a$, 
$x_{N_1+2} = a$, \dots, until $\disc(\bx^{N_1+M_1},\mu) > N^{-\alpha}$. 
Then set $x_{N_1+M_1+1} = v_{N_1+1}$, $x_{N_1+M_1+2} = v_{N_1+2}$, \dots, 
until once again $\disc(\bx^{N_1+M_1+N_2},\mu) < N^{-\alpha}$. Repeat 
indefinitely. We will show first, that the two steps are possible, and 
that nowhere does $\disc(\bx^N,\mu)$ differ by too much from $N^{-\alpha}$. 

Note that $\frac{M+1}{N+M+1} - \frac{M}{N+M} \leqslant N^{-1}$. This tells 
us that when we are adding $a$s at the end of $\bx^N$, the discrepancy of 
$\bx_{\leqslant N},a_{\leqslant M}$ increases by at most $N^{-1}$ at each 
step. So if $\disc(\bx^N,\mu) < N^{-\alpha}$, we can ensure that 
$\disc(\bx^N:a^M,\mu)$ is at most $N^{-1}$ greater than $N^{-\alpha}$. 

Moreover, we know that $\disc(\bx^N:a,\mu)$ is at most 
$\frac{2}{N+1}$ away from $\disc(\bx^N,\mu)$. So when adding van der Corput 
elements to the end of the sequence, its' discrepancy cannot decay any faster 
than by $\frac{2}{N+1}$ per $a$ added. This yields 
\[
	\left|\disc(\bx^N,\mu) - N^{-\alpha}\right| \ll N^{-1} , 
\]
which is even stronger than we need.
\end{proof}





\section{The Koksma--Hlawka inequality}

Here we summarize the results of the paper \cite{okten-1999}, generalizing them 
as needed for our context. Recall that a function $f$ on $[0,\infty)^d$ is 
said to be of \emph{bounded variation} if there is a finite Radon measure $\nu$ 
such that $f(x) - f(0) = \nu[0,x]$. In such a case we write 
$\Var(f) = |\nu|$. If the appropriate differentiability conditions are 
satisfied, then 
\[
	\Var(f) = \int_{[0,\infty)^d} \left|\frac{\dd^d f}{\dd x_1 \dots \dd x_d} \right|.
\]

\begin{theorem}[Koksma--Hlawka]
Let $\mu$ be a probability measure on $[0,\infty)^d$, $f$ a function of 
bounded variation. Then for any sequence $\bx$ in $[0,\infty)^d$, we have 
\[
	\left| \frac{1}{\pi(x)} \sum_{p\leqslant x} f(x_p) - \int f\, \dd\mu \right| \leqslant \Var(f) \disc(\bx^N,\mu) .
\]
\end{theorem}
\begin{proof}
By our assumptions there is a Radon measure $\nu$ such that 
$f(y) - f(0) = \nu[0,y]$. What follows is essentially trivial, noting that 
$1_{[0,x]}(y) = 1_{[y,\infty)^d}(x)$. 
\begin{align*}
	\frac{1}{\pi(x)} \sum_{p\leqslant x} f(x_p) - \int f\, \dd\mu 
		&= \frac{1}{\pi(x)} \sum_{p\leqslant x} \left(f(x_p) - f(0)\right) - \int \left(f - f(0)\right)\, \dd\mu \\
		&= \frac{1}{\pi(x)} \sum_{p\leqslant x} \int 1_{[y,\infty)^d}(x_p)\, \dd \nu(y) - \int \int 1_{[0,y]}\, \dd\nu \, \dd\mu(y) \\
		&= \int \frac{1}{\pi(x)} \sum_{p\leqslant x} 1_{[y,\infty)^d}(x_p) - \int 1_{[y,\infty)^d}\, \dd\mu \, \dd\nu(y)
\end{align*}
It follows that 
\[
	\left| \frac{1}{\pi(x)} \sum_{p\leqslant x} f(x_p) - \int f\, \dd\mu \right|
		\leqslant \sup_{y\in [0,\infty)} \left| \frac{1}{\pi(x)} \sum_{p\leqslant x} 1_{[y,\infty)}(x_p) - \int 1_{[y,\infty)}\, \dd\mu\right| \cdot |\nu| .
\]
The supremum in question is clearly bounded above by $\disc(\bx^N,\mu)$, so the 
proof is complete. 
\end{proof}





\section{Comparing sequences}

\begin{lemma}
Let $\bx$ and $\by$ be sequences in $[0,\infty)$. Suppose 
$\nu = f\cdot \lambda$ for $f$ a bounded continuous function and $\lambda$ the 
Lebesgue measure. Then 
\[
	\left|\disc^\star(\bx^N, \nu) - \disc^\star(\by^N,\nu)\right| \leqslant \|f\|_\infty \epsilon + \frac{\#\{ n\leqslant N : |x_n - y_n| \geqslant \epsilon\}}{N} .
\]
\end{lemma}
\begin{proof}
Let $\epsilon>0$ and $t\in [0,\infty)$ be arbitrary. For all $n\leqslant N$ 
such that $y_n<t$, either $x_n < t+\epsilon$ or 
$|x_n - y_n| \geqslant \epsilon$. It follows that 
\[
	\by^N[0,t) \leqslant \bx^N[0,t+\epsilon) + \frac{\#\{ n\leqslant N : |x_n - y_n| \geqslant \epsilon\}}{N} .
\]
Moreover, we trivially have 
$\left| \bx^N[0,t+\epsilon) - \nu[0,t+\epsilon)\right| \leqslant \disc^\star(\bx^N,\nu)$. Putting these together, we get: 
\begin{align*}
	\by^N[0,t) - \nu[0,t) 
		&\leqslant \bx^N[0,t+\epsilon) - \nu[0,t) + \frac{\#\{ n\leqslant N : |x_n - y_n| \geqslant \epsilon\}}{N} \\
		&\leqslant \nu[t,t+\epsilon) + \disc^\star(\bx^N,\nu) + \frac{\#\{ n\leqslant N : |x_n - y_n| \geqslant \epsilon\}}{N} \\
		&\leqslant \|f\|_\infty \epsilon + \disc^\star(\bx^N,\nu) + \frac{\#\{ n\leqslant N : |x_n - y_n| \geqslant \epsilon\}}{N} 
\end{align*}
This tells us that 
\[
	\disc^\star(\by^N,\nu) \leqslant \|f\|_\infty \epsilon + \disc^\star(\bx^N,\nu) + \frac{\#\{ n\leqslant N : |x_n - y_n| \geqslant \epsilon\}}{N} .
\]
Reversing the roles of $\bx$ and $\by$, we obtain the desired result. 
\end{proof}

\begin{lemma}
Let $\sigma$ be an isometry of $\bR$, and $\bx$ a sequence in $[0,\infty)$ 
such that $\sigma(\bx)$ is also in $[0,\infty)$. Let $\nu$ be an absolutely 
continuous measure on $[0,\infty)$ such that $\sigma_\ast \nu$ is also 
supported on $[0,\infty)$. Then 
\[
	\left|\disc(\bx^N, \nu) - \disc(\sigma_\ast \bx^N, \sigma_\ast \nu)\right| \leqslant \frac{2}{\pi(N)} .
\]
\end{lemma}
\begin{proof}
Every isometry of $\bR$ is a combination of translations and reflections. 
The statement is clear with translations (the two discrepancies are equal). So, 
suppose $\sigma(t) = a - t$ for some $a>0$. Since $\nu$ is absolutely 
continuous, $\nu\{t\}=0$ for all $t\geqslant 0$. In particular, 
$\nu[s,t) = \nu(s,t]$. In contrast, $\bx^N\{t\}\leqslant \pi(N)^{-1}$. For any 
interval $[s,t)$ in $[0,\infty)$, we know that 
\[
	\left| \bx^N[s,t) - \bx^N(s,t]\right| \leqslant \frac{2}{\pi(N)}  ,
\]
hence 
\[
	\left| \bx^N[s,t) - \nu[s,t) - (\sigma_\ast \bx^N)[a-t,a-s) - (\sigma_\ast \nu)[a-t,a-s)\right| \leqslant \frac{2}{\pi(N)} .
\]
This proves the result. 
\end{proof}

A trick we will use throughout this thesis involves comparing the discrepancy 
of a sequence with the discrepancy of a pushforward sequence, with respect to 
the pushforward measure. 

\begin{lemma}\label{lem:push-discrepancy}
Let $f$ be an order isomorphism $f\colon [a,b] \to [c,d]$. If 
$\bx$ is a sequence on $[a,b]$ and $\mu$ is a probability measure on 
$[a,b]$, then 
$\disc(\bx^N, \mu) = \disc(f(\bx)^N, f_\ast \mu)$, 
and likewise for star discrepancy. 
\end{lemma}
\begin{proof}
This is a simple computation, which we only check for star discrepancy: 
\begin{align*}
	\disc(f(\bx)^N, f_\ast \mu)
		&= \sup_{t\in [c,d]} \left| \frac{\#\{n\leqslant N : f(x_n) \leqslant t\}}{N} - (f_\ast \mu)[a,t]\right| \\
		&= \sup_{t\in [a,b]} \left| \frac{\#\{n\leqslant N : x_n \leqslant f^{-1}(t)\}}{N} - \mu[a,f^{-1}(t)]\right| \\
		&= \disc(\bx^N, \mu) .
\end{align*}
\end{proof}





\section{Combining sequences}

\begin{definition}
Let $\bx$ and $\by$ be sequences in $[0,\infty)^d$. We write $\bx\wr\by$ for 
the interleaved sequence 
\[
	(x_2,y_2,x_3,y_3,x_5,y_5,\dots,x_p,y_p,\dots) .
\]
\end{definition}

For the interleaved sequence $\bx\wr\by$, we write $(\bx\wr\by)^N$ for the 
empirical measure 
\[
	(\bx\wr\by)^N = \frac{1}{2\pi(N)} \sum_{p\leqslant N} \delta_{x_p} + \delta_{y_p} .
\]

\begin{theorem}
Let $I$ and $J$ be disjoint open boxes in $[0,\infty)^d$, and let $\mu$, 
$\nu$ be absolutely continuous probability measures on $I$ and $J$, 
respectively. Let $\bx$ be a sequence in $I$ and $\by$ be a sequence in $J$. 
Then 
\[
	\max\{\disc(\bx^N,\mu),\disc(\by^N,\nu)\} \leqslant \disc((\bx\wr\by)^N, \mu+\nu) \leqslant \disc(\bx^N,\mu) + \disc(\by^N,\nu)
\]
\end{theorem}
\begin{proof}
Any half-open box in $[0,\infty)^d$ can be split by a coordinate 
hyperplane into two disjoint half-open boxes $[a,b)\sqcup [s,t)$, each of which 
intersects at most one of $I$ and $J$. We may assume that 
$[a,b)\cap J=\varnothing$ and $[s,t)\cap I = \varnothing$. Then 
\begin{align*}
	\left| (\bx\wr\by)^N([a,b)\sqcup [s,t)) - (\mu+\nu)([a,b)\sqcup[s,t))\right| 
		&\leqslant |\bx^N[a,b) - \mu[a,b)| + |\by^N[s,t) - \nu[s,t)| \\
		&\leqslant \disc(\bx^N,\mu) + \disc(\by^N,\nu) .
\end{align*}
This yields the second inequality in the statement of the theorem. To see the 
first, assume that the maximum discrepancy is $\disc(\bx^N,\mu)$, and let 
$[s,t)$ be a half-open box such that $|\bx^N[s,t) - \mu[s,t)|$ is within an 
arbitrary $\epsilon$ of $\disc(\bx^N,\mu)$. We can assume that $[s,t)$ does not 
intersect $J$, and thus 
\[
	\left|(\bx\wr\by)^N[s,t) - (\mu+\nu)[s,t)\right| = |\bx^N[s,t) - \mu[s,t)| ,
\]
which yields the result. 
\end{proof}

% !TEX root = thesis.tex

\chapter{Strange Dirichlet series}





\section{Definitions}

We start by considering a very general class of Dirichlet series. In fact, they 
are all Dirichlet series that admit a product formula with degree-1 factors, 
but in this thesis they will be called strange Dirichlet series. The motivating 
example was suggested by Ramakrishna. Let $E_{/\bQ}$ be an elliptic curve and 
let 
\[
	L_{\sgn}(E,s) = \prod_p \frac{1}{1-\sgn(a_p) p^{-s}} .
\]
How much can we say about the behavior of $L_{\sgn}(E,s)$? For example, does it 
``know'' the rank of $E$?

\begin{definition}
Let $\bz=(z_2,z_3,z_5,\dots)$ be a sequence of complex numbers indexed by the 
primes. The associated \emph{strange Dirichlet series} is 
\[
	L(\bz,s) = \prod_p \frac{1}{1- z_p p^{-s}} .
\]
\end{definition}

If $z_p$ is only defined for all but finitely many primes, then we tacitly set 
$\bz_p = 0$ for all primes for which $z_p$ is not defined. 

\begin{lemma}
Let $\bz$ be a sequence with $\|\bz\|_\infty \leqslant 1$. Then $L(\bz,s)$ 
defines a holomorphic function on the region $\{\Re s>1\}$. Moreover, on that 
region, 
\[
	\log L(\bz,s) = \sum_{p^r} \frac{z_p^n}{n p^{n s}} .
\]
\end{lemma}
\begin{proof}
Expanding the product for $L(\bz,s)$ formally, we have 
\[
	L(\bz,s) = \sum_{n\geqslant 1} \frac{\prod_p z_p^{v_p(n)}}{n^s} .
\]
An easy comparison with the Riemann zeta function tells us that this sum 
is holomorphic on $\{\Re s>1\}$. By \cite[Th.~11.7]{apostol-1976}, the 
product formula holds in the same region. The formula for $\log L(\bz,s)$ 
comes from \cite[11.9 Ex.2]{apostol-1976}. 
\end{proof}

\begin{lemma}[Abel summation]\label{lem:abel-sum}
Let $\bz=(z_2,z_3,z_5,\dots)$ be a sequence of complex numbers, $f$ a smooth 
complex-valued function on $\bR$. Then 
\[
	\sum_{p\leqslant N} f(p) z_p = f(N) \sum_{p\leqslant N} z_p - \int_2^N f'(x) \sum_{p\leqslant x} z_p\, \dd x .
\]
\end{lemma}
\begin{proof}
Simply note that if $p_1,\dots,p_n$ is an enumeration of the primes 
$\leqslant N$, we have 
\begin{align*}
	\int_2^N f'(x) \sum_{p\leqslant x} z_p\, \dd x 
		&= \sum_{p\leqslant N} z_p \int_{p_n}^N f' + \sum_{i=1}^{n-1} \sum_{p\leqslant p_{i+1}} z_p \int_{p_i}^{p_{i+1}} f' \\
		&= (f(N) - f(p_n)) \sum_{p\leqslant N} z_p + \sum_{i=1}^{n-1} (f(p_{i+1}) - f(p_i)) \sum_{p\leqslant p_{i+1}} z_p \\
		&= f(N) \sum_{p\leqslant N} z_p - \sum_{p\leqslant N} f(p) z_p ,
\end{align*}
as desired. 
\end{proof}

\begin{theorem}
Assume $|\sum_{p\leqslant x} z_p| \ll x^{\alpha+\epsilon}$ for some 
$\alpha\in [\frac 1 2,1]$. Then the series for $\log L(\bz,s)$ converges to a 
holomorphic function on the region $\{\Re s>\alpha\}$. 
\end{theorem}
\begin{proof}
Formally split the sum for $\log L(\bz,s)$ into two pieces: 
\[
	\log L(\bz,s) = \sum_p \frac{z_p}{p^s} + \sum_p \sum_{r\geqslant 2} \frac{z_p^r}{r p^{r s}} .
\]
For each $p$, we have 
\[
	\left| \sum_{r\geqslant 2} \frac{z_p^r}{r p^{r s}}\right| \leqslant \sum_{r\geqslant 2} p^{- r \Re s} = p^{-2 \Re s} \frac{1}{1-p^{-\Re s}} .
\]
Elementary analysis gives 
\[
	1 \leqslant \frac{1}{1-p^{-\Re s}} \leqslant 2 + 2\sqrt 2 ,
\]
so the second piece of $\log L(\bz,s)$ converges absolutely when 
$\Re s>\frac 1 2$. We could simply cite \cite[II.1 Th.~10]{tenenbaum-1995}; 
instead we prove directly that $\sum_p \frac{z_p}{p^s}$ converges absolutely 
to a holomorphic function on the region $\{\Re s>\alpha\}$. 

By Lemma \ref{lem:abel-sum} with $f(x) = x^{-s}$, we have 
\begin{align*}
	\sum_{p\leqslant N} \frac{z_p}{p^s}
		&= N^{-s} \sum_{p\leqslant N} z_p + s \int_2^N \sum_{p\leqslant x} z_p\, \frac{\dd x}{x^{s+1}} \\
		&\ll N^{-\Re s + \alpha + \epsilon} + s \int_2^N x^{\alpha+\epsilon} \frac{\dd x}{x^{s+1}} .
\end{align*}
Since $\alpha-\Re s < 0$, the first term is bounded. Since $s+1-\alpha > 1$ and 
$\epsilon$ is arbitrary, the integral converges absolutely, and the proof is 
complete. 
\end{proof}

\begin{theorem}
Let $\bz=(z_2,z_3,\dots)$ be a sequence with $\|\bz\|_\infty\leqslant 1$, and 
assume $\log L(\bz,s)$ has analytic continuation to $\{\Re s>\alpha\}$ for some 
$\alpha\in \frac 1 2,1]$, and that for $\sigma>\alpha$, we have 
$|\log L(\bz,\sigma+i t)| \ll |t|^{1-\epsilon}$ (implied constant independent 
of $\sigma$.) Then $|\sum_{p\leqslant N} z_p| \ll N^{\alpha+\epsilon}$. 
\end{theorem}
\begin{proof}
Recall that we can write 
\[
	\log L(\bz,s) = \sum_p \frac{z_p}{p^s} + \sum_p \sum_{r\geqslant 2} \frac{z_p^r}{r p^{r s}} = \sum_p \frac{z_p}{p^s} + O(\zeta(2 \Re s)) .
\]
Thus, for any $\epsilon>0$, analytic continuation and the bound on 
$|\log L(\bz,\sigma+i t)|$ implies the same analytic continuation and bound for 
$\sum \frac{z_p}{p^s}$ on $\{\Re s>\alpha+\epsilon\}$. 

For any $T>0$, let 
$\gamma_T = \gamma_{1,T} + \gamma_{2,T} + \gamma_{3,T} + \gamma_{4,T}$ be the 
following contour: 
\begin{align*}
	\gamma_{1,T}(t) &= (\alpha+\epsilon)+i t\qquad t\in [-T,T] \\
	\gamma_{2,T}(t) &= t+i T \qquad t\in [\alpha+\epsilon,1+\epsilon] \\
	\gamma_{3,T}(t) &= (1+\epsilon) + i t \qquad t\in [T,-T] \\
	\gamma_{4,T}(t) &= t - i T \qquad t\in [1+\epsilon,\alpha+\epsilon] .
\end{align*}
Graphically, the contour looks like this: 
\begin{center}
\begin{tikzpicture}[
	decoration={%
		markings,
		mark=at position 2cm with {\arrow[line width=1pt]{>}},
		}]
	\draw [help lines,->] (-1,0) -- (4,0) coordinate (xaxis);
	\draw [help lines,->] (0,-4) -- (0,4) coordinate (yaxis);
	\node [below] at (xaxis) {$\bR$};
	\node [left] at (yaxis) {$i\bR$};
	\node at (0.6,1) {$\gamma_{1,T}$};
	\node at (2, 2.5) {$\gamma_{2,T}$};
	\node at (3.6, 1) {$\gamma_{3,T}$};
	\node at (-2.5, 2) {$\gamma_{4,T}$};
	\path[draw,postaction=decorate] 
		(1,-3) node[below] {$\alpha+\epsilon - i T$} -- 
		(1,3) node[above] {$\alpha+\epsilon + i T$} -- 
		(3,3) node[above] {$1+\epsilon + i T$} -- 
		(3,-3) node[below] {$1+\epsilon - i T$} --
		(1,-3);
\end{tikzpicture}
\end{center}
By Perron's formula \cite[Th.~11.18]{apostol-1976}, 
\[
	\lim_{T\to \infty} \frac{1}{2\pi i} \int_{-\gamma_{3,T}} \sum_p \frac{z_p}{p^s} N^z\, \frac{\dd z}{z} = \frac 1 2 \sum_{p\leqslant N} z_p .
\]
for $N\in \bZ$, and the same without the $\frac 1 2$ on the right-hand side 
when $N\notin \bZ$. 

Let $h(s)$ be the analytic continuation of $\sum z_p p^{-s}$ to 
$\{\Re s>\alpha\}$. Since $\int_{\gamma_T} h(s)\, \frac{\dd s}{s}=0$, we obtain 
\[
	\left|\sum_{p\leqslant N} z_p\right| 
		\ll \lim_{T\to \infty} \left(\left| \int_{\gamma_{1,T}} h(s) N^s\frac{\dd s}{s}\right| + \left|\int_{\gamma_{2,T}} h(s) N^s \frac{\dd s}{s}\right| + \left|\int_{\gamma_{4,T}} h(s) N^s \frac{\dd s}{s}\right| \right).
\]
We know that $|h(\sigma+i t)| \ll |t|^{1-\epsilon}$, so we can bound 
\[
	\left| \int_{\gamma_{2,T}} h(s)N^s \frac{\dd s}{s}\right| = \left| \int_{\alpha+\epsilon}^{1+\epsilon} \frac{h(t+i T) N^{t+i T}}{t+i T}\, \dd t\right| \ll \frac{N^{1+\alpha}}{T^\epsilon} ,
\]
and similarly for $\gamma_{4,T}$. Finally, note that 
\[
	\left| \int_{\gamma_{1,T}} h(s) N^s\, \frac{\dd s}{s}\right| \ll \int_{-T}^T |t|^{1-\epsilon} \frac{N^{\alpha+\epsilon}}{(\alpha+\epsilon)^2 + t^2} \, \dd t \ll N^{\alpha+\epsilon} .
\]
Letting $T\to \infty$ we obtain the desired result. 
\end{proof}

In this thesis, we are interested in the following sort of strange Dirichlet 
series. Let $X$ be a space, $f\colon X\to \bC$ a function with 
$\|f\|_\infty\leqslant 1$, and $\bx=(x_2,x_3,\dots)$ a sequence in $X$. Write 
\[
	L_f(\bx,s) = \prod_p \frac{1}{1-f(x_p) p^{-s}} ,
\]
for the associated strange Dirichlet series. 





\section{Relation to automorphic and motivic \texorpdfstring{$L$}{L}-functions}


To-do: show that \cite[A.1]{serre-1989} works for $L_f(\bx,s)$, $f$ 
almost-everywhere continuous. 





\section{Discrepancy of sequences and the Riemann Hypothesis}

To-do: show that A--T implies RH. 

To-do: define ``Riemann Hypothesis'' for strange Dirichlet series. 





\section{Strange Dirichlet series over function fields}

To-do: summarize \cite[Ch.~9]{katz-sarnak-1999} and 
\cite{niederreiter-1991}. 

% !TEX root = main.tex

\chapter{Irrationality exponents}

% !TEX root = main.tex





\chapter{Deformation theory}





\section{Category of test objects}

This section summarizes the theory in 
\cite[VII\textsubscript{B}, \S 0--1]{sga3-1}, adapting it to the deformation 
theory of Galois representations. All rings are commutative with unit. 

\begin{definition}
Let $\Lambda$ be a ring. A topological $\Lambda$-module $M$ is 
\emph{pseudocompact} if it is a filtered inverse limit of discrete 
finite-length $\Lambda$-modules. The ring $\Lambda$ is pseudocompact if it 
is pseudocompact as a module over itself. 
\end{definition}

Let $\Lambda$ be a topological ring. Given a pseudocompact $\Lambda$-algebra 
$A$, write $\Art_\Lambda$ for the opposite 
of the category of $\Lambda$-algebras which have finite length as 
$\Lambda$-modules. Given such a $\Lambda$-algebra $A$, write $X=\spf(A)$ for 
the corresponding object of $\Art_\Lambda$, and we put $A=\sO(X)$. 

\begin{lemma}
Let $\Lambda$ be a pseudocompact ring, $\Art_\Lambda$ as above. Then 
$\Art_\Lambda$ is closed under finite limits and colimits. 
\end{lemma}
\begin{proof}
That $\Art_\Lambda$ is closed under finite colimits follows from the fact that 
finite-length $\Lambda$-algebras are closed under finite limits (the 
underlying modules are closed under finite limits). Moreover, since the tensor 
product of finite length modules also has finite length, and quotients of 
length modules have finite length, $\Art_\Lambda$ is closed under finite 
limits. 
\end{proof}

\begin{lemma}
Let $\Lambda$ be a pseudocompact local ring. Then $\Lambda$ is henselian, in 
any of the following senses:
\begin{enumerate}
\item
Every finite $\Lambda$-algebra is a product of local $\Lambda$-algebras.

\item
The first condition is satisfied for $\Lambda$-algebras of the form 
$\Lambda[t]/f$, where $f$ is monic. 

\item
Let $\fm$ be the maximal ideal of $\Lambda$. Then $A\mapsto A/\fm$ is an 
equivalence of categories from finite \'etale $\Lambda$-algebras to 
finite \'etale $\Lambda/\fm$-algebras. 
\end{enumerate}
\end{lemma}
\begin{proof}
The conditions are equivalent by \cite[18.5.11]{ega4-4}. Recall that 
$\Lambda = \varprojlim \Lambda/\fa$, where $\fa$ ranges over closed ideals of 
finite index. Let $A$ be a pseudocompact $\Lambda$-algebra. For any 
ideal $\fa\subset \Lambda$, the ring $\Lambda/\fa$ is henselian by 
\cite[18.5.14]{ega4-4}, so $A/\fa$ is a product of local 
$\Lambda/\fa$-algebras. Moreover, by \cite[18.5.4]{ega4-4}, the map 
$A/\fa \to A/\fm$ is a bijection on idempotents. The inverse limit of 
these compatible systems of idempotents decompose $A$ into a product of local 
$\Lambda$-algebras. 
\end{proof}

Following Grothendieck, if $\cC$ is an arbitrary category, we write 
$\widehat\cC=\hom(\cC^\circ,\sets)$ for the category of contravariant functors 
$\cC\to \sets$. We regard $\cC$ as a full subcategory of $\widehat\cC$ via the 
Yoneda embedding, so for $X,Y\in \cC$, we write $X(Y) = \hom_\cC(Y,X)$. With 
this notation, the Yoneda Lemma states that $\hom_{\widehat\cC}(X,P) = P(X)$ 
for all $X\in \cC$. 

\begin{lemma}\label{thm:ind-object-def}
Let $\cX\in\widehat{\Art_\Lambda}$. Then $\cX$ is left exact if and only 
if there exists a filtered system $\{X_i\}_{i\in I}$ in $\cC_\Lambda$ together 
with a natural isomorphism $\cX(\cdot)\simeq \varinjlim X_i(\cdot)$. Write 
$\Ind(\Art_\Lambda)$ for the category of such functors. Then 
$\Ind(\Art_\Lambda)$ is closed under colimits, and the 
Yoneda embedding $\Art_\Lambda\monic\Ind(\Art_\Lambda)$ 
preserves filtered colimits.
\end{lemma}
\begin{proof}
This follows from the results of \cite[6.1]{kashiwara-schapira-2006}. 
\end{proof}

\begin{lemma}\label{lem:ind-cat-left-exact}
The functors $\Art_\Lambda \to \Ind(\Art_\Lambda) \to \widehat{\Art_\Lambda}$ 
are left exact.
\end{lemma}
\begin{proof}
This is \cite[6.1.17]{kashiwara-schapira-2006}. 
\end{proof}

If $R$ is a pseudocompact $\Lambda$-algebra, write $\spf(R)$ for the object of 
$\widehat{\Art_\Lambda}$ defined by $\spf(R)(A)=\hom_{\cts/\Lambda}(R,A)$, 
the set of continuous $\Lambda$-algebra homomorphisms. 

\begin{lemma}
The funtor $\spf$ induces an (anti-)equivalence between the category of 
pseudocompact $\Lambda$-algebras and $\Ind(\Art_\Lambda)$. 
\end{lemma}
\begin{proof}
This is \cite[VII\textsubscript{B} 0.4.2 Prop.]{sga3-1}. 
\end{proof}

So $\Ind(\Art_\Lambda)$ is the category of pro-representable 
functors on finite length $\Lambda$-algebras. \emph{Warning}: in many papers, 
for example the foundational \cite{mazur-1995}, one reserves the term 
\emph{pro-representable} for functors of the form $\spf(R)$, where $R$ is 
\emph{noetherian}. We do not make this restriction. 

\begin{lemma}\label{thm:exponential-ideal}
The category $\Ind(\Art_\Lambda)$ is an exponential ideal in 
$\widehat{\Art_\Lambda}$. 
\end{lemma}
\begin{proof}
By this we mean the following. Let $\cX\in \Ind(\Art_\Lambda)$, 
$P\in \widehat{\Art_\Lambda}$. Then the functor $\cX^P$ defined by 
\[
	\cX^P(S) = \hom_{\widehat{\Art_\Lambda}_{/S}}(P_{/S},\cX_{/S}) 
\]
is also in $\Ind(\Art_\Lambda)$. Given the characterization of 
$\Ind(\Art_\Lambda)$ as left exact functors, this is easy to 
prove, see e.g.~\cite[4.2.3]{johnstone-2002}. 
\end{proof}

If $\cC$ is a category, we write $\Gp(\cC)$ for the category of group 
objects in $\cC$. 

\begin{corollary}\label{thm:framed-deformation}
Let $\Gamma\in \Gp(\widehat{\Art_\Lambda})$ and 
$\cG\in \Gp(\Ind(\Art_\Lambda))$, then the functor 
$[\Gamma,\cG]$ defined by 
\[
	[\Gamma,\cG](S) = \hom_{\Gp/S}(\Gamma_{/S},\cG_{/S}) 
\]
is in $\Ind(\Art_\Lambda)$. In particular, if $\Gamma$ is a 
profinite group, then the functor 
\[
	[\Gamma,\cG](S) = \hom_{\cts/\Gp}(\Gamma,\cG(S))
\]
is in $\Ind(\Art_\Lambda)$. 
\end{corollary}
\begin{proof}
The first claim follows easily from Lemma \ref{thm:exponential-ideal} and 
Lemma \ref{lem:ind-cat-left-exact}. Just note that $[\Gamma,\cG]$ is the 
equalizer:
\[
\begin{tikzcd}
	{[\Gamma,\cG]} \ar[r]
		& \cG^\Gamma \ar[r, "m_\Gamma^\ast", shift left=.5ex] \ar[r, "m_{\cG\ast}"', shift right=.5ex]
		& \cG^{\Gamma\times \Gamma} ,
\end{tikzcd}
\]
that is, those $f\colon \Gamma\to \cG$ such that 
$f\circ m_\Gamma = m_\cG\circ(f\times f)$. The latter claim is just 
a special case. 
\end{proof}





\section{Quotients in the flat topology}

If $\Lambda$ is a pseudocompact ring, the category 
$\Ind(\Art_\Lambda)$ has nice ``geometric'' properties. However, 
for operations like taking quotients, we will embed it into the larger category 
$\Sh_\fl(\Art_\Lambda)$ of flat sheaves. We call a collection 
$\{U_i\to X\}$ of morphisms in $\Art_\Lambda$ a \emph{flat cover} if each 
ring map $\sO(X)\to \sO(U_i)$ is flat, and moreover 
$\sO(X)\to \prod \sO(U_i)$ is faithfully flat. By \cite[IV 6.3.1]{sga3-1}, this 
is a subcanonical Grothendieck topology on $\Art_\Lambda$. We call it the 
\emph{flat topology}, even though finite presentation comes for free because 
all the rings are finite length. 

\begin{lemma}
Let $\Sh_\fl(\Art_\Lambda)$ be the category of sheaves (of sets) on 
$\Art_\Lambda$ with respect to the flat topology. Then a presheaf 
$P\in \widehat{\Art_\Lambda}$ lies in $\Sh_\fl(\Art_\Lambda)$ if 
and only if $P(\coprod U_i) = \prod P(U_i)$ and moreover, whenever 
$U \to X$ is a flat cover where $\sO(U)$ and $\sO(X)$ are local rings, the 
sequence 
\[
\begin{tikzcd}
	P(X) \ar[r]
		& P(U) \ar[r, shift left=.5ex] \ar[r, shift right=.5ex]
		& P(U\times_X U) .
\end{tikzcd}
\]
is exact. Moreover, 
$\Ind(\Art_\Lambda)\subset \Sh_\fl(\Art_\Lambda)$. 
\end{lemma}
\begin{proof}
The first claim is the content of \cite[IV 6.3.1(ii)]{sga3-1}. For the second, 
note that any $\cX\in \Ind(\Art_\Lambda)$ will, by 
\ref{thm:ind-object-def}, convert (arbitrary) colimits into limits. Thus 
$\cX(\coprod U_i) = \prod \cX(U_i)$. If $U\to X$ is a flat cover, then by (loc.~cit.), $U\times_X U\rightrightarrows U\to X$ is a coequalizer diagram in 
$\Art_\Lambda$, hence 
$\cX(X)\to \cX(U)\rightrightarrows \cX(U\times_X U)$ is an equalizer. 
\end{proof}

Our main reason for introducing the category $\Sh_\fl(\Art_\Lambda)$ 
is that, as a (Grothendieck) topos, it is closed under arbitrary colimits. 
Recall that in an \emph{equivalence relation} in $\widehat{\Art_\Lambda}$ 
is a morphism $R\to X\times X$ such that, for all $S$, the map 
$R(S)\to X(S)\times X(S)$ is an injection whose image is an equivalence 
relation on $X(S)$. We define the quotient $X/R$ to be the coequalizer 
\[
\begin{tikzcd}
	R \ar[r, shift left=.5ex] \ar[r, shift right=.5ex]
		& X \ar[r]
		& X/R .
\end{tikzcd}
\]
By Giraud's Theorem \cite[App.]{maclane-moerdijk-1994}, for any 
$S\in \Art_\Lambda$, the natural map $X(S)/R(S)\to (X/R)(S)$ is injective. 
It will not be surjective in general. 

We let $\Sh_\fl(\Art_\Lambda)$ inherit definitions from 
$\Art_\Lambda$ as follows. If $P$ is a property of maps in 
$\Art_\Lambda$ (for example, ``flat,'' or ``smooth,'') and 
$f\colon X\to Y$ is a morphism in $\Sh_\fl(\Art_\Lambda)$, we say 
that $f$ has $P$ if for all $S\in \Art_\Lambda$ and $y\in Y(S)$, the 
pullback $X_S=X\times_Y S$ lies in $\Art_\Lambda$, and the pullback map 
$X_S\to S$ has property $P$. For example, if $X=\spf(R')$ and $Y=\spf(R)$, then 
$X\to Y$ has property $P$ if and only if for all finite length $A$ and 
continuous $\Lambda$-algebra maps $R\to A$, the induced map 
$A\to R'\otimes_R A$ has $P$.

\begin{theorem}\label{thm:quotients-ind}
Let $\cR\to \cX\times \cX$ be an equivalence relation in 
$\Ind(\Art_\Lambda)$ such that one of the maps $\cR\to \cX$ is 
flat. Then the quotient $\cX/\cR$ lies in $\Ind(\Art_\Lambda)$, and 
$\cX\to \cX/\cR$ is a flat cover. 
\end{theorem}
\begin{proof}
This is \cite[VII\textsubscript{B} 1.4]{sga3-1}. 
\end{proof}

By \cite[29.7]{matsumura-1989}, if $k$ is a field and $R$ is a complete regular 
local $k$-algebra, then $R\simeq k\pow{t_1,\dots,t_n}$. In particular, $R$ 
admits an augmentation $\epsilon\colon R\to k$. There is a general analogue of 
this result, but first we need a definition. 

\begin{definition}
A map $f\colon \cX\to \cY$ in $\Ind(\Art_\Lambda)$ is a 
\emph{residual isomorphism} if for all $S=\spf(k)\in \Art_\Lambda$ where 
$k$ is a field, the map $f\colon \cX(S)\to \cY(S)$ is a bijection. 
\end{definition}

\begin{lemma}\label{thm:smooth-section}
Let $f\colon \cX\to \cY$ be a smooth map in $\Ind(\Art_\Lambda)$ 
that is a residual isomorphism. Then $f$ admits a section. 
\end{lemma}
\begin{proof}
By \cite[VII\textsubscript{B} 0.1.1]{sga3-1}, it suffices to prove the result 
when $\cX=\spf(R')$, $\cY=\spf(R)$, for local $\Lambda$-algebras $R\to R'$ 
with the same residue field. Let $k=R/\fm_R \iso R'/\fm_{R'}$ be their common 
residue field. From the diagram 
\[
\begin{tikzcd}
	R' \ar[r, dotted] \ar[dr, two heads]
		& R \ar[d, two heads] \\
	R \ar[u] \ar[ur, equal] \ar[r, two heads]
		& k ,
\end{tikzcd}
\]
the definition of (formal) smoothness, and a limiting argument involving the 
finite length quotients $R/\fa$, we obtain the result. 
\end{proof}

\begin{corollary}\label{thm:quotients-good}
Let $\cR\to \cX\times \cX$ be an equivalence relation satisfying the hypotheses 
of Theorem \ref{thm:quotients-ind}. Suppose further that 
\begin{enumerate}
\item
One of the maps $\cR\to \cX$ is smooth, and 

\item
The projection $\cX\to \cX/\cR$ is a residual isomorphism. 
\end{enumerate}
Then $\cX\to \cX/\cR$ admits a section, so $\cX(S)/\cR(S)\iso (\cX/\cR)(S)$ 
for all $S\in \Art_\Lambda$. 
\end{corollary}
\begin{proof}
By \ref{thm:smooth-section}, it suffices to prove that $\cX\to \cX/\cR$ is 
smooth. By \cite[17.7.3(ii)]{ega4-4}, smoothness can be detected after flat 
descent. So base-change with respect to the projection $\cX\to \cX/\cR$. In the 
following commutative diagram 
\[
\begin{tikzcd}[column sep=small, row sep=small]
	\cR \ar[dr, equal] \ar[drr, bend left] \ar[ddr, bend right] \\
	& \cX\times_{\cX/\cR}\cX \ar[r] \ar[d]
		& \cX \ar[d] \\
	& \cX \ar[r]
		& \cX/\cR
\end{tikzcd}
\]
we can ensure the smoothness of $\cR\to \cX$ by our hypotheses. Since  
$\cX\to \cX/\cR$ is smooth after flat base-change, the original map is smooth. 
\end{proof}

\begin{example}
The hypothesis on residue fields in \ref{thm:quotients-good} is necessary. To 
see this, let $\Lambda=k$ be a field, $k\monic K$ a finite Galois extension 
with Galois group $G$. Then $G\times \spf(K)\rightrightarrows \spf(K)$ has 
quotient $\spf(k)$, but the map $\spf(K)(S)\to \spf(k)(S)$ is \emph{not} 
surjective for all $S\in \Art_k$, e.g.~it is not for $S=\spf(k)$. 
\end{example}

\begin{example}
The hypothesis of smoothness in Theorem \ref{thm:quotients-good} is necessary. 
To see this, let $k$ be a field of characteristic $p>0$. Then the formal 
additive group $\widehat\Ga=\spf(k\pow{t})$ has a subgroup $\balpha_p$ defined 
by 
\[
  \balpha_p(S) = \{s\in \sO(S)\colon s^p=0\} .
\]
The quotient $\widehat\Ga/\balpha_p$ has as affine coordinate ring 
$k\pow{t^p}$. In particular, the following sequence is exact in the flat 
topology:
\[
\begin{tikzcd}
	0 \ar[r]
		& \balpha_p \ar[r]
		& \widehat\Ga \ar[r, "(\cdot)^p"]
		& \widehat\Ga \ar[r]
		& 0 .
\end{tikzcd}
\]
It follows that 
$\balpha_p\times \widehat\Ga\rightrightarrows\widehat\Ga\xrightarrow{(\cdot)^p} \widehat\Ga$
is a coequalizer in $\Sh_\fl(\Art_k)$ satisfying all the hypothese 
of \ref{thm:quotients-good} except smoothness. And indeed, as one sees by 
letting $S=\spf(A)$ for any non-perfect $k$-algebra $A$, the map 
$(\cdot)^p\colon \widehat\Ga(S)\to \widehat\Ga(S)$ is \emph{not} surjective for 
all $S$. 
\end{example}





\section{Deformations of group representations}

Relate to \cite{bockle-2013}. 

Let $\Gamma\in \Gp(\widehat{\Art_\Lambda})$ and $\cG\in \Ind(\Art_\Lambda)$. By 
\ref{thm:framed-deformation}, the functor 
\[
  \Rep^\square(\Gamma,\cG)(S) = \hom_{\Gp/S}(\Gamma_S,\cG_S)
\]
is in $\Ind(\Art_\Lambda)$. We would like to define an ind-scheme 
$\Rep(\Gamma,\cG)$ as ``$\Rep^\square(\Gamma,\cG)$ modulo conjugation,'' but 
this requires some care. The conjugation action of $\cG$ on 
$\Rep^\square(\Gamma,\cG)$ will have fixed points, so the quotient will be 
badly behaved. We loosely follow \cite{tilouine-1996}. 

Assume $\Lambda$ is local, with maximal ideal $\fm$ and residue field $\kk$. 
Fix $\bar\rho\in \Rep^\square(\Gamma,\cG)(\kk)$, i.e.~a residual representation 
$\bar\rho\colon \Gamma\to \cG(\kk)$. Let $\Rep^\square(\Gamma,\cG)_{\bar\rho}$ 
be the connected component of $\bar\rho$ in $\Rep^\square(\Gamma,\cG)$. Assume 
that $\cG$ and $\zentrum(\cG)$ are smooth; then the quotient 
$\cG^\ad=\cG/\zentrum(\cG)$ is also smooth. Let $\cG^{\ad,\circ}$ be the 
connected component of $1$ in $\cG^\ad$. 

\begin{theorem}
Suppose $(\Lambda,\fm,\kk)$ is local. If $\cX,\cY\in \Ind(\Art_\Lambda)$ are 
connected and $\cX(\kk)\ne\varnothing$, then $\cX\times_\Lambda \cY$ is 
connected. 
\end{theorem}
\begin{proof}
We are reduced to proving the following result from commutative algebra: if 
$R,S$ are local pro-artinian $\Lambda$-algebras and $R$ has residue field 
$\kk$, then $R\widehat\otimes_\Lambda S$ is local. Since 
$R\widehat\otimes_\Lambda S = \varprojlim (R/\fr)\otimes_\Lambda (S/\fs)$, 
$\fr$ (resp.~$\fs$) ranges over all open ideals in $R$ (resp.~$S$), we may 
assume that both $R$ and $S$ are artinian. The rings $R$ and $S$ are 
henselian, so $R\otimes S$ is local if and only if 
$(R/\fm_R)\otimes (S/\fm_S) = S/\fm_S$ is local, which it is. 
\end{proof}

We conclude that the action of $\cG^{\ad,\circ}$ on $\Rep^\square(\Gamma,\cG)$ 
preserves $\Rep^\square(\Gamma,\cG)_{\bar\rho}$. Thus we may put 
\[
	\Rep(\Gamma,\cG)_{\bar\rho} = \Rep^\square(\Gamma,\cG)_{\bar\rho} / \cG^{\ad,\circ} .
\]
If $\cG^{\ad,\circ}$ acts faithfully on $\Rep^\square(\Gamma,\cG)_{\bar\rho}$, 
then we recover the classical notion of the deformation functor. 

\begin{theorem}
Let $\Gamma$ be a profinite group, $\bar\rho\colon \Gamma\to \cG(\kk)$ a 
representation with $\h^0(\Gamma,\Ad\bar\rho)=0$. Then 
$\Rep(\Gamma,\cG)_{\bar\rho}$ exists and is what you expect. 
\end{theorem}
\begin{proof}
To-do: this shouldn't be hard. 

Need assumptions on $\zentrum(\cG)$, $\cG$ should be smooth. 

Need $\zentrum(\cG)=\ker(\cG\to \GL(\fg))$ in connected case. This should use 
$\fg=\lie(\aut \cG)$, via deviations in \cite{sga3-1}. 

Recall first that $\Rep^\square(\Gamma,\cG)_{\bar\rho}$\ldots. Main things: 
need a residual isomorphism (this one can check directly) and faithful 
action (do this!). 
\end{proof}





\section{Tangent spaces and obstruction theory}

To-do: define tangent spaces, show that they're isomorphic to $\h^1(-)$. 

For $S_0\in \Art_\Lambda$, let $\Exal_{S_0}$ be the category of square-zero 
thickenings of $S_0$. An object of $\Exal_{S_0}$ is a closed embedding 
$S_0\hookrightarrow S$ whose ideal of definition has square zero. Should be 
``exponential exact sequence''
\[
\begin{tikzcd}
	0 \ar[r]
		& \fg(I) \ar[r]
		& \cG(S) \ar[r]
		& \cG(S_0) \ar[r]
		& 1
\end{tikzcd}
\]
This gives us a class $\exp\in \h^2(\cG(S_0),\fg(I))$. For 
$\rho_0\colon \Gamma\to \cG(S_0)$, the obstruction class is 
$o(\rho_0,I) = \rho_0^\ast(\exp)\in \h^2(\Gamma,\fg(I))$. It's easy to check 
that $o(\rho_0,I)=0$ if and only if $\rho_0$ lifts to $\rho$. So obstruction 
theory naturally for $\Rep^\square(\Gamma,\cG)$. 

[Use \cite[6.6.4]{weibel-1994}. Given setting as above, $\rho_0^\ast(\exp)$ is 
the pullback by $\rho_0$:
\[
\begin{tikzcd}
	0 \ar[r] 
		& \fg(I) \ar[r] \ar[d, equals]
		& \cG(S)\times_{\cG(S_0)} \Gamma \ar[r] \ar[d] 
		& \Gamma \ar[d, "\rho_0"] \ar[r]
		& 1 \\
	0 \ar[r] 
		& \fg(I) \ar[r]
		& \cG(S) \ar[r] 
		& \cG(S_0) \ar[r] 
		& 1
\end{tikzcd}
\]
Computing explicitly, we see the result. 
]

\begin{proposition}
Let $f\colon G\to H$ be a morphism of profinite groups. Suppose $M$ is a 
discrete $H$-module and $c\in \h^2(H,M)$ corresponds to the extension 
\[
\begin{tikzcd}
	0 \ar[r]
		& M \ar[r]
		& \widetilde H \ar[r]
		& H \ar[r] 
		& 1 .
\end{tikzcd}
\]
Then $f^\ast c=0$ in $\h^2(G,M)$ if and only if there is a map 
$\widetilde f\colon G\to \widetilde H$ making the following diagram commute: 
\[
\begin{tikzcd}[row sep=small]
	& \widetilde H \ar[dd] \\
	G \ar[ur, "\widetilde f"] \ar[dr, "f"] \\
	& H .
\end{tikzcd}
\]
\end{proposition}
\begin{proof}
By \cite[6.6.4]{weibel-1994}, the class $f^\ast c$ corresponds to the pullback 
diagram: 
\[
\begin{tikzcd}
	0 \ar[r]
		& M \ar[r] \ar[d, equals]
		& G\times_H \widetilde{H} \ar[r] \ar[d] 
		& G \ar[r] \ar[d, "f"]
		& 1 \\
	0 \ar[r]
		& M \ar[r]
		& \widetilde H \ar[r]
		& H \ar[r] 
		& 1 .
\end{tikzcd}
\]
Writing explicitly what it means for $G\times_H \widetilde H \to G$ to split 
yields the result. 
\end{proof}

% !TEX root = main.tex

\chapter{Constructing Galois representations}\label{ch:construct-Galois}





\section{Notation and necessary results}

In this chapter we loosely summarize, and adapt as needed, the results of 
\cite{khare-larsen-ramakrishna-2005,pande-2011}. Throughout, if $F$ is a field, 
$M$ a $G_F$-module, we write $\h^i(F,M)$ in place of $\h^1(G_F,M)$. All Galois 
representations will be to $\GL_2(\bZ/l^n)$ or $\GL_2(\bZ_l)$ for $l$ a (fixed) 
rational prime, and all deformations will have fixed determinant, so we only 
consider the cohomology of $\Ad^0\bar\rho$, the induced representation on 
trace-zero matrices by conjugation. 

If $S$ is a set of rational primes, $\bQ_S$ denotes the largest extension of 
$\bQ$ unramified outside $S$. So $\h^i(\bQ_S,-)$ is what is usually written as 
$\h^1(G_{\bQ,S},-)$. If $M$ is a $G_\bQ$-module and $S$ a finite set of primes, 
write 
\[
	\sha^i_S(M) = \ker\left( \h^i(\bQ_S,M) \to \prod_{p\in S} \h^i(\bQ_p,M)\right) .
\]
If $l$ is a rational prime and $S$ a finite set of primes containing $l$, then 
for any $\bF_l[G_{\bQ_S}]$-module $M$, write $M^\vee=\hom_{\bF_l}(M,\bF_l)$ 
with the obvious $G_{\bQ_S}$-action, and write $M^\ast = M^\vee(1)$ for the 
Cartier dual. By \cite[Th.~8.6.7]{neukirch-schmidt-winberg-2008}, there is an 
isomorphism $\sha^1_S(M^\ast) = \sha_S^2(M)^\vee$. 

\begin{definition}
A \emph{good residual representation} is an odd, absolutely irreducible, 
weight-$2$ representation $\bar\rho\colon G_{\bQ_S} \to \GL_2(\bF_l)$, where 
$l\geqslant 7$ is a rational prime. 
\end{definition}

Roughly, ``good residual representations'' have enough properties that we can 
prove quite a lot about their lifts. By results of Khare--Wintenberger, we know 
that good residual representations have characteristic-zero lifts. Even better, 
they admit $\bZ_l$-lifts. 

\begin{theorem}\label{thm:always-can-lift}
Let $\bar\rho\colon G_{\bQ_S} \to \GL_2(\bF_l)$ be a good residual 
representation. Then there exists a weight-$2$ lift of $\bar\rho$ to $\bZ_l$. 
\end{theorem}
\begin{proof}
This is \cite[Th.~1]{ramakrishna-2002}, taking into account that the paper in 
question allows for arbitrary fixed determinants. 
\end{proof}

\begin{definition}
Let $\bar\rho\colon G_{\bQ_S} \to \GL_2(\bF_l)$ be a good residual 
representation. A prime $p\not\equiv \pm 1\pmod l$ is \emph{nice} if 
$\Ad^0\bar\rho\simeq \bF_l \oplus \bF_l(1)\oplus \bF_l(-1)$, i.e.~if the 
eigenvalues of $\bar\rho(\frob_p)$ have ratio $p$. 
\end{definition}

\begin{theorem}
Let $\bar\rho$ be a good residual representation and $p$ a nice prime. Then 
any deformation of $\left.\bar\rho\right|_{G_{\bQ_p}}$ is induced by 
$G_{\bQ_p} \to \GL_2(\bZ_l\pow{a,b} / \langle a b\rangle)$, sending 
\[
	\frob_p \mapsto \smat{p(1+a)}{}{}{(1+a)^{-1}} \qquad \tau_p \mapsto \smat{1}{b}{}{1} ,
\]
where $\tau_p\in G_{\bQ_p}$ is a generator for tame inertia. 
\end{theorem}
\begin{proof}
This is mentioned in KLR, find the real proof. 
\end{proof}

We close this section by introducing some new terminology and notation to 
condense the lifting process used in \cite{khare-larsen-ramakrishna-2005}. 

Fix a good residual representation $\bar\rho$. We will consider weight-$2$ 
deformations of $\bar\rho$ to $\bZ/l^n$ and $\bZ_l$. Call such a deformation a 
``lift of $\bar\rho$ to $\bZ/l^n$ (resp.~$\bZ_l$).'' We will often restrict the 
local behavior of such lifts, i.e.~the restrictions of a lift to $G_{\bQ_p}$ 
for $p$ in some set of primes. The necessary constraints are captured in the 
following definition. 

\begin{definition}
Let $\bar\rho$ be a good residual representation, $h\colon \bR^+ \to \bR^+$ a 
function decreasing to zero. An \emph{$h$-bounded lifting datum} is a tuple 
$(\rho_n,R,U,\{\rho_p\}_{p\in R\cup U})$, where 
\begin{enumerate}
\item
$\rho_n\colon G_{\bQ_R} \to \GL_2(\bZ/l^n)$ is a lift of $\bar\rho$.

\item
$R$ and $U$ are finite sets of primes, $R$ containing $l$ and all primes at 
which $\rho_n$ ramifies. 

\item
$\pi_R(x)\leqslant h(x)\pi(x)$ for all $x$. 

\item
$\sha_R^1(\Ad^0\bar\rho) = \sha_R^2(\Ad^0\bar\rho) = 0$. 

\item
For all $p\in R\cup U$, 
$\rho_p\equiv \left. \rho_n\right|_{G_{\bQ_p}}\pmod{l^n}$. 

\item
For all $p\in R$, $\rho_p$ is ramified. 

\item
$\rho_n$ admits a lift to $\bZ/l^{n+1}$. 
\end{enumerate}
\end{definition}

If $(\rho_n,R,U,\{\rho_p\})$ is an $h$-bounded lifting datum, we call 
another $h$-bounded lifting datum $(\rho_{n+1},R',U',\{\rho_p\})$ a \emph{lift 
of $(\rho_n,R,U,\{\rho_p\})$} if $U\subset U'$, $R\subset R'$, and for all 
$p\in R\cup U$, the two possible ``$\rho_p$'' agree. 

\begin{theorem}\label{thm:lifting-datum}
Let $\bar\rho$ be a good residual representation, $h\colon \bR^+ \to \bR^+$ 
decreasing to zero. If $(\rho_n,R,U,\{\rho_p\})$ is an $h$-bounded lifting 
datum, $U'\supset U$ is a finite set of primes disjoint from $R$, and 
$\{\rho_p\}_{p\in U'}$ extends $\{\rho_p\}_{p\in U}$, then there exists an 
$h$-bounded lift $(\rho_{n+1},R',U',\{\rho_p\})$ of 
$(\rho_n,R,U,\{\rho_p\})$. 
\end{theorem}
\begin{proof}
Note that we do not bound the size of $R'\smallsetminus R$. It is possible that 
this can be done, using unpublished results of Ramakrishna, but that is not 
necessary for the results that follow. 

By \cite[Lem.~8]{khare-larsen-ramakrishna-2005}, there exists a finite set 
$N$ of what they call \emph{nice primes}, such that the map 
\begin{equation}\label{eq:h1-isom}
	\h^1(\bQ_{R\cup N},\Ad^0\bar\rho) \to \prod_{p\in R} \h^1(\bQ_p,\Ad^0\bar\rho) \times \prod_{p\in U'} \h_\nr^1(\bQ_p,\Ad^0\bar\rho) 
\end{equation}
is an isomorphism. In fact, $\# N = h^1(\bQ_{R\cup N},\Ad^0\bar\rho^\ast)$, and 
the primes in $N$ are chosen, one at a time, from Chebotarev sets. This means we 
can force them to be large enough to ensure that the bound 
$\pi_{R\cup N}(x) \leqslant h(x) \pi(x)$ continues to hold. 

By our hypothesis, $\rho_n$ admits a lift to $\bZ/l^{n+1}$; call one such lift 
$\rho^\ast$. For each $p\in R\cup U'$, $\h^1(\bQ_p,\Ad^0\bar\rho)$ acts simply 
transitively on lifts of $\left.\rho_n\right|_{G_{\bQ_p}}$ to $\bZ/l^{n+1}$. In 
particular, there are cohomology classes $f_p\in \h^1(\bQ_p,\Ad^0\bar\rho)$ 
such that $f_p\cdot \rho^\ast \equiv \rho_p\pmod{l^{n+1}}$ for all 
$p\in R\cup U'$. Moreover, for all $p\in U'$, the class $f_p$ is unramified. 
Since the map in \eqref{eq:h1-isom} is an isomorphism, there exists 
$f\in \h^1(\bQ_{R\cup N},\Ad^0\bar\rho)$ such that 
$\left.f\cdot \rho^\ast\right|_{G_{\bQ_p}}\equiv \rho_p\pmod{l^{n+1}}$ for all 
$p\in R\cup U'$. 

Clearly $\left. f\cdot \rho^\ast\right|_{G_{\bQ_p}}$ admits a lift to $\bZ_l$ 
for all $p\in R\cup U'$, but it does not necessarily admit such a lift for 
$p\in N$. By repeated applications of \cite[Prop.~3.10]{pande-2011}, there 
exists a set $N'\supset N$, with $\# N'\leqslant 2\# N$, of nice primes and 
$g\in \h^1(\bQ_{R\cup N'},\Ad^0\bar\rho)$ such that 
$(g+f)\cdot \rho^\ast$ still agrees with $\rho_p$ for $p\in R\cup U'$, and 
$(g+f)\cdot \rho^\ast$ is nice for all $p\in N'$. As above, the primes in $N'$ 
are chosen one at a time from Chebotarev sets, so we can continue to ensure the 
bound $\pi_{R\cup N'}(x)\leqslant h(x) \pi(x)$. Let 
$\rho_{n+1} = (g+f) \cdot \rho^\ast$. Let $R' = R\cup N'$. For each 
$p\in R'\smallsetminus R$, choose a ramified lift $\rho_p$ of 
$\left. \rho_{n+1}\right|_{G_{\bQ_p}}$ to $\bZ_l$. 

Since $\left.\rho_{n+1}\right|_{G_{\bQ_p}}$ admits a lift to $\bZ/l^{n+2}$ (in 
fact, it admits a lift to $\bZ_l$) for each $p$, and 
$\sha_{R'}^2(\Ad^0\bar\rho) = 0$, the deformation $\rho_{n+1}$ admits a lift to 
$\bZ/l^{n+2}$. Thus $(\rho_{n+1},R',U',\{\rho_p\})$ is the desired lift of 
$(\rho_n,R,U,\{\rho_p\})$. 
\end{proof}





\section{Galois representations with specified Satake parameters}

Fix a good residual representation $\bar\rho$. We 
consider weight-$2$ deformations of $\bar\rho$. The final deformation, 
$\rho\colon G_\bQ \to \GL_2(\bZ_l)$, will be constructed as the inverse limit 
of a compatible collection of lifts $\rho_n\colon G_\bQ \to \GL_2(\bZ/l^n)$. At 
any given stage, we will be concerned with making sure that a) there exists a 
lift to the next stage, and b) there is a lift with the necessary properties. 
Fix a sequence $\bx=(x_1,x_2,\dots)$ in $[-1,1]$. The set of unramified primes 
of $\rho$ is not determined at the beginning, but at each stage there will be 
a large finite set $U$ of primes which we know will remain unramified. 
Re-indexing $\bx$ by these unramified primes, we will construct $\rho$ so that 
for all unramified primes $p$, $\tr\rho(\frob_p)\in \bZ$, satisfies the Hasse 
bound, and has $\tr\rho(\frob_p) \approx x_p$. Moreover, we can ensure that the 
set of ramified primes has density zero in a very strong sense (controlled by a 
parameter function $h$) and that our trace of Frobenii are very close to 
specified values (the ``closeness'' again controlled by a parameter function 
$b$). 

Given any deformation $\rho$, write $\pi_{\ram(\rho)}(x)$ for the function 
which counts $\rho_n$-ramified primes $\leqslant x$. 

\begin{theorem}\label{thm:master-Galois}
Let $l$, $\bar\rho$, $\bx$ be as above. Fix functions 
$h\colon \bR^+\to \bR^+$ (resp.~$b\colon \bR^+ \to \bR_{\geqslant 1}$) which 
decrease to zero (resp.~increase to infinity). Then there exists a weight-$2$ 
deformation $\rho$ of $\bar\rho$, such that 
\begin{enumerate}
\item
$\pi_{\ram(\rho)}(x) \ll h(x) \pi(x)$. 

\item
For each unramified prime $p$, $a_p=\tr\rho(\frob_p)\in \bZ$ and satisfies the 
Hasse bound. 

\item
For each unramified prime $p$, 
$\left| \frac{a_p}{2\sqrt p} - x_p\right| \leqslant \frac{l b(p)}{2\sqrt p}$. 
\end{enumerate}
\end{theorem}
\begin{proof}
Begin with $\rho_1= \bar\rho$. By \cite[Lem.~6]{khare-larsen-ramakrishna-2005}, 
there exists a finite set $R$, containing the set of primes at which $\bar\rho$ 
ramifies, such that $\sha_R^1(\Ad^0\bar\rho) = \sha_R^2(\Ad^0\bar\rho) = 0$. 
Let $R_2$ be the union of $R$ and all primes $p$ with 
$\frac{l}{2\sqrt p} > 2$. For all $p\notin R_2$ and any $a\in \bF_l$, there 
exists $a_p\in \bZ$ satisfying the Hasse bound with $a_p\equiv a\pmod l$. In 
fact, given any $x_p\in [-1,1]$, there exists $a_p\in \bZ$ satisfying the Hasse 
bound such that 
$\left| \frac{a_p}{2\sqrt p} - x_p\right| \leqslant \frac{l}{2\sqrt p}$.
Choose, for all primes $p\in R_2$, a ramified 
lift $\rho_p$ of $\left. \rho_1\right|_{G_{\bQ_p}}$. Let $U_2$ be the set of 
primes not in $R_2$ such that 
$\frac{l^2}{2\sqrt p} > \min\left(2, \frac{l b(p)}{2\sqrt p}\right)$. 
For each $p\in U_2$, there exists $a_p\in \bZ$, satisfying the 
Hasse bound, such that 
\[
	\left| \frac{a_p}{2\sqrt p} - x_p\right| \leqslant \frac{l}{2\sqrt p} \leqslant \frac{l b(p)}{2\sqrt p} ,
\]
and moreover $a_p\equiv \tr\bar\rho(\frob_p)\pmod l$. For each $p\in U_2$, let 
$\rho_p$ be an unramified lift of $\left.\bar\rho\right|_{G_{\bQ_p}}$ with 
$a_p\equiv\tr\rho_p(\frob_p)\pmod l$. It may not be that 
$\pi_{R_2}(x) \leqslant h(x) \pi(x)$ for all $x$, but there is a scalar 
multiple $h^\ast$ of $h$ so that $\pi_{R_2}(x) \leqslant h^\ast(x) \pi(x)$ for 
all $x$. 

We have constructed our first $h^\ast$-bounded lifting datum 
$(\rho_1,R_2,U_2,\{\rho_p\})$. We proceed to construct 
$\rho = \varprojlim \rho_n$ inductively, by constructing a new $h^\ast$-bounded 
lifting datum for each $n$. We ensure that $U_n$ contains all primes for which 
$\frac{l^n}{2\sqrt p} > \min\left(2, \frac{l b(p)}{2\sqrt p}\right)$, so there 
are always integral $a_p$ satisfying the Hasse bound which satisfy any 
mod-$l^n$ constraint, and that can always choose these $a_p$ so as to preserve 
statement 2 in the theorem. 

The base case is already complete, so suppose we are given 
$(\rho_n,R_n,U_n,\{\rho_p\})$. We may assume that $U_n$ contains all primes for 
which $\frac{l^n}{2\sqrt p} > \min\left(2, \frac{l b(p)}{2\sqrt p}\right)$. Let 
$U_{n+1}$ be the set of all primes not in $R_n$ such that 
$\frac{l^{n+1}}{2\sqrt p} > \min\left(2, \frac{l b(p)}{2\sqrt p}\right)$. For 
each $p\in U_{n+1}\smallsetminus U_n$, there is an integer $a_p$, satisfying 
the Hasse bound, such that $a_p\equiv \rho_n(\frob_p)\pmod{l^n}$, and moreover 
$\left|\frac{a_p}{2\sqrt p} - x_p\right| \leqslant \frac{l b(p)}{2\sqrt p}$. 
For such $p$, let $\rho_p$ be an unramified lift of 
$\left. \rho_n\right|_{G_{\bQ_p}}$ such that 
$a_p\equiv\tr\rho_n(\frob_p)\pmod{l^n}$. By Theorem \ref{thm:lifting-datum}, 
there exists an $h^\ast$-bounded lifting datum 
$(\rho_{n+1},R_{n+1},U_{n+1},\{\rho_p\})$ extending and lifting 
$(\rho_n,R_n,U_n,\{\rho_p\})$. This completes the inductive step.  
\end{proof}

% !TEX root = main.tex

\chapter{Counterexample via Diophantine Approximation}

% !TEX root = main.tex

\chapter{Direct counterexample}





\section{Main ideas}

This chapter has two parts. First, for any reasonable measure $\mu$ on 
$[0,\pi]$ invariant under the same ``flip'' automorphism as the Sato--Tate 
measure, there is a sequence $\{a_p\}$ of integers satisfying the Hasse 
bound $|a_p|\leqslant 2\sqrt p$, such that for 
$\theta_p = \cos^{-1}\left(\frac{a_p}{2\sqrt p}\right)$, the discrepancy 
$\disc(\{\theta_p\}_{p\leqslant x},\mu)$ behaves like $x^{-\alpha}$ for 
predetermined $\alpha\in (0,1/2]$, while for any smooth $f$ satisfying 
$f(\pi-\theta) = -f(\theta)$ (and hence $\int f\, \dd\mu = 0$), the 
strange Dirichlet series $L_f(\{\theta_p\},s)$ satisfies the Riemann 
Hypothesis. 

In the second part of this chapter, we associate (infinitely ramified) Galois 
representations to the fake Satake parameters above, using techniques from 
\cite{pande-2011,khare-larsen-ramakrishna-2005}. 





\section{Construction}

To begin with, let $\mu$ be a probability measure on $[0,\pi]$ such that 
$\cos_\ast \mu$, supported on $[-1,1]$, has continuous, strictly increasing 
cdf, and fix $\alpha\in (0,1/2)$. Then we can apply Theorem 
\ref{thm:discrepancy-arbitrary} to find a sequence $\bx$ such that 
$\disc(\bx^N,\cos_\ast \mu) = \Theta(\pi(N)^{-\alpha})$. For each prime $p$, 
there exists an integer $a_p$ such that $|a_p|\leqslant 2\sqrt p$ and 
$\left| \frac{a_p}{2\sqrt p} - x_p\right| \leqslant p^{-1/2}$. Let 
$y_p = \frac{a_p}{2\sqrt p}$. Now apply 
Lemma \ref{lem:disc-of-two-seq} with $\epsilon = N^{-1/2}$. We obtain 
\[
	\left| \disc(\bx^N,\cos_\ast \mu) - \disc(\by^N, \cos_\ast \mu)\right| \ll  N^{-1/2} + \frac{\pi(N^{1/2})}{\pi(N)} ,
\]
which tells us that $\disc(\by^N,\cos_\ast\mu) = \Theta(\pi(N)^{-\alpha})$. 
Now let $\btheta = \cos^{-1}(\by)$. Then we see that 
$\disc(\btheta^N,\mu) = \Theta(\pi(N)^{-\alpha})$. 

We can improve this example by controlling the behavior of sums of the form 
$\sum_{p\leqslant N} f(\theta_p)$, at least for ``odd'' $f$. 

Let $\sigma$ be the automorphism of $[0,\pi]$ given by 
$\sigma(\theta) = \pi-\theta$. Note that $\sigma_\ast \ST = \ST$. 
Fix a measure $\mu$ on $[0,\pi]$ such that $\sigma_\ast \mu = \mu$. If $f$ is a 
function on $[0,\pi]$ satisfying $f\circ \sigma = - f$, then $\mu(f) = 0$. Let 
$\nu = \cos_\ast \left. \mu\right|_{[0,\pi/2]}$; this is a measure on $[0,1]$. 
Assume that $\nu$ has a bounded continuous density function. (For example, this 
is true if $\nu = \ST$.) Finally, fix $\alpha\in (0,1/2)$. Then we can apply 
Theorem \ref{thm:discrepancy-arbitrary} to find a sequence $\bx$ such that 
$\disc(\bx^N,\nu) = \Theta(\pi(N)^{-\alpha})$. For each prime $p$, there 
exists an integer $a_p\in \bZ$ such that $|a_p| \leqslant 2\sqrt p$ and 
$\left| \frac{a_p}{2\sqrt p} - x_p\right| \leqslant p^{-1/2}$. Applying 
? to the sequence $\ba$ with $\epsilon = ?$, we get 
\[
	\disc(\ba^N,\nu) = \ldots .
\]
We'd like to do better. 





\section{Associated Galois representation}



Fix, for the remainder of this section, a continuous representation 
\[
	\bar\rho_l \colon G_\bQ \to \GL_2(\bF_l) .
\]
For each $p$ at which $\bar\rho_l$ is unramified, we write 
\[
	\Theta_p(\bar\rho_l) = \left\{\cos^{-1}\left(\frac{a}{2\sqrt p}\right) : a\in \bZ\text{, }|a|\leqslant 2\sqrt p\text{, and }a\equiv \tr\bar\rho_l(\frob_p)\pmod l\right\} .
\]
For the finitely many primes $p$ for which $\Theta_p(\bar\rho_l)$ is empty, 
redefine $\Theta_p(\bar\rho_l)$ to include some elements for which 
$|a|>2\sqrt p$. We have a sequence of $\Theta_p(\bar\rho_l)$ for which at most 
finitely many do not satisfy the Hasse bound. 

\begin{theorem}
There exists a choice of $\theta_p\in \Theta_p(\bar\rho_l)$ for odd-indexed 
primes $\{2,5,11,\dots\}$ such that 
\begin{enumerate}
\item
$\theta_p\in [0,\pi/2)$ for all but finitely many $p$. 

\item
$\disc\left(\btheta_\odd^N, \left.\ST\right|_{[0,\pi/2)}\right) \to 0$, 
but is not $\ll N^{-\epsilon}$ for any $\epsilon>0$. 
\end{enumerate}
\end{theorem}
\begin{proof}
This is intuitively obvious, but a bit tricky to prove rigorously. 

Two key ideas: 

1. If we're given a ``bad'' finite distribution $\nu$, we can choose ``good'' 
$\theta_p$'s to make the combined distribution close enough (discrepancy-wise) 
to $\ST$. 

2. If we're given a ``good'' finite distribution $\nu$, we can choose ``bad'' 
$\theta_p\sim \pi/2$ to make the combined distribution far away 
(discrepancy-wise) from $\ST$. 
\end{proof}

Claim: let $\mu,\nu$ be two absolutely continuous distributions. Suppose there 
is a sequence $\{T_p\}$ of $\mu$-distributed sets, such that 
$\disc(T_p,\mu) \ll p^{-1/2}$. Suppose moreover that $\mu/\nu$ is bounded away 
from zero (at the pdf side). Then we can choose $t_p\in T_p$ so that 
$\{t_p\}$ is $\nu$-equidistributed with good discrepancy. 

Let $\mu$ be an absolutely continuous measure on $[0,\pi]$ such that the 
pushforward $\cos_\ast \mu$ is bounded (this is true for the Sato--Tate 
measure). Fix a prime $l\geqslant 5$ and a constant $\alpha\in (0,1/2]$. We 
want to construct a weight-$2$ Galois representation 
$\rho_l\colon G_\bQ \to \GL_2(\bZ_l)$, ramified at a density zero set of 
primes, such that 
\begin{enumerate}
\item
If $\rho_l$ is unramified at $p$, then $a_p = \tr \rho_l(\frob_p)\in \bZ$ and 
satisfies the Hasse bound $|a_p| \leqslant 2\sqrt p$. 

\item
If we write $\theta_p = \cos^{-1}(a_p / 2\sqrt p)$ for the Satake parameters at 
unramified primes, then $\disc(\btheta^N, \mu) \ll N^{-\alpha+\epsilon}$ and 
$\disc(\btheta^N,\mu) = \Omega(N^{-\alpha-\epsilon})$. 
\end{enumerate}

Recall the van der Corput sequence $\{x_p\}$ satisfies 
$\disc(\bx^N) \ll N^{-1+\epsilon}$. Let $\nu = \cos_\ast \mu$; this is an 
absolutely continuous measure supported on $[-1,1]$. By transforming the 
van der Corput sequence by a continuous map, we may assume that in fact 
$\disc(\bx^N,\nu) \ll N^{-1+\epsilon}$. In fact, by alternating between 
``van der Corput elements'' and ``bad elements'' we can ensure that not 
only does $\disc(\bx^N,\nu) \ll N^{-\alpha+\epsilon}$, but also 
$\disc(\bx^N,\nu) = \Omega(N^{-\alpha-\epsilon})$. 

We start by choosing a modular mod-$l$ representation 
$\rho_1\colon G_\bQ \to \GL_2(\bZ/l)$, which is ramified at a finite set of 
primes $S_1$. Let $R_1 = \{p\leqslant r_1 : p \notin S_1\}$. For $p\in R_1$, 
we can choose $a_p\in \bZ$ subject only to the condition 
$a_p \equiv \tr \rho_1(\frob_p)\pmod l$. For any $p\in R_1$, the set 
\[
	T_p(l) = \left\{\frac{a}{2\sqrt p} : |a| \leqslant 2\sqrt p\text{ and }a\equiv \tr \rho_1(\frob_p)\pmod l\right\}
\]
has an element within $l p^{-1/2}$ of any element of $[-1,1]$. Choose 
$a_p\in T_p(l)$ so that 
$\left|\frac{a_p}{2\sqrt p} - x_p\right| \leqslant l p^{-1/2}$. It follows 
that for $p\in R_1$, we have 
\[
	|\disc(\{a_p/2\sqrt p\}_{p\leqslant N}, \nu) - \disc(\bx^N,\nu)| \ll l N^{-1/2}
\]
We get a lift of $\rho_1$ to $\rho_2\colon G_\bQ \to \GL_2(\bZ/l^2)$ respecting 
our choices of the $a_p$ for $p\in R_1$, which is ramified at one (perhaps two) 
extra primes. 


What happens next is in stages. We'll already have a mod-$l^{n+1}$ 
representation $\rho_{n+1}\colon G_\bQ \to \GL_2(\bZ/l^n)$, together with 
choices of $a_p$ for $p\in R_1\cup \cdots \cup R_n$ that ensure 
$|\disc(\{a_p / 2\sqrt p\}_{p\leqslant N}, \nu) - \disc(\bx^N,\nu)| \ll ?$


The main question is: how do we choose $r_1$, and the later $r_n$? We ensure 
that a) the set $T_p(l^n)$ are non-empty, and that b) $l^n < \log(r_n)$. This 
gives us that for $N\leqslant r_n$, we have 
\[
	|\disc(\{a_p / 2\sqrt p\}_{p\leqslant N}, \nu) - \disc(\bx^N,\nu)| \ll N^{-\frac 1 2+\epsilon} .
\]

Todo: can I make $\sum a_p = ?$ anything from $-\infty$ to $\infty$?

What if I make a fake modular form with these ``bad'' Satake parameters? What 
can I say about it?





\section{Informal approach}

This discussion is inspired by \cite{pande-2011}. Throughout, all Galois 
representations have weight $2$, i.e.~determinant is the cyclotomic power. 

Fix a prime $l\geqslant 5$ and a (modular) representation 
$\rho_1\colon G_\bQ \to \GL_2(\bZ/l)$. We claim that there is a finite set 
$S$ such that $\sha_S^1(\Ad^0\rho_1) = \sha_S^2(\Ad^0 \rho_1) = 0$. Moreover, 
all the local deformation spaces are smooth? (Why?)

Set $S=S_2$. Choose lifts $\rho_p\colon G_{\bQ_p} \to \GL_2(\bZ_l)$ of 
$\left.\rho_1\right|_{G_{\bQ_p}}$ for all $p\in S_2$. We can ensure that the 
$\rho_p$ are ramified. (Can we also ensure that their characteristic 
polynomials are well behaved? By \cite{khare-rajan-2001}, these 
characteristic polynomials are well-defined for all but finitely many primes.)
Now let $R_2 = \{p\notin S_2 : p\leqslant r_2\}$, where $r_2$ is a yet 
unspecified large constant (say $l^{l^{100}}$). Choose $a_p$ for all 
$p\in R_2$. By \cite[Lem.~5.1]{pande-2011}, there is a set $Q_2$ (bound the 
size of $Q_2$!) 

% !TEX root = main.tex

\chapter{Computational evidence for the Akiyama--Tanigawa conjecture}

% !TEX root = main.tex

\chapter{Concluding remarks and future directions}






\printbibliography
\end{document}
