\documentclass[phd,tocprelim]{cornell}
\let\ifpdf\relax 

\usepackage{
	amsmath,
	amssymb,
	amsthm,
	mathpazo, % Palatino font
	stmaryrd, % llbracket
	thesis,   % custom math commands
	tikz-cd   % commutative diagrams
}
\SetSymbolFont{stmry}{bold}{U}{stmry}{m}{n} % Kills stmaryrd warning.
\usepackage[hidelinks]{hyperref}

\tolerance=9999

\bibliographystyle{alpha}

\renewcommand{\topfraction}{0.85}
\renewcommand{\textfraction}{0.1}
\renewcommand{\floatpagefraction}{0.75}

\title{Counterexamples related to the Sato--Tate conjecture}
\author{Daniel Miller}
\conferraldate{May}{2017}

\begin{document}

\maketitle
\makecopyright





\begin{abstract}
Let $E_{/\bQ}$ be an elliptic curve. The Sato--Tate conjecture, now a theorem, 
tells us that the angles $\theta_p =\cos^{-1}\left(\frac{a_p}{2\sqrt p}\right)$ 
are equidistributed in $[0,\pi]$ with respect to the measure 
$\frac{2}{\pi}\sin^2\theta\, \dd\theta$ if $E$ is non-CM
(resp.~$\frac{1}{2\pi} \dd \theta + \frac 1 2 \delta_{\pi/2}$ if $E$ is CM). 
In the non-CM case, Akiyama and Tanigawa conjecture that the discrepancy 
\[
	D_N = \sup_{x\in [0,\pi]} \left| \frac{1}{\pi(N)} \sum_{p\leqslant N} 1_{[0,x]}(\theta_p) - \int_0^x \frac{2}{\pi}\sin^2\theta\, \dd\theta\right| 
\]
asymptotically decays like $N^{-\frac 1 2+\epsilon}$, as is suggested by computational 
evidence and certain reasonable heuristics on the Kolmogorov--Smirnov 
statistic. This conjecture implies the Riemann hypothesis 
for all $L$-functions associated with $E$. It is natural to assume that the 
converse (``generalized Riemann hypothesis implies discrepancy estimate'') holds, 
as is suggested by analogy with Artin $L$-functions. We construct, for CM abelian 
varieties, ``fake Satake parameters'' yielding $L$-functions which satisfy 
the generalized Riemann hypothesis, but for which the discrepancy decays like 
$N^{-\epsilon}$ for any fixed $\epsilon>0$. This provides evidence that in the 
CM case, the converse to ``Akiyama--Tanigawa conjecture implies generalized 
Riemann hypothesis'' does not hold. 

We also show that there are Galois representations 
$\rho\colon \Gal(\overline \bQ /\bQ) \to \GL_2(\bZ_l)$, ramified at an 
arbitrarily thin (but still infinite) set of primes, whose Satake parameters 
can be made to converge at any specified rate to any fixed measure $\mu$ on 
$[0,\pi]$ for which $\cos_\ast\mu$ is absolutely continuous with bounded 
derivative. 
\end{abstract}





\begin{biosketch}
Daniel Miller was born in St.~Paul, Minnesota. He completed his Bachelor of 
Science at the University of Nebraska Omaha. In addition to his studies there, 
he played the piano competitively and attended Cornell's Summer Mathematics 
Institute. He started his Ph.D.~at Cornell planning on a career in academia. 
Halfway through he had a change of heart, and will be joining Microsoft's 
Analysis and Experimentation team as a data scientist after graduation. He is 
happily married to Ivy Lai Miller, and owns a cute but grumpy cat named Socrates. 
\end{biosketch}





\begin{dedication}
This thesis is dedicated to my undergraduate adviser, Griff Elder. He is the 
reason I considered a career in mathematics, and his infectious enthusiasm 
for number theory has inspired me more than I can say. 
\end{dedication}





\begin{acknowledgements}
I could not have completed this thesis without help and support from many 
people. I would like to offer my sincerest thanks to the following people, and 
my sincerest apologies to anyone whose name I have forgotten to include here. 

My parents Jay and Cindy, for noticing and fostering my mathematical 
interests early on, and for being unfailingly loving and supportive. 

My undergraduate thesis advisor, Griffith Elder. Without his encouragement 
and inspiration I probably would have never considered a career in math. 

Tara Holm, Jason Boynton, and Anthony Weston, for making Cornell's 2011 Summer 
Mathematics Institute the fantastic experience it was. 

My fellow graduate students Sasha Patotski, Bal\'azs Elek, and Sergio Da 
Silva, for sharing my early love of algebraic geometry, laughing with me at the 
absurdities of academic life, and listening to my ramblings about number theory. 

The mathematics department at Cornell, where many professors were generous 
with their time and ideas. I appreciate Yuri Berest, John Hubbard, Farbod 
Shokrieh, Birget Speh, and David Zywina for letting me bounce ideas off them, 
helping me add rigor to half-baked ideas, and pointing my research in new and 
interesting directions. 

My adviser Ravi Ramakrishna. He kindled my first love for number theory, 
stayed supportive as my research bounced all over the place, and kept me 
focused, grounded, and concrete when I needed to be. 

Most importantly, my wife Ivy for being there for me through the highs 
and the lows, when I prematurely thought my thesis was complete, and when I 
thought my results were completely in shambles. I couldn't have done it without 
her. 
\end{acknowledgements}





\contentspage





\symlist

\begin{tabular}{ll}
$1_S$     
	& characteristic function of a set $S$. \\
$l$       
	& rational prime $\geqslant 7$. \\
$f_\ast\mu$
	& pushforward measure $(f_\ast\mu)(S) = \mu(f^{-1}(S))$. \\
$\Re z$
	& real part of $z$. \\
$\Re > \alpha$
	& half-plane $\{z\in \bC : \Re z > \alpha\}$. \\
$f \ll g$ 
	& there is a constant $C>0$ such that $f \leqslant C g$. \\
$f(x) \ll x^{\alpha+\epsilon}$ 
	& for all $\epsilon>0$, $f(x) \ll x^{\alpha+\epsilon}$ (the constant may depend on $\epsilon$). \\
$\frob_p$
	& conjugacy class of arithmetic Frobenius at $p$. \\
$G^\natural$
	& space of conjugacy classes of a group $G$. \\
$\smat{a}{}{}{b}$
	& shorthand for $\smat{a}{0}{0}{b}$. \\
$\ST$
	& Sato--Tate measure $\frac{2}{\pi} \sin^2\theta\, \dd\theta$ on $[0,\pi]$. \\
$\bx$
	& sequence $(x_1,x_2,x_3,\dots)$ or $(x_2,x_3,x_5,\dots)$. \\
$\bvx$
	& sequence of vectors $(\vx_1,\vx_2,\dots)$ or $(\vx_2,\vx_3,\vx_5,\dots)$. \\
$f = o(g)$
	& means $\limsup \frac f g = 0$. \\
$|\cdot|_\infty$
	& supremum norm. \\
$[\vx,\vy)$
	& half-open box $[x_1,y_1) \times \cdots \times [x_d,y_d)$. \\
$\mu[a,b]$
	& shorthand for $\mu([a,b])$ if $\mu$ is a measure. \\
$\D(\mu,\nu)$
	& discrepancy between $\mu$ and $\nu$. \\
$\D^\star(\mu,\nu)$
	& star discrepancy between $\mu$ and $\nu$. \\
$P_{\bx,N}$
	& empirical measure associated to the set $\{x_\alpha\}_{\alpha\leqslant N}$. \\
$\D_N(\bx,\mu)$
	& discrepancy between $P_{\bx,N}$ and $\mu$. \\
$\bT^d$
	& $d$-dimensional real torus $(\bR/\bZ)^d$. \\
$\Var(f)$
	& total variation of $f$. \\
$\frac{\dd\mu}{\dd\lambda}$
	& Radon--Nikodym derivative of $\mu$. \\
$\cdf_\mu(x)$
	& cumulative distribution function $x\mapsto \mu[-\infty,x]$. \\	
$\bx\wr\by$
	& interleaved sequence $(x_1,y_1,x_2,y_2,\dots)$. \\
$f = \Theta(g)$
	& there exist constants $0 < C_1 < C_2$ such that $C_1 g \leqslant f \leqslant C_2 g$. \\
$\bx_{\leqslant N} : a^M$
	& shorthand for $(x_1,\dots,x_N,a,\dots,a)$ ($M$ copies of $a$). \\
$U_k(\theta)$
	& $\tr\sym^k\smat{e^{i\theta}}{}{}{e^{-i\theta}} = \frac{\sin((k+1)\theta)}{\sin\theta}$ on $\SU(2)^\natural = [0,\pi]$. \\
$L(\bx,s)$
	& Dirichlet series associated to a sequence $\bx$ in $\bC$. \\
$L(\rho(\bx),s)$
	& Dirichlet series associated to a representation. \\
$\omega_i(\vx)$
	& $i$-th irrationality measure of $\vx$. \\
$\langle \cdot,\cdot\rangle$
	& standard inner product on $\bR^d$. \\
$r(\vm)$
	& shorthand for $\max(1,|m_1|) \cdots \max(1,|m|_d)$. \\
$f = \Omega(g)$
	& means $\limsup \frac f g > 0$ (Hardy--Littlewood convention). \\
$\R_{F/\bQ} \Gm$
	& Weil restriction of scalars of $\Gm$. \\
$\h^i(F,M)$
	& Galois cohomology $\h^i(G_F,M)$. \\
$\Sha^i_S(M)$
	& Tate--Shafarevich group of $M$. \\
$M^\ast$
	& Cartier dual of $M$. \\
$\h_\nr^1(\bQ_p,M)$
	& unramified cohomology classes in $\h^1(\bQ_p,M)$. 
\end{tabular}
\newpage





\normalspacing
\setcounter{page}{1}
\pagenumbering{arabic}
\pagestyle{cornell}
\addtolength{\parskip}{0.5\baselineskip}





% !TEX root = Daniel-Miller-thesis.tex

\chapter{Introduction}





\section{Motivation from classical analytic number theory}

Start with a problem central to the history of number theory---counting 
prime numbers. As usual, let $\pi(x)$ be the number of rational primes 
$\leqslant x$ and $\Li(x) = \int_2^x \frac{\dd t}{\log t}$ be the 
logarithmic integral. For any $x\geqslant 2$, there is a (normalized) 
empirical measure capturing the distribution of those primes $\leqslant x$: 
\[
	P_x = \frac{1}{\pi(x)} \sum_{p\leqslant x} \delta_{p/x} ,
\]
which is supported on the unit interval $[0,1]$. The prime number theorem 
tells us that as $x\to \infty$, these empirical measures weakly converge to the 
``true'' measure $L_x = \frac{\Li(t x)}{\Li(x)}\, \dd t$. The standard approach 
to proving the prime number theorem is by showing that the Riemann 
$\zeta$-function has meromorphic continuation past $\Re = 1$, with no zeros 
on that line. 

\begin{theorem}
The function $\zeta(s)$ admits a meromorphic continuation past $\Re = 1$ with 
at most a simple pole at $s=1$ and no zeros on $\Re = 1$, if and only if 
$P_x \to L_x$ weakly. 
\end{theorem}

Since $\zeta(s)$ does have the desired properties, the prime number 
theorem holds. 
It is natural to try to quantify the rate of converge of $P_x$ to $L_x$. One 
way to do this is via the (star) discrepancy 
\[
	\disc^\star(P_x,L_x) 
		= \sup_{t\in [0,1]} \left| P_x[0,t] - L_x[0,t]\right|
		= \sup_{t\in [0,1]} \left| \frac{\pi(t x)}{\pi(x)} - \frac{\int_2^{tx} \frac{\dd s}{\log s}}{\int_2^x \frac{\dd s}{\log s}}\right| .
\]
Numerical experiments suggest that 
$\disc^\star(P_x,L_x) \ll x^{-\frac 1 2+\epsilon}$, and in fact we have the 
following result. 

\begin{theorem}
The Riemann Hypothesis is true if and only if 
$\disc^\star(P_x,L_x) \ll x^{-\frac 1 2+\epsilon}$. 
\end{theorem}

Of course, neither side of this equivalence is known for certain to be true! 

The above discussion finds a natural generalization in Artin $L$-functions. 
Let $K/\bQ$ be a finite Galois extension with group $G=\Gal(K/\bQ)$. For any 
irreducible representation $\rho\colon G\to \GL_d(\bC)$, there is a 
corresponding $L$-function defined as 
\[
	L(\rho,s) = \prod_p \frac{1}{\det(1-\rho(\frob_p) p^{-s})} ,
\]
where here (and for the remainder of this thesis) we tacitly omit from the 
product those primes at which $\rho$ is ramified. Given a cutoff $x$, there is 
a natural empirical measure 
\[
	P_x = \frac{1}{\pi(x)} \sum_{p\leqslant x} \delta_{\frob_p} ,
\]
where $\frob_p$ is the $p$-th Frobenius conjugacy class in $G$. Let 
\[
	\disc(P_x) = \sup_{S\subset G^\natural} \left| P_x(S) - \frac{\# S}{\# G^\natural}\right|,
\]
where $G^\natural$ is the set of conjugacy classes in $G$. 

\begin{theorem}
The measure $P_x$ converge weakly to the uniform measure on $G^\natural$ if and 
only if the function $L(\rho,s)$ admits analytic continuation past $\Re =1$ for 
all nontrivial $\rho$. 
\end{theorem}

Both sides of this equivalence are true, and known as the Chebotarev density 
theorem. Moreover, there is a version of the strong Prime Number Theorem in 
this context. 

\begin{theorem}
The bound $\disc(P_x) \ll x^{-\frac 1 2+\epsilon}$ holds if and only if each
$L(\rho,s)$, $\rho$ nontrivial, satisfies the Riemann Hypothesis. 
\end{theorem}

This whole discussion generalizes to a more complicated set of Galois 
representations---those arising from elliptic curves. 





\section{Discrepancy and Riemann Hypothesis for elliptic curves}

Let $E_{/\bQ}$ be an elliptic curve. For any prime $l$, there is an 
$l$-adic Galois representation $\tate_l E$ associated to $E$, known as the 
Tate module. This is a rank-$2$ $\bZ_l$-module with continuous $G_\bQ$-action, 
so it induces a continuous representation 
$\rho_{E,l} \colon G_\bQ \to \GL_2(\bZ_l)$. It is known 
\cite[Th.~V.1.1]{silverman-2009} that the quantities 
$a_p(E) = \tr \rho_l(\frob_p)$ lie in $\bZ$ and satisfy the Hasse bound 
$|a_p(E)| \leqslant 2\sqrt p$. Thus we can define, for each unramified prime 
$p$, the corresponding Satake parameter for $E$: 
\[
	\theta_p(E) = \cos^{-1}\left(\frac{a_p(E)}{2\sqrt p}\right) \in [0,\pi) .
\]
The Satake parameters are packaged into an $L$-function as follows:
\[
	L(E,s) = \prod_p \frac{1}{(1 - e^{i \theta_p(E)} p^{-s})(1- e^{-i \theta_p(E)} p^{-s})} = \prod_p \frac{1}{1 - \det\smat{e^{i\theta_p}}{}{}{e^{-i \theta_p}}p^{-s}}.
\]
More generally we have, for each irreducible representation of $\SU(2)$, which 
will be $\sym^k$ for some $k\geqslant 1$, the $k$-th symmetric power 
$L$-function 
\[
	L(\sym^k E, s) = \prod_p \prod_{j=0}^k \frac{1}{1 - e^{i (k - 2j) \theta_p(E)} p^{-s}} = \prod_p \frac{1}{1-\det \sym^k \smat{e^{i\theta_p}}{}{}{e^{-i \theta_p}}p^{-s}}.
\]

Numerical experiments suggest that the Satake parameters are distributed with 
respect to the Sato--Tate distribution 
$\ST = \frac{2}{\pi} \sin^2\theta\, \dd\theta$. Indeed, for any cutoff $x$, let 
$P_x$ be the empirical measure 
\[
	P_x = \frac{1}{\pi(x)} \sum_{p\leqslant x} \delta_{\theta_p} .
\]
The convergence of the $P_x$ to the Sato--Tate measure is closely related to 
the analytic properties of the $L(\sym^k E,s)$. First, here is the famous 
Sato--Tate Conjecture (now a theorem) in our notation. 

\begin{theorem}[Sato--Tate conjecture]
If $E$ is non-CM, the measures $P_x$ converge weakly to $\ST$. 
\end{theorem}

\begin{theorem}
Let Sato--Tate conjecture holds for (a non-CM) $E$ if and only if each of 
the functions $L(\sym^k E,s)$ have analytic continuation past $\Re = 1$. 
\end{theorem}

The stunning recent proof of the Sato--Tate conjecture 
\cite{clozel-harris-taylor-2008,taylor-2008,harris-shepherd-barron-taylor-2010} 
in fact showed that the functions $L(\sym^k E,s)$ were potentially automorphic, 
which gives analytic continuation. 

The ``usual'' Riemann Hypothesis, and its generalization to Artin 
$L$-functions, have a natural generalization to elliptic curves. In this 
context, the discrepancy of the set $\{\theta_p\}_{p\leqslant x}$ is 
\[
	\disc\left(\{\theta_p\}_{p\leqslant x},\ST\right) = \sup_{t\in [0,\pi]} \left| P_x[0,t] - \ST[0,t]\right| .
\]
The following conjecture is first made in \cite{akiyama-tanigawa-1999}: for 
$E_{/\bQ}$ a non-CM elliptic curve, the bound 
$\disc\left(\{\theta_p\}_{p\leqslant x},\ST\right)\ll x^{-\frac 1 2+\epsilon}$ 
holds. The authors go on to prove what is essentially the following theorem 
(fully fleshed out in \cite{mazur-2008}). 

\begin{theorem}
If $\disc\left(\{\theta_p\}_{p\leqslant x},\ST\right)\ll x^{-\frac 1 2+\epsilon}$, 
then all the functions $L(\sym^k E, s)$ satisfy the Riemann Hypothesis. 
\end{theorem}

This discussion also makes sense when $E$ has complex multiplication (say, 
defined over $\bQ$), but the Sato--Tate measure is instead the Haar measure 
on $\SO(2)$, i.e.~the uniform measure on $[0,\pi]$. Instead of symmetric power 
$L$-functions, one takes $L$-functions for each representation of $\SO(2)$. 

It is natural to assume that the converse to the implication 
``Strong Sato--Tate implies General Riemann Hypothesis'' holds. David Zywina 
first suggested to the author that it might not. In this thesis, we construct a 
range of counterexamples to the implication ``Strong Sato--Tate implies 
Riemann Hypothesis.'' We also construct a broader conjectural framework 
generalizing 
Akiyama--Tanigawa's conjecture to more general motives. Moreover, we generalize 
the results of \cite{pande-2011} to show that there can be no purely 
Galois-theoretic proof of the Sato--Tate conjecture, for the are Galois 
representations with arbitrary Sato--Tate distributions! We also show that 
some of the results of \cite{sarnak-2007} about sums of the form 
$\sum_{p\leqslant x} \frac{a_p}{\sqrt p}$ cannot be generalized to general 
Galois representations. 





\section{Notational conventions}

Throughout, whenever $l$ is mentioned it is a rational prime $\geqslant 7$. 

The symbol $f=\Omega(g)$ (in the convention of Hardy--Littlewood) means 
the negation of $f = O(g)$.

The symbol $f = \Theta(g)$ means there exist non-zero constants $0<C_1<C_2$
such that $C_1 g \leqslant f \leqslant C_2 f$. 

If $\mu$ is a measure on $\bR$, then $\mu[a,b] = \mu([a,b])$ and similarly 
for $[a,b)$, $(a,b])$, etc. 

If $\mu$ is a measure on $\bR$, then \emph{cumulative distribution function of 
$\mu$} is given by $\cdf_\mu(x) = \mu[-\infty,x]$. 

If $z\in \bC$, write $\Re z$ for the real part of $z$. 

If $\alpha\in \bR$, we write $\Re > \alpha$ for the half-plane of complex 
numbers with real part $> \alpha$. 

We write $\bx = (x_1,x_2,\dots)$ for infinite sequences and 
$\vx = (x_1,\dots,x_d)$ for vectors. Sometimes we will have a sequence of 
vectors, written as $\bx = (\vx_1,\vx_2,\dots)$. 

% !TEX root = main.tex

\chapter{Discrepancy}





\section{Definitions and first results}

regular discrepancy

star discrepancy

Euclidean space vs.~torus





\section{Comparing sequences}

If $\{x_n\}\subset [0,\pi/2)$ has some discrepancy with respect to some 
measure, then the ``flipped'' sequence $\{\pi/2-x_n\}$ has the same discrepancy 
with respect to the ``flipped'' measure. 





\section{Combining sequences}

If $\{x_n\}$ and $\{y_n\}$ are sequences supported on $[0,\pi/2)$ and 
$[\pi/2,\pi)$ respectively, and both are equidistributed with respect to 
measures supported on their respective intervals, then the ``interleaved'' 
sequence $(x_1,y_1,x_2,y_2,\dots)$ also has equidistribution (with respect to 
the combined measure) and discrepancy which decays no faster than the slower of 
the two. 

\include{3_Dirichlet}
\include{4_irrationality}
% !TEX root = Daniel-Miller-thesis.tex

\chapter{Pathological Galois representations}\label{ch:construct-Galois}





\section{Notation and supporting results}

In this section we loosely summarize and adapt the results of 
\cite{khare-larsen-ramakrishna-2005,pande-2011}. Throughout, if $F$ is a field 
and $M$ a $G_F$-module, we write $\h^\bullet(F,M)$ in place of 
$\h^\bullet(G_F,M)$. All Galois representations will take values in 
$\GL_2(\bZ/l^n)$ or $\GL_2(\bZ_l)$ for $l$ a (fixed) rational prime, and 
all deformations will have fixed determinant. So we consider the cohomology of 
$\Ad^0\bar\rho$, the induced representation on trace-zero matrices by 
conjugation. 

If $S$ is a set of rational primes, $\bQ_S$ denotes the largest extension of 
$\bQ$ unramified outside $S$. So $\h^i(\bQ_S,-)$ is what is usually written as 
$\h^1(G_{\bQ,S},-)$. If $M$ is a $G_\bQ$-module and $S$ a finite set of primes, 
denote the corresponding Tate--Shafarevich group by 
\[
	\Sha^i_S(M) = \ker\left( \h^i(\bQ_S,M) \to \prod_{p\in S} \h^i(\bQ_p,M)\right) .
\]
If $l$ is a rational prime and $S$ a finite set of primes containing $l$, then 
for any $\bF_l[G_{\bQ_S}]$-module $M$, write $M^\vee=\hom_{\bF_l}(M,\bF_l)$ 
with the obvious $G_{\bQ_S}$-action, and write $M^\ast = M^\vee(1)$ for the 
Cartier dual of $M^\vee$. By \cite[Th.~8.6.7]{neukirch-schmidt-winberg-2008}, 
there is an isomorphism $\Sha^1_S(M^\ast) \simeq \Sha_S^2(M)^\vee$. As a 
result, if $\Sha_S^1(M)$ and $\Sha_S^2(M)$ are trivial, and $S\subset T$, then 
$\Sha_T^1(M)$ and $\Sha_T^2(M)$ are also both trivial. 

% The l>=7 comes from \cite{ramakrishna-2002}. 
\begin{definition}
A \emph{good residual representation} is an odd, absolutely irreducible, 
weight-$2$ representation $\bar\rho\colon G_{\bQ_S} \to \GL_2(\bF_l)$, where 
$l\geqslant 7$ is a rational prime. 
\end{definition}

Recall that $\bar\rho$ is weight-$2$ if $\det\bar\rho$ is the mod-$l$ 
cyclotomic character. Similarly, $\rho\colon G_\bQ \to \GL_2(\bZ_l)$ is 
weight-$2$ if $\det\rho$ is the $l$-adic cyclotomic character. 
Roughly, ``good residual representations'' have enough properties that we can 
prove meaningful theorems about their lifts without assuming the modularity 
results of Khare--Wintenberger. 

\begin{theorem}\label{thm:always-can-lift}
Let $\bar\rho\colon G_\bQ \to \GL_2(\bF_l)$ be a good residual 
representation. Then there exists a weight-$2$ lift of $\bar\rho$ to $\bZ_l$, 
ramified at the same set of primes as $\bar\rho$. 
\end{theorem}
\begin{proof}
This is \cite[Th.~1]{ramakrishna-2002}, taking into account that the paper in 
question allows for arbitrary fixed determinants. 
\end{proof}

\begin{definition}
Let $\bar\rho\colon G_{\bQ_S} \to \GL_2(\bF_l)$ be a good residual 
representation. A prime $p\not\equiv \pm 1\pmod l$ is \emph{nice} if 
$\Ad^0\bar\rho\simeq \bF_l \oplus \bF_l(1)\oplus \bF_l(-1)$, i.e.~if the 
eigenvalues of $\bar\rho(\frob_p)$ have ratio $p$. 
\end{definition}

Taylor allows $p\equiv -1\pmod l$, but the results of \cite{pande-2011} 
require $p\not\equiv -1\pmod l$. The following theorem gives a complete 
description of the versal deformation ring for 
$\left.\bar\rho\right|_{G_{\bQ_p}}$ when $p$ is nice.

\begin{theorem}[\cite{ramakrishna-1999}]
Let $\bar\rho$ be a good residual representation and $p$ a nice prime. Then 
any deformation of $\left.\bar\rho\right|_{G_{\bQ_p}}$ is induced by 
$G_{\bQ_p} \to \GL_2(\bZ_l\pow{a,b} / \langle a b\rangle)$, sending 
\[
	\frob_p \mapsto \smat{p(1+a)}{}{}{(1+a)^{-1}} \qquad \tau_p \mapsto \smat{1}{b}{}{1} ,
\]
where $\tau_p\in G_{\bQ_p}$ is a generator for tame inertia. 
\end{theorem}

We close this section by introducing some new terminology and notation to 
condense the lifting process used in \cite{khare-larsen-ramakrishna-2005}. 

Fix a good residual representation $\bar\rho$. We will consider weight-$2$ 
deformations of $\bar\rho$ to $\bZ/l^n$ and $\bZ_l$. Call such a deformation a 
``lift of $\bar\rho$ to $\bZ/l^n$ (resp.~$\bZ_l$).'' We will often restrict the 
local behavior of such lifts, i.e.~the restrictions of a lift to $G_{\bQ_p}$ 
for $p$ in some set of primes. The necessary constraints are captured in the 
following definition. 

\begin{definition}
Let $\bar\rho$ be a good residual representation, 
$h\colon \bR^+ \to \bR_{\geqslant 1}$ an 
increasing function. An \emph{$h$-bounded lifting datum} is a tuple 
$(\rho_n,R_n,U_n,\{\rho_p\}_{p\in R_n\cup U_n})$, where 
\begin{enumerate}
\item
$\rho_n\colon G_{\bQ_{R_n}} \to \GL_2(\bZ/l^n)$ is a lift of $\bar\rho$.

\item
$R_n$ and $U_n$ are finite sets of primes, $R_n$ containing $l$ and all primes at 
which $\rho_n$ ramifies. 

\item
$\pi_{R_n}(x)\leqslant h(x)$ for all $x$. 

\item
Both $\Sha_{R_n}^1(\Ad^0\bar\rho)$ and $\Sha_{R_n}^2(\Ad^0\bar\rho)$ are 
trivial. 

\item
For all $p\in R_n\cup U_n$, 
$\rho_p\colon G_{\bQ_p} \to \GL_2(\bZ_l)$ satisfies 
$\rho_p\equiv \left. \rho_n\right|_{G_{\bQ_p}}\pmod{l^n}$. 

\item
For all $p\in R_n$, $\rho_p$ is ramified. 

\item
$\rho_n$ admits a lift to $\bZ/l^{n+1}$. 
\end{enumerate}
\end{definition}

If $(\rho_n,R_n,U_n,\{\rho_p\})$ is an $h$-bounded lifting datum, we call 
another $h$-bounded lifting datum $(\rho_{n+1},R_{n+1},U_{n+1},\{\rho_p\})$ a 
\emph{lift} of $(\rho_n,R_n,U_n,\{\rho_p\})$ if $U_n\subset U_{n+1}$, 
$R_n\subset R_{n+1}$, and for all 
$p\in R_n\cup U_n$, the two possible $\rho_p$ agree. 

\begin{theorem}\label{thm:lifting-datum}
Let $\bar\rho$ be a good residual representation, 
$h\colon \bR^+ \to \bR_{\geqslant 1}$ 
increasing to infinity. If $(\rho_n,R_n,U_n,\{\rho_p\})$ is an $h$-bounded lifting 
datum, $U_{n+1}\supset U_n$ is a finite set of primes disjoint from $R_n$, and 
$\{\rho_p\}_{p\in U_{n+1}}$ extends $\{\rho_p\}_{p\in U_n}$, then there exists an 
$h$-bounded lift $(\rho_{n+1},R_{n+1},U_{n+1},\{\rho_p\})$ of 
$(\rho_n,R_n,U_n,\{\rho_p\})$. 
\end{theorem}
\begin{proof}
By \cite[Lem.~8]{khare-larsen-ramakrishna-2005}, there exists a finite set 
$N$ of nice primes such that the map 
\begin{equation}\label{eq:h1-isom}
	\h^1(\bQ_{R_n\cup N},\Ad^0\bar\rho) \to \prod_{p\in R_n} \h^1(\bQ_p,\Ad^0\bar\rho) \times \prod_{p\in U_{n+1}} \h_\nr^1(\bQ_p,\Ad^0\bar\rho) 
\end{equation}
is an isomorphism. In fact, 
$\# N = \dim\h^1(\bQ_{R_n\cup U_n},\Ad^0\bar\rho^\ast)$, and the primes in $N$ are 
chosen, one at a time, from Chebotarev sets. Since $\pi_{R_n}(x)$ is eventually 
constant and $h(x)$ increases to infinity, $h(x) \geqslant \pi_{R_n}(x)+1$ for all 
$x\geqslant C_1$ for some $C_1$. Choose the first prime $p$ in $N$ to be 
$\geqslant C_1$; then $\pi_{R_n\cup\{p\}}(x) \leqslant h(x)$ for all $x$. Repeat 
this process for for all the other primes in $N$. We can ensure 
that the bound $\pi_{R_n\cup N}(x) \leqslant h(x)$ continues to hold. We also 
choose the primes in $N$ to be larger than any prime in $U_{n+1}$. 

By our hypothesis, $\rho_n$ admits a lift to $\bZ/l^{n+1}$; call one such lift 
$\rho^\ast$. For each $p\in R_n\cup U_{n+1}$, $\h^1(\bQ_p,\Ad^0\bar\rho)$ acts 
transitively on lifts of $\left.\rho_n\right|_{G_{\bQ_p}}$ to $\bZ/l^{n+1}$. In 
particular, there are cohomology classes $f_p\in \h^1(\bQ_p,\Ad^0\bar\rho)$ 
such that $f_p\cdot \rho^\ast \equiv \rho_p\pmod{l^{n+1}}$ for all 
$p\in R_n\cup U_{n+1}$. Moreover, for all $p\in U_{n+1}$, the class $f_p$ is 
unramified. Since the map \eqref{eq:h1-isom} is an isomorphism, there exists 
$f\in \h^1(\bQ_{R_n\cup N},\Ad^0\bar\rho)$ such that 
$\left.f\cdot \rho^\ast\right|_{G_{\bQ_p}}\equiv \rho_p\pmod{l^{n+1}}$ for all 
$p\in R_n\cup U_{n+1}$. 

Clearly $\left. f\cdot \rho^\ast\right|_{G_{\bQ_p}}$ admits a lift to $\bZ_l$ 
for all $p\in R_n\cup U_{n+1}$, but it does not necessarily admit such a lift for 
$p\in N$. By repeated applications of \cite[Prop.~3.10]{pande-2011}, there 
exists a set $N'\supset N$, with $\# N'\leqslant 2\# N$, of nice primes and 
$g\in \h^1(\bQ_{R_n\cup N'},\Ad^0\bar\rho)$ such that 
$(g+f)\cdot \rho^\ast$ still agrees with $\rho_p$ for $p\in R_n\cup U'$, and 
$(g+f)\cdot \rho^\ast$ is nice for all $p\in N'$. As above, the primes in $N'$ 
are chosen one at a time from Chebotarev sets, so we can continue to ensure the 
bound $\pi_{R_n\cup N'}(x)\leqslant h(x)$ and also that all primes in $N'$ 
are larger than those in $U_{n+1}$. Let $\rho_{n+1} = (g+f) \cdot \rho^\ast$. Let 
$R_{n+1} = R_n\cup \{p\in N' : \rho_{n+1}\text{ is ramified at }p\}$. For each 
$p\in R_{n+1}\smallsetminus R_n$, choose a lift $\rho_p$ of 
$\left. \rho_{n+1}\right|_{G_{\bQ_p}}$ to $\bZ_l$. 

Since $\left.\rho_{n+1}\right|_{G_{\bQ_p}}$ admits a lift to $\bZ/l^{n+2}$ (in 
fact, it admits a lift to $\bZ_l$) for each $p$, and 
$\Sha_{R_{n+1}}^1(\Ad^0\bar\rho)$, $\Sha_{R_{n+1}}^2(\Ad^0\bar\rho)$ are 
trivial, the deformation $\rho_{n+1}$ admits a lift to $\bZ/l^{n+2}$. The tuple 
$(\rho_{n+1},R_{n+1},U_{n+1},\{\rho_p\})$ is the desired lift of 
$(\rho_n,R_n,U_n,\{\rho_p\})$ to $\bZ/l^{n+1}$. 
\end{proof}





\section{Galois representations with specified Satake parameters}

Fix a good residual representation $\bar\rho$, and consider weight-$2$ 
deformations of $\bar\rho$. The final deformation, 
$\rho\colon G_\bQ \to \GL_2(\bZ_l)$, will be constructed as the inverse limit 
of a compatible collection of lifts $\rho_n\colon G_\bQ \to \GL_2(\bZ/l^n)$. At 
any given stage, we will be concerned with making sure that there exists a 
lift to the next stage, and that there is a lift with the necessary properties. 
Fix a sequence $\bx=(x_1,x_2,\dots)$ in $[-1,1]$. The set of unramified primes 
of $\rho$ is not determined at the beginning, but at each stage there will be 
a large finite set $U$ of primes which we know will remain unramified. 
Reindexing $\bx$ by these unramified primes, we will construct $\rho$ so that 
for all unramified primes $p$, $\tr\rho(\frob_p)\in \bZ$, satisfies the Hasse 
bound, and has $\tr\rho(\frob_p) \approx x_p$. Moreover, we can ensure that the 
set of ramified primes has density zero in a very strong sense (controlled by a 
parameter function $h$) and that our trace of Frobenii are very close to 
specified values. 

Given any deformation $\rho$, write $\pi_{\ram(\rho)}(x)$ for the function 
which counts $\rho_n$-ramified primes $\leqslant x$. Since we will have 
$\pi_{\ram(\rho)}(x)\ll h(x)$ and bounds of this form are only helpful 
if $h(x) = o(\pi(x))$, we will usually assume $h(x) \ll x^\epsilon$, 
e.g.~$h(x) = \log x$ or something which grows even slower (for example, the 
inverse of the Ackerman function). In \cite{khare-rajan-2001}, it is proved 
that for \emph{any} continuous semisimple $\rho\colon G_\bQ \to \GL_2(\bZ_l)$, 
we will have $\pi_{\ram(\rho)}(x) = o(\pi(x))$. That is, \emph{any} continuous 
Galois representation we consider will be ramified at a density zero set of 
primes. However, by \cite[Th.~19]{khare-larsen-ramakrishna-2005}, it is 
possible for $\pi_{\ram(\rho)}(x)$ to be 
$\Omega(\frac{x}{\log(x)^{1+\epsilon}})$. This means the ability to bound 
$\pi_{\ram(\rho)}(x)$ in the following result is non-trivial. 

\begin{theorem}\label{thm:master-Galois}
Let $l$, $\bar\rho$, $\bx$ be as above. Fix a function 
$h\colon \bR^+ \to \bR_{\geqslant 1}$ which increases to infinity. Then there 
exists a weight-$2$ deformation $\rho$ of $\bar\rho$, such that: 
\begin{enumerate}
\item
$\pi_{\ram(\rho)}(x) \ll h(x)$. 

\item
For each unramified prime $p$, $a_p=\tr\rho(\frob_p)\in \bZ$ and satisfies the 
Hasse bound. 

\item
For each unramified prime $p$, 
$\left| \frac{a_p}{2\sqrt p} - x_p\right| \leqslant \frac{l h(p)}{\sqrt p}$. 
\end{enumerate}
\end{theorem}
\begin{proof}
Begin with $\rho_1= \bar\rho$. By \cite[Lem.~6]{khare-larsen-ramakrishna-2005}, 
there exists a finite set $R$, containing the set of primes at which $\bar\rho$ 
ramifies, such that $\Sha_R^1(\Ad^0\bar\rho)$ and $\Sha_R^2(\Ad^0\bar\rho)$ are 
trivial. 
Let $R_1$ be the union of $R$ and all primes $p$ with 
$\frac{l}{2\sqrt p} > 2$. Since $\frac{l}{2\sqrt p} \to 0$ as $p\to \infty$, 
the set $R_1$ is finite. For all $p\notin R_1$ and any $a\in \bF_l$, there 
exists $a_p\in \bZ$ satisfying the Hasse bound with $a_p\equiv a\pmod l$. In 
fact, given any $x_p\in [-1,1]$, there exists $a_p\in \bZ$ satisfying the Hasse 
bound such that 
$\left| \frac{a_p}{2\sqrt p} - x_p\right| \leqslant \frac{l}{\sqrt p}$.
Choose, for all primes $p\in R_1$, a ramified 
lift $\rho_p$ of $\left. \rho_1\right|_{G_{\bQ_p}}$. Let $U_1$ be the set of 
primes $p$ not in $R_1$ such that 
$\frac{l^2}{2\sqrt p} > \min\left(2, \frac{l h(p)}{2\sqrt p}\right)$; this is 
finite because $\frac{l^2}{2\sqrt p} \to 0$ and also eventually 
$h(p) \geqslant l$. If $U_1$ is empty, then the next few sentences of the 
proof are superfluous, but the theorem still holds. 
For each $p\in U_1$, there exists $a_p\in \bZ$, satisfying the 
Hasse bound, such that 
\[
	\left| \frac{a_p}{2\sqrt p} - x_p\right| \leqslant \frac{l}{\sqrt p} \leqslant \frac{l h(p)}{\sqrt p} ,
\]
and moreover $a_p\equiv \tr\bar\rho(\frob_p)\pmod l$. For each $p\in U_1$, let 
$\rho_p$ be an unramified lift of $\left.\bar\rho\right|_{G_{\bQ_p}}$ with 
$\tr\rho_p$ being the desired $a_p$. It may not be that 
$\pi_{R_1}(x) \leqslant h(x)$ for all $x$. Let 
$C = \max\left\{\pi_{R_1}(x)\right\}$; this is finite because 
$R_1$ is and $\pi_{R_1}(x)$ is constant past the largest prime in $R_1$. Then 
for $h^\ast = C h$, we have $\pi_{R_1}(x) \leqslant h^\ast(x)$ for all $x$. 

We have constructed our first $h^\ast$-bounded lifting datum 
$(\rho_1,R_1,U_1,\{\rho_p\})$. We proceed to construct 
$\rho = \varprojlim \rho_n$ inductively, by constructing a new $h^\ast$-bounded 
lifting datum for each $n$. We ensure that $U_n$ contains all primes for which 
$\frac{l^{n+1}}{2\sqrt p} > \min\left(2, \frac{l h(p)}{2\sqrt p}\right)$, so 
there are always integral $a_p$ satisfying the Hasse bound which satisfy any 
mod-$l^{n+1}$ constraint, and that can always choose these $a_p$ so as to 
preserve statement 2 in the theorem. 

The base case is already complete, so suppose we are given 
$(\rho_{n-1},R_{n-1},U_{n-1},\{\rho_p\})$. We may assume that $U_{n-1}$ 
contains all primes for 
which $\frac{l^n}{2\sqrt p} > \min\left(2, \frac{l h(p)}{2\sqrt p}\right)$. Let 
$U_n$ be the set of all primes not in $R_{n-1}$ such that 
$\frac{l^{n+1}}{2\sqrt p} > \min\left(2, \frac{l h(p)}{2\sqrt p}\right)$. For 
each $p\in U_n\smallsetminus U_{n-1}$, there is an integer $a_p$, satisfying 
the Hasse bound, such that $a_p\equiv \rho_n(\frob_p)\pmod{l^n}$, and moreover 
$\left|\frac{a_p}{2\sqrt p} - x_p\right| \leqslant \frac{l h(p)}{\sqrt p}$. 
For such $p$, let $\rho_p$ be an unramified lift of 
$\left. \rho_n\right|_{G_{\bQ_p}}$ such that $\tr\rho_n(\frob_p)$ is the 
desired $a_p$. By Theorem \ref{thm:lifting-datum}, there exists an 
$h^\ast$-bounded lifting datum $(\rho_n,R_n,U_n,\{\rho_p\})$ extending and 
lifting $(\rho_{n-1},R_{n-1},U_{n-1},\{\rho_p\})$. This completes the inductive 
step.  
\end{proof}

The implied constant in the bound $\pi_{\ram(\rho)}(x)\ll h(x)$ depends on 
$\bar\rho$ (and hence $l$) but not on $h$. 
We will apply this theorem to construct Galois representations with specified 
Sato--Tate distributions in the next section, but for now here is a small 
consequence, which addresses the results in \cite{sarnak-2007}. Sarnak remarks 
that for $E_{/\bQ}$ a non-CM elliptic curve with rank $r$, the partial sums 
$\frac{\log x}{\sqrt x} \sum_{p\leqslant x} \frac{a_p}{\sqrt p}$ approach a 
limiting distribution with mean $1 - 2 r$. 

\begin{corollary}
Let $L \in [-\infty,\infty]$ and $\epsilon>0$ be given. Then there exists a 
weight 2 Galois representation $\rho\colon G\to \GL_2(\bZ_l)$, such that 
each $a_p = \tr\rho(\frob_p)\in \bZ$ satisfies the Hasse bound, 
\[
	L = \lim_{N\to \infty} \frac{\log N}{\sqrt N}\sum_p \frac{a_p}{\sqrt p} 
\]
and $\pi_{\ram(\rho)}(x) \ll \log(x)$. 
\end{corollary}
\begin{proof}
Begin with a sequence $(x_p)$ in $[-1,1]$ such that 
$\lim_{N\to \infty} \frac{\log N}{\sqrt N}\sum_{p\leqslant N} x_p = L$. If 
$L=\pm \infty$, we can choose $x_p = \pm 1$. By Theorem 
\ref{thm:master-Galois}, there exists $\rho\colon G_\bQ \to \GL_2(\bZ_l)$ with 
$\pi_{\ram(\rho)}(x) \ll \log(x)$, and such that for each unramified $p$, 
$a_p = \tr\rho(\frob_p)\in \bZ$, satisfies the Hasse bound, has 
$\left| \frac{a_p}{2\sqrt p} - x_p\right| < \frac{l \log p}{\sqrt p}$, and by 
inspecting the proof, we can even ensure that 
$\sum \left(\frac{a_p}{2\sqrt p} - x_p\right)$ converges conditionally (make 
sure the sign of $\frac{a_p}{2\sqrt p} - b_p$ alternates). Note that
\begin{align*}
	\left|\frac{\log N}{\sqrt N}\sum_{p\leqslant N} \frac{a_p}{2\sqrt p} - \frac{\log N}{\sqrt N} \sum_{p\leqslant N} x_p\right| 
		&\leqslant \frac{\log N}{\sqrt N}\left|\sum_{\substack{p\leqslant N \\ p\text{ ram.}}} \left(\frac{a_p}{2 \sqrt p} - x_p\right) + \sum_{\substack{p\leqslant N \\ p\text{ n.r.}}} \left(\frac{a_p}{2 \sqrt p} - x_p\right) \right|  \\
		&\ll \frac{\log N}{\sqrt N} \left(\pi_{\ram(\rho)}(N) + \left|\sum_{p\leqslant N} \left(\frac{a_p}{2 \sqrt p} - x_p\right)\right|\right) ,
\end{align*}
which converges to zero. 
When $L\ne \pm\infty$, this shows that the limit in question exists and is $L$. 
When $L=\pm \infty$, this shows that the the sums in question diverge to $L$. 
\end{proof}





\section{Galois representations with specified Sato--Tate distributions}

For $k\geqslant 1$, let 
\[
	U_k(\theta) = \tr\sym^k\smat{e^{i\theta}}{}{}{e^{- i \theta}} = \frac{\sin((k+1)\theta)}{\sin\theta} .
\]
Then $U_k(\cos^{-1} t)$ is the $k$-th Chebyshev polynomial of the 2nd kind. 
Moreover, $\{1\}\cup\{U_k\}$ forms an orthonormal basis for 
$L^2([0,\pi],\ST) = L^2(\SU(2)^\natural)$. 

This section has two parts. First, for any reasonable measure $\mu$ on 
$[0,\pi]$ invariant under the same ``flip'' automorphism as the Sato--Tate 
measure, there is a sequence $(a_p)$ of integers satisfying the Hasse 
bound $|a_p|\leqslant 2\sqrt p$, such that for 
$\theta_p = \cos^{-1}\left(\frac{a_p}{2\sqrt p}\right)$, the discrepancy 
$\D_N(\btheta,\mu)$ behaves like $\pi(N)^{-\alpha}$ for predetermined 
$\alpha\in \left(0,\frac 1 2\right)$, while for any odd $k$, the strange 
Dirichlet series $L_{U_k}(\btheta,s)$, which we will write as 
$L(\sym^k \btheta,s)$, satisfies the Riemann hypothesis. In the second part of 
this section, we associate Galois representations to these fake Satake 
parameters. 

\begin{definition}
Let $\mu = f(\theta)\, \dd \theta$ be an absolutely continuous probability 
measure on $[0,\pi]$. If $f(\theta) \ll \sin(\theta)$, then $\mu$ is a 
\emph{Sato--Tate compatible measure}. 
\end{definition}

Recall that $\cos_\ast\mu = \frac{f(\cos^{-1} t)}{\sqrt{1-t^2}}\, \dd t$. So 
the Radon--Nikodym derivative of $\cos_\ast\mu$ is bounded if and only if 
$\frac{f(\cos^{-1} t)}{\sqrt{1-t^2}}$ is bounded. Plugging in 
$t = \cos\theta$, we see that $\cos_\ast\mu$ has bounded Radon--Nikodym 
derivative if and only if $\frac{f(\theta)}{\sin\theta}$ is bounded, 
i.e.~$f(\theta) \ll \sin\theta$. So we could rephrase the definition of a 
Sato--Tate compatible measure to be ``an absolutely continuous measure $\mu$ 
such that $\cos_\ast\mu$ has bounded Radon--Nikodym derivative.'' 
Since $\ST = \frac 2 \pi \sin^2\theta\, \dd\theta$ clearly satisfies this 
definition, the Sato--Tate measure is itself Sato--Tate compatible. 

If $\mu$ is Sato--Tate compatible, then $\cos_\ast\mu$ 
satisfies the hypotheses of Theorem \ref{thm:discrepancy-arbitrary}, so 
there are ``$N^{-\alpha}$-decaying van der Corput sequences'' for 
$\cos_\ast\mu$, and also that since $\cos\colon [0,\pi] \to [-1,1]$ is 
strictly decreasing, we know that for any sequence $\bx$ on $[-1,1]$, 
$\D_N(\bx,\cos_\ast\mu) \approx \D_N(\cos^{-1}\bx,\mu)$, with the 
difference being $O(N^{-1})$.  Finally, the 
Radon--Nikodym derivative of $\mu$ (and also $\cos_\ast\mu$) is bounded , so 
Lemma \ref{lem:disc-of-two-seq} applies to both $\mu$ and $\cos_\ast\mu$. 
Recall that for deceasing functions 
$\varphi_1,\varphi_2$, we write $\varphi_1(N) = \Theta(\varphi_2(N))$ if 
there exists constants $0 < C_1 < C_2$ such that 
$C_1 \varphi_2(N) \leqslant \varphi_1(N) \leqslant C_2 \varphi_2(N)$. 


\begin{theorem}\label{thm:integral-a_p-alpha}
Let $\mu$ be a Sato--Tate compatible measure, and fix 
$\alpha\in \left(0,\frac 1 2\right)$. 
Then there exists a sequence of integers $a_p$ satisfying the Hasse bound, 
such that if we set $\theta_p = \cos^{-1}\left(\frac{a_p}{2\sqrt p}\right)$, 
then $\D_N(\btheta,\mu) = \Theta(\pi(N)^{-\alpha})$. 
\end{theorem}
\begin{proof}
Apply Theorem \ref{thm:discrepancy-arbitrary} to find a sequence $\bx$ such 
that $\D_N(\bx,\cos_\ast \mu) = \Theta(\pi(N)^{-\alpha})$. For each prime 
$p$, there exists an integer $a_p$ such that $|a_p|\leqslant 2\sqrt p$ and 
$\left| \frac{a_p}{2\sqrt p} - x_p\right| \leqslant p^{-1/2}$. Let 
$y_p = \frac{a_p}{2\sqrt p}$, and apply Lemma \ref{lem:disc-of-two-seq} with 
$\epsilon = N^{-1/2}$. We obtain 
\[
	\left| \D_N(\bx,\cos_\ast \mu) - \D_N(\by, \cos_\ast \mu)\right| \ll  N^{-1/2} + \frac{\pi(N^{1/2})}{\pi(N)} ,
\]
which tells us that $\D_N(\by,\cos_\ast\mu) = \Theta(\pi(N)^{-\alpha})$. 
Now let $\btheta = \cos^{-1}(\by)$. Apply Lemma \ref{lem:push-discrepancy} to 
$\btheta = \cos^{-1}(\by)$, and we see that 
$\D_N(\btheta,\mu) = \Theta(\pi(N)^{-\alpha})$. 
\end{proof}

We can improve this example by controlling the behavior of the sums 
$\sum_{p\leqslant N} U_k(\theta_p)$ for odd $k$. Let $\sigma$ be 
the involution of $[0,\pi]$ given by $\sigma(\theta) = \pi-\theta$. Note that 
$\sigma_\ast \ST = \ST$. Moreover, note that for any odd $k$, 
$U_k\circ\sigma = - U_k$, so $\int U_k\, \dd\ST = 0$. Of course, 
$\int U_k \,\dd\ST = 0$ for the reason that $U_k$ is the trace of a 
non-trivial unitary representation, but we will directly use the ``oddness'' 
of $U_k$ in what follows.

\begin{theorem}\label{thm:int-flip-seq}
Let $\mu$ be a $\sigma$-invariant Sato--Tate compatible measure. Fix 
$\alpha\in \left(0,\frac 1 2\right)$. Then there is a sequence of integers 
$a_p$, satisfying the Hasse bound, such that for 
$\theta_p =\cos^{-1}\left( \frac{a_p}{2\sqrt p}\right)$, we have
\begin{enumerate}
\item
$\D_N(\btheta,\mu) = \Theta(\pi(N)^{-\alpha})$. 

\item
For all odd $k$, 
$\left| \sum_{k\leqslant N} U_k(\theta_p)\right| \ll \pi(N)^{1/2}$. 
\end{enumerate}
\end{theorem}
\begin{proof}
The basic ideas is as follows. Enumerate the primes 
\[
	p_1 = 2, q_1 = 3, p_2 = 5, q_2 = 7, p_3 = 11, q_3 = 13, \dots .
\]
Consider the measure $\left.\mu\right|_{[0,\pi/2)}$. An argument 
nearly identical to the proof of Theorem \ref{thm:integral-a_p-alpha} shows 
that we can choose $a_{p_i}$ satisfying the Hasse bound so that 
\[
	\D_N\left( \left\{\theta_{p_i}\right\},\left.\mu\right|_{[0,\pi/2)}\right) = \Theta(N^{-\alpha}) .
\]
We can also choose the $a_{q_i}$ such that 
$\frac{a_{q_i}}{2\sqrt{q_i}}\in [\pi/2,\pi]$ and 
$\left| \frac{a_{p_i}}{2\sqrt{p_i}} + \frac{a_{q_i}}{2\sqrt{q_i}}\right| \ll \frac{1}{\sqrt{p_i}}$. 
Let $\bx$ be the sequence of the $\frac{a_{p_i}}{2\sqrt{p_i}}$ and $\by$  
the corresponding sequence with the $q_i$-s. Then 
Lemma \ref{lem:flip-discrepancy} tells us that the discrepancy of 
$\by$ decays at the same rate (within $O(N^{-1})$) as $-\by$, and then Lemma  \ref{lem:disc-of-two-seq} with $\epsilon\approx N^{-1/2}$ tells us that 
the discrepancy of $-\by$ decays at the same rate (within $O(N^{-1/2})$) as 
the discrepancy of $\bx$. Thus the discrepancies of both $\bx$ and $\by$ decay 
as $\Theta(N^{-\alpha})$. Finally, Theorem \ref{thm:wreath-seq} tell us that 
$\D_N(\bx\wr\by,\mu) = \Theta(N^{-\alpha})$. 

The function $U_k(\cos^{-1} t)$ is an odd polynomial in $t$, so for 
$t_1,t_2\in [-1,1]$, 
\[
	|U_k(\cos^{-1} t_1) + U_k(\cos^{-1} t_2)| = |U_k(\cos^{-1} t_1) - U_k(\cos^{-1}(-t_1))| \ll |t_1 - (-t_2)|.
\]
It follows that since 
$\left|\frac{a_{p_i}}{2\sqrt{p_i}} - \left(- \frac{a_{q_i}}{2\sqrt{q_i}}\right)\right| \ll p_i^{-1/2}$, 
then $|U_k(\theta_{p_i}) + U_k(\theta_{q_i})|\ll p_i^{-1/2}$. 
We can then bound 
\[
	\left| \sum_{i\leqslant N} \left(U_k(\theta_{p_i}) + U_k(\theta_{q_i})\right)\right| \ll \sum_{p\leqslant N} p^{-1/2} \ll \pi(N)^{1/2} .
\]
\end{proof}

Note that this proof actually shows that for any $f\in C([0,\pi])$ such 
that $f\circ \cos^{-1}$ is Lipschitz, and $f(\pi-\theta) = -f(\theta)$, the 
estimate $\left| \sum_{p\leqslant N} f(\theta_p)\right| \ll \pi(N)^{1/2}$ 
holds. 

\begin{theorem}\label{thm:bad-Galois}
Let $\mu$ be a Sato--Tate compatible $\sigma$-invariant measure on $[0,\pi]$. 
Fix $\alpha\in \left(0,\frac 1 2\right)$ and a good residual representation 
$\bar\rho\colon G_\bQ \to \GL_2(\bF_l)$. Then there exists a weight-$2$ lift 
$\rho\colon G_\bQ \to \GL_2(\bZ_l)$ of $\bar\rho$ such that 
\begin{enumerate}
\item
$\pi_{\ram(\rho)}(x) \ll \log(x)$. 

\item
For each unramified prime $p$, $a_p = \tr\rho(\frob_p)\in \bZ$ and satisfies 
the Hasse bound. 

\item
If, for unramified $p$ we set 
$\theta_p = \cos^{-1}\left(\frac{a_p}{2\sqrt p}\right)$, then 
$\D_N(\btheta,\mu) = \Theta(\pi(N)^{-\alpha})$. 

\item
For each odd $k$, the function $L(\sym^k \rho,s)$ satisfies the Riemann 
hypothesis. 
\end{enumerate}
\end{theorem}
\begin{proof}
Let $\bx$ be an $N^{-\alpha}$-decay van der Corput sequence for 
$\cos_\ast \left.\mu\right|_{[0,\pi/2)}$, so that $\bx$ is contained in 
$(0,1]$. Let $\by = -\bx$ (contained in $[-1,0)$), and put 
$\bz = \bx\wr\by$, reindexed by the prime numbers. We have
$\D_N(\bz,\cos_\ast\mu) = \Theta(\pi(N)^{-\alpha})$ just as in the proof 
of Theorem \ref{thm:int-flip-seq}. Set $h(x) = \log(x)$. By Theorem 
\ref{thm:master-Galois}, there is a $\rho\colon G_\bQ \to \GL_2(\bZ_l)$ lifting 
$\bar\rho$ such that $\pi_{\ram(\rho)}(x) \ll \log x$, the $\tr \rho(\frob_p)$ 
are integral, satisfy the Hasse bound, and 
$\left| \frac{a_p}{2\sqrt p} - z_p\right| \leqslant \frac{l \log p}{2\sqrt p}$. 
This implies, just as in the proof of Theorem \ref{thm:master-Galois}, that 
the discrepancy of the sequence $\left\{\frac{a_p}{2\sqrt p}\right\}$ decays 
as $\Theta(\pi(N)^{-\alpha})$ and by Lemma \ref{lem:push-discrepancy}, the 
discrepancies of $\left\{\frac{a_p}{2\sqrt p}\right\}$ and $\{\theta_p\}$ 
decay asymptotically at the same rate. 

We've proved statements 1--3 in the theorem, so all that remains is to prove 
the Riemann hypothesis for odd symmetric powers. The proof of Theorem 
\ref{thm:int-flip-seq} gives us an estimate 
$\left| \sum_{p\leqslant N} U_k(\theta_p)\right| \ll N^{\frac 1 2+\epsilon}$, 
and this combined with Theorem \ref{thm:AT->RH:gp} yields the result. 
\end{proof}

This entire discussion also works with absolutely continuous measure $\mu$, 
supported on a proper subinterval of $[0,\pi]$, so long as $\cdf_\mu$ is 
strictly increasing on that interval. Let $I$ be an arbitrarily small 
subinterval of $[0,\pi]$ (for example, 
$I = \left[\frac \pi 2 - \epsilon,\frac \pi 2 + \epsilon\right]$), let $B_I(t)$ 
be a bump function for $I$, normalized to have total mass one. Then Theorem 
\ref{thm:bad-Galois} gives Galois representations with empirical Sato--Tate 
distribution converging at any specified rate to $\mu_I = B_I(t)\, \dd t$. This 
is a strictly stronger result than \cite[Th.~5.2]{pande-2011}. Moreover, the 
proof of Theorem \ref{thm:int-flip-seq} shows that in fact for any 
$f\in C([0,\pi])$ with $f\circ \cos^{-1}\in C^1([-1,1])$ and 
$f(\pi-\theta) = -f(\theta)$, the Dirichlet series 
$L_f(\rho,s) = \prod \left( 1 - f(\theta_p) p^{-s}\right)^{-1}$ satisfies the 
Riemann hypothesis. 

\include{6_conclusion}





\bibliography{thesis}

\end{document}
