\documentclass[phd,cornellheadings,tocprelim]{cornell}

\usepackage{
	amsmath,
	amssymb,
	amsthm,
	mathrsfs, % mathscr
	stmaryrd, % llbracket
	thesis,   % custom math commands
	tikz-cd   % commutative diagrams
}

\SetSymbolFont{stmry}{bold}{U}{stmry}{m}{n} % Kills stmaryrd warning.
\usepackage[hidelinks]{hyperref}

\usepackage[
	backend  = bibtex,    % use bibtex instead of biber
	sorting  = nyt,       % sort by (name, year, title)
	style    = alphabetic % citations look like [Har77]
]{biblatex}

\tolerance=9999

\DeclareFieldFormat{postnote}{#1}
\DeclareFieldFormat{multipostnote}{#1}
\addbibresource{thesis.bib}

\title{Counterexamples related to the Sato--Tate conjecture}
\author{Daniel Miller}
\conferraldate{May}{2017}





\begin{document}
\maketitle
\makecopyright

\begin{abstract}
Let $E_{/\bQ}$ be an elliptic curve. The Sato--Tate conjecture, now a theorem, 
tells us that the angles $\theta_p =\cos^{-1}\left(\frac{a_p}{2\sqrt p}\right)$ 
are equidistributed in $[0,\pi]$ with respect to the measure 
$\frac{2}{\pi}\sin^2\theta\, \dd\theta$ if $E$ is non-CM
(resp.~$\frac{1}{2\pi} \dd \theta + \frac 1 2 \delta_{\pi/2}$ if $E$ is CM). 
In the non-CM case, Akiyama and Tanigawa conjecture that the discrepancy 
\[
	D_N = \sup_{x\in [0,\pi]} \left| \frac{1}{\pi(N)} \sum_{p\leqslant N} 1_{[0,x]}(\theta_p) - \int_0^x \frac{2}{\pi}\sin^2\theta\, \dd\theta\right| 
\]
asymptotically decays like $N^{-\frac 1 2+\epsilon}$, as is suggested by computational 
evidence and certain reasonable heuristics on the Kolmogorov--Smirnov 
statistic. This conjecture implies the Riemann hypothesis 
for all $L$-functions associated with $E$. It is natural to assume that the 
converse (``generalized Riemann hypothesis implies discrepancy estimate'') holds, 
as is suggested by analogy with Artin $L$-functions. We construct, for CM abelian 
varieties, ``fake Satake parameters'' yielding $L$-functions which satisfy 
the generalized Riemann hypothesis, but for which the discrepancy decays like 
$N^{-\epsilon}$. This provides evidence that in the CM case, the converse to 
``Akiyama--Tanigawa conjecture implies generalized Riemann hypothesis'' does 
not hold. 

We also show that there are Galois representations 
$\rho\colon \Gal(\overline \bQ /\bQ) \to \GL_2(\bZ_l)$, ramified at an 
arbitrarily thin set of primes, whose Satake parameters can be made to 
converge at any specified rate to any fixed measure $\mu$ on $[0,\pi]$ for 
which $\cos_\ast\mu$ is absolutely continuous. 
\end{abstract}

\begin{biosketch}
Daniel Miller was born in St.~Paul, Minnesota. He completed his Bachelor of 
Science at the University of Nebraska--Omaha. In addition to his studies there, 
he played the piano competitively and attended Cornell's Summer Mathematics 
Institute. He started his Ph.D.~at Cornell planning on a career in academia. 
Halfway through he had a change of heart, and will be joining Microsoft's 
Analysis and Experimentation team as a data scientist after graduation. He is 
happily married to Ivy Lai Miller, and has a cute but grumpy cat named Socrates. 
\end{biosketch}

\begin{dedication}
This thesis is dedicated to my undergraduate adviser, Griff Elder. He is the 
reason I considered a career in mathematics, and his infectious enthusiasm 
for number theory has inspired me more than I can say. 
\end{dedication}

\begin{acknowledgements}
I could not have completed this thesis without help and support from many 
people. I would like to offer my sincerest thanks to the following people. 

My parents Jay and Cindy for noticing and fostering my mathematical 
interests early on, and for being unfailingly loving and supportive. 

My undergraduate thesis advisor, Griffith Elder. Without his encouragement 
and inspiration I probably would have never considered a career in math. 

Tara Holm, Jason Boynton, and Anthony Weston for making Cornell's 2011 Summer 
Mathematics Institute the fantastic introduction to research it was. 

My fellow graduate students Sasha Patotski, Bal\'azs Elek, and Sergio Da 
Silva for sharing my early love of algebraic geometry, laughing with me at the 
absurdities of academic life, and listening to my ramblings about number theory. 

The mathematics department at Cornell, where many professors were generous 
with their time and ideas. I appreciate Yuri Berest, John Hubbard, Farbod 
Shokrieh, Birget Speh, and David Zywina for letting me bounce ideas off them, 
helping me add rigor to half-baked ideas, and pointing my research in new and 
interesting directions. 

My adviser Ravi Ramakrishna. He kindled my first love for number theory, 
stayed supportive as my research bounced all over the place, and kept me 
focused, grounded, and concrete when I needed to be. 

Most importantly, my wife Ivy for being there for me through the highs 
and the lows, when I prematurely thought my thesis was complete, and when I 
thought my results were completely in shambles. I couldn't have done it without 
her. 
\end{acknowledgements}

\contentspage
\normalspacing
\setcounter{page}{1}
\pagenumbering{arabic}
\pagestyle{cornell}
\addtolength{\parskip}{0.5\baselineskip}





% !TEX root = Daniel-Miller-thesis.tex

\chapter{Introduction}





\section{Motivation from classical analytic number theory}

Start with an old problem central to number theory: counting 
prime numbers. As usual, let $\pi(x)$ be the prime counting function and 
$\Li(x) = \int_2^x \frac{\dd t}{\log t}$ be the Eulerian logarithmic integral. 
The prime number theorem tells us that as $x\to \infty$, 
$\frac{\pi(x)}{\Li(x)} \to 1$. The standard approach 
to proving the prime number theorem is by showing that the Riemann 
$\zeta$-function has non-vanishing meromorphic continuation to $\Re = 1$.

\begin{theorem}
The function $\zeta(s)$ admits a non-vanishing meromorphic continuation to 
$\Re = 1$ with a simple pole at $s=1$, if and only if 
$\lim_{x\to \infty} \frac{\pi(x)}{\Li(x)} = 1$. 
\end{theorem}

Since $\zeta(s)$ does have the desired properties, the prime number 
theorem is true. It is natural to try to bound the difference 
$\pi(x) - \Li(x)$. Numerical experiments dating back to Gauss suggest that 
$|\pi(x) - \Li(x)| \ll x^{\frac 1 2+\epsilon}$. By this we mean that the 
estimate holds for any $\epsilon>0$, though the implied constant may depend on 
$\epsilon$. In fact, we have the following result. 

\begin{theorem}[{\cite[Th., p.~90]{edwards-1974}}]
The Riemann hypothesis is true if and only if 
$|\pi(x) - \Li(x)| \ll x^{\frac 1 2+\epsilon}$. 
\end{theorem}

Neither side of this equivalence is known for certain to be true! 

There is an analogue of the above discussion for Artin $L$-functions. 
Let $K/\bQ$ be a nontrivial finite  Galois extension with group 
$G=\Gal(K/\bQ)$. For any 
rational prime $p$ at which $K$ is unramified, let $\frob_p$ be the conjugacy 
class of the Frobenius at $p$ in $G$. For any complex irreducible representation 
$\rho$ of $G$, there is a corresponding $L$-function defined as 
\[
	L(\rho,s) = \prod_p \det\left(1-\rho(\frob_p) p^{-s}\right)^{-1} ,
\]
where here, and for the remainder of this thesis, we tacitly omit from the 
product those primes at which $\rho$ is ramified. If $\rho$ is the trivial 
representation, then $L(1,s) = \zeta(s)$. For each $p$, let $\delta_{\frob_p}$ 
be the Dirac delta measure concentrated at $\frob_p$ on $G^\natural$, the set 
of conjugacy classes of $G$. Given a cutoff $x$, there is a 
natural empirical measure 
$P_x = \frac{1}{\pi(x)} \sum_{p\leqslant x} \delta_{\frob_p}$ on $G^\natural$.
Let $\mu$ be the normalized Haar measure 
on $G^\natural$ (induced from the uniform measure on $G$), and let 
$\D(P_x) = \max_{S\subset G^\natural} \left| P_x(S) - \mu(S)\right|$. 
Then $P_x$ converges weakly to the Haar measure on $G^\natural$ if and 
only if $\D(P_x) \to 0$. Recall that weak convergence of $P_x$ to $\mu$ means 
$\int f\, \dd P_x \to \int f\, \dd\mu$ for all continuous functions $f$ on 
$G^\natural$. Since $G^\natural$ is a finite set, all functions on $G^\natural$ 
are continuous, but later on we will consider weak convergence on more general 
spaces.  

\begin{theorem}[{\cite[Th.~2 Cor., A.1]{serre-1989}}]
The measures $P_x$ converge weakly to the Haar measure on $G^\natural$ if and 
only if the function $L(\rho,s)$ admits a non-vanishing analytic continuation 
to $\Re = 1$ for all nontrivial $\rho$. 
\end{theorem}

Both sides of this equivalence are true, and known as the Chebotarev density 
theorem. If $K = \bQ$, so that $G$ (and hence $\rho$) are trivial, then the 
``Frobenius elements'' are all the identity, so equidistribution holds 
trivially. However, $\zeta(s)$ does not admit a non-vanishing \emph{analytic} 
continuation to $\Re = 1$, for it has a simple pole at $s = 1$. So the 
result is only true when $K/\bQ$ is a nontrivial extension. Returning to that 
case ($K\ne \bQ$), there is a version of the strong prime number theorem. It is 
known that Artin $L$-functions admit a meromorphic continuation to the complex 
plane, and that this continuation satisfies a functional equation. However, in 
this thesis, we will consider Dirichlet series for which no such continuation 
or functional equation exist---even conjecturally. As a result, in this thesis, 
by the ``Riemann hypothesis'' for a Dirichlet series $L(s)$ we mean the 
statement that $L(s)$ admits a non-vanishing analytic continuation to 
$\Re > \frac 1 2$. 

\begin{theorem}
The bound $\D(P_x) \ll x^{-\frac 1 2+\epsilon}$ holds if and only if each
$L(\rho,s)$, $\rho$ nontrivial, satisfies the Riemann hypothesis. 
\end{theorem}

The forward implication follows from Theorem \ref{thm:AT->RH:gp}, while the reverse implication is a result of Serre \cite[Th.~4]{serre-1981}. 
This whole discussion generalizes to a more complicated set of Galois 
representations---those arising from elliptic curves and more general motives.  





\section{Discrepancy and the Riemann hypothesis for elliptic curves}

For background on the Galois representations and $L$-functions associated to 
elliptic curves, see \cite[III\S7, C\S17]{silverman-2009}. Throughout this 
thesis, what we call the $L$-function of an elliptic curve (motive, etc.) is 
the normalized (i.e.~analytic instead of algebraic) $L$-function. 
Let $E_{/\bQ}$ be a non-CM elliptic curve. For any prime $l$, the $l$-adic Tate 
module of $E$ induces a continuous representation 
$\rho_l \colon G_\bQ \to \GL_2(\bZ_l)$. It is known that for $p$ not dividing 
either $l$ or the conductor of $E$, the quantities 
$a_p = \tr \rho_l(\frob_p)$ lie in $\bZ$, are independent of $l$, and satisfy 
the Hasse bound $|a_p| \leqslant 2\sqrt p$. For each unramified prime $p$, the 
corresponding Satake parameter for $E$ is 
$\theta_p = \cos^{-1}\left(\frac{a_p}{2\sqrt p}\right) \in [0,\pi]$. 
These parameters are packaged into an $L$-function as follows:
\[
	L(E,s) = \prod_p \frac{1}{(1 - e^{i \theta_p} p^{-s})(1- e^{-i \theta_p} p^{-s})} = \prod_p \det\left(1 - \smat{e^{i\theta_p}}{}{}{e^{-i \theta_p}}p^{-s}\right)^{-1}.
\]
More generally we have, for each irreducible representation $\sym^k$ of 
$\SU(2)$, the $k$-th symmetric power $L$-function: 
\[
	L(\sym^k E, s) = \prod_p \prod_{j=0}^k \frac{1}{1 - e^{i (k - 2j) \theta_p} p^{-s}} = \prod_p \det\left(1-\sym^k \smat{e^{i\theta_p}}{}{}{e^{-i \theta_p}}p^{-s}\right)^{-1}.
\]

Numerical experiments suggest that the Satake parameters are equidistributed 
with respect to the Sato--Tate distribution 
$\ST = \frac{2}{\pi} \sin^2\theta\, \dd\theta$. Indeed, for any cutoff $x$, let 
$P_x$ be the empirical measure 
$\frac{1}{\pi(x)} \sum_{p\leqslant x} \delta_{\theta_p}$. 
The convergence of $P_x$ to the Sato--Tate measure is closely related to 
the analytic properties of the $L(\sym^k E,s)$. First, here is the famous 
Sato--Tate conjecture, now a theorem, in our notation. 

\begin{theorem}[{\cite[Cor.~8.9]{bght-2011}}]
The measures $P_x$ converge weakly to $\ST$. 
\end{theorem}

\begin{theorem}[{\cite[Th.~2 Cor.]{serre-1989}}]
The Sato--Tate conjecture holds for $E$ if and only if each of 
the functions $L(\sym^k E,s)$ have analytic continuation to $\Re = 1$. 
\end{theorem}

The stunning recent proof of the Sato--Tate conjecture over totally real fields 
showed that the functions $L(\sym^k E,s)$ have the desired analytic 
continuation. Moreover, it showed that for all $k$, $L(\sym^k E,s)$ has 
meromorphic continuation to the whole complex plane. Even better, when $k$ is 
odd, the $L$-function is potentially automorphic. See Theorem 
\ref{thm:bad-Galois} for a result in this thesis where more can be said about 
odd symmetric power $L$-functions than even ones. 

The Riemann hypothesis, and its analogue for Artin $L$-functions, has a natural 
generalization to elliptic curves. In this context, the discrepancy of the set 
$\{\theta_p\}_{p\leqslant x}$ is 
\[
	\D_x\left(E,\ST\right) = \sup_{t\in [0,\pi]} \left| P_x[0,t) - \ST[0,t)\right| .
\]
The following conjecture is made in \cite{akiyama-tanigawa-1999}.

\begin{conjecture}[Akiyama--Tanigawa]\label{conj:akiyama-tanigawa}
$\D_x\left(E,\ST\right)\ll x^{-\frac 1 2+\epsilon}$.
\end{conjecture}

Akiyama and Tanigawa provide computational evidence for their conjecture, then 
go on to prove a special case of the following theorem, proved in full 
generality by Mazur. 

\begin{theorem}[{\cite[\S3.4]{mazur-2008}}]
If $\D_x\left(E,\ST\right)\ll x^{-\frac 1 2+\epsilon}$, 
then all the functions $L(\sym^k E, s)$ satisfy the Riemann hypothesis. 
\end{theorem}

This discussion also makes sense when $E$ has complex multiplication (for 
simplicity, we consider $E_{/F}$ where $F$ is the field of definition of the 
complex multiplication). The Sato--Tate measure for such $E$ is the Haar 
measure on $\SO(2)$, i.e.~the uniform measure on $[0,\pi]$. Instead of 
symmetric power $L$-functions, there is an $L$-function for each character of 
$\SO(2)$. Once again, there is a theorem ``Akiyama--Tanigawa conjecture implies 
generalized Riemann hypothesis.'' For a precise statement and proof, see Section 
\ref{sec:Satake-CM}.

It is natural to assume that the converse to the implication 
``Akiyama--Tanigawa conjecture implies Riemann hypothesis'' holds. 
In this thesis, we construct a range of counterexamples to the implication 
``generalized Riemann hypothesis implies fast discrepancy decay '' for sequences 
in compact real tori. This suggests that for CM abelian varieties, proving the 
converse to ``Akiyama--Tanigawa implies generalized Riemann hypothesis'' is not 
as straightforward as in the case of Artin $L$-functions. 

Moreover, we generalize the results of \cite{pande-2011} to show that there are 
(infinitely ramified) Galois representations whose Satake parameters exist and 
are equidistributed with respect to essentially arbitrary specified measures. 
Moreover, the rate of decay of discrepancy can be prescribed, and for ``odd'' 
measures, all the odd symmetric-power $L$-functions can be made to satisfy the 
Riemann hypothesis. We also show 
that some of the results of \cite{sarnak-2007} about sums of the form 
$\sum_{p\leqslant x} \frac{a_p}{\sqrt p}$ cannot be generalized to 
general---in particular, infinitely ramified---Galois representations. 

% !TEX root = main.tex

\chapter{Discrepancy}





\section{Definitions and first results}

regular discrepancy

star discrepancy

Euclidean space vs.~torus





\section{The Koksma--Hlawka inequality}

d





\section{Comparing sequences}

If $\{x_n\}\subset [0,\pi/2)$ has some discrepancy with respect to some 
measure, then the ``flipped'' sequence $\{\pi/2-x_n\}$ has the same discrepancy 
with respect to the ``flipped'' measure. 





\section{Combining sequences}

If $\{x_n\}$ and $\{y_n\}$ are sequences supported on $[0,\pi/2)$ and 
$[\pi/2,\pi)$ respectively, and both are equidistributed with respect to 
measures supported on their respective intervals, then the ``interleaved'' 
sequence $(x_1,y_1,x_2,y_2,\dots)$ also has equidistribution (with respect to 
the combined measure) and discrepancy which decays no faster than the slower of 
the two. 

% !TEX root = Daniel-Miller-thesis.tex

\chapter{Dirichlet series with Euler product}





\section{Definitions}

We start by considering a very general class of Dirichlet series, namely all 
Dirichlet series that admit a product formula with degree-1 factors. The 
motivating example was suggested to the author by Ravi Ramakrishna. Let 
$E_{/\bQ}$ be an elliptic curve and let 
\[
	L_{\sgn}(E,s) = \prod_p \frac{1}{1-\sgn(a_p) p^{-s}} .
\]
How much can we say about the behavior of $L_{\sgn}(E,s)$? For example, does it 
admit analytic continuation past $\Re = 1$? Can the rank of $E$ be found from 
$L_{\sgn}(E,s)$?

\begin{definition}
Let $\bx=(x_2,x_3,x_5,\dots)$ be a sequence of complex numbers indexed by the 
primes. The associated Dirichlet series is 
\[
	L(\bx,s) = \prod_p \frac{1}{1- x_p p^{-s}} .
\]
\end{definition}

If $x_p$ is defined only for a subset of the primes, we tacitly set $x_p = 0$ 
(so the Euler factor is $1$) at all pries for which $x_p$ is not defined. 

\begin{lemma}
Let $\bx$ be a sequence with $\|\bx\|_\infty \leqslant 1$. Then $L(\bx,s)$ 
defines a holomorphic function on the region $\Re > 1$. Moreover, on that 
region, 
\[
	\log L(\bx,s) = \sum_{p^r} \frac{x_p^n}{n p^{n s}} .
\]
\end{lemma}
\begin{proof}
Expanding the product for $L(\bx,s)$ formally, we have 
\[
	L(\bx,s) = \sum_{n\geqslant 1} \frac{\prod_p x_p^{v_p(n)}}{n^s} .
\]
An easy comparison with the Riemann zeta function tells us that this sum 
is holomorphic on $\Re > 1$. By \cite[Th.~11.7]{apostol-1976}, the 
product formula holds in the same region. The formula for $\log L(\bx,s)$ 
comes from \cite[11.9 Ex.2]{apostol-1976}. 
\end{proof}

\begin{lemma}[Abel summation]\label{lem:abel-sum}
Let $\bx=(x_2,x_3,x_5,\dots)$ be a sequence of complex numbers, $f$ a smooth 
complex-valued function on $\bR$. Then 
\[
	\sum_{p\leqslant N} f(p) x_p = f(N) \sum_{p\leqslant N} x_p - \int_2^N f'(t) \sum_{p\leqslant t} x_p\, \dd t .
\]
\end{lemma}
\begin{proof}
Simply note that if $p_1,\dots,p_n$ is an enumeration of the primes 
$\leqslant N$, we have 
\begin{align*}
	\int_2^N f'(t) \sum_{p\leqslant t} x_p\, \dd t 
		&= \sum_{p\leqslant N} x_p \int_{p_n}^N f'(t)\, \dd t + \sum_{i=1}^{n-1} \sum_{p\leqslant p_{i+1}} x_p \int_{p_i}^{p_{i+1}} f'(t)\, \dd t \\
		&= \left(f(N) - f(p_n)\right) \sum_{p\leqslant N} x_p + \sum_{i=1}^{n-1} \left(f(p_{i+1}) - f(p_i)\right) \sum_{p\leqslant p_{i+1}} x_p \\
		&= f(N) \sum_{p\leqslant N} x_p - \sum_{p\leqslant N} f(p) x_p ,
\end{align*}
as desired. 
\end{proof}

\begin{theorem}\label{thm:AT->RH}
Assume $|\sum_{p\leqslant N} x_p| \ll N^{\alpha+\epsilon}$ for some 
$\alpha\in [\frac 1 2,1]$. Then the series for $\log L(\bx,s)$ converges 
conditionally to a holomorphic function on the region $\Re > \alpha$. 
\end{theorem}
\begin{proof}
Formally split the sum for $\log L(\bx,s)$ into two pieces: 
\[
	\log L(\bx,s) = \sum_p \frac{x_p}{p^s} + \sum_p \sum_{r\geqslant 2} \frac{x_p^r}{r p^{r s}} .
\]
For each $p$, we have 
\[
	\left| \sum_{r\geqslant 2} \frac{x_p^r}{r p^{r s}}\right| \leqslant \sum_{r\geqslant 2} p^{- r \Re s} = p^{-2 \Re s} \frac{1}{1-p^{-\Re s}} .
\]
Elementary analysis gives 
\[
	1 \leqslant \frac{1}{1-p^{-\Re s}} \leqslant 2 + 2\sqrt 2 ,
\]
so the second piece of $\log L(\bz,s)$ converges absolutely when 
$\Re s>\frac 1 2$. We could simply cite \cite[II.1 Th.~10]{tenenbaum-1995} to 
finish the proof; 
instead we prove directly that $\sum \frac{x_p}{p^s}$ converges absolutely 
to a holomorphic function on the region $\Re > \alpha$. 

By Lemma \ref{lem:abel-sum} with $f(t) = t^{-s}$, we have 
\begin{align*}
	\sum_{p\leqslant N} \frac{x_p}{p^s}
		&= N^{-s} \sum_{p\leqslant N} x_p + s \int_2^N \sum_{p\leqslant t} x_p\, \frac{\dd t}{t^{s+1}} \\
		&\ll N^{-\Re s + \alpha + \epsilon} + s \int_2^N t^{\alpha+\epsilon} \frac{\dd t}{t^{\Re s+1}} .
\end{align*}
Since $\alpha-\Re s < 0$, the first term is bounded. Since 
$\Re s+1-\alpha > 1$ and 
$\epsilon$ is arbitrary, the integral converges absolutely, and the proof is 
complete. 
\end{proof}

Note that the proof of Theorem \ref{thm:AT->RH} actually gives an absolutely 
convergent expression for $\log L(\bx,s)$ on the region $\Re >\alpha$. Namely, 
\[
	\log L(\bx,s) = s \int_2^\infty t^{-s-1}\sum_{p\leqslant t} x_p \dd t + \sum_p \sum_{r\geqslant 2} \frac{x_p^r}{r p^{r s}} . 
\]

Let $X$ be a space, $f\colon X\to \bC$ a function with 
$\|f\|_\infty\leqslant 1$, and $\bx=(x_2,x_3,\dots)$ a sequence in $X$. Write 
\[
	L_f(\bx,s) = \prod_p \frac{1}{1-f(x_p) p^{-s}} ,
\]
for the associated Dirichlet series. In the remainder, we will 
exclusively focus on strange Dirichlet series of this type. 





\section{Relation to automorphic and motivic \texorpdfstring{$L$}{L}-functions}

Suppose $G$ is a compact group, $G^\natural$ the space of conjugacy classes in 
$G$. If $\bx = (x_2,x_3,x_5,\dots)$ is a sequence in $G^\natural$ and $\rho$ is 
a finite-dimensional representation of $G$, put 
\[
	L(\rho(\bx),s) = \prod_p \frac{1}{\det(1-\rho(x_p) p^{-s})} .
\]
Clearly $L((\rho_1\oplus \rho_2)(\bx),s) = L(\rho_1(\bx),s) L(\rho_2(\bx),s)$. 
Now, let $T\subset G$ be a maximal torus, and recall that 
$T\twoheadrightarrow G^\natural$. The representation 
$\left.\rho\right|_T$ decomposes as $\bigoplus \chi^{\oplus m_\chi}$, where 
$\chi$ ranges over characters of $T$ and the entire expression is 
$W$-invariant. We may regard the $x_p$ as lying in $T/W$, so we have 
\[
	L(\rho(\bx),s) = \prod_\chi L(\chi(\bx),s)^{m_\chi} .
\]
If the trivial representation appears in $\left.\rho\right|_T$, this product 
formula will include a copy (possibly several) of $\zeta(s)$. 





\section{Discrepancy of sequences and the Riemann Hypothesis}

\begin{definition}
We say the \emph{Riemann Hypothesis} for $L(\bx,s)$ holds if the function 
$\log L(\bx,s)$ admits analytic continuation to $\Re > \frac 1 2$. 
\end{definition}

Under reasonable analytic hypotheses, namely conditional convergence of 
$\log L(\bx,s)$ on $\Re > \frac 1 2$, \cite[II.1 Th.~10]{tenenbaum-1995} gives 
an estimate $|\sum_{p\leqslant N} x_p| \ll N^{\frac 1 2 + \epsilon}$. 

\begin{theorem}
Let $(X,\mu)$ be a probability space in which discrepancy makes sense, and let 
$\bx=(x_2,x_3,x_5,\dots)$ be a sequence in $X$ with 
$\disc(\bx^N,\mu) \ll N^{-\frac 1 2+\epsilon}$. Then for any function $f$ on 
$X$ of bounded variation, the strange Dirichlet series $L_f(\bx,s)$ satisfies 
the Riemann Hypothesis. 
\end{theorem}
\begin{proof}
The bound on discrepancy yields (by the Koksma--Hlawka inequality) the estimate 
\[
	\left| \sum_{p\leqslant N} f(x_p)\right| \ll N^{-\frac 1 2+\epsilon} .
\]
By Theorem \ref{thm:AT->RH}, the Riemann Hypothesis holds for $L_f(\bx,s)$. 
\end{proof}

The same proof shows that if $\disc(\bx^N,\mu)\ll N^{-\alpha+\epsilon}$, then 
$\log L(\bx,s)$ conditionally converges to a holomorphic function on 
$\Re > 1 - \alpha$. 

Let $F = \bF_q(t)$ be a function field, $E_{/F}$ a generic elliptic curve. 
There is, for every prime $\fp$ of $F$, a Satake parameter 
$\theta_\fp \in [0,\pi]$, defined in the usual way. It is known 
\cite[Ch.~3]{katz-1988} that
\[
	\left|\sum_{\N(\fp) \leqslant x} U_k(\theta_\fp) \right| \ll k \sqrt{x} .
\]
This tells us that for any $f\in C(\SU(2))$ with 
$\sum_{k\geqslant 1} |\widehat f(\sym^k)|<\infty$, the strange Dirichlet 
series $L_f(\btheta,s)$ satisfies the Riemann Hypothesis. 

For function fields, the best estimate on discrepancy is found in 
\cite{niederreiter-1991}, where it is shown that $D_N \ll N^{-1/4}$ 
by applying a generalization of the Koksma--Hlawka inequality to 
$\SU(2)^\natural$. Namely, for any odd $r$, we have 
\[
	\disc(\btheta^x,\ST) \ll \frac 1 r + \sum_{k=1}^{2 r -1} \frac 1 k \left| \frac{1}{\pi_F(x)} \sum_{\N(\fp)\leqslant x} U_k(\btheta_\fp)\right| .
\]
Using the above estimate on $\sum U_k(\btheta_\fp)$, he is able to derive 
$D_N \ll N^{-1/4}$. This fits the results of 
\cite{bucar-kedlaya-2015,rouse-thorner-2016}, both of which derive estimates 
of the form $D_N \ll N^{-1/4+\epsilon}$ under GRH + functional equation for 
the (non-CM) elliptic curve in question. 

% !TEX root = Daniel-Miller-thesis.tex

\chapter{Irrationality exponents}\label{chapter:irrationality-exponent}





\section{Definitions and first results}

We follow the notation of \cite{laurent-2009}. Fix a dimension 
$d\geqslant 1$, and let $\vx=(x_1,\dots,x_d)\in \bR^d$ be such that the $x_i$ 
are linearly independent over $\bQ$. If $d = 1$, the \emph{irrationality 
exponent} of $x\in \bR$ is the supremum of the set of $w\in \bR^+$ such that 
there infinitely many rational numbers $\frac p q$ ($p,q$ relatively prime) with 
$\left| x - \frac p q\right| \leqslant q^{-w}$. If $x$ is rational, then it has 
irrationality exponent $1$. If $x$ is an algebraic irrational, then Roth's 
theorem tells us its irrationality exponent is $2$. Liouiville constructed 
transcendental numbers with arbitrarily large irrationality exponent. In fact, 
a measure-zero set of reals have infinite irrationality exponent, for example 
the Louiville number $\sum_{r\geqslant 1} 10^{-r!}$. In the results below, we 
will only consider reals with finite irrationality constant. When 
$d\geqslant 1$, there are a $d$ natural measures of irrationality, but we will 
use only two of them. 

For the remainder of this thesis, $\langle \cdot,\cdot\rangle$ is the standard 
inner product on $\bR^d$. 

\begin{definition}\label{def:approx-exp}
Let $\omega_0(\vx)$ (resp.~$\omega_{d-1}(\vx)$) be the supremum of the set of 
real numbers $w$ for which there exist infinitely many 
$(n,\vm)\in \bZ\times\bZ^d$ such that 
\begin{align*}
	|n \vx - \vm|_\infty 
		&\leqslant |(n,\vm)|_\infty^{-w}  \qquad\text{(resp.} \\
	|n +\langle \vm,\vx\rangle| 
		&\leqslant |(n,\vm)|_\infty^{-w} \text{).}
\end{align*}
\end{definition}

It is easy to see that both $\omega_0(\vx)$ and $\omega_{d-1}(\vx)$ are 
nonnegative. These two quantities are related by Khintchine's transference 
principle \cite[Th.~2]{laurent-2009}, namely 
\[
	\frac{\omega_{d-1}(\vx)}{(d-1) \omega_{d-1}(\vx)+d} \leqslant \omega_0(\vx) \leqslant \frac{\omega_{d-1}(\vx)-d+1}{d} .
\]
Moreover, the second of these inequalities is sharp in a very strong sense. 

% For Jarnik's paper online
% http://pldml.icm.edu.pl/pldml/element/bwmeta1.element.bwnjournal-article-aav2i1p1bwm
\begin{theorem}[\cite{jarnik-1936}]\label{thm:jarnik}
Let $w\geqslant 1/d$. Then there exists $\vx\in \bR^d$ such that 
$\omega_0(\vx)=w$ and $\omega_{d-1}(\vx) = d w+d-1$. 
\end{theorem}

\begin{theorem}
If $d=1$ and $x\in (0,1)$, then $\omega_0(x) = \mu-1$, where $\mu$ is the 
traditional irrationality measure of $x$. 
\end{theorem}
\begin{proof}
First we show that $\omega_0(x)\geqslant \mu-1$. Suppose there exist infinitely 
many $p/q$ with $\left| x - \frac p q\right| \leqslant q^{-w}$. Since $x<1$ we 
may assume that for infinitely many of the $p/q$, $p<q$. Then 
$| q x - p| \leqslant q^{-w + 1} = \max(p,q)^{-w}$, which tells us that 
$\omega_0(x) \geqslant \mu - 1$. 

To-do: complete proof.
\end{proof}

\begin{theorem}[\cite{roth-1955}]
Let $x\in (\overline\bQ\cap \bR)\smallsetminus \bQ$. Then 
$\omega_0(x) = 1$. 
\end{theorem}

Given $\vx\in \bR^d$, write 
$\dd(\vx,\bZ^d)=\min_{\vm\in \bZ^d} |\vx-\vm|_\infty$. Note that 
$\dd(\vx,\bZ^d)=0$ if and only if $\vx\in \bZ^d$. Moreover, $\dd(-,\bZ^d)$ 
is well-defined for elements of $(\bR/\bZ)^d$. 

\begin{lemma}\label{lem:bound-distance}
Let $\vx\in \bR^d$ with $|\vx|_\infty< 1$ and $\omega_0(\vx)$ 
(resp.~$\omega_{d-1}(\vx)$) finite. Then 
\begin{align*}
	\frac{1}{\dd(n \vx,\bZ^d)} 
		&\ll |n|^{\omega_0(\vx)+\epsilon}\qquad \text{ for $n\in \bZ$ (resp.} \\
	\frac{1}{\dd\left(\langle \vm,\vx\rangle, \bZ\right)} 
		&\ll |\vm|_\infty^{\omega_{d-1}(\vx)+\epsilon} \qquad\text{ for $\vm\in \bZ^d$).}
\end{align*}
\end{lemma}
\begin{proof}
Let $\epsilon>0$. Then there are only finitely many $n\in \bZ$ 
(resp.~$\vm\in \bZ^d$) such that the inequalities in Definition 
\ref{def:approx-exp} hold with $w = \omega_0(x)+\epsilon$ 
(resp.~$w = \omega_{d-1}(\vx)+\epsilon$). In other words, there exist constants 
$C_0, C_{d-1}>0$, depending on $\vx$ and $\epsilon$, such that 
\begin{align*}
	|n \vx - \vm|_\infty 
		&\geqslant C_0 |(n,\vm)|_\infty^{-\omega_0(\vx)-\epsilon} ,\\
	|n + \langle \vm,\vx\rangle| 
		&\geqslant C_{d-1} |(n,\vm)|_\infty^{-\omega_{d-1}(\vx)-\epsilon} 
\end{align*}
for all $(n,\vm)\ne 0$ in $\bZ\times\bZ^d$. 

Start with the first inequality. Fix $n$, and let $\vm$ be a lattice point 
achieving the minimum $|n \vx - \vm|_\infty$; then 
$\dd(n \vx,\bZ^d) \geqslant C_0 |(n,\vm)|_\infty^{-\omega_0(\vx)-\epsilon}$. 
Since $|n\vx - \vm|_\infty < 1$, the reverse triangle inequality gives 
$\left| |n| - \frac{|\vm|_\infty}{|\vx|_\infty}\right| \leqslant \frac{1}{|\vx|_\infty}$. So $|n|$ and $|\vm|$ are bounded above and below by scalar multiples 
of each other, which tells us that 
$\dd(n \vx,\bZ^d) \geqslant C_0' |n|^{-\omega_0(\vx)-\epsilon}$ for $C_0'$ 
depending on $\vx$. It follows that 
$\frac{1}{\dd(n \vx,\bZ^d)} \ll |n|^{\omega_0(\vx)+\epsilon}$, the 
implied constant depending on $x$ and $\epsilon$.

Now we consider the second inequality. Note 
that $\dd(\langle \vm,\vx\rangle,\bZ) = |n + \langle \vm,\vx\rangle|$ for 
some $n$ with $|n| \leqslant |\vm|_2\cdot |\vx|_2 + 1$. Thus 
$|(n,\vm)|_\infty \ll |\vm|_2 \ll |\vm|_\infty$ with the implied constants 
depending on $d$ and $x$, because any two norms on a 
finite-dimensional Banach space are equivalent. This gives us 
$\dd(\langle\vm,\vx\rangle,\bZ) \geqslant C_{d-1}' |\vm|_\infty^{-\omega_{d-1}(\vx)-\epsilon}$, 
for some constant $C_{d-1}'$. This implies 
\[
	\frac{1}{d(\langle \vm,\vx\rangle,\bZ)} \ll |\vm|_\infty^{\omega_{d-1}(\vx)+\epsilon} ,
\]
the implied constant depending on $\vx$ and $\epsilon$.
\end{proof}





\section{Irrationality exponents and discrepancy}

Let $\vx=(x_1,\dots,x_d)\in \bR^d$. The sequence 
$(\vx\mod \bZ^d,2\vx\mod\bZ^d,\dots)$ will be equidistributed in a subtorus of 
$(\bR/\bZ)^d$ whose rank is equal to the dimension of the $\bQ$-vector space 
spanned by $\{x_1,\dots,x_d\}$. We are interested in the case where this 
sequence is equidistributed in the whole torus $(\bR/\bZ)^d$, so assume 
$x_1,\dots,x_d$ are linearly independent over $\bQ$ (this condition also 
makes sense for elements of $(\bR/\bZ)^d$). For $\vx\in (\bR/\bZ)^d$, we wish 
to control the discrepancy of the sequence $(\vx,2\vx,3\vx,\dots)$ with respect 
to the Haar measure of $(\bR/\bZ)^d$. 

\begin{theorem}[Erd\"os--Tur\'an--Koksma]
Let $\bvx$ be a sequence in $(\bR/\bZ)^d$ and $h$ an arbitrary integer. Then 
\[
	\D_N(\bvx) \ll \frac 1 h + \sum_{0\leqslant |\vm|_\infty \leqslant h} \frac{1}{r(\vm)} \left| \frac 1 N \sum_{n\leqslant N} e^{2\pi i \langle \vm,\vx_n\rangle}\right| ,
\]
where the first sum ranges over $\vm\in \bZ^d$, 
$r(\vm) = \prod \max\{1,|m_i|\}$, and the implied constant depends only on $d$. 
\end{theorem}
\begin{proof}
This is \cite[Th.~1.21]{drmota-tichy-1997}. 
\end{proof}

\begin{lemma}\label{lem:bound-exp-sum}
Let $x\in \bR$. Then 
\[
	\left| \sum_{n\leqslant N} e^{2\pi i n x}\right| \leqslant \frac{2}{\dd(x, \bZ)} .
\]
\end{lemma}
\begin{proof}
We begin with an easy bound: 
\[
	\left| \sum_{n\leqslant N} e^{2\pi i n x}\right| = \frac{|e^{2\pi i (N+1) x} - e^{2\pi i x}|}{|e^{2\pi i x} - 1|} \leqslant \frac{2}{|e^{2\pi i x} - 1|} .
\]
Since $|e^{2\pi i x} - 1| = \sqrt{2-2\cos(2\pi x)}$ and 
$\cos(2\theta) = 1-2\sin^2\theta$, we obtain 
\[
	\left|\sum_{n\leqslant N} e^{2\pi i n x}\right| \leqslant \frac{1}{|\sin(\pi x)|} .
\]
It is easy to check that $|\sin(\pi x)| \geqslant \dd(x,\bZ)$, whence the result.  
\end{proof}

\begin{corollary}\label{cor:bound-disc-distance}
Let $\vx\in (\bR/\bZ)^d$ with $(x_1,\dots,x_d)$ linearly independent over $\bQ$. 
Then for $\bx=(\vx,2\vx,3\vx,\dots)$, we have 
\[
	\D_N(\bx) \ll \frac 1 h + \frac 1 N \sum_{0<|\vm|_\infty \leqslant h} \frac{2}{r(\vm) \dd(\langle \vm,\vx\rangle,\bZ)} 
\]
for any integer $h$, with the implied constant depending only on $d$. 
\end{corollary}
\begin{proof}
Apply the Erd\"os--Tur\'an--Koksma inequality, and bound the exponential sums 
using Lemma \ref{lem:bound-exp-sum}. 
\end{proof}

\begin{theorem}\label{thm:disc-upper-bound}
Let $\bx=(\vx,2\vx,3\vx,\dots)$ in $(\bR/\bZ)^d$. Then 
\[
	\D_N(\bx) \ll N^{-\frac{1}{\omega_{d-1}(\vx)+1}+\epsilon} .
\]
\end{theorem}
\begin{proof}
Fix $\epsilon>0$ smaller than $\frac{1}{\omega_{d-1}(\vx) - 1}$, and choose 
$\delta>0$ such that 
$\frac{1}{\omega_{d-1}(\vx)+1+\delta} = \frac{1}{\omega_{d-1}(\vx)+1} - \epsilon$. 
By Corollary \ref{cor:bound-disc-distance}, we know that 
\[
	\D_N(\bx) \ll \frac 1 h + \frac 1 N \sum_{0<|\vm|_\infty \leqslant h} \frac{1}{r(\vm) \dd(\langle \vm,\vx\rangle,\bZ)} ,
\]
and by Lemma \ref{lem:bound-distance}, we know that 
$\dd(\langle \vm,\vx\rangle,\bZ)^{-1} \ll |\vm|^{\omega_{d-1}(\vx)+\delta}$. 
It follows that 
\[
	\D_N(\bx) \ll \frac 1 h + \frac 1 N \sum_{0 < |\vm|_\infty \leqslant h} \frac{|\vm|_\infty^{\omega_{d-1}(\vx)+\delta}}{r(\vm)} .
\]
All that remains is to bound the sum. Clearly 
\[
	\sum_{0 < |\vm|_\infty \leqslant h} \frac{|\vm|_\infty^{\omega_{d-1}(\vx) + \delta}}{r(m)} \ll \int_1^h \int_1^h \dots \int_1^h \frac{\max(|t_1|,\dots,|t_d|)^{\omega_{d-1}(\vx)+\delta}}{t_1 \dots t_d}\, \dd t_1 \dots \dd t_d .
\]
For each permutation $\sigma$ of $\{1,\dots,d\}$, call $I_\sigma$ the set of 
all $(t_1,\dots,t_d)$ in $[1,\infty)^d$ with 
$t_{\sigma(1)} < \dots < t_{\sigma(d)}$. Then 
$[1,\infty)^d = \bigcup_{\sigma\in S_d} I_\sigma$, and each integral over 
$I_\sigma$ is easy to bound. For example, the integral over $I_1$ is 
\begin{align*}
	\int_1^h \int_1^{t_d} \dots \int_1^{t_2} \frac{t_d^{\omega_{d-1}(\vx)+\delta}}{t_1 \dots t_d}\, \dd t_1 \dots \dd t_d 
		&\ll \int_1^h t^{\omega_{d-1}(\vx)+\delta-1}\, \dd t \prod_{j=1}^{d-1} \int_1^h \frac{\dd t}{t} \\
		&\ll (\log h)^{d-1} h^{\omega_{d-1}(\vx)+\delta} .
\end{align*}
It follows that 
$\D_N(\bx) \ll \frac 1 h + \frac 1 N (\log h)^{d-1} h^{\omega_{d-1}(\vx)+\delta}$. 
Setting $h\approx N^{\frac{1}{1+\omega_{d-1}(\vx)+\delta}}$, we see that 
$D_N(\bx) \ll N^{-\frac{1}{\omega_{d-1}(\vx)+1+\delta}} = N^{-\frac{1}{\omega_{d-1}(\vx)+1} + \epsilon}$. 

For a slightly different proof of a similar result, given as a sequence of 
exercises, see  \cite[Ch.~2, Ex.~3.15, 16, 17]{kuipers-niederreiter-1974}. 
Also, this estimate is quite coarse, but a better one would only have a smaller 
leading coefficient, which would no doubt be useful for computational 
purposes, but does not strengthen any of the results in this thesis.  
\end{proof}

\begin{theorem}\label{thm:disc-lower-bound}
Let $\vx\in \bR^d$ be such that $x_1,\dots,x_d$ are linearly independent over 
$\bQ$, and let $\bx=(\vx,2\vx,3\vx,\dots)$ in $(\bR/\bZ)^d$. Then 
$\D_N(\bx) = \Omega\left(N^{-\frac{d}{\omega_0(\vx)}-\epsilon} \right)$. 
\end{theorem}
\begin{proof}
We follow the proof of \cite[Ch.~2, Th.~3.3]{kuipers-niederreiter-1974}, 
modifying it as needed for our context. Given $\epsilon>0$, there exists 
$\delta>0$ such that 
$\frac{d}{\omega_0(\vx)-\delta} = \frac{d}{\omega_0(\vx)} + \epsilon$. 

By the definition of $\omega_0(\vx)$, there exist infinitely many 
$(q,\vm)$ with $q>0$ such that 
$|q \vx - \vm|_\infty \leqslant |(q,\vm)|_\infty^{-\omega_0(\vx)+\delta/2}$. 
Since $|(q,\vm)|_\infty \geqslant q$, we derive the stronger 
statement that for infinitely many $q$, there exists 
$\vm\in \bZ^d$ such that 
$|q \vx-\vm|_\infty \leqslant q^{-\omega_0(\vx)+\delta/2}$ or, equivalently, 
$|\vx-q^{-1}\vm| \leqslant q^{-1-\omega_0(\vx)+\delta/2}$. Fix one such $q$, 
and let $N=\lfloor q^{\omega_0(\vx)-\delta}\rfloor$. For each $n\leqslant N$, 
we have 
\[
	\left|n \vx - n q^{-1} \vm\right|_\infty 
		\leqslant n \left|\vx - q^{-1} \vm\right|_\infty
		\leqslant n q^{-1-\omega_0(\vx)+\delta/2}
		\leqslant q^{-1-\delta/2}. 
\]
Thus, for each $n\leqslant N$, $n \vx$ is within $q^{-1-\delta/2}$ of the 
grid $\frac 1 q \bZ^d\subset (\bR/\bZ)^d$. So no element of 
$\{\vx,\dots,N \vx\}$ lies any the half-open box 
$I_q = \left[ q^{-1 - 2\delta / 3}, q^{-1} - q^{-1 - 2\delta / 3}\right)^d$. 
Moreover, $I_q$ has volume $\left(q^{-1} - 2q^{-1 - 2\delta / 3}\right)^d$. 
For $q$ sufficiently large, the volume of $I_q$ is bounded below by 
$2^{-d} q^{-d}$, so the discrepancy $\D_N^\star(\bvx)$ is 
bounded below by $2^{-d} q^{-d}$. Since $q^{\omega_0(\vx)-\delta} \leqslant 2 N$, 
the discrepancy $\D_N^\star(\bvx)$ is bounded below by 
\[
	2^{-d} \left( (2 N)^{\frac{1}{\omega_0(\vx)+\delta}}\right)^{-d} 
		= 2^{-d-\frac{d}{\omega_0(\vx)+\delta}} N^{-\frac{d}{\omega_0(\vx)+\delta}}
		= 2^{-d\left(1+\frac{1}{\omega_0(\vx)}\right)-\epsilon} N^{-\frac{d}{\omega_0(\vx)}-\epsilon} .
\]
Since $\D_N^\star(\bvx)$ can, for $N\to \infty$, be bounded below by a constant 
multiple of $N^{-\frac{d}{\omega_0(\vx)}}$, the proof is complete after an 
application of Lemma \cite{lem:star-reg-disc}. 
\end{proof}






\section{Pathological Satake parameters for CM abelian varieties}

We apply the results of the previous sections to $L$-functions associated to 
CM abelian varieties. For background, see 
\cite{serre-tate-1968,serre-1994,yu-2015}. 
Recall that for $E$ a non-CM elliptic curve, the 
Akiyama--Tanigawa conjecture implies the Riemann Hypothesis for all 
$L(\sym^k E,s)$, $k\geqslant 1$. The appearance of $\sym^k$ is dictated by the 
classification of irreducible representations of $\SU(2)$, the Sato--Tate group 
of $E$. If $A$ is a CM abelian variety, there should be an $L$-function (and 
Galois representation) for each irreducible representation of the Sato--Tate 
group of $A$, which we denote by $\ST(A)$. In the 
CM case, $\ST(A)$ is a real torus, so things can be described relatively 
explicitly. 

Let $K/\bQ$ be a finite Galois extension, $A_{/K}$ a $g$-dimensional abelian 
variety with complex multiplication by $F$, defined over $K$, that is, 
$F = \End_K(A)_\bQ$. Since the action of $F$ commutes with 
$\rho_l\colon G_\bQ \to \GL_{2g}(\bQ_l)$, the Galois representation coming 
from the $l$-adic Tate module of $A$ takes values in $\R_{F/\bQ}\Gm(\bQ_l)$, 
where $R_{F/\bQ} \Gm$ is the Weil restriction of scalars of the multiplicative 
group from $F$ to $\bQ$. The functor of points of $\R_{F/\bQ}\Gm$ is 
$R\mapsto (R\otimes F)^\times$. It follows that the Sato--Tate group of $A$ is 
a subgroup of the maximal compact torus inside $\R_{F/\bQ}\Gm(\bC)$. 

Recall, following \cite{serre-1994}, that the motivic Galois group of $A$ 
should be a subgroup $G_A\subset \R_{F/\bQ}\Gm$ such that for all primes $l$, 
the image $\rho_l(G_\bQ)$ lies inside $G_A(\bQ_l)$, and is open in 
$G_A(\bQ_l)$. For general abelian varieties, the existence of the motivic 
Galois group is a matter of conjecture, but for CM abelian varieties, it can be 
described directly. Let $\fa=\lie(A)$, and 
$\det_\fa\colon \R_{K/\bQ}\Gm \to \R_{F/\bQ}\Gm$ be the map induced by the 
determinant of the action of $K$ on $\fa$ (viewed as an $F$-vector space). Then 
$G_A = \im(\det_\fa)$ \cite{yu-2015}, and $\ST(A)$ is a maximal compact 
subgroup of $G_A^1(\bC) = G_A^{\N_{F/\bQ} = 1}(\bC)$. So 
$\ST(A) \simeq (\bR/\bZ)^d$ for some $1\leqslant d \leqslant g$, and every 
unitary character of $\ST(A)$ is induced by an algebraic character of 
$G_A^1$. Any character of a subtorus extends to the whole torus, so any 
character of $G_A^1$ is the restriction of a character of $\R_{F/\bQ}\Gm$. 

Let $\fp$ be a prime of $K$ at which $A$ has good reduction. Then 
$F = \End(A)_\bQ\hookrightarrow \End(A_{/\bF_\fp})_\bQ$, and the Frobenius 
element $\frob_\fp\in \End(A_{/\bF_\fp})_\bQ$ comes from an element 
$\pi_\fp\in F$. In other words, $\rho_l(\frob_\fp) = \pi_\fp$. The element 
$\pi_\fp\in F$ is $\fp$-Weil of weight $1$, 
i.e.~$|\sigma(\pi_\fp)| = \N(\fp)^{1/2}$ for all embeddings 
$\sigma\colon F\hookrightarrow \bC$. The normalized element 
$\theta_\fp = \frac{\pi_\fp}{\N(\fp)^{1/2}}$ lies in $\ST(A)$, and we call this 
the Satake parameter for $A$ at $\fp$. For the Satake parameters to be 
equidistributed in $\ST(A)$, it is necessary and sufficient for the 
$L$-function $L(r\circ \rho_l,s)$ to have non-vanishing analytic continuation 
to $\Re =1$ for each $r\in \X^\ast(\R_{F/\bQ}\Gm)$ which has nontrivial 
restriction to $\ST(A)$. By the Wiener--Ikehara Tauberian theorem, this is 
equivalent to an estimate 
$\left| \sum_{\N(fp)\leqslant x} r(\theta_\fp)\right| = o(\pi_K(x))$. 

\begin{theorem}[Shimura--Taniyama, Weil, Hecke]
The elements $\theta_\fp\in \ST(A)$ are equidistributed with respect to the 
Haar measure. 
\end{theorem}
\begin{proof}
By \cite[Th.~10, 11]{serre-tate-1968}, for every 
$r\in \X^\ast(\R_{F/\bQ}\Gm)$, there exists a Hecke character $\omega_r$ of $K$ 
such that $L(r\circ \rho_l,s) = L(s,\omega_r)$, and moreover $\omega_r$ is 
nontrivial if and only if $\left.r\right|_{\ST(A)}$ is. Since $L$-functions of 
Hecke characters have the desired analytic continuation and nonvanishing, the 
result follows. 
\end{proof}

Recall that 
$L(r\circ \rho_l,s) = \prod \left(1 - r(\theta_\fp) \N(\fp)^{-s}\right)^{-1}$ 
(this is the normalized $L$-function, not the algebraic $L$-function). 
As in Chapter \ref{ch2:discrepancy}, the choice of an isomorphism
$(\bR/\bZ)^d\simeq \ST(A)$ gives rise to a notion of discrepancy 
$\D_N(\btheta)$. We call the ``Akiyama--Tanigawa conjecture for $A$'' the 
estimate $\D_N(\btheta) \ll N^{-\frac 1 2+\epsilon}$. 

\begin{theorem}
Let $A_{/K}$ be a CM abelian variety. The Akiyama--Tanigawa conjecture for $A$ 
implies the Riemann Hypothesis for all $L(r\circ \rho_l,s)$ with 
$\left. r\right|_{\ST(A)}$ nontrivial. 
\end{theorem}
\begin{proof}
The Akiyama--Tanigawa estimate implies, via the Koksma--Hlawka inequality, an 
estimate 
$\left| \sum_{\N(\fp)\leqslant N} r(\theta_\fp)\right| \ll N^{-\frac 1 2+\epsilon}$. 
By Theorem \ref{thm:AT->RH}, the function $L(r\circ \rho_l,s)$ satisfies the 
Riemann Hypothesis. 
\end{proof}

It is natural to ask: does the Riemann Hypothesis for all $L(r\circ \rho_l,s)$ 
imply the Akiyama--Tanigawa conjecture for $A$? We proceed to construct 
$L$-functions coming from ``fake Satake parameters'' which provide evidence to 
the contrary. 

Give $(\bR/\bZ)^d$ the Haar measure normalized to have total mass one. 
Recall that for any $f\in L^1((\bR/\bZ)^d)$, the Fourier coefficients of $f$ 
are, for $\vm\in \bZ^d$: 
\[
	\widehat f(\vm) = \int_{(\bR/\bZ)^d} e^{2\pi i \langle \vm,\vx\rangle} \, \dd \vx ,
\]
where $\langle \vm,\vx\rangle = m_1 x_1 + \cdots + m_d x_d$ is the usual inner 
product. Typically, if $f$ is a function on $(\bR/\bZ)^d$, sums of the form 
$\sum_{n\leqslant N} f(n \vx)$ will be $o(N)$. When $f$ is a character of the 
torus, there is a much stronger bound. 

\begin{theorem}
Fix $\vx\in (\bR/\bZ)^d$ with $\omega_{d-1}(\vx)$ finite. Then 
\[
	\left| \sum_{n\leqslant N} e^{2\pi i \langle \vm, n \vx\rangle}\right| \ll |\vm|_\infty^{\omega_{d-1}(\vx) + \epsilon} 
\]
as $\vm$ ranges over $\bZ^d\smallsetminus 0$. 
\end{theorem}
\begin{proof}
From Lemma \ref{lem:bound-exp-sum} we know that 
$\left| \sum_{n\leqslant N} e^{2\pi i \langle \vm, n \vx\rangle}\right| \ll \dd(\langle \vm, \vx\rangle,\bZ)^{-1}$, 
and from Lemma \ref{lem:bound-distance}, we know that 
$\dd(\langle \vm,\vx\rangle, \bZ)^{-1} \ll |\vm|_\infty^{\omega_{d-1}(x)+\epsilon}$. 
The result follows. 
\end{proof}

\begin{theorem}\label{thm:translates-bound-sum}
Let $\vx\in \bR^d$ with $\omega_{d-1}(\vx)$ finite. Fix 
$f\in C^\infty((\bR/\bZ)^d)$ with $\widehat f(\vzero)=0$. Then 
$\left| \sum_{n\leqslant N} f(n \vx)\right| \ll 1$. 
\end{theorem}
\begin{proof}
Write $f$ as a Fourier series:
$f(\vx) = \sum_{\vm\in \bZ^d} \widehat f(\vm) e^{2\pi i \langle \vm,\vx\rangle}$. 
Since $\widehat f(0)=0$, we can compute:
\begin{align*}
	\left| \sum_{n\leqslant N} f(n \vx)\right| 
		&= \left| \sum_{n\leqslant N} \sum_{\vm\in \bZ^d\smallsetminus 0} \widehat f(\vm) e^{2\pi i n \langle \vm,\vx\rangle}\right| \\
		&\leqslant \sum_{\vm\in \bZ^d\smallsetminus 0} \left|\widehat f(\vm)\right|\cdot \left| \sum_{n\leqslant N} e^{2\pi i n \langle \vm,\vx\rangle}\right| \\
		&\ll \sum_{\vm\in \bZ^d\smallsetminus 0} \left|\widehat f(\vm)\right|\cdot |\vm|_\infty^{\omega_{d-1}(\vx) + \epsilon} .
\end{align*}
The sum converges since the Fourier coefficients $\widehat f(\vm)$ converge to 
zero faster than the reciprocal of any polynomial. 
\end{proof}

The function $f$ does not need to be smooth---so long as its Fourier 
coefficients decay sufficiently rapidly so as to force 
$\sum |\widehat f(\vm)| \cdot |\vm|_\infty^{\omega_{d-1}(\vx)+\epsilon}$ to 
converge, the proof works. 

Enumerate the primes of $K$ as $\fp_1,\fp_2,\fp_3,\dots$.  
Let $\vy\in \bR^d$ with $y_1,\dots,y_d$ linearly independent over $\bQ$. The 
associated sequence of ``fake Satake parameters'' is 
$\bvx = (\vy, 2\vy, 3 \vy, 4 \vy, \dots)$, 
where we put $\vx_{\fp_n} = n \vy\mod \bZ^d$. By Theorem \ref{thm:jarnik}, we can 
arrange for $\omega_0(\vy) = w$ and $\omega_{d-1}(\vy) = d w + d - 1$. 

\begin{theorem}
The sequence $\bvx$ is equidistributed in $(\bR/\bZ)^d$, with discrepancy 
decaying as $\D_N(\bvx) \ll N^{-\frac{1}{d w+d} + \epsilon}$, and for which 
$\D_N(\bvx) = \Omega\left(N^{-\frac{d}{w} - \epsilon}\right)$. 
However, for any $f\in C^\infty((\bR/\bZ)^d)$ with $\widehat f(0)=0$, the 
Dirichlet series  $L_f(\bx,s)$ satisfies the Riemann Hypothesis. 
\end{theorem}
\begin{proof}
The upper bound on discrepancy is Theorem \ref{thm:disc-upper-bound}, and 
the lower bound is Theorem \ref{thm:disc-lower-bound}. For the functions $f$ in 
question, Theorem \ref{thm:translates-bound-sum} gives an estimate (stronger 
than) $\left| \sum_{\N(\fp)\leqslant N} f(\vx_\fp)\right|\ll N^{\frac 1 2}$, and 
Theorem \ref{thm:AT->RH} tells us this estimate implies the Riemann Hypothesis. 
\end{proof}

This shows that even if each $L(r_\ast \rho_l,s)$ satisfies the Riemann 
Hypothesis, we may not conclude that the Akiyama--Tanigawa 
conjecture holds for $A$. Note also that Theorem \ref{thm:AT->RH} does 
\emph{not} tell us that $L_f(\vx,s)$ has analytic continuation to $\Re > 0$, or 
that there are no zeros in $\Re > 0$. For, the term 
$\sum_\fp \sum_{r\geqslant 2} \frac{f(x_\fp^r}{r p^{r s}}$ will not converge 
past $\Re > \frac 1 2$.

% !TEX root = Daniel-Miller-thesis.tex

\chapter{Pathological Galois representations}\label{ch:construct-Galois}





\section{Notation and supporting results}

In this section we loosely summarize and adapt the results of 
\cite{khare-larsen-ramakrishna-2005,pande-2011}. Throughout, if $F$ is a field 
and $M$ a $G_F$-module, we write $\h^\bullet(F,M)$ in place of 
$\h^\bullet(G_F,M)$. All Galois representations will take values in 
$\GL_2(\bZ/l^n)$ or $\GL_2(\bZ_l)$ for $l$ a (fixed) rational prime, and 
all deformations will have fixed determinant. So we consider the cohomology of 
$\Ad^0\bar\rho$, the induced representation on trace-zero matrices by 
conjugation. 

If $S$ is a set of rational primes, $\bQ_S$ denotes the largest extension of 
$\bQ$ unramified outside $S$. So $\h^i(\bQ_S,-)$ is what is usually written as 
$\h^1(G_{\bQ,S},-)$. If $M$ is a $G_\bQ$-module and $S$ a finite set of primes, 
denote the corresponding Tate--Shafarevich group by 
\[
	\Sha^i_S(M) = \ker\left( \h^i(\bQ_S,M) \to \prod_{p\in S} \h^i(\bQ_p,M)\right) .
\]
If $l$ is a rational prime and $S$ a finite set of primes containing $l$, then 
for any $\bF_l[G_{\bQ_S}]$-module $M$, write $M^\vee=\hom_{\bF_l}(M,\bF_l)$ 
with the obvious $G_{\bQ_S}$-action, and write $M^\ast = M^\vee(1)$ for the 
Cartier dual of $M^\vee$. By \cite[Th.~8.6.7]{neukirch-schmidt-winberg-2008}, 
there is an isomorphism $\Sha^1_S(M^\ast) \simeq \Sha_S^2(M)^\vee$. As a 
result, if $\Sha_S^1(M)$ and $\Sha_S^2(M)$ are trivial, and $S\subset T$, then 
$\Sha_T^1(M)$ and $\Sha_T^2(M)$ are also trivial. 

% The l>=7 comes from \cite{ramakrishna-2002}. 
\begin{definition}
A \emph{good residual representation} is an odd, absolutely irreducible, 
weight-$2$ representation $\bar\rho\colon G_{\bQ_S} \to \GL_2(\bF_l)$, where 
$l\geqslant 7$ is a rational prime. 
\end{definition}

Recall that $\bar\rho$ is weight-$2$ if $\det\bar\rho$ is the mod-$l$ 
cyclotomic character. Similarly, $\rho\colon G_\bQ \to \GL_2(\bZ_l)$ is 
weight-$2$ if $\det\rho$ is the $l$-adic cyclotomic character. 
Roughly, ``good residual representations'' have enough properties that we can 
prove meaningful theorems about their lifts without assuming the modularity 
results of Khare--Wintenberger. 

\begin{theorem}\label{thm:always-can-lift}
Let $\bar\rho\colon G_\bQ \to \GL_2(\bF_l)$ be a good residual 
representation. Then there exists a weight-$2$ lift of $\bar\rho$ to $\bZ_l$, 
ramified at the same set of primes as $\bar\rho$. 
\end{theorem}
\begin{proof}
This is \cite[Th.~1]{ramakrishna-2002}, taking into account that the paper in 
question allows for arbitrary fixed determinants. 
\end{proof}

\begin{definition}
Let $\bar\rho\colon G_{\bQ_S} \to \GL_2(\bF_l)$ be a good residual 
representation. A prime $p\not\equiv \pm 1\pmod l$ is \emph{nice} if 
$\Ad^0\bar\rho\simeq \bF_l \oplus \bF_l(1)\oplus \bF_l(-1)$, i.e.~if the 
eigenvalues of $\bar\rho(\frob_p)$ have ratio $p$. 
\end{definition}

Taylor allows $p\equiv -1\pmod l$, but the results of \cite{pande-2011} 
require $p\not\equiv -1\pmod l$. The following theorem gives a complete 
description of the versal deformation ring for 
$\left.\bar\rho\right|_{G_{\bQ_p}}$ when $p$ is nice.

\begin{theorem}[\cite{ramakrishna-1999}]
Let $\bar\rho$ be a good residual representation and $p$ a nice prime. Then 
any deformation of $\left.\bar\rho\right|_{G_{\bQ_p}}$ is induced by 
$G_{\bQ_p} \to \GL_2(\bZ_l\pow{a,b} / \langle a b\rangle)$, sending 
\[
	\frob_p \mapsto \smat{p(1+a)}{}{}{(1+a)^{-1}} \qquad \tau_p \mapsto \smat{1}{b}{}{1} ,
\]
where $\tau_p\in G_{\bQ_p}$ is a generator for tame inertia. 
\end{theorem}

We close this section by introducing some new terminology and notation to 
condense the lifting process used in \cite{khare-larsen-ramakrishna-2005}. 

Fix a good residual representation $\bar\rho$. We will consider weight-$2$ 
deformations of $\bar\rho$ to $\bZ/l^n$ and $\bZ_l$. Call such a deformation a 
``lift of $\bar\rho$ to $\bZ/l^n$ (resp.~$\bZ_l$).'' We will often restrict the 
local behavior of such lifts, i.e.~the restrictions of a lift to $G_{\bQ_p}$ 
for $p$ in some set of primes. The necessary constraints are captured in the 
following definition. 

\begin{definition}
Let $\bar\rho$ be a good residual representation, 
$h\colon \bR^+ \to \bR_{\geqslant 1}$ an 
increasing function. An \emph{$h$-bounded lifting datum} is a tuple 
$(\rho_n,R,U,\{\rho_p\}_{p\in R\cup U})$, where 
\begin{enumerate}
\item
$\rho_n\colon G_{\bQ_R} \to \GL_2(\bZ/l^n)$ is a lift of $\bar\rho$.

\item
$R$ and $U$ are finite sets of primes, $R$ containing $l$ and all primes at 
which $\rho_n$ ramifies. 

\item
$\pi_R(x)\leqslant h(x)$ for all $x$. 

\item
$\Sha_R^1(\Ad^0\bar\rho)$ and $\Sha_R^2(\Ad^0\bar\rho)$ are trivial. 

\item
For all $p\in R\cup U$, 
$\rho_p\equiv \left. \rho_n\right|_{G_{\bQ_p}}\pmod{l^n}$. 

\item
For all $p\in R$, $\rho_p$ is ramified. 

\item
$\rho_n$ admits a lift to $\bZ/l^{n+1}$. 
\end{enumerate}
\end{definition}

If $(\rho_n,R,U,\{\rho_p\})$ is an $h$-bounded lifting datum, we call 
another $h$-bounded lifting datum $(\rho_{n+1},R',U',\{\rho_p\})$ a \emph{lift 
of $(\rho_n,R,U,\{\rho_p\})$} if $U\subset U'$, $R\subset R'$, and for all 
$p\in R\cup U$, the two possible $\rho_p$ agree. 

\begin{theorem}\label{thm:lifting-datum}
Let $\bar\rho$ be a good residual representation, 
$h\colon \bR^+ \to \bR_{\geqslant 1}$ 
increasing to infinity. If $(\rho_n,R,U,\{\rho_p\})$ is an $h$-bounded lifting 
datum, $U'\supset U$ is a finite set of primes disjoint from $R$, and 
$\{\rho_p\}_{p\in U'}$ extends $\{\rho_p\}_{p\in U}$, then there exists an 
$h$-bounded lift $(\rho_{n+1},R',U',\{\rho_p\})$ of 
$(\rho_n,R,U,\{\rho_p\})$. 
\end{theorem}
\begin{proof}
By \cite[Lem.~8]{khare-larsen-ramakrishna-2005}, there exists a finite set 
$N$ of nice primes such that the map 
\begin{equation}\label{eq:h1-isom}
	\h^1(\bQ_{R\cup N},\Ad^0\bar\rho) \to \prod_{p\in R} \h^1(\bQ_p,\Ad^0\bar\rho) \times \prod_{p\in U'} \h_\nr^1(\bQ_p,\Ad^0\bar\rho) 
\end{equation}
is an isomorphism. In fact, 
$\# N = \dim\h^1(\bQ_{R\cup N},\Ad^0\bar\rho^\ast)$, and the primes in $N$ are 
chosen, one at a time, from Chebotarev sets. Since $\pi_R(x)$ is eventually 
constant and $h(x)$ increases to infinity, $h(x) \geqslant \pi_R(x)+1$ for all 
$x\geqslant C_1$ for some $C_1$. Choose the first prime $p$ in $N$ to be 
$\geqslant C_1$; then $\pi_{R\cup\{p\}}(x) \leqslant h(x)$ for all $x$. Repeat 
this process for allor all the other primes in $N$. We can ensure 
that the bound $\pi_{R\cup N}(x) \leqslant h(x)$ continues to hold. We also 
choose the primes in $N$ to be larger than any prime in $U'$. 

By our hypothesis, $\rho_n$ admits a lift to $\bZ/l^{n+1}$; call one such lift 
$\rho^\ast$. For each $p\in R\cup U'$, $\h^1(\bQ_p,\Ad^0\bar\rho)$ acts 
transitively on lifts of $\left.\rho_n\right|_{G_{\bQ_p}}$ to $\bZ/l^{n+1}$. In 
particular, there are cohomology classes $f_p\in \h^1(\bQ_p,\Ad^0\bar\rho)$ 
such that $f_p\cdot \rho^\ast \equiv \rho_p\pmod{l^{n+1}}$ for all 
$p\in R\cup U'$. Moreover, for all $p\in U'$, the class $f_p$ is unramified. 
Since the map \eqref{eq:h1-isom} is an isomorphism, there exists 
$f\in \h^1(\bQ_{R\cup N},\Ad^0\bar\rho)$ such that 
$\left.f\cdot \rho^\ast\right|_{G_{\bQ_p}}\equiv \rho_p\pmod{l^{n+1}}$ for all 
$p\in R\cup U'$. 

Clearly $\left. f\cdot \rho^\ast\right|_{G_{\bQ_p}}$ admits a lift to $\bZ_l$ 
for all $p\in R\cup U'$, but it does not necessarily admit such a lift for 
$p\in N$. By repeated applications of \cite[Prop.~3.10]{pande-2011}, there 
exists a set $N'\supset N$, with $\# N'\leqslant 2\# N$, of nice primes and 
$g\in \h^1(\bQ_{R\cup N'},\Ad^0\bar\rho)$ such that 
$(g+f)\cdot \rho^\ast$ still agrees with $\rho_p$ for $p\in R\cup U'$, and 
$(g+f)\cdot \rho^\ast$ is nice for all $p\in N'$. As above, the primes in $N'$ 
are chosen one at a time from Chebotarev sets, so we can continue to ensure the 
bound $\pi_{R\cup N'}(x)\leqslant h(x)$ and also that all primes in $N'$ 
are larger than those in $U'$. Let $\rho_{n+1} = (g+f) \cdot \rho^\ast$. Let 
$R' = R\cup \{p\in N' : \rho_{n+1}\text{ is ramified at }p\}$. For each 
$p\in R'\smallsetminus R$, choose a lift $\rho_p$ of 
$\left. \rho_{n+1}\right|_{G_{\bQ_p}}$ to $\bZ_l$. 

Since $\left.\rho_{n+1}\right|_{G_{\bQ_p}}$ admits a lift to $\bZ/l^{n+2}$ (in 
fact, it admits a lift to $\bZ_l$) for each $p$, and 
$\Sha_{R'}^1(\Ad^0\bar\rho)$, $\Sha_{R'}^2(\Ad^0\bar\rho)$ are trivial, the 
deformation $\rho_{n+1}$ admits a lift to $\bZ/l^{n+2}$. Thus 
$(\rho_{n+1},R',U',\{\rho_p\})$ is the desired lift of 
$(\rho_n,R,U,\{\rho_p\})$. 
\end{proof}





\section{Galois representations with specified Satake parameters}

Fix a good residual representation $\bar\rho$, and consider weight-$2$ 
deformations of $\bar\rho$. The final deformation, 
$\rho\colon G_\bQ \to \GL_2(\bZ_l)$, will be constructed as the inverse limit 
of a compatible collection of lifts $\rho_n\colon G_\bQ \to \GL_2(\bZ/l^n)$. At 
any given stage, we will be concerned with making sure that there exists a 
lift to the next stage, and that there is a lift with the necessary properties. 
Fix a sequence $\bx=(x_1,x_2,\dots)$ in $[-1,1]$. The set of unramified primes 
of $\rho$ is not determined at the beginning, but at each stage there will be 
a large finite set $U$ of primes which we know will remain unramified. 
Reindexing $\bx$ by these unramified primes, we will construct $\rho$ so that 
for all unramified primes $p$, $\tr\rho(\frob_p)\in \bZ$, satisfies the Hasse 
bound, and has $\tr\rho(\frob_p) \approx x_p$. Moreover, we can ensure that the 
set of ramified primes has density zero in a very strong sense (controlled by a 
parameter function $h$) and that our trace of Frobenii are very close to 
specified values. 

Given any deformation $\rho$, write $\pi_{\ram(\rho)}(x)$ for the function 
which counts $\rho_n$-ramified primes $\leqslant x$. Since we will have 
$\pi_{\ram(\rho)}(x)\ll h(x)$ and bounds of this form are only helpful 
if $h(x) = o(\pi(x))$, we will usually assume $h(x) \ll x^\epsilon$, 
e.g.~$h(x) = \log x$ or something which grows even slower (for example, the 
inverse of the Ackerman function). 

\begin{theorem}\label{thm:master-Galois}
Let $l$, $\bar\rho$, $\bx$ be as above. Fix a function 
$h\colon \bR^+ \to \bR_{\geqslant 1}$ which increases to infinity. Then there 
exists a weight-$2$ deformation $\rho$ of $\bar\rho$, such that: 
\begin{enumerate}
\item
$\pi_{\ram(\rho)}(x) \ll h(x)$. 

\item
For each unramified prime $p$, $a_p=\tr\rho(\frob_p)\in \bZ$ and satisfies the 
Hasse bound. 

\item
For each unramified prime $p$, 
$\left| \frac{a_p}{2\sqrt p} - x_p\right| \leqslant \frac{l h(p)}{2\sqrt p}$. 
\end{enumerate}
\end{theorem}
\begin{proof}
Begin with $\rho_1= \bar\rho$. By \cite[Lem.~6]{khare-larsen-ramakrishna-2005}, 
there exists a finite set $R$, containing the set of primes at which $\bar\rho$ 
ramifies, such that $\Sha_R^1(\Ad^0\bar\rho)$ and $\Sha_R^2(\Ad^0\bar\rho)$ are 
trivial. 
Let $R_1$ be the union of $R$ and all primes $p$ with 
$\frac{l}{2\sqrt p} > 2$. Since $\frac{l}{2\sqrt p} \to 0$ as $p\to \infty$, 
the set $R_1$ is finite. For all $p\notin R_1$ and any $a\in \bF_l$, there 
exists $a_p\in \bZ$ satisfying the Hasse bound with $a_p\equiv a\pmod l$. In 
fact, given any $x_p\in [-1,1]$, there exists $a_p\in \bZ$ satisfying the Hasse 
bound such that 
$\left| \frac{a_p}{2\sqrt p} - x_p\right| \leqslant \frac{l}{2\sqrt p}$.
Choose, for all primes $p\in R_1$, a ramified 
lift $\rho_p$ of $\left. \rho_1\right|_{G_{\bQ_p}}$. Let $U_1$ be the set of 
primes $p$ not in $R_1$ such that 
$\frac{l^2}{2\sqrt p} > \min\left(2, \frac{l h(p)}{2\sqrt p}\right)$; this is 
finite because $\frac{l^2}{2\sqrt p} \to 0$ and also eventually 
$h(p) \geqslant l$. If $U_1$ is empty, then the next few sentences of the 
proof are superfluous, but the theorem still holds. 
For each $p\in U_1$, there exists $a_p\in \bZ$, satisfying the 
Hasse bound, such that 
\[
	\left| \frac{a_p}{2\sqrt p} - x_p\right| \leqslant \frac{l}{2\sqrt p} \leqslant \frac{l h(p)}{2\sqrt p} ,
\]
and moreover $a_p\equiv \tr\bar\rho(\frob_p)\pmod l$. For each $p\in U_1$, let 
$\rho_p$ be an unramified lift of $\left.\bar\rho\right|_{G_{\bQ_p}}$ with 
$\tr\rho_p$ being the desired $a_p$. It may not be that 
$\pi_{R_1}(x) \leqslant h(x)$ for all $x$. Let 
$C = \max\left\{\pi_{R_1}(x)\right\}$; this is finite because 
$R_1$ is and $\pi_{R_1}(x)$ is constant past the largest prime in $R_1$. Then 
for $h^\ast = C h$, we have $\pi_{R_1}(x) \leqslant h^\ast(x)$ for all $x$. 

We have constructed our first $h^\ast$-bounded lifting datum 
$(\rho_1,R_1,U_1,\{\rho_p\})$. We proceed to construct 
$\rho = \varprojlim \rho_n$ inductively, by constructing a new $h^\ast$-bounded 
lifting datum for each $n$. We ensure that $U_n$ contains all primes for which 
$\frac{l^{n+1}}{2\sqrt p} > \min\left(2, \frac{l h(p)}{2\sqrt p}\right)$, so 
there are always integral $a_p$ satisfying the Hasse bound which satisfy any 
mod-$l^{n+1}$ constraint, and that can always choose these $a_p$ so as to 
preserve statement 2 in the theorem. 

The base case is already complete, so suppose we are given 
$(\rho_{n-1},R_{n-1},U_{n-1},\{\rho_p\})$. We may assume that $U_{n-1}$ 
contains all primes for 
which $\frac{l^n}{2\sqrt p} > \min\left(2, \frac{l h(p)}{2\sqrt p}\right)$. Let 
$U_n$ be the set of all primes not in $R_{n-1}$ such that 
$\frac{l^{n+1}}{2\sqrt p} > \min\left(2, \frac{l h(p)}{2\sqrt p}\right)$. For 
each $p\in U_n\smallsetminus U_{n-1}$, there is an integer $a_p$, satisfying 
the Hasse bound, such that $a_p\equiv \rho_n(\frob_p)\pmod{l^n}$, and moreover 
$\left|\frac{a_p}{2\sqrt p} - x_p\right| \leqslant \frac{l h(p)}{2\sqrt p}$. 
For such $p$, let $\rho_p$ be an unramified lift of 
$\left. \rho_n\right|_{G_{\bQ_p}}$ such that $\tr\rho_n(\frob_p)$ is the 
desired $a_p$. By Theorem \ref{thm:lifting-datum}, there exists an 
$h^\ast$-bounded lifting datum $(\rho_n,R_n,U_n,\{\rho_p\})$ extending and 
lifting $(\rho_{n-1},R_{n-1},U_{n-1},\{\rho_p\})$. This completes the inductive 
step.  
\end{proof}

The implied constant in the bound $\pi_{\ram(\rho)}(x)\ll h(x)$ depends on 
$\bar\rho$ (and hence $l$) but not on $h$. 
We will apply this theorem to construct Galois representations with specified 
Sato--Tate distributions in the next section, but for now here is a small 
consequence, which addresses the results in \cite{sarnak-2007}. Sarnak remarks 
that for $E_{/\bQ}$ a non-CM elliptic curve with rank $r$, the partial sums 
$\frac{\log x}{\sqrt x} \sum_{p\leqslant x} \frac{a_p}{\sqrt p}$ approach a 
limiting distribution with mean $1 - 2 r$. 

\begin{corollary}
Let $L \in [-\infty,\infty]$ and $\epsilon>0$ be given. Then there exists a 
weight 2 Galois representation $\rho\colon G\to \GL_2(\bZ_l)$, such that 
each $a_p = \tr\rho(\frob_p)\in \bZ$ satisfies the Hasse bound, 
\[
	L = \lim_{N\to \infty} \frac{\log N}{\sqrt N}\sum_p \frac{a_p}{\sqrt p} 
\]
and $\pi_{\ram(\rho)}(x) \ll \log(x)$. 
\end{corollary}
\begin{proof}
Begin with a sequence $(x_p)$ in $[-1,1]$ such that 
$\lim_{N\to \infty} \frac{\log N}{\sqrt N}\sum_{p\leqslant N} x_p = L$. If 
$L=\pm \infty$, we can choose $x_p = \pm 1$. By Theorem 
\ref{thm:master-Galois}, there exists $\rho\colon G_\bQ \to \GL_2(\bZ_l)$ with 
$\pi_{\ram(\rho)}(x) \ll \log(x)$, and such that for each unramified $p$, 
$a_p = \tr\rho(\frob_p)\in \bZ$, satisfies the Hasse bound, has 
$\left| \frac{a_p}{2\sqrt p} - x_p\right| < \frac{l \log p}{\sqrt p}$, and by 
inspecting the proof, we can even ensure that 
$\sum \left(\frac{a_p}{2\sqrt p} - x_p\right)$ converges conditionally (make 
sure the sign of $\frac{a_p}{2\sqrt p} - b_p$ alternates). Note that
\begin{align*}
	\left|\frac{\log N}{\sqrt N}\sum_{p\leqslant N} \frac{a_p}{2\sqrt p} - \frac{\log N}{\sqrt N} \sum_{p\leqslant N} x_p\right| 
		&\leqslant \frac{\log N}{\sqrt N}\left|\sum_{\substack{p\leqslant N \\ p\text{ ramified}}} \left(\frac{a_p}{2 \sqrt p} - x_p\right) + \sum_{\substack{p\leqslant N \\ p\text{ unramified}}} \left(\frac{a_p}{2 \sqrt p} - x_p\right) \right|  \\
		&\ll \frac{\log N}{\sqrt N} \left(\pi_{\ram(\rho)}(N) + \left|\sum_{p\leqslant N} \left(\frac{a_p}{2 \sqrt p} - x_p\right)\right|\right) ,
\end{align*}
which converges to zero. 
When $L\ne \pm\infty$, this shows that the limit in question exists and is $L$. 
When $L=\pm \infty$, this shows that the the sums in question diverge to $L$. 
\end{proof}





\section{Galois representations with specified Sato--Tate distributions}

For $k\geqslant 1$, let 
\[
	U_k(\theta) = \tr\sym^k\smat{e^{i\theta}}{}{}{e^{- i \theta}} = \frac{\sin((k+1)\theta)}{\sin\theta} .
\]
Then $U_k(\cos^{-1} t)$ is the $k$-th Chebyshev polynomial of the 2nd kind. 
Moreover, $\{1\}\cup\{U_k\}$ forms an orthonormal basis for 
$L^2([0,\pi],\ST) = L^2(\SU(2)^\natural)$. 

This section has two parts. First, for any reasonable measure $\mu$ on 
$[0,\pi]$ invariant under the same ``flip'' automorphism as the Sato--Tate 
measure, there is a sequence $(a_p)$ of integers satisfying the Hasse 
bound $|a_p|\leqslant 2\sqrt p$, such that for 
$\theta_p = \cos^{-1}\left(\frac{a_p}{2\sqrt p}\right)$, the discrepancy 
$\D_N(\btheta,\mu)$ behaves like $\pi(N)^{-\alpha}$ for predetermined 
$\alpha\in \left(0,\frac 1 2\right)$, while for any odd $k$, the strange 
Dirichlet series $L_{U_k}(\btheta,s)$, which we will write as 
$L(\sym^k \btheta,s)$, satisfies the Riemann hypothesis. In the second part of 
this section, we associate Galois representations to these fake Satake 
parameters. 

\begin{definition}
Let $\mu = f(t)\, \dd t$ be an absolutely continuous measure with continuous 
cdf on $[0,\pi]$. If $f(t) \ll \sin(t)$, then $\mu$ is a \emph{Sato--Tate compatible measure}. 
\end{definition}

The key facts about Sato--Tate compatible measures are that $\cos_\ast\mu$ 
satisfies the hypotheses of Theorem \ref{thm:discrepancy-arbitrary}, so 
there are ``$N^{-\alpha}$-decaying van der Corput sequences'' for 
$\cos_\ast\mu$, and also that since $\cos\colon [0,\pi] \to [-1,1]$ is an 
strictly decreasing, we know that for any sequence $\bx$ on $[-1,1]$, 
$\D_N(\bx,\cos_\ast\mu) \approx \D_N(\cos^{-1}\bx,\mu)$. Finally, the 
Radon--Nikodym derivative $\mu$ (and also $\cos_\ast\mu$) is bounded , so Lemma 
\ref{lem:disc-of-two-seq} applies. Recall that for deceasing functions 
$\varphi_1,\varphi_2$, we write $\varphi_1(N) = \Theta(\varphi_2(N))$ if 
there exists constants $0 < C_1 < C_2$ such that 
$C_1 \varphi_2(N) \leqslant \varphi_1(N) \leqslant C_2 \varphi_2(N)$. 


\begin{theorem}\label{thm:integral-a_p-alpha}
Let $\mu$ be a Sato--Tate compatible measure, and fix 
$\alpha\in \left(0,\frac 1 2\right)$. 
Then there exists a sequence of integers $a_p$ satisfying the Hasse bound, 
such that if we set $\theta_p = \cos^{-1}\left(\frac{a_p}{2\sqrt p}\right)$, 
then $\D_N(\btheta,\mu) = \Theta(\pi(N)^{-\alpha})$. 
\end{theorem}
\begin{proof}
Apply Theorem \ref{thm:discrepancy-arbitrary} to find a sequence $\bx$ such 
that $\D_N(\bx,\cos_\ast \mu) = \Theta(\pi(N)^{-\alpha})$. For each prime 
$p$, there exists an integer $a_p$ such that $|a_p|\leqslant 2\sqrt p$ and 
$\left| \frac{a_p}{2\sqrt p} - x_p\right| \leqslant p^{-1/2}$. Let 
$y_p = \frac{a_p}{2\sqrt p}$, and apply Lemma \ref{lem:disc-of-two-seq} with 
$\epsilon = N^{-1/2}$. We obtain 
\[
	\left| \D_N(\bx,\cos_\ast \mu) - \D_N(\by, \cos_\ast \mu)\right| \ll  N^{-1/2} + \frac{\pi(N^{1/2})}{\pi(N)} ,
\]
which tells us that $\D_N(\by,\cos_\ast\mu) = \Theta(\pi(N)^{-\alpha})$. 
Now let $\btheta = \cos^{-1}(\by)$. Apply Lemma \ref{lem:push-discrepancy} to 
$\btheta = \cos^{-1}(\by)$, and we see that 
$\D_N(\btheta,\mu) = \Theta(\pi(N)^{-\alpha})$. 
\end{proof}

We can improve this example by controlling the behavior of the sums 
$\sum_{p\leqslant N} U_k(\theta_p)$ for odd $k$. Let $\sigma$ be 
the involution of $[0,\pi]$ given by $\sigma(\theta) = \pi-\theta$. Note that 
$\sigma_\ast \ST = \ST$. Moreover, note that for any odd $k$, 
$U_k\circ\sigma = - U_k$, so $\int U_k\, \dd\ST = 0$. Of course, 
$\int U_k \,\dd\ST = 0$ for the reason that $U_k$ is the trace of a 
non-trivial unitary representation, but we will directly use the ``oddness'' 
of $U_k$ in what follows.

\begin{theorem}\label{thm:int-flip-seq}
Let $\mu$ be a $\sigma$-invariant Sato--Tate compatible measure. Fix 
$\alpha\in \left(0,\frac 1 2\right)$. Then there is a sequence of integers 
$a_p$, satisfying the Hasse bound, such that for 
$\theta_p =\cos^{-1}\left( \frac{a_p}{2\sqrt p}\right)$, we have
\begin{enumerate}
\item
$\D_N(\btheta,\mu) = \Theta(\pi(N)^{-\alpha})$. 

\item
For all odd $k$, 
$\left| \sum_{k\leqslant N} U_k(\theta_p)\right| \ll \pi(N)^{1/2}$. 
\end{enumerate}
\end{theorem}
\begin{proof}
The basic ideas is as follows. Enumerate the primes 
\[
	p_1 = 2, q_1 = 3, p_2 = 5, q_2 = 7, p_3 = 11, q_3 = 13, \dots .
\]
Consider the measure $\left.\mu\right|_{[0,\pi/2)}$. An argument 
nearly identical to the proof of Theorem \ref{thm:integral-a_p-alpha} shows 
that we can choose $a_{p_i}$ satisfying the Hasse bound so that 
\[
	\D_N\left( \left\{\theta_{p_i}\right\},\left.\mu\right|_{[0,\pi/2)}\right) = \Theta(N^{-\alpha}) .
\]
We can also choose the $a_{q_i}$ such that 
$\frac{a_{q_i}}{2\sqrt{q_i}}\in [\pi/2,\pi]$ and 
$\left| \frac{a_{p_i}}{2\sqrt{p_i}} + \frac{a_{q_i}}{2\sqrt{q_i}}\right| \ll \frac{1}{\sqrt{p_i}}$. 
Let $\bx$ be the sequence of the $\frac{a_{p_i}}{2\sqrt{p_i}}$ and $\by$  
the corresponding sequence with the $q_i$-s. Then 
Lemma \ref{lem:flip-discrepancy} tells us that the discrepancy of 
$\by$ decays at the same rate (within $O(N^{-1})$) as $-\by$, and then Lemma  \ref{lem:disc-of-two-seq} with $\epsilon\approx N^{-1/2}$ tells us that 
the discrepancy of $-\by$ decays at the same rate (within $O(N^{-1/2})$) as 
the discrepancy of $\bx$. Thus the discrepancies of both $\bx$ and $\by$ decay 
as $\Theta(N^{-\alpha})$. Finally, Theorem \ref{thm:wreath-seq} tell us that 
$\D_N(\bx\wr\by,\mu) = \Theta(N^{-\alpha})$. 

The function $U_k(\cos^{-1} t)$ is an odd polynomial in $t$, so for 
$t_1,t_2\in [-1,1]$, 
\[
	|U_k(\cos^{-1} t_1) + U_k(\cos^{-1} t_2)| = |U_k(\cos^{-1} t_1) - U_k(\cos^{-1}(-t_1))| \ll |t_1 - (-t_2)|.
\]
It follows that since 
$\left|\frac{a_{p_i}}{2\sqrt{p_i}} - \left(- \frac{a_{q_i}}{2\sqrt{q_i}}\right)\right| \ll p_i^{-1/2}$, 
then $|U_k(\theta_{p_i}) + U_k(\theta_{q_i})|\ll p_i^{-1/2}$. 
We can then bound 
\[
	\left| \sum_{i\leqslant N} \left(U_k(\theta_{p_i}) + U_k(\theta_{q_i})\right)\right| \ll \sum_{p\leqslant N} p^{-1/2} \ll \pi(N)^{1/2} .
\]
\end{proof}

Note that this proof actually shows that for any $f\in C([0,\pi])$ such 
that $f\circ \cos^{-1}$ is Lipschitz, and $f(\pi-\theta) = -f(\theta)$, the 
estimate $\left| \sum_{p\leqslant N} f(\theta_p)\right| \ll \pi(N)^{1/2}$ 
holds. 

\begin{theorem}\label{thm:bad-Galois}
Let $\mu$ be a Sato--Tate compatible $\sigma$-invariant measure on $[0,\pi]$. 
Fix $\alpha\in \left(0,\frac 1 2\right)$ and a good residual representation 
$\bar\rho\colon G_\bQ \to \GL_2(\bF_l)$. Then there exists a weight-$2$ lift 
$\rho\colon G_\bQ \to \GL_2(\bZ_l)$ of $\bar\rho$ such that 
\begin{enumerate}
\item
$\pi_{\ram(\rho)}(x) \ll \log(x)$. 

\item
For each unramified prime $p$, $a_p = \tr\rho(\frob_p)\in \bZ$ and satisfies 
the Hasse bound. 

\item
If, for unramified $p$ we set 
$\theta_p = \cos^{-1}\left(\frac{a_p}{2\sqrt p}\right)$, then 
$\D_N(\btheta,\mu) = \Theta(\pi(N)^{-\alpha})$. 

\item
For each odd $k$, the function $L(\sym^k \rho,s)$ satisfies the Riemann 
hypothesis. 
\end{enumerate}
\end{theorem}
\begin{proof}
Let $\bx$ be an $N^{-\alpha}$-decay van der Corput sequence for 
$\cos_\ast \left.\mu\right|_{[0,\pi/2)}$, so that $\bx$ is contained in 
$(0,1]$. Let $\by = -\bx$ (contained in $[-1,0)$), and put 
$\bz = \bx\wr\by$, reindexed by the prime numbers. We have
$\D_N(\bz,\cos_\ast\mu) = \Theta(\pi(N)^{-\alpha})$ just as in the proof 
of Theorem \ref{thm:int-flip-seq}. Set $h(x) = \log(x)$. By Theorem 
\ref{thm:master-Galois}, there is a $\rho\colon G_\bQ \to \GL_2(\bZ_l)$ lifting 
$\bar\rho$ such that $\pi_{\ram(\rho)}(x) \ll \log x$, the $\tr \rho(\frob_p)$ 
are integral, satisfy the Hasse bound, and 
$\left| \frac{a_p}{2\sqrt p} - z_p\right| \leqslant \frac{l \log p}{2\sqrt p}$. 
This implies, just as in the proof of Theorem \ref{thm:master-Galois}, that 
the discrepancy of the sequence $\left\{\frac{a_p}{2\sqrt p}\right\}$ decays 
as $\Theta(\pi(N)^{-\alpha})$ and by Lemma \ref{lem:push-discrepancy}, the 
discrepancies of $\left\{\frac{a_p}{2\sqrt p}\right\}$ and $\{\theta_p\}$ 
decay asymptotically at the same rate. 

We've proved statements 1--3 in the theorem, so all that remains is to prove 
the Riemann hypothesis for odd symmetric powers. The proof of Theorem 
\ref{thm:int-flip-seq} gives us an estimate 
$\left| \sum_{p\leqslant N} U_k(\theta_p)\right| \ll N^{\frac 1 2+\epsilon}$, 
and this combined with Theorem \ref{thm:AT->RH:gp} yields the result. 
\end{proof}

This entire discussion also works with absolutely continuous measure $\mu$, 
supported on a proper subinterval of $[0,\pi]$, so long as $\cdf_\mu$ is 
strictly increasing on that interval. Let $I$ be an arbitrarily small 
subinterval of $[0,\pi]$ (for example, 
$I = \left[\frac \pi 2 - \epsilon,\frac \pi 2 + \epsilon\right]$), let $B_I(t)$ 
be a bump function for $I$, normalized to have total mass one. Then Theorem 
\ref{thm:bad-Galois} gives Galois representations with empirical Sato--Tate 
distribution converging at any specified rate to $\mu_I = B_I(t)\, \dd t$. This 
is a strictly stronger result than \cite[Th.~5.2]{pande-2011}. Moreover, the 
proof of Theorem \ref{thm:int-flip-seq} shows that in fact for any 
$f\in C([0,\pi])$ with $f\circ \cos^{-1}\in C^1([-1,1])$ and 
$f(\pi-\theta) = -f(\theta)$, the Dirichlet series 
$L_f(\rho,s) = \prod \left( 1 - f(\theta_p) p^{-s}\right)^{-1}$ satisfies the 
Riemann hypothesis. 

% !TEX root = Daniel-Miller-thesis.tex

\chapter{Concluding remarks and future directions}





\section{Fake modular forms}

The Galois representations of Theorem \ref{thm:bad-Galois} have ``fake modular 
forms'' associated to them. Namely, there is a representation of $\GL_2(\bA)$ 
with the specified Satake parameters at each prime (for now, set $\theta_p = 0$ 
at ramified primes). It is natural to ask if these ``fake modular forms'' have 
any interesting properties. For example, we know that all their odd symmetric 
powers satisfy the Riemann hypothesis. The author is unaware of any further 
results (say about analytic continuation or functional equation) concerning 
these fake modular forms. 





\section{Dense free subgroups of compact semisimple groups}

Let $G$ be a compact semisimple Lie group, for example $\SU(2)$. By 
\cite{breuillard-gelander-2003}, $G$ contains a dense free subgroup 
$\Gamma = \langle \gamma_1,\gamma_2\rangle$. We will now follow the argument of 
\cite{arnold-krylov-1963} to hint at how $\Gamma$ may yield equidistributed 
sequences with ``bad'' discrepancy and small character sums. 

Given an integer $N$, let $B_N$ be the ``closed ball of size $N$'' in $\Gamma$, 
that is the set of products $\gamma_{\sigma(1)} \dots \gamma_{\sigma(n)}$, 
where $n\leqslant N$ and $\sigma\colon \{1,\dots,n\} \to \{1,2\}$ is a 
function. We will write $\sigma\colon [n] \to [2]$ in this case. Given an 
irreducible unitary representation $\rho\in \widehat G$, we wish to control 
the behavior of $\sum_{\gamma\in B_N} \tr\rho(\gamma)$, ideally to show an 
estimate of the form 
\[
	\left| \sum_{\gamma\in B_N} \tr\rho(\gamma)\right| \ll \left(\# B_N\right)^{\frac 1 2 + \epsilon} .
\]
In fact, $\# B_N = \sum_{n=0}^N 2^n = 2^{N+1} - 1$. We can encode these sums 
in terms of convolutions of a measure as follows. Let $\mu$ be the measure 
$\delta_{\gamma_1^{-1}} + \delta_{\gamma_2^{-1}}$ on $G$. 
If $\rho$ is any unitary representation (not necessarily irreducible or even 
finite-dimensional) then $\mu$ acts on $\rho$ via 
$\rho(\mu) \int \rho\, \dd\mu$. So, if $\rho = L^2(G)$ via the left regular 
representation, then $(\mu\cdot f)(x) = f(\gamma_1 x) + f(\gamma_2 x)$, while 
if $\rho\in \widehat G$ and $v\in \rho$, then 
$\mu\cdot v = \rho(\gamma_1) v + \rho(\gamma_2) v$. Note that 
\[
	\mu^{\ast n} = \sum_{\sigma\colon [n] \to [2]} \delta_{\gamma_{\sigma(1)} \dots \gamma_{\sigma(n)}} .
\]
This tells us that 
$\sum_{\gamma\in B_N} f(\gamma) = \sum_{n\leqslant N} \mu^{\ast n}(f)$. So 
we really only need to study how $\mu$ and its powers act on the functions 
$\tr\rho$, $\rho\in \widehat G$. 

First note that $\tr\rho$ generates a subrepresentation of $L^2(G)$ which is 
isomorphic to $\rho$. On that representation, we claim that $\mu$ is 
invertible, hence 
$\sum_{n=0}^N \mu^{\ast n} = (\mu^{\ast(N+1)} - 1)(\mu - 1)^{-1}$.It 
follows that 
$\| \sum_{n=0}^N \mu^{\ast n}\| \leqslant \frac{\|\mu\|^{N+1}}{\|\mu - 1\|}$, 

Note that $\|\mu\|^{N+1} \leqslant 2^{(N+1)\alpha}$ if and only if 
$\|\mu\| \leqslant 2^\alpha$. In other words, to get the Riemann hypothesis for 
$L$-functions coming from $\Gamma$, we need $\|\mu\| \leqslant \sqrt 2$. 
If $v\in \rho$ has norm $1$, then 
\begin{align*}
	\|\rho(\mu) v\|^2
		&= \langle \rho(\gamma_1^{-1}) v + \rho(\gamma_2^{-1}) v, \rho(\gamma_1^{-1}) v + \rho(\gamma_2^{-1}) v\rangle \\
%		&= 2 \|v\|^2 + 2 \Re\langle \rho(\gamma_1^{-1}) v,\rho(\gamma_2^{-1}) v\rangle \\
		&= 2\|v\|^2 + 2\Re \langle \rho(\gamma_2 \gamma_1^{-1}) v,v\rangle .
\end{align*}
So, we want $\Re \langle \rho(\gamma_2 \gamma_1^{-1}) v,v\rangle \leqslant 0$ 
for all irreducible $\rho$. Sadly, even for $\SU(2)$, this is not possible. 

Write $\gamma = \gamma_2 \gamma_1^{-1}$, then the identity 
$\langle \rho(\gamma)\rho(\delta)v,\rho(\delta)v\rangle = \langle \rho(\delta^{-1} \gamma\delta) v,v\rangle$ tells us that we can restrict our search to 
$\gamma$ of the form $\smat{a}{}{}{\overline a}$ with $|a|=1$. Now 
\[
	\left\langle \smat{a}{}{}{\overline a} \svec{u}{v}, \svec{u}{v}\right\rangle = \Re(a) ,
\]
which appears to be promising. But a similar computation with $\sym^2$ shows 
that one can always get $\langle \sym^2 \gamma  v,v\rangle = 1$, so the above 
approach fails. 

There may be alternative ways of bounding the sums $\sum\mu^{\ast n}(\tr\rho)$, 
but we do not investigate them here. 






\printbibliography[heading=bibintoc]
\end{document}
