% !TEX root = Daniel-Miller-thesis.tex

\chapter{Irrationality exponents}\label{chapter:irrationality-exponent}





\section{Definitions and first results}

We follow the notation of \cite{laurent-2009}. Fix a dimension 
$d\geqslant 1$, and let $\vx=(x_1,\dots,x_d)\in \bR^d$ be such that the $x_i$ 
are linearly independent over $\bQ$. If $d = 1$, the \emph{irrationality 
exponent} of $x\in \bR$ is the supremum of the set of $w\in \bR^+$ such that 
there infinitely many rational numbers $\frac p q$ with 
$\left| x - \frac p q\right| \leqslant q^{-w}$. If $x$ is rational, then it has 
irrationality exponent $1$. If $x$ is an algebraic irrational, then Roth's 
theorem says its irrationality exponent is $2$. Liouiville constructed 
transcendental numbers with arbitrarily large irrationality exponent. Only 
a measure-zero set of reals, for example the Louiville number 
$\sum_{r\geqslant 1} 10^{-r!}$, have infinite irrationality exponent. In the 
results below, we will only consider reals with finite irrationality exponent. 
When $d\geqslant 1$, there are a $d$ natural measures of irrationality, but we 
will use only two of them. 

For the remainder of this thesis, $\langle \cdot,\cdot\rangle$ is the standard 
inner product on $\bR^d$. 

\begin{definition}\label{def:approx-exp}
Let $\omega_0(\vx)$ (resp.~$\omega_{d-1}(\vx)$) be the supremum of the set of 
real numbers $w$ for which there exist infinitely many 
$(n,\vm)\in \bZ\times\bZ^d$ such that 
\begin{align*}
	|n \vx - \vm|_\infty 
		&\leqslant |(n,\vm)|_\infty^{-w}  \qquad\text{(resp.} \\
	|n +\langle \vm,\vx\rangle| 
		&\leqslant |(n,\vm)|_\infty^{-w} \text{).}
\end{align*}
\end{definition}

It is easy to see that both $\omega_0(\vx)$ and $\omega_{d-1}(\vx)$ are 
nonnegative. Even better, by \cite[Th.~2 Cor]{laurent-2009}, 
$\omega_0(\vx) \geqslant \frac 1 d$ and $\omega_{d-1}(\vx) \geqslant d$. 
These two quantities are related by Khintchine's transference 
principle \cite[Th.~2]{laurent-2009}, namely 
\[
	\frac{\omega_{d-1}(\vx)}{(d-1) \omega_{d-1}(\vx)+d} \leqslant \omega_0(\vx) \leqslant \frac{\omega_{d-1}(\vx)-d+1}{d} .
\]
Moreover, the second of these inequalities is sharp in a very strong sense. 

% For Jarnik's paper online
% http://pldml.icm.edu.pl/pldml/element/bwmeta1.element.bwnjournal-article-aav2i1p1bwm
\begin{theorem}[\cite{jarnik-1936}]\label{thm:jarnik}
Let $w\geqslant 1/d$. Then there exists $\vx\in \bR^d$ such that 
$\omega_0(\vx)=w$ and $\omega_{d-1}(\vx) = d w+d-1$. 
\end{theorem}

We can relate the traditional irrationality exponent and the invariant 
$\omega_0$ in the special case $d = 1$. 

\begin{theorem}\label{thm:omega-irrationality}
If $d=1$, then $\omega_0(x) = \mu-1$, where $\mu$ is the 
traditional irrationality exponent of $x$. 
\end{theorem}
\begin{proof}
Both $\mu$ and $\omega_0$ are invariant under translation by $\bZ$, so without 
loss of generality we may assume $x\in [0,1)$. 

First we show that $\omega_0(x)\geqslant \mu-1$. Suppose there exist infinitely 
many $p/q$ with $\left| x - \frac p q\right| \leqslant q^{-w}$. Since $x<1$ we 
may assume that for infinitely many of the $p/q$, $p<q$. Then 
$| q x - p| \leqslant q^{-(w - 1)} = \max(p,q)^{-w}$, which tells us that 
$\omega_0(x) \geqslant \mu - 1$. 

Now, we show that $\mu \geqslant \omega_0(x) + 1$. Suppose there exist 
infinitely many $(n,m)$ with $|n x - m| \leqslant \max(|n|,|m|)^{-w}$. By the 
reverse triangle inequality, 
$\left| |n x| - |m|\right| \leqslant \max(|n|,|m|)^{-w}$, and since 
$x<1$, for $n$ sufficiently large this implies $|n| \geqslant |m|$. It follows 
that for infinitely many $(n,m)$, we have 
$\left| x - \frac m n\right| \leqslant n^{-(w + 1)}$, which implies 
$\mu \geqslant \omega_0(x) + 1$. 
\end{proof}

Here is a statement of Roth's theorem in the current context. 

\begin{theorem}[Roth]
Let $x\in (\overline\bQ\cap \bR)\smallsetminus \bQ$. Then 
$\omega_0(x) = 1$. 
\end{theorem}
\begin{proof}
This follows directly from \cite{roth-1955} and Theorem 
\ref{thm:omega-irrationality}.
\end{proof}

Given $\vx\in \bR^d$, write 
$\dd(\vx,\bZ^d)=\min_{\vm\in \bZ^d} |\vx-\vm|_\infty$. Note that 
$\dd(\vx,\bZ^d)=0$ if and only if $\vx\in \bZ^d$. Moreover, $\dd(-,\bZ^d)$ 
is well-defined for elements of $(\bR/\bZ)^d$. 

\begin{lemma}\label{lem:bound-distance}
Let $\vx\in \bR^d$ with $|\vx|_\infty< 1$ and $\omega_0(\vx)$ 
(resp.~$\omega_{d-1}(\vx)$) finite. Then 
\begin{align*}
	\frac{1}{\dd(n \vx,\bZ^d)} 
		&\ll |n|^{\omega_0(\vx)+\epsilon}\qquad \text{ for $n\in \bZ$ (resp.} \\
	\frac{1}{\dd\left(\langle \vm,\vx\rangle, \bZ\right)} 
		&\ll |\vm|_\infty^{\omega_{d-1}(\vx)+\epsilon} \qquad\text{ for $\vm\in \bZ^d$).}
\end{align*}
\end{lemma}
\begin{proof}
Let $\epsilon>0$. Then there are only finitely many $n\in \bZ$ 
(resp.~$\vm\in \bZ^d$) such that the inequalities in Definition 
\ref{def:approx-exp} hold with $w = \omega_0(x)+\epsilon$ 
(resp.~$w = \omega_{d-1}(\vx)+\epsilon$). In other words, there exist constants 
$C_0, C_{d-1}>0$, depending on $\vx$ and $\epsilon$, such that 
\begin{align*}
	|n \vx - \vm|_\infty 
		&\geqslant C_0 |(n,\vm)|_\infty^{-\omega_0(\vx)-\epsilon} ,\\
	|n + \langle \vm,\vx\rangle| 
		&\geqslant C_{d-1} |(n,\vm)|_\infty^{-\omega_{d-1}(\vx)-\epsilon} 
\end{align*}
for all $(n,\vm)\ne 0$ in $\bZ\times\bZ^d$. 

Start with the first inequality. Fix $n$, and let $\vm$ be a lattice point 
achieving the minimum $|n \vx - \vm|_\infty$; then 
$\dd(n \vx,\bZ^d) \geqslant C_0 |(n,\vm)|_\infty^{-\omega_0(\vx)-\epsilon}$. 
Since $|n\vx - \vm|_\infty < 1$, the reverse triangle inequality gives 
$\left| |n| - \frac{|\vm|_\infty}{|\vx|_\infty}\right| \leqslant \frac{1}{|\vx|_\infty}$. So $|n|$ and $|\vm|$ are bounded above and below by scalar multiples 
of each other, which tells us that 
$\dd(n \vx,\bZ^d) \geqslant C_0' |n|^{-\omega_0(\vx)-\epsilon}$ for $C_0'$ 
depending on $\vx$. It follows that 
$\frac{1}{\dd(n \vx,\bZ^d)} \ll |n|^{\omega_0(\vx)+\epsilon}$, the 
implied constant depending on $x$ and $\epsilon$.

Now we consider the second inequality. Note 
that $\dd(\langle \vm,\vx\rangle,\bZ) = |n + \langle \vm,\vx\rangle|$ for 
some $n$ with $|n| \leqslant |\vm|_2\cdot |\vx|_2 + 1$. Thus 
$|(n,\vm)|_\infty \ll |\vm|_2 \ll |\vm|_\infty$ with the implied constants 
depending on $d$ and $x$, because any two norms on a 
finite-dimensional Banach space are equivalent. This gives us 
$\dd(\langle\vm,\vx\rangle,\bZ) \geqslant C_{d-1}' |\vm|_\infty^{-\omega_{d-1}(\vx)-\epsilon}$, 
for some constant $C_{d-1}'$. This implies 
\[
	\frac{1}{d(\langle \vm,\vx\rangle,\bZ)} \ll |\vm|_\infty^{\omega_{d-1}(\vx)+\epsilon} ,
\]
the implied constant depending on $\vx$ and $\epsilon$.
\end{proof}





\section{Irrationality exponents and discrepancy}

Let $\vx=(x_1,\dots,x_d)\in \bR^d$. The sequence 
$(\vx\mod \bZ^d,2\vx\mod\bZ^d,\dots)$ will be equidistributed in a subtorus of 
$(\bR/\bZ)^d$ whose rank is equal to the dimension of the $\bQ$-vector space 
spanned by $\{x_1,\dots,x_d\}$. We are interested in the case where this 
sequence is equidistributed in the whole torus $(\bR/\bZ)^d$, so assume 
$x_1,\dots,x_d$ are linearly independent over $\bQ$ (this condition also 
makes sense for elements of $(\bR/\bZ)^d$). For $\vx\in (\bR/\bZ)^d$, we wish 
to control the discrepancy of the sequence $(\vx,2\vx,3\vx,\dots)$ with respect 
to the Haar measure of $(\bR/\bZ)^d$. 

\begin{theorem}[Erd\"os--Tur\'an--Koksma]
Let $\bvx$ be a sequence in $(\bR/\bZ)^d$ and $h$ an arbitrary integer. Then 
\[
	\D_N(\bvx) \ll \frac 1 h + \sum_{0\leqslant |\vm|_\infty \leqslant h} \frac{1}{r(\vm)} \left| \frac 1 N \sum_{n\leqslant N} e^{2\pi i \langle \vm,\vx_n\rangle}\right| ,
\]
where the first sum ranges over $\vm\in \bZ^d$, 
$r(\vm) = \prod \max\{1,|m_i|\}$, and the implied constant depends only on $d$. 
\end{theorem}
\begin{proof}
This is \cite[Th.~1.21]{drmota-tichy-1997}. 
\end{proof}

\begin{lemma}\label{lem:bound-exp-sum}
Let $x\in \bR$. Then 
$\left| \sum_{n\leqslant N} e^{2\pi i n x}\right| \leqslant \frac{2}{\dd(x, \bZ)}$. 
\end{lemma}
\begin{proof}
We begin with an easy bound: 
\[
	\left| \sum_{n\leqslant N} e^{2\pi i n x}\right| = \frac{|e^{2\pi i (N+1) x} - e^{2\pi i x}|}{|e^{2\pi i x} - 1|} \leqslant \frac{2}{|e^{2\pi i x} - 1|} .
\]
Since $|e^{2\pi i x} - 1| = \sqrt{2-2\cos(2\pi x)}$ and 
$\cos(2\theta) = 1-2\sin^2\theta$, we obtain 
\[
	\left|\sum_{n\leqslant N} e^{2\pi i n x}\right| \leqslant \frac{1}{|\sin(\pi x)|} .
\]
It is easy to check that $|\sin(\pi x)| \geqslant \dd(x,\bZ)$, whence the result.  
\end{proof}

\begin{corollary}\label{cor:bound-disc-distance}
Let $\vx\in (\bR/\bZ)^d$ with $(x_1,\dots,x_d)$ linearly independent over $\bQ$. 
Then for $\bvx=(\vx,2\vx,3\vx,\dots)$, we have 
\[
	\D_N(\bvx) \ll \frac 1 h + \frac 1 N \sum_{0<|\vm|_\infty \leqslant h} \frac{2}{r(\vm) \dd(\langle \vm,\vx\rangle,\bZ)} 
\]
for any integer $h$, with the implied constant depending only on $d$. 
\end{corollary}
\begin{proof}
Apply the Erd\"os--Tur\'an--Koksma inequality, and bound the exponential sums 
using Lemma \ref{lem:bound-exp-sum}. 
\end{proof}

We combine the above results to estimate an upper bound on the discrepancy of 
the sequence $\bvx$. 

\begin{theorem}\label{thm:disc-upper-bound}
Let $\bvx=(\vx,2\vx,3\vx,\dots)$ in $(\bR/\bZ)^d$. Then 
\[
	\D_N(\bvx) \ll N^{-\frac{1}{\omega_{d-1}(\vx)+1}+\epsilon} .
\]
\end{theorem}
\begin{proof}
Fix $\epsilon>0$ smaller than $\frac{1}{\omega_{d-1}(\vx) - 1}$, and choose 
$\delta>0$ such that 
$\frac{1}{\omega_{d-1}(\vx)+1+\delta} = \frac{1}{\omega_{d-1}(\vx)+1} - \epsilon$. 
By Corollary \ref{cor:bound-disc-distance}, we know that 
\[
	\D_N(\bvx) \ll \frac 1 h + \frac 1 N \sum_{0<|\vm|_\infty \leqslant h} \frac{1}{r(\vm) \dd(\langle \vm,\vx\rangle,\bZ)} ,
\]
and by Lemma \ref{lem:bound-distance}, we know that 
$\dd(\langle \vm,\vx\rangle,\bZ)^{-1} \ll |\vm|^{\omega_{d-1}(\vx)+\delta}$. 
It follows that 
\[
	\D_N(\bvx) \ll \frac 1 h + \frac 1 N \sum_{0 < |\vm|_\infty \leqslant h} \frac{|\vm|_\infty^{\omega_{d-1}(\vx)+\delta}}{r(\vm)} .
\]
All that remains is to bound the sum. Clearly 
\[
	\sum_{0 < |\vm|_\infty \leqslant h} \frac{|\vm|_\infty^{\omega_{d-1}(\vx) + \delta}}{r(m)} \ll \int_1^h \int_1^h \dots \int_1^h \frac{\max(|t_1|,\dots,|t_d|)^{\omega_{d-1}(\vx)+\delta}}{t_1 \dots t_d}\, \dd t_1 \dots \dd t_d .
\]
For each permutation $\sigma$ of $\{1,\dots,d\}$, call $I_\sigma$ the set of 
all $(t_1,\dots,t_d)$ in $[1,\infty)^d$ with 
$t_{\sigma(1)} < \dots < t_{\sigma(d)}$. Then 
$[1,\infty)^d = \bigcup_{\sigma\in S_d} I_\sigma$, and each integral over 
$I_\sigma$ is easy to bound. For example, the integral over $I_1$ is 
\begin{align*}
	\int_1^h \int_1^{t_d} \dots \int_1^{t_2} \frac{t_d^{\omega_{d-1}(\vx)+\delta}}{t_1 \dots t_d}\, \dd t_1 \dots \dd t_d 
		&\ll \int_1^h t^{\omega_{d-1}(\vx)+\delta-1}\, \dd t \prod_{j=1}^{d-1} \int_1^h \frac{\dd t}{t} \\
		&\ll (\log h)^{d-1} h^{\omega_{d-1}(\vx)+\delta} .
\end{align*}
It follows that 
$\D_N(\bvx) \ll \frac 1 h + \frac 1 N (\log h)^{d-1} h^{\omega_{d-1}(\vx)+\delta}$. 
Setting $h\approx N^{\frac{1}{1+\omega_{d-1}(\vx)+\delta}}$, we see that 
$D_N(\bvx) \ll N^{-\frac{1}{\omega_{d-1}(\vx)+1+\delta}} = N^{-\frac{1}{\omega_{d-1}(\vx)+1} + \epsilon}$. 

For a slightly different proof of a similar result, given as a sequence of 
exercises, see  \cite[Ch.~2, Ex.~3.15, 16, 17]{kuipers-niederreiter-1974}. 
Also, this estimate is quite coarse, but a better one would only have a smaller 
leading coefficient, which no doubt would be useful for computational 
purposes, but does not strengthen any of the results in this thesis.  
\end{proof}

\begin{theorem}\label{thm:disc-lower-bound}
Let $\vx\in \bR^d$ be such that $x_1,\dots,x_d$ are linearly independent over 
$\bQ$, and let $\bvx=(\vx,2\vx,3\vx,\dots)$ in $(\bR/\bZ)^d$. Then 
$\D_N(\bvx) = \Omega\left(N^{-\frac{d}{\omega_0(\vx)}-\epsilon} \right)$. 
\end{theorem}
\begin{proof}
We follow the proof of \cite[Ch.~2, Th.~3.3]{kuipers-niederreiter-1974}, 
modifying it as needed for our context. Given $\epsilon>0$, there exists 
$\delta>0$ such that 
$\frac{d}{\omega_0(\vx)-\delta} = \frac{d}{\omega_0(\vx)} + \epsilon$. 

By the definition of $\omega_0(\vx)$, there exist infinitely many 
$(q,\vm)$ with $q>0$ such that 
$|q \vx - \vm|_\infty \leqslant |(q,\vm)|_\infty^{-\omega_0(\vx)+\delta/2}$. 
Since $|(q,\vm)|_\infty \geqslant q$, we derive the seemingly stronger 
statement that for infinitely many $q$, there exists 
$\vm\in \bZ^d$ such that 
$|q \vx-\vm|_\infty \leqslant q^{-\omega_0(\vx)+\delta/2}$ or, equivalently, 
$|\vx-q^{-1}\vm| \leqslant q^{-1-\omega_0(\vx)+\delta/2}$. Fix one such $q$, 
and let $N=\lfloor q^{\omega_0(\vx)-\delta}\rfloor$. For each $n\leqslant N$, 
we have 
\[
	\left|n \vx - n q^{-1} \vm\right|_\infty 
		\leqslant n \left|\vx - q^{-1} \vm\right|_\infty
		\leqslant n q^{-1-\omega_0(\vx)+\delta/2}
		\leqslant q^{-1-\delta/2}. 
\]
Thus, for each $n\leqslant N$, $n \vx$ is within $q^{-1-\delta/2}$ of the 
grid $\frac 1 q \bZ^d\subset (\bR/\bZ)^d$. So no element of 
$\{\vx,\dots,N \vx\}$ lies in the half-open box 
$I_q = \left[ q^{-1 - \delta / 3}, q^{-1} - q^{-1 - \delta / 3}\right)^d$. 
Moreover, $I_q$ has volume $\left(q^{-1} - 2q^{-1 - \delta / 3}\right)^d$. 
For $q$ sufficiently large, the volume of $I_q$ is bounded below by 
$2^{-d} q^{-d}$, so the discrepancy $\D_N(\bvx)$ is 
bounded below by $2^{-d} q^{-d}$. Since $q^{\omega_0(\vx)-\delta} \leqslant 2 N$, 
the discrepancy $\D_N(\bvx)$ is bounded below by 
\[
	2^{-d} \left( (2 N)^{\frac{1}{\omega_0(\vx)+\delta}}\right)^{-d} 
		= 2^{-d-\frac{d}{\omega_0(\vx)+\delta}} N^{-\frac{d}{\omega_0(\vx)+\delta}}
		= 2^{-d\left(1+\frac{1}{\omega_0(\vx)}\right)-\epsilon} N^{-\frac{d}{\omega_0(\vx)}-\epsilon} .
\]
Since $\D_N(\bvx)$ can, as $N\to \infty$, be bounded below by a constant 
multiple of $N^{-\frac{d}{\omega_0(\vx)}-\epsilon}$, the proof is complete.
\end{proof}






\section{Pathological Satake parameters for CM abelian varieties}

We apply the results of the previous sections to $L$-functions associated to 
CM abelian varieties. For background on the motivic Galois group and Sato--Tate 
group of an abelian variety, see \cite{serre-tate-1968,serre-1994,yu-2015}. 
Recall that for $E$ a non-CM elliptic curve, the 
Akiyama--Tanigawa conjecture implies the Riemann hypothesis for all 
$L(\sym^k E,s)$, $k\geqslant 1$. The appearance of $\sym^k$ is dictated by the 
classification of irreducible representations of $\SU(2)$, the Sato--Tate group 
of $E$. If $A$ is a CM abelian variety, there should be an $L$-function (and 
Galois representation) for each irreducible representation of the Sato--Tate 
group of $A$, which we denote by $\ST(A)$. In the 
CM case, $\ST(A)$ is a real torus, so things can be described relatively 
explicitly. 

Let $K/\bQ$ be a finite Galois extension, $A_{/K}$ a $g$-dimensional abelian 
variety with complex multiplication by $F$, defined over $K$, that is, 
$F = \End_K(A)_\bQ$. Since the action of $F$ commutes with 
$\rho_l\colon G_\bQ \to \GL_{2g}(\bQ_l)$, the Galois representation coming 
from the $l$-adic Tate module of $A$ takes values in $\R_{F/\bQ}\Gm(\bQ_l)$, 
where $R_{F/\bQ} \Gm$ is the Weil restriction of scalars of the multiplicative 
group from $F$ to $\bQ$. The functor of points of $\R_{F/\bQ}\Gm$ is 
$R\mapsto (R\otimes F)^\times$. It follows that the Sato--Tate group of $A$ is 
a subgroup of the maximal compact torus inside $\R_{F/\bQ}\Gm(\bC)$. 

Recall, following \cite{serre-1994}, that the motivic Galois group of $A$ 
should be a subgroup $G_A\subset \R_{F/\bQ}\Gm$ such that for all primes $l$, 
the image $\rho_l(G_\bQ)$ lies inside $G_A(\bQ_l)$, and is open in 
$G_A(\bQ_l)$. For general abelian varieties, the existence of the motivic 
Galois group is a matter of conjecture, but for CM abelian varieties, it can be 
described directly. Let $\fa=\lie(A)$, and 
$\det_\fa\colon \R_{K/\bQ}\Gm \to \R_{F/\bQ}\Gm$ be the map induced by the 
determinant of the action of $K$ on $\fa$ (viewed as an $F$-vector space). Then 
$G_A = \im(\det_\fa)$ \cite{yu-2015}, and $\ST(A)$ is a maximal compact 
subgroup of $G_A^1(\bC) = G_A^{\N_{F/\bQ} = 1}(\bC)$. So 
$\ST(A) \simeq (\bR/\bZ)^d$ for some $1\leqslant d \leqslant g$, and every 
unitary character of $\ST(A)$ is induced by an algebraic character of 
$G_A^1$. Any character of a subtorus extends to the whole torus, so any 
character of $G_A^1$ is the restriction of a character of $\R_{F/\bQ}\Gm$. 

Let $\fp$ be a prime of $K$ at which $A$ has good reduction. Then 
$F = \End(A)_\bQ\hookrightarrow \End(A_{/\bF_\fp})_\bQ$, and the Frobenius 
element $\frob_\fp\in \End(A_{/\bF_\fp})_\bQ$ comes from an element 
$\pi_\fp\in F$. In other words, $\rho_l(\frob_\fp) = \pi_\fp$. The element 
$\pi_\fp\in F$ is $\fp$-Weil of weight $1$, 
i.e.~$|\sigma(\pi_\fp)| = \N(\fp)^{1/2}$ for all embeddings 
$\sigma\colon F\hookrightarrow \bC$. The normalized element 
$\theta_\fp = \frac{\pi_\fp}{\N(\fp)^{1/2}}$ lies in $\ST(A)$, and we call this 
the Satake parameter for $A$ at $\fp$. For the Satake parameters to be 
equidistributed in $\ST(A)$, it is necessary and sufficient for the 
$L$-function $L(r\circ \rho_l,s)$ to have non-vanishing analytic continuation 
to $\Re =1$ for each $r\in \X^\ast(\R_{F/\bQ}\Gm)$ which has nontrivial 
restriction to $\ST(A)$. By the Wiener--Ikehara Tauberian theorem, this is 
equivalent to an estimate 
$\left| \sum_{\N(fp)\leqslant x} r(\theta_\fp)\right| = o(\pi_K(x))$. 

\begin{theorem}[Shimura--Taniyama, Weil, Hecke]
The elements $\theta_\fp\in \ST(A)$ are equidistributed with respect to the 
Haar measure. 
\end{theorem}
\begin{proof}
By \cite[Th.~10, 11]{serre-tate-1968}, for every 
$r\in \X^\ast(\R_{F/\bQ}\Gm)$, there exists a Hecke character $\omega_r$ of $K$ 
such that $L(r\circ \rho_l,s) = L(s,\omega_r)$, and moreover $\omega_r$ is 
nontrivial if and only if $\left.r\right|_{\ST(A)}$ is. Since $L$-functions of 
Hecke characters have the desired analytic continuation and nonvanishing, the 
result follows. 
\end{proof}

Recall that 
$L(r\circ \rho_l,s) = \prod \left(1 - r(\theta_\fp) \N(\fp)^{-s}\right)^{-1}$ 
(this is the normalized $L$-function, not the algebraic $L$-function). 
As in Chapter \ref{ch2:discrepancy}, the choice of an isomorphism
$(\bR/\bZ)^d\simeq \ST(A)$ yields a definition of discrepancy 
for sequences in $\ST(A)$. We call the ``Akiyama--Tanigawa conjecture for $A$'' 
the estimate $\D_N(\btheta) \ll N^{-\frac 1 2+\epsilon}$, where 
$\btheta = (\theta_\fp)_\fp$ is the sequence of Satake parameters of $A$. 

\begin{theorem}\label{AT->RH:AB}
Let $A_{/K}$ be a CM abelian variety. The Akiyama--Tanigawa conjecture for $A$ 
implies the Riemann hypothesis for all $L(r\circ \rho_l,s)$ with 
$\left. r\right|_{\ST(A)}$ nontrivial. 
\end{theorem}
\begin{proof}
The Akiyama--Tanigawa estimate implies, via the Koksma--Hlawka inequality, an 
estimate 
$\left| \sum_{\N(\fp)\leqslant N} r(\theta_\fp)\right| \ll N^{-\frac 1 2+\epsilon}$. 
By Theorem \ref{thm:AT->RH:gp}, the function $L(r\circ \rho_l,s)$ satisfies 
the Riemann hypothesis. 
\end{proof}

It is natural to ask: does the Riemann hypothesis for all $L(r\circ \rho_l,s)$ 
imply the Akiyama--Tanigawa conjecture for $A$? We proceed to construct 
$L$-functions coming from ``fake Satake parameters'' which provide evidence to 
the contrary for nonmotivic (non-automorphic, in fact) Satake parameters. 

Give $(\bR/\bZ)^d$ the Haar measure normalized to have total mass one. 
Recall that for any $f\in L^1((\bR/\bZ)^d)$, the Fourier coefficients of $f$ 
are, for $\vm\in \bZ^d$: 
\[
	\widehat f(\vm) = \int_{(\bR/\bZ)^d} e^{2\pi i \langle \vm,\vx\rangle} \, \dd \vx ,
\]
where $\langle \vm,\vx\rangle = m_1 x_1 + \cdots + m_d x_d$ is the usual inner 
product. Typically, if $f$ is a function on $(\bR/\bZ)^d$, sums of the form 
$\sum_{n\leqslant N} f(n \vx)$ will be $o(N)$. When $f$ is a character of the 
torus, there is a much stronger bound. 

\begin{theorem}
Fix $\vx\in (\bR/\bZ)^d$ with $\omega_{d-1}(\vx)$ finite. Then 
\[
	\left| \sum_{n\leqslant N} e^{2\pi i \langle \vm, n \vx\rangle}\right| \ll |\vm|_\infty^{\omega_{d-1}(\vx) + \epsilon} 
\]
as $\vm$ ranges over $\bZ^d\smallsetminus 0$. 
\end{theorem}
\begin{proof}
From Lemma \ref{lem:bound-exp-sum} we know that 
$\left| \sum_{n\leqslant N} e^{2\pi i \langle \vm, n \vx\rangle}\right| \ll \dd(\langle \vm, \vx\rangle,\bZ)^{-1}$, 
and from Lemma \ref{lem:bound-distance}, we know that 
$\dd(\langle \vm,\vx\rangle, \bZ)^{-1} \ll |\vm|_\infty^{\omega_{d-1}(x)+\epsilon}$. 
The result follows. 
\end{proof}

By writing any function as a Fourier series, we can apply this result to sums 
of the form $\sum_{n\leqslant N} f(n \vx)$. 

\begin{theorem}\label{thm:translates-bound-sum}
Let $\vx\in \bR^d$ with $\omega_{d-1}(\vx)$ finite. Fix 
$f\in C^\infty((\bR/\bZ)^d)$ with $\widehat f(\vzero)=0$. Then 
$\left| \sum_{n\leqslant N} f(n \vx)\right| \ll 1$. 
\end{theorem}
\begin{proof}
Write $f$ as a Fourier series:
$f(\vx) = \sum_{\vm\in \bZ^d} \widehat f(\vm) e^{2\pi i \langle \vm,\vx\rangle}$. 
Since $\widehat f(\vzero)=0$, we can compute:
\begin{align*}
	\left| \sum_{n\leqslant N} f(n \vx)\right| 
		&= \left| \sum_{n\leqslant N} \sum_{\vm\in \bZ^d\smallsetminus \vzero} \widehat f(\vm) e^{2\pi i n \langle \vm,\vx\rangle}\right| \\
		&\leqslant \sum_{\vm\in \bZ^d\smallsetminus \vzero} \left|\widehat f(\vm)\right|\cdot \left| \sum_{n\leqslant N} e^{2\pi i n \langle \vm,\vx\rangle}\right| \\
		&\ll \sum_{\vm\in \bZ^d\smallsetminus \vzero} \left|\widehat f(\vm)\right|\cdot |\vm|_\infty^{\omega_{d-1}(\vx) + \epsilon} .
\end{align*}
The sum converges since the Fourier coefficients $\widehat f(\vm)$ converge to 
zero faster than the reciprocal of any polynomial. 
\end{proof}

The function $f$ does not need to be smooth---so long as its Fourier 
coefficients decay sufficiently rapidly as to force 
$\sum |\widehat f(\vm)| \cdot |\vm|_\infty^{\omega_{d-1}(\vx)+\epsilon}$ to 
converge, the proof works. 

Enumerate the primes of $K$ with increasing norms as $\fp_1,\fp_2,\fp_3,\dots$.  
Let $\vy\in \bR^d$ with $y_1,\dots,y_d$ linearly independent over $\bQ$. The 
associated sequence of ``fake Satake parameters'' is 
$\bvx = (\vy, 2\vy, 3 \vy, 4 \vy, \dots)$, 
where we put $\vx_{\fp_n} = n \vy\mod \bZ^d$. For any fixed 
$w\geqslant \frac 1 d$, by Theorem \ref{thm:jarnik}, we can find $\vy$ with 
$\omega_0(\vy) = w$ and $\omega_{d-1}(\vy) = d w + d - 1$. 

\begin{theorem}
The sequence $\bvx$ is equidistributed in $(\bR/\bZ)^d$, with discrepancy 
decaying as $\D_N(\bvx) \ll N^{-\frac{1}{d w+d} + \epsilon}$, and for which 
$\D_N(\bvx) = \Omega\left(N^{-\frac{d}{w} - \epsilon}\right)$. 
However, for any $f\in C^\infty((\bR/\bZ)^d)$ with $\widehat f(0)=0$, the 
Dirichlet series  $L_f(\bx,s)$ satisfies the Riemann hypothesis. 
\end{theorem}
\begin{proof}
The upper bound on discrepancy is Theorem \ref{thm:disc-upper-bound}, and 
the lower bound is Theorem \ref{thm:disc-lower-bound}. For the functions $f$ in 
question, Theorem \ref{thm:translates-bound-sum} gives an estimate (stronger 
than) $\left| \sum_{\N(\fp)\leqslant N} f(\vx_\fp)\right|\ll N^{\frac 1 2}$, and 
Theorem \ref{thm:AT->RH:gp} tells us this estimate implies the Riemann 
hypothesis. 
\end{proof}

This shows that even if each $L(r_\ast \rho_l,s)$ satisfies the Riemann 
hypothesis, we may not conclude that the Akiyama--Tanigawa 
conjecture holds for $A$. Note also that Theorem \ref{thm:AT->RH:gp} does 
\emph{not} tell us that $L_f(\bvx,s)$ has analytic continuation to $\Re > 0$, or 
that there are no zeros in $\Re > 0$. For, the term 
$\sum_\fp \sum_{r\geqslant 2} \frac{f(\vx_\fp)^r}{r \N(\fp)^{r s}}$ will not 
converge past $\Re > \frac 1 2$.
