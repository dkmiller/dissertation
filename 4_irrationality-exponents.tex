% !TEX root = Daniel-Miller-thesis.tex

\chapter{Irrationality exponents}\label{chapter:irrationality-exponent}





\section{Definitions and first results}

We follow the notation of \cite{laurent-2009}. Fix a dimension 
$d\geqslant 1$, and let $\vx=(x_1,\dots,x_d)\in \bR^d$ be such that the $x_i$ 
are $\bQ$-linearly independent. 

\begin{definition}\label{def:approx-exp}
Let $\omega_0(\vx)$ (resp.~$\omega_{d-1}(\vx)$) be the supremum of the set of 
real numbers $\omega$ for which there exist infinitely many 
$(n,\vm)\in \bZ^{d+1}$ such that 
\begin{align*}
	|n \vx - \vm|_\infty 
		&\leqslant |(n,\vm)|_\infty^{-\omega}  \qquad\text{(resp.} \\
	|n +\langle \vm,\vx\rangle| 
		&\leqslant |(n,\vm)|_\infty^{-\omega} \text{).}
\end{align*}
\end{definition}

These two quantities are related by Khintchine's Transference Principle, namely 
\[
	\frac{\omega_{d-1}(\vx)}{(d-1) \omega_{d-1}(\vx)+d} \leqslant \omega_0(\vx) \leqslant \frac{\omega_{d-1}(\vx)-d+1}{d} .
\]
Moreover, these inequalities are sharp in a very strong sense. 

\begin{theorem}[Jarn\'ik]\label{thm:jarnik}
Let $w\geqslant 1/d$. Then there exists $\vx\in \bR^d$ such that 
$\omega_0(\vx)=w$ and $\omega_{d-1}(\vx) = d w+d-1$. 
\end{theorem}

\begin{theorem}
If $d=1$, then $\omega_0(x) = \mu-1$, where $\mu$ is the traditional 
irrationality measure of $x$. 
\end{theorem}

So Roth's Theorem tells us that for $x$ an algebraic irrational, 
$\omega_0(x) = 1$. 

Given $\vx\in \bR^d$, we write 
$\dd(\vx,\bZ^d)=\min_{\vm\in \bZ^d} |\vx-\vm|_\infty$. Note that 
$\dd(\vx,\bZ^d)=0$ if and only if $\vx\in \bZ^d$. Moreover, $\dd(\cdot,\bZ^d)$ 
is well-defined for elements of $(\bR/\bZ)^d$. 

\begin{lemma}\label{lem:bound-distance}
Let $\vx\in \bR^d$ with $|\vx|_\infty\leqslant 1$ and $\omega_0(\vx)$ 
(resp.~$\omega_{d-1}(\vx)$) finite. Then 
\begin{align*}
	\frac{1}{\dd(n \vx,\bZ^d)} 
		&\ll |n|^{\omega_0(\vx)+\epsilon}\qquad \text{ for $n\in \bZ$ (resp.} \\
	\frac{1}{\dd\left(\langle \vm,\vx\rangle, \bZ\right)} 
		&\ll |\vm|_\infty^{\omega_{d-1}(\vx)+\epsilon} \qquad\text{ for $\vm\in \bZ^d$).}
\end{align*}
\end{lemma}
\begin{proof}
Let $\epsilon>0$. Then there are only finitely many $n\in \bZ$ 
(resp.~$\vm\in \bZ^d$) such that the inequalities in Definition 
\ref{def:approx-exp} hold with $\omega_0(x)+\epsilon$ 
(resp.~$\omega_{d-1}(\vx)+\epsilon$). In other words, there exist constants 
$C_0, C_{d-1}>0$ such that 
\begin{align*}
	|n \vx - \vm|_\infty 
		&\geqslant C_0 |(n,\vm)|_\infty^{-\omega_0(\vx)-\epsilon} ,\\
	|n + \langle \vm,\vx\rangle| 
		&\geqslant C_{d-1} |(n,\vm)|_\infty^{-\omega_{d-1}(\vx)-\epsilon} 
\end{align*}
for all $(n,\vm)\ne 0$ in $\bZ^{d+1}$. 

Start with the first inequality in the statement of the result. Let 
$\vm$ be the lattice point achieving the minimum $|n \vx - \vm|_\infty$; 
then 
$\dd(n \vx,\bZ^d) \geqslant C_0 |(n,\vm)|_\infty^{-\omega_0(\vx)-\epsilon}$. 
Since $|n\vx - \vm|_\infty < 1$, 
$|n| \geqslant \frac{|\vm|_\infty}{|\vx|_\infty}-1$ , which gives 
$\dd(n \vx,\bZ^d) \geqslant C_0' |n|^{-\omega_0(\vx)-\epsilon}$ for $C_0'$ 
depending on $\vx$. It follows that 
$\frac{1}{\dd(n \vx,\bZ^d)} \ll |n|^{\omega_0(\vx)+\epsilon}$, the 
implied constant depending on $x$ and $\epsilon$.

Now let's consider the second inequality in the statement of the result. Note 
that $\dd(\langle \vm,\vx\rangle,\bZ) = |n + \langle \vm,\vx\rangle|$ for 
some $n$ with $|n| \leqslant \|\vm\|_2 \|\vx\|_2 + 1$. Thus 
$|(n,\vm)|_\infty \ll |\vm|_2 \ll |\vm|_\infty$ (any two norms on a 
finite-dimensional Banach space are equivalent), which gives us 
$\dd(\langle\vm,\vx\rangle,\bZ) \geqslant C_{d-1}' |\vm|^{-\omega_{d-1}(\vx)-\epsilon}$, for some constant $C_{d-1}'$. This implies 
\[
	\frac{1}{d(\langle \vm,\vx\rangle,\bZ)} \ll |\vm|^{\omega_{d-1}(\vx)+\epsilon} ,
\]
the implied constant depending on $\vx$ and $\epsilon$.
\end{proof}





\section{Irrationality exponents and discrepancy}

Let $x\in \bR^d$ with $x_1,\dots,x_d$ linearly independent over $\bQ$. We wish 
to control the discrepancy of the sequence $\{x,2x,3x,\dots\}$ in 
$(\bR/\bZ)^d$. 

\begin{theorem}[Erd\"os--Tur\'an--Koksma]
Let $\bx$ be a sequence in $\bR^d$ and $h$ an arbitrary integer. Then 
\[
	\D(\bx^N) \ll \frac 1 h + \sum_{0\leqslant \|\vm\|_\infty \leqslant h} \frac{1}{r(\vm)} \left| \frac 1 N \sum_{n\leqslant N} e^{2\pi i \langle \vm,\vx_n\rangle}\right| ,
\]
where the first sum ranges over $\vm\in \bZ^d$, 
$r(\vm) = \prod \max\{1,|m_i|\}$, and the implied constant depends only on $d$. 
\end{theorem}
\begin{proof}
This is \cite[Th.~1.21]{drmota-tichy-1997}. 
\end{proof}

\begin{lemma}\label{lem:bound-exp-sum}
Let $x\in \bR$. Then 
\[
	\left| \sum_{n\leqslant N} e^{2\pi i n x}\right| \leqslant \frac{2}{\dd(x, \bZ)} .
\]
\end{lemma}
\begin{proof}
We begin with an easy bound: 
\[
	\left| \sum_{n\leqslant N} e^{2\pi i n x}\right| = \frac{|e^{2\pi i (N+1) x} - e^{2\pi i x}|}{|e^{2\pi i x} - 1|} \leqslant \frac{2}{|e^{2\pi i x} - 1|} .
\]
Since $|e^{2\pi i x} - 1| = \sqrt{2-2\cos(2\pi x)}$ and 
$\cos(2\theta) = 1-2\sin^2\theta$, we obtain 
\[
	\left|\sum_{n\leqslant N} e^{2\pi i n x}\right| \leqslant \frac{1}{|\sin(\pi x)|} .
\]
It is easy to check that $|\sin(\pi x)| \geqslant \dd(x,\bZ)$, whence the result.  
\end{proof}

\begin{corollary}\label{cor:bound-disc-distance}
Let $\vx\in (\bR/\bZ)^d$ with $(x_1,\dots,x_d)$ linearly independent over $\bQ$. 
Then for $\bx=(\vx,2\vx,3\vx,\dots)$, we have 
\[
	\D_N(\bx) \ll \frac 1 h + \frac 1 N \sum_{0<|\vm|_\infty \leqslant h} \frac{1}{r(\vm) \dd(\langle \vm,\vx\rangle,\bZ)} 
\]
for any integer $h$, with the implied constant depending only on $d$. 
\end{corollary}
\begin{proof}
Apply the Erd\"os--Tur\'an--Koksma inequality and bound the exponential sums 
using Lemma \ref{lem:bound-exp-sum}. 
\end{proof}

\begin{theorem}
Let $\bx=(\vx,2\vx,3\vx,\dots)$ in $(\bR/\bZ)^d$. Then 
\[
	\D_N(\bx) \ll N^{-\frac{1}{\omega_{d-1}(\vx)+1}+\epsilon} .
\]
\end{theorem}
\begin{proof}
Choose $\delta>0$ such that 
$\frac{1}{\omega_{d-1}(\vx)+1+\delta} = \frac{1}{\omega_{d-1}(\vx)+1} - \epsilon$. 
By Corollary \ref{cor:bound-disc-distance}, we know that 
\[
	\D_N(\bx) \ll \frac 1 h + \frac 1 N \sum_{0<|\vm|_\infty \leqslant h} \frac{1}{r(\vm) \dd(\langle \vm,\vx\rangle,\bZ)} ,
\]
and by Lemma \ref{lem:bound-distance}, we know that 
$\dd(\langle \vm,\vx\rangle,\bZ)^{-1} \ll |\vm|^{\omega_{d-1}(\vx)+\delta}$. 
It follows that 
\[
	\D_N(\bx) \ll \frac 1 h + \frac 1 N \sum_{0 < |\vm|_\infty \leqslant h} \frac{|\vm|^{\omega_{d-1}(\vx)+\delta}}{r(\vm)} .
\]
All that remains is to bound the sum. 
\begin{align*}
	\sum_{0< |\vm|_\infty \leqslant h} \frac{|\vm|_\infty^{\omega_{d-1}(\vx)+\delta}}{r(m)} 
		&\ll \int_1^h \int_1^{t_d} \dots \int_1^{t_2} \frac{t_d^{\omega_{d-1}(\vx)+\delta}}{t_1 \dots t_d}\, \dd t_1 \dots \dd t_d \\
		&\ll \int_1^h t^{\omega_{d-1}(\vx)+\delta-1}\, \dd t \prod_{j=1}^{d-1} \int_1^h \frac{\dd t}{t} \\
		&\ll (\log h)^{d-1} h^{\omega_{d-1}(\vx)+\delta} .
\end{align*}
It follows that 
$\D_N(\bx) \ll \frac 1 h + \frac 1 N (\log h)^{d-1} h^{\omega_{d-1}(\vx)+\delta}$. 
Setting $h\approx N^{\frac{1}{1+\omega_{d-1}(\vx)+\delta}}$, we see that 
$D_N(\bx) \ll N^{-\frac{1}{\omega_{d-1}(\vx)+1+\delta}} = N^{-\frac{1}{\omega_{d-1}(\vx)+1} + \epsilon}$. 

For a slightly different proof of a similar result (given as a sequence of 
exercises), see  \cite[Ch.~2, Ex.~3.15, 16, 17]{kuipers-niederreiter-1974}. 
\end{proof}

\begin{theorem}
Let $\vx\in \bR^d$ be such that $x_1,\dots,x_d$ are linearly independent over 
$\bQ$, and let $\bx=(\vx,2\vx,3\vx,\dots)$ in $(\bR/\bZ)^d$. Then 
\[
	\D_N(\bx) = \Omega\left(N^{-\frac{d}{\omega_0(\vx)}-\epsilon} \right).
\]
\end{theorem}
\begin{proof}
Here $f=\Omega(g)$ in the sense of Hardy, namely that $\limsup \frac f g>0$. We 
follow the proof of \cite[Ch.~2, Th.~3.3]{kuipers-niederreiter-1974}. Given 
$\epsilon>0$, there exists $\delta>0$ such that 
$\frac{d}{\omega_0(\vx)-\delta} = \frac{d}{\omega_0(\vx)} + \epsilon$. 

By the definition of $\omega_0(\vx)$, there exist infinitely many 
$(q,\vm)$ with $q>0$ such that 
$|q \vx - \vm|_\infty \leqslant |(q,\vm)|_\infty^{-\omega_0(\vx)+\delta/2}$. 
Since $|(q,\vm)|_\infty \geqslant q$, we derive the stronger 
statement that for infinitely many $q\to \infty$, there exists 
$\vm\in \bZ^d$ such that 
$|q \vx-\vm|_\infty \leqslant q^{-\omega_0(\vx)+\delta/2}$ or, equivalently, 
$|\vx-q^{-1}\vm| \leqslant q^{-1-\omega_0(\vx)+\delta/2}$. Pick such a $q$, and 
let $N=\lfloor q^{\omega_0(\vx)-\delta}\rfloor$. Then for each $n\leqslant N$, we 
have $|n \vx - \frac n q \vm|_\infty \leqslant q^{-1-\delta/2}$. Thus, for each 
$n\leqslant N$, each $n \vx$ is within $q^{-1-\delta/2}$ of the grid 
$\frac 1 q \bZ^d\subset (\bR/\bZ)^d$. Thus, they miss a box with side lengths 
$q^{-1} - 2 q^{-1-\delta/2}$. For $q$ sufficiently large, 
$q^{-1} - 2 q^{-1-\delta/2} \geqslant 1/2q$, so the discrepancy is 
bounded below by $2^{-d} q^{-d}$. Since $q^{\omega_0(\vx)-\delta} \leqslant 2 N$, 
the discrepancy at $N$ is bounded below by 
\[
	2^{-d} \left( (2 N)^{-\frac{1}{\omega_0(\vx)+\delta}}\right)^{-d} 
		= 2^{-d-\frac{d}{\omega_0(\vx)+\delta}} N^{-\frac{d}{\omega_0(\vx)+\delta}}
		= 2^{-d\left(1+\frac{1}{\omega_0(\vx)}\right)-\epsilon} N^{-\frac{d}{\omega_0(x)}-\epsilon} .
\]
\end{proof}






\section{Pathological Satake parameters for CM abelian varieties}

Give $(\bR/\bZ)^d$ the natural Haar measure normalized to have total mass one. 
Recall that for any $f\in L^1((\bR/\bZ)^d)$, the Fourier coefficients of $f$ 
are, for $\vm\in \bZ^d$ 
\[
	\widehat f(\vm) = \int_{(\bR/\bZ)^d} e^{2\pi i \langle \vm,\vx\rangle} \, \dd \vx ,
\]
where $\langle \vm,\vx\rangle = m_1 x_1 + \cdots + m_d x_d$ is the usual inner 
product. 

\begin{theorem}
Fix $\vx\in (\bR/\bZ)^d$ with $\omega_{d-1}(\vx)$ finite. Then 
\[
	\left| \sum_{n\leqslant N} e^{2\pi i \langle \vm, n \vx\rangle}\right| \ll |\vm|_\infty^{\omega_{d-1}(\vx) + \epsilon} 
\]
as $\vm$ ranges over $\bZ^d\smallsetminus 0$. 
\end{theorem}
\begin{proof}
From Lemma \ref{lem:bound-exp-sum} we know that 
\[
	\left| \sum_{n\leqslant N} e^{2\pi i \langle \vm, n \vx\rangle}\right| \ll \frac{1}{\dd(\langle \vm, \vx\rangle,\bZ)} ,
\]
and from Lemma \ref{lem:bound-distance}, we know that 
$\dd(\langle \vm,\vx\rangle, \bZ)^{-1} \ll |\vm|_\infty^{\omega_{d-1}(x)+\epsilon}$. 
The result follows. 
\end{proof}

\begin{theorem}
Let $\vx\in \bR^d$ with $\omega_{d-1}(\vx)$ finite. Then let 
$f\in L^1((\bR/\bZ)^d)$ with $\widehat f(0)=0$ and suppose the Fourier 
coefficients of $f$ satisfy the bound 
$|\widehat f(\vm)| \ll |\vm|_\infty^{-\frac{1}{d-1} - \omega_{d-1}(\vx)-\epsilon}$. 
Then 
\[
	\left| \sum_{n\leqslant N} f(n \vx)\right| \ll 1. 
\]
\end{theorem}
\begin{proof}
Write $f$ as a Fourier series:
\[
	f(x) = \sum_{\vm\in \bZ^d} \widehat f(\vm) e^{2\pi i \langle \vm,\vx\rangle} .
\]
Since $\widehat f(0)=0$, we can compute:
\begin{align*}
	\left| \sum_{n\leqslant N} f(n \vx)\right| 
		&= \left| \sum_{n\leqslant N} \sum_{\vm\in \bZ^d\smallsetminus 0} \widehat f(\vm) e^{2\pi i \langle \vm,\vx\rangle}\right| \\
		&\leqslant \sum_{\vm\in \bZ^d\smallsetminus 0} |\widehat f(\vm)| \left| \sum_{n\leqslant N} e^{2\pi i n \langle \vm,\vx\rangle}\right| \\
		&\ll \sum_{\vm\in \bZ^d\smallsetminus 0} |\vm|_\infty^{-\frac{1}{d-1} - \omega_{d-1}(\vx) - \epsilon} |m|^{\omega_{d-1}(\vx) + \epsilon/2} \\
		&\ll \sum_{\vm\in \bZ^d\smallsetminus 0} |\vm|_\infty^{-\frac{1}{d-1} - \epsilon/2} .
\end{align*}
The sum converges since the exponent is less than $-\frac{1}{d-1}$, and it 
doesn't depend on $N$, hence the result. 
\end{proof}


Let $p_1 = 2, p_2 = 3, p_3 = 5, \dots$ be an enumeration of the prime numbers. 
Let $\vy\in \bR^d$ with $y_1,\dots,y_d$ linearly independent over $\bQ$. The 
associated sequence of ``fake Satake parameters'' is 
\[
	\bx = (\vy, 2\vy, 3 \vy, 4 \vy, \dots) ,
\]
where we put $\vx_{p_n} = n \vy\mod \bZ^d$. By Theorem \ref{thm:jarnik}, we can 
arrange for $\omega_0(\vy) = w$ and $\omega_{d-1}(\vy) = d w + d - 1$. 

\begin{theorem}
The sequence $\bx$ is equidistributed in $(\bR/\bZ)^d$, with discrepancy 
decaying as 
\[
	\D_N(\bx) \ll N^{-\frac{1}{d w+d} + \epsilon} 
\]
and for which 
\[
	\D_N(\bx) = \Omega\left(N^{-\frac{d}{w} - \epsilon}\right) .
\]
However, for any $f\in C^\infty((\bR/\bZ)^d)$ with $\widehat f(0)=0$, the 
Dirichlet series  $L_f(\bx,s)$ satisfies the Riemann Hypothesis. 
\end{theorem}

Let's apply this theorem to abelian varieties with complex multiplication. Let 
$K/\bQ$ be a finite Galois extension, $A_{/K}$ an abelian variety with complex 
multiplication by $F$, defined over $K$. Let $\fa = \lie(A)$ be the Lie algebra 
of $A$; this is a $K$-vector space with natural $F$-action. The determinant of 
this action gives a homomorphism 
$\det_\fa\colon \R_{K/\bQ}\Gm \to \R_{F/\bQ}\Gm$; the image of this map is 
the motivic Galois group $G_A$ \cite{yu-2015}. Let 
$\N_{F/\bQ}\colon \R_{F/\bQ}\Gm \to \Gm$ be the norm map; then 
$G_A^1 = \im(\det_\fa)^{\N_{F/\bQ} = 1}$, and $\ST(A)$ is a maximal compact 
subgroup of $G_A^1(\bC)$. Every unitary representation of $\ST(A)$ is the 
restriction of a character of $G_A$; in fact it is the restriction of a 
character of $\R_{F/\bQ} \Gm$. 

Fix a rational prime $l$; then the Galois representation associated to $A$ is 
$\rho_l\colon G_\bQ \to G_A(\bQ_l)\subset (F\otimes\bQ_l)^\times$. For 
$\sigma\colon F\hookrightarrow \bC$, also 
write $\sigma$ for the corresponding character of $\R_{F/\bQ}\Gm$. Shimura and 
Taniyama proved \cite{serre-tate-1968} that there exists a Hecke character 
$\chi_\sigma\colon \bA_K^\times / K^\times \to \bC^\times$ such that 
$L^\alg(\sigma\circ \rho_l,s) = L(s,\chi_\sigma)$. The analytic $L$-function of 
$A$ is normalized as $L(A,s) = \prod_\sigma L^\alg(\sigma\circ \rho_l,s+1/2)$, 
so for $\omega_\sigma = \chi_\sigma \|\cdot\|^{-1/2}$, we have 
\[
	L(A,s) = \prod_{\sigma\colon F\hookrightarrow \bC} L(s,\omega_\sigma) .
\]
A character $r$ of $\R_{F/\bQ} \Gm$ is of the form $r = \sum a_\sigma \sigma$ 
with $a\in \bZ$. Write $\omega_r = \prod \omega_\sigma^{a_\sigma}$. Then 
\[
	L(r_\ast \rho_l,s) = L(s,\omega_r) .
\]
The Riemann Hypothesis holds for all $L(r_\ast \rho_l,s)$ if and only if it 
holds for each $L(\sigma \circ \rho_l,s) = L(s,\omega_\sigma)$. Since we 
already know analytic continuation of $\log L(s,\omega_\sigma)$ past 
$\Re = 1$, the Sato--Tate conjecture holds for $A$. However, the above theorem 
shows that even if each $L(r_\ast \rho_l,s)$ satisfies the Riemann Hypothesis, 
we may not immediately conclude that the Akiyama--Tanigawa conjecture holds for 
$A$. 

